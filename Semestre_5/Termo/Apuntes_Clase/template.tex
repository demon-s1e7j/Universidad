\documentclass{report}

\documentclass[12pt]{article}
\usepackage{array}
\usepackage{color}
\usepackage{amsthm}
\usepackage{eufrak}
\usepackage{lipsum}
\usepackage{pifont}
\usepackage{yfonts}
\usepackage{amsmath}
\usepackage{amssymb}
\usepackage{ccfonts}
\usepackage{comment} \usepackage{amsfonts}
\usepackage{fancyhdr}
\usepackage{graphicx}
\usepackage{listings}
\usepackage{mathrsfs}
\usepackage{setspace}
\usepackage{textcomp}
\usepackage{blindtext}
\usepackage{enumerate}
\usepackage{microtype}
\usepackage{xfakebold}
\usepackage{kantlipsum}
%\usepackage{draftwatermark}
\usepackage[spanish]{babel}
\usepackage[margin=1.5cm, top=2cm, bottom=2cm]{geometry}
\usepackage[framemethod=tikz]{mdframed}
\usepackage[colorlinks=true,citecolor=blue,linkcolor=red,urlcolor=magenta]{hyperref}

%//////////////////////////////////////////////////////
% Watermark configuration
%//////////////////////////////////////////////////////
%\SetWatermarkScale{4}
%\SetWatermarkColor{black}
%\SetWatermarkLightness{0.95}
%\SetWatermarkText{\texttt{Watermark}}

%//////////////////////////////////////////////////////
% Frame configuration
%//////////////////////////////////////////////////////
\newmdenv[tikzsetting={draw=gray,fill=white,fill opacity=0},backgroundcolor=none]{Frame}

%//////////////////////////////////////////////////////
% Font style configuration
%//////////////////////////////////////////////////////
\renewcommand{\familydefault}{\ttdefault}
\renewcommand{\rmdefault}{tt}

%//////////////////////////////////////////////////////
% Bold configuration
%//////////////////////////////////////////////////////
\newcommand{\fbseries}{\unskip\setBold\aftergroup\unsetBold\aftergroup\ignorespaces}
\makeatletter
\newcommand{\setBoldness}[1]{\def\fake@bold{#1}}
\makeatother

%//////////////////////////////////////////////////////
% Default font configuration
%//////////////////////////////////////////////////////
\DeclareFontFamily{\encodingdefault}{\ttdefault}{%
  \hyphenchar\font=\defaulthyphenchar
  \fontdimen2\font=0.33333em
  \fontdimen3\font=0.16667em
  \fontdimen4\font=0.11111em
  \fontdimen7\font=0.11111em}


\input{macros}
\input{letterfonts}

\title{\Huge{Some Class}\\Random Examples}
\author{\huge{Your Name}}
\date{}

\begin{document}

\maketitle
\newpage% or \cleardoublepage
% \pdfbookmark[<level>]{<title>}{<dest>}
\pdfbookmark[section]{\contentsname}{toc}
\tableofcontents
\pagebreak

\chapter{Apuntes}
\section{12-04-23}
Para este segundo caso tenemos que
\ex{}{Para dos fluidos con $u_1$ y  $u_2$ deseamos minimizar la energía por lo tanto escriba el diferencial total y encuentre la condición de equilibrio 

\sol

Para este caso lo primero que nos interesa es la derivada por lo que debemos tomar $U_1$ y $U_2$ y derivarlos con respecto a cada una de sus variables. El resultado es  \[
  dU = \frac{\partial U_1}{\partial S_1}dS_1 + \frac{\partial U_1}{V_1}dV_1 + \frac{\partial U_1}{N_1} + \frac{\partial U_2}{\partial S_2}dS_2 + \frac{\partial U_2}{V_2}dV_2 + \frac{\partial U_2}{N_2}
.\] Sin embargo, dadas las condiciones del problema nos queda
\begin{align*}
  dU &= \frac{\partial U_1}{\partial U_2}dS_1 - \frac{\partial U_2}{\partial S_2}\\
     &= \frac{\partial U_1}{\partial S_{1}}-\frac{\partial U_2}{\partial S_2} dS_1 = 0\\
  T_1&=T_2
.\end{align*}
}

Ahora bien en el ejemplo uno se trabaja con $S$ sin embargo, esto es una cosa suprema mente incomoda para el laboratorio. Por lo tanto queremos traducirlo a  $U(T,V,N)$ y  $U(T,P,N)$.
Para esto utilizaremos las transformaciones de Legendre

\subsection{Transformaciones de Legendre}
Partamos desde una función \[
y=y(x)=x^2+2;
.\] 
pero entonces podemos desarrollar de la siguiente manera
\begin{align*}
  \p&=\frac{dy}{dx}=2x \to x = \frac{p}{2}
.\end{align*}
Sin embargo, con esto no podemos distinguir entre funciones con distinto intercepto. Por lo tanto, vamos a tomar la pendiente y su intersección por lo tanto nos queda \[
  \psi(p) = y(p) + px(p)
.\] 

\end{document}

\documentclass{report}

\documentclass[12pt]{article}
\usepackage{array}
\usepackage{color}
\usepackage{amsthm}
\usepackage{eufrak}
\usepackage{lipsum}
\usepackage{pifont}
\usepackage{yfonts}
\usepackage{amsmath}
\usepackage{amssymb}
\usepackage{ccfonts}
\usepackage{comment} \usepackage{amsfonts}
\usepackage{fancyhdr}
\usepackage{graphicx}
\usepackage{listings}
\usepackage{mathrsfs}
\usepackage{setspace}
\usepackage{textcomp}
\usepackage{blindtext}
\usepackage{enumerate}
\usepackage{microtype}
\usepackage{xfakebold}
\usepackage{kantlipsum}
%\usepackage{draftwatermark}
\usepackage[spanish]{babel}
\usepackage[margin=1.5cm, top=2cm, bottom=2cm]{geometry}
\usepackage[framemethod=tikz]{mdframed}
\usepackage[colorlinks=true,citecolor=blue,linkcolor=red,urlcolor=magenta]{hyperref}

%//////////////////////////////////////////////////////
% Watermark configuration
%//////////////////////////////////////////////////////
%\SetWatermarkScale{4}
%\SetWatermarkColor{black}
%\SetWatermarkLightness{0.95}
%\SetWatermarkText{\texttt{Watermark}}

%//////////////////////////////////////////////////////
% Frame configuration
%//////////////////////////////////////////////////////
\newmdenv[tikzsetting={draw=gray,fill=white,fill opacity=0},backgroundcolor=none]{Frame}

%//////////////////////////////////////////////////////
% Font style configuration
%//////////////////////////////////////////////////////
\renewcommand{\familydefault}{\ttdefault}
\renewcommand{\rmdefault}{tt}

%//////////////////////////////////////////////////////
% Bold configuration
%//////////////////////////////////////////////////////
\newcommand{\fbseries}{\unskip\setBold\aftergroup\unsetBold\aftergroup\ignorespaces}
\makeatletter
\newcommand{\setBoldness}[1]{\def\fake@bold{#1}}
\makeatother

%//////////////////////////////////////////////////////
% Default font configuration
%//////////////////////////////////////////////////////
\DeclareFontFamily{\encodingdefault}{\ttdefault}{%
  \hyphenchar\font=\defaulthyphenchar
  \fontdimen2\font=0.33333em
  \fontdimen3\font=0.16667em
  \fontdimen4\font=0.11111em
  \fontdimen7\font=0.11111em}


\input{macros}
\input{letterfonts}

\title{\Huge{Maquinas de Turing}}
\author{\huge{SergiOS}}
\date{18-04-2023}

\begin{document}
\maketitle
\newpage% or \cleardoublepage
% \pdfbookmark[<level>]{<title>}{<dest>}
\pdfbookmark[section]{\contentsname}{toc}
\tableofcontents
\pagebreak

\chapter{Introducción}

Este texto fue escrito para el semillero de Computación Cuántica de la universidad de los Andes. Tiene como objetivo ser un texto auxiliar a la presentación realizada por los estudiantes Sergio David López y Sergio Montoya. Fue realizado siguiendo las secciones $3.1$ y $3.2$ del libro \textit{Introduction to the Theory of Computation} de \textit{Michael Sipser}.
\chapter{Definición}
\section{Presentación: Un relato a grandes rasgos de lo que es una maquina de Turing}
Una maquina de Turing es un modelo mas potente pero similar a un autómata. Su diferencia principal es que este cuenta con una memoria ilimitada y no restringida. Este modelo es mas util a la hora de aproximar una computadora convencional y soluciona muchos de los problemas que tienen los autómatas vistos previamente. Sin embargo, aun existen problemas que salen de las capacidades de una maquina de Turing.
\section{Intuición: Una muestra de que es una maquina de Turing en términos que un humano pueda entender}

Una maquina de Turing se compone de dos cosas
\begin{enumerate}
  \item Cinta: Esta cumple el trabajo de la memoria. Es ilimitada y no esta restringida. Inicialmente solo contiene el string de entrada y esta vacía en todos los otros lugares. Si la maquina necesitara guardar información puede escribirla en la cinta.
  \item Head: es esencialmente el elemento encargado de leer y modificar la cinta. Puede moverse arbitrariamente sobre esta.
\end{enumerate}

Las maquinas de Turing Tienen un estado inicial y se les determina arbitrariamente un estado final de aceptación o rechazo. En el caso de que estos estados no existan la maquina quedara computando para siempre.

\ex{}{Sea $M_1$ una maquina de Turing que recibirá el lenguaje  $B=\{w\#w|w\in\{0,1\}^*\}$. Queremos que $M_1$ acepte si su imput hace pertenece a B y lo rechace en caso contrario. Para que esto sea mas fácil pongámonos en el lugar de $M_1$. Nos encontramos encima de una lista inmensa de numero y te piden comprobar que esta lista consiste de dos numeros iguales separados por un $\#$. La estrategia mas obvia es ir en zig-zag a ambos lados del  $\#$ buscando que sean iguales o no.

  Si diseñamos a  $M_1$ para que trabaje de esa manera el algoritmo nos quedaria algo asi
  \begin{algorithm}[H]
    \KwIn{.. \textvisiblespace w\#w \textvisiblespace ..}
    \KwOut{.. \textvisiblespace x\#x \textvisiblespace ..}
    \tcc{Algoritmo que permite que $M_1$ acepte strings que pertenezcan a $B$}
    Inicie en la primera celda\;
    Lea la celda\;
    Marque la celda como $x$\;
    avance hasta encontrar un  $\#$ \;
    avance hasta que encuentre una celda diferente de $x$\;
    Compruebe la celda\;
    Si la celda es diferente rechace, de lo contrario marque como $x$\;
    Recorra de regreso hasta encontrar un $\#$\;
    Avance hasta encontrar una $x$\;
    Avance 1\;
    Si encuentra $\#$  lea la primera entrada que no sea $x$\;
    Si esta entrada  es vació acepte de lo contrario rechace\;
    En caso contrario repita desde el paso 2\;
  \end{algorithm}
}
\section{Definición Formal: Definición en términos que los matemáticos entiendan}

\dfn{Maquina de Turing}{
  Una maquina de Turing es una 7-tupla $(Q,\Sigma,\Gamma,\delta,q_0,q_{a},q_{r})$ donde $Q,\Sigma,\Gamma$ son conjuntos finitos y
  \begin{enumerate}
    \item $Q$ es el conjunto de estados
    \item $\Sigma$ es el alfabeto del input no incluyendo \textvisiblespace
    \item $\Gamma$ es el alfabeto de la cinta donde \textvisiblespace $\in \Gamma$ y $\Sigma \subseteq \Gamma$
    \item $\delta$ : $Q\times\Gamma \to Q\times\Gamma\times\{L,R\}$ es la función de transición
    \item $q_0\in Q$ es el estado inicial
    \item  $q_{accept}\in Q$ es el estado de aceptación
    \item $q_{reject}\in Q$ es el estado de rechazo donde $q_a\neq q_r$ 
  \end{enumerate}
}

El corazón de la definición de una maquina de Turing es $\delta$ o función de transformación. Esta es una función  $Q\times \Gamma \to Q\times\Gamma\times\left\{ L,R \right\} $ que nos indica para cada estado que debemos hacer.

\qs{}{Se deja al lector encontrar la representación en definición formal de la maquina de Turing mostrada en el ejemplo 1.}

\subsection{Configuraciones}

Una configuración de una maquina de Turing se define como una 3-tupla $(w,q,i)$ donde:
 \begin{itemize}
  \item $w$ es la cadena que se encuentra en la cinta en un momento dado.
  \item  $q$ es el estado actual de la maquina de Turing.
  \item $i$ es la posición actual de la cabeza de lectura/escritura en la cinta.
\end{itemize}

Se dice que $C_1$ produce $C_2$ si la maquina puede ir de $C_1$ a $C_2$ en un solo paso. 

\subsubsection{Ejemplo}
\begin{align*}
  1011q_{7}01111
.\end{align*}
\begin{itemize}
  \item la cinta es $101101111$
  \item el estado actual es $q_7$ 
  \item la cabeza se encuentra en el segundo 0
\end{itemize}
\section{Definiciones Adicionales}
\dfn{Turing-Reconocible}{Se llama a un lenguaje Turing-Reconocible si una maquina de Turing lo reconoce.}
\dfn{Turing-Decidible}{Se le llama a un lenguaje Turing-Decidible o simplemente decidible si una maquina de Turing decide. Es decir que nunca se queda corriendo infinitamente y por tanto siempre llega a un estado final o inicial.}
\chapter{Variantes de la Maquina de Turing}



\end{document}

  \documentclass[12pt]{exam}
\usepackage{amsthm}
\usepackage{libertine}
\usepackage[utf8]{inputenc}
\usepackage[margin=1in]{geometry}
\usepackage{amsmath,amssymb}
\usepackage{multicol}
\usepackage[shortlabels]{enumitem}
\usepackage{siunitx}
\usepackage{cancel}
\usepackage{graphicx}
\usepackage{pgfplots}
\usepackage{listings}
\usepackage{tikz}


\pgfplotsset{width=10cm,compat=1.9}
\usepgfplotslibrary{external}
\tikzexternalize

\newcommand{\class}{Moderna - Magistral} % This is the name of the course 
\newcommand{\examnum}{Preguntas} % This is the name of the assignment
\newcommand{\examdate}{\today} % This is the due date
\newcommand{\timelimit}{}





\begin{document}
\pagestyle{plain}
\thispagestyle{empty}

\noindent
\begin{tabular*}{\textwidth}{l @{\extracolsep{\fill}} r @{\extracolsep{6pt}} l}
	\textbf{\class} & \textbf{Name:} & \textit{Sergio Montoya}\\ %Your name here instead, obviously 
	\textbf{\examnum} &&\\
	\textbf{\examdate} &&
\end{tabular*}\\
\rule[2ex]{\textwidth}{2pt}
% ---

\section{Mis Soluciones}
\begin{enumerate}
  \item Importancia del Oscilador Armónico.
  \item Dad que $\psi_0,\psi_1$ son soluciones ortogonales del TISE cual es la solución General:
    \begin{enumerate}
      \item \textbf{Propuesta 1:}
	Una solución general seria combinación lineal de $\psi_0$ y $\psi_1$ dado que por ser ortogonales deben ser linealmente independientes y por (Un teorema de álgebra lineal que no recuerdo el nombre de quien tiene)
    \end{enumerate}
  \item En este segundo caso para un par $a,b$ definidos la probabilidad estaría descrita por  \[
  |\psi_2|^2=|a|^2+|b|^2
  .\]
\end{enumerate}
\section{Soluciones Reales}

\end{document}

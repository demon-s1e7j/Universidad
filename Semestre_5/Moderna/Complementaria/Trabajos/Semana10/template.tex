\documentclass{report}

\documentclass[12pt]{article}
\usepackage{array}
\usepackage{color}
\usepackage{amsthm}
\usepackage{eufrak}
\usepackage{lipsum}
\usepackage{pifont}
\usepackage{yfonts}
\usepackage{amsmath}
\usepackage{amssymb}
\usepackage{ccfonts}
\usepackage{comment} \usepackage{amsfonts}
\usepackage{fancyhdr}
\usepackage{graphicx}
\usepackage{listings}
\usepackage{mathrsfs}
\usepackage{setspace}
\usepackage{textcomp}
\usepackage{blindtext}
\usepackage{enumerate}
\usepackage{microtype}
\usepackage{xfakebold}
\usepackage{kantlipsum}
%\usepackage{draftwatermark}
\usepackage[spanish]{babel}
\usepackage[margin=1.5cm, top=2cm, bottom=2cm]{geometry}
\usepackage[framemethod=tikz]{mdframed}
\usepackage[colorlinks=true,citecolor=blue,linkcolor=red,urlcolor=magenta]{hyperref}

%//////////////////////////////////////////////////////
% Watermark configuration
%//////////////////////////////////////////////////////
%\SetWatermarkScale{4}
%\SetWatermarkColor{black}
%\SetWatermarkLightness{0.95}
%\SetWatermarkText{\texttt{Watermark}}

%//////////////////////////////////////////////////////
% Frame configuration
%//////////////////////////////////////////////////////
\newmdenv[tikzsetting={draw=gray,fill=white,fill opacity=0},backgroundcolor=none]{Frame}

%//////////////////////////////////////////////////////
% Font style configuration
%//////////////////////////////////////////////////////
\renewcommand{\familydefault}{\ttdefault}
\renewcommand{\rmdefault}{tt}

%//////////////////////////////////////////////////////
% Bold configuration
%//////////////////////////////////////////////////////
\newcommand{\fbseries}{\unskip\setBold\aftergroup\unsetBold\aftergroup\ignorespaces}
\makeatletter
\newcommand{\setBoldness}[1]{\def\fake@bold{#1}}
\makeatother

%//////////////////////////////////////////////////////
% Default font configuration
%//////////////////////////////////////////////////////
\DeclareFontFamily{\encodingdefault}{\ttdefault}{%
  \hyphenchar\font=\defaulthyphenchar
  \fontdimen2\font=0.33333em
  \fontdimen3\font=0.16667em
  \fontdimen4\font=0.11111em
  \fontdimen7\font=0.11111em}


\input{macros}
\input{letterfonts}

\title{\Huge{Some Class}\\Random Examples}
\author{\huge{Your Name}}
\date{}

\begin{document}

\maketitle
\newpage% or \cleardoublepage
% \pdfbookmark[<level>]{<title>}{<dest>}
\pdfbookmark[section]{\contentsname}{toc}
\tableofcontents
\pagebreak

\chapter{}
\section{Recuento de lo ocurrido}
Lo que hemos visto es mecánica cuántica que nos sirve para  mostrar una realidad física. Para esto, tomamos en cuenta la ecuación de Schr\"{o}dinger que encontramos como solución en 
\begin{enumerate}
  \item Pozo Infinito
  \item Pozo finito
  \item Oscilador armónico simple
\end{enumerate}
La primera solución que se dio fue la del átomo mas simple, el átomo de hidrógeno.
\subsection{Átomo de hidrógeno}
Un atomo de hidrogeno es esencialemente una particula oscilando a otra. Segun la teoria clasica este no deberia existir pero si lo hacen. Desarrollemos.

En un atomo la energía potencial que funciona es el potencial de coulomb que funciona para los hidrógenoides.
\begin{equation}
  V = \frac{-ze^2}{4\pi\epsilon_0|\Vec{r}_2-\Vec{r}_1}
.\end{equation}
Ahora bien, respecto al origen tenemos que 
\begin{equation}
  -\frac{h^2}{2m_1}\nabla^2_1 \psi - \frac{h^2}{2m_2}\nabla^2_2\psi+V(\Vec{r}_2-\Vec{r}_1)\psi=E\psi
.\end{equation}
Pero esto no tendria sentido pues no se puede conseguir dado que dependemos de la distancia entre $\Vec{r}_2-\Vec{r}_1$

Ahora bien con esto la distancia al centro de masa nos queda
 \begin{equation}
   R=\frac{Mr_1+m_2r}{M}=r_1+\frac{m_2}{M}r\sim r_1=R-\frac{m_2}{M}r
.\end{equation}
Masa reducida es $\mu = \frac{m_1m_2}{Mo}$ Con esto la ecuación de Schr\"{o}dinger nos queda \[
  \frac{h^2}{2M}\nabla_R^2 \psi_{cM}= E_{cM}\psi_cN \to \psi_{cM}=Ae^{i(k\cdotx-\omega t}
.\] 

Ahora bien para solucionar esto mas fácilmente combine utilizar coordenadas esféricas. Para este necesitamos $\nabla^2 \psi$ el cual es
\begin{equation}
  \nabla ^2 \psi = \frac{1}{r^2}\frac{\partial}{\partial r}\left( r^2\frac{ \partial\psi}{\partial r}  \right) + \frac{1}{r^2\sin^2\theta}\frac{\partial^2\psi}{\partiall\phi}+ \frac{1}{r^2\sin\theta}\frac{\partial}{\partial \theta} 
.\end{equation}

\end{document}

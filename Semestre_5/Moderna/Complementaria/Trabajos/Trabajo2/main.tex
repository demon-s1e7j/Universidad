\documentclass[12pt]{exam}
\usepackage{amsthm}
\usepackage{libertine}
\usepackage[utf8]{inputenc}
\usepackage[margin=1in]{geometry}
\usepackage{amsmath,amssymb}
\usepackage{multicol}
\usepackage[shortlabels]{enumitem}
\usepackage{siunitx}
\usepackage{cancel}
\usepackage{graphicx}
\usepackage{pgfplots}
\usepackage{listings}
\usepackage{tikz}


\pgfplotsset{width=10cm,compat=1.9}
\usepgfplotslibrary{external}
\tikzexternalize

\newcommand{\class}{Moderna - Complementaria} % This is the name of the course 
\newcommand{\examnum}{Taller 2} % This is the name of the assignment
\newcommand{\examdate}{03/02/2023} % This is the due date
\newcommand{\timelimit}{}





\begin{document}
\pagestyle{plain}
\thispagestyle{empty}

\noindent
\begin{tabular*}{\textwidth}{l @{\extracolsep{\fill}} r @{\extracolsep{6pt}} l}
\textbf{\class} & \textbf{Name:} & \textit{Sergio Montoya Ramirez}\\ %Your name here instead, obviously 
	\textbf{\examnum} && \textit{David Pachon Ballen}\\
\textbf{\examdate} &&\\
\end{tabular*}\\
\rule[2ex]{\textwidth}{2pt}
% ---




\begin{enumerate} %You can make lists!
	\item Halle la velocidad que a la que debe ir el cohete para que los astronautas percivan 40 años de ida y vuelta.

		Primero, tenemos que saber que $L = \gamma L'$ donde $L$ es la distancia medida desde el planeta ($40$ años$\cdot c$), $\gamma$ es el factor de cambio y $L'$ la distancia de los astronautas. Ademas, sabemos que $L' = v\cdot t'$ donde V es la velocidad y $t'$ es el tiempo percivido por los astronautas ($40$ añs)
		\begin{align*}
			& 40\text{ años}\cdot c = \gamma v \cdot 40\text{ años}\\
			& c = \gamma \cdot v\\
			& \gamma = \frac{1}{\sqrt{1-\frac{v^2}{c^2}}}\\
			& c = \frac{v}{\sqrt{1-\frac{v^2}{c^2}}}\\
			& c(\sqrt{1-\frac{v^2}{c^2}}) = v\\
			& c^2(1-\frac{v^2}{c^2}) = v^2\\
			& c^2 - v^2 = v^2\\
			& c^2 = 2v^2\\
			& v^2 = \frac{c^2}{2}\\
			& v = \frac{c}{\sqrt{2}} = \frac{c\sqrt{2}}{2}
		\end{align*}

		Halle el tiempo que vivieron los cientificos en tierra.

		Para esto vamos a hallar $\gamma$ (que ya podemos dado que ya tenemos la velocidad) y utilizaremos el hecho de que $t = \gamma\cdot t'$ 
		\begin{align*}
			& \gamma = \frac{1}{\sqrt{1-\frac{v^2}{c^2}}}\\
			& \gamma = \frac{1}{\sqrt{1-\frac{\left(\frac{c}{\sqrt{2}}\right)^2}{c^2}}}\\
			& \gamma = \frac{1}{\sqrt{1 - \frac{\frac{c^2}{2}}{c^2}}}\\
			& \gamma = \frac{1}{\sqrt{1 - \frac{1}{2}}}\\
			& \gamma = \frac{1}{\sqrt{\frac{1}{2}}}\\
			& \gamma = \sqrt{2}\\
			& t = \gamma t'\\
			& t = \sqrt{2}\cdot40 \text{ años}
		\end{align*}
	\item Hallar la velocidad relativa de B y C vistos desde A.
		Primero asumimos que la velocidad de A es $0.6c$ y la de B $-0.6c$ esto dado que tienen direcciónes opuestas y con esta configuración se nos facilitan los calculos.
		\begin{enumerate}
			\item $V_{ab}$
				\begin{align*}
					& V_{ab} = \frac{V - U}{1-\frac{V\cdot U}{c^2}}\\
					& = \frac{0.6c - 0}{1 - \frac{0.6c\cdot 0}{c^2}}\\
					& = \frac{0.6c}{1}\\
					& V_{ab} = 0.6c
				\end{align*}
			\item $V_{ac}$
				\begin{align*}
					& V_{ac} = \frac{V - U}{1 - \frac{v\cdot U}{c^2}}\\
					& V_{ac} = \frac{0.6c - (-0.6c)}{1-\frac{(0.6c)(-0.6c)}{c^2}}\\
					& V_{ac} = \frac{1.2c}{1-(-0.36)}\\
					& V_{ac} = \frac{1.2c}{1.36}\\
					& V_{ac} = 0.882c
				\end{align*}
		\end{enumerate}
\end{enumerate}




\end{document}

\documentclass[12pt]{exam}
\usepackage{amsthm}
\usepackage{libertine}
\usepackage[utf8]{inputenc}
\usepackage[margin=1in]{geometry}
\usepackage{amsmath,amssymb}
\usepackage{multicol}
\usepackage[shortlabels]{enumitem}
\usepackage{siunitx}
\usepackage{cancel}
\usepackage{graphicx}
\usepackage{pgfplots}
\usepackage{listings}
\usepackage{tikz}


\pgfplotsset{width=10cm,compat=1.9}
\usepgfplotslibrary{external}
\tikzexternalize

\newcommand{\class}{Complementaria Moderna} % This is the name of the course 
\newcommand{\examnum}{Taller 6} % This is the name of the assignment
\newcommand{\examdate}{\today} % This is the due date
\newcommand{\timelimit}{}





\begin{document}
\pagestyle{plain}
\thispagestyle{empty}

\noindent
\begin{tabular*}{\textwidth}{l @{\extracolsep{\fill}} r @{\extracolsep{6pt}} l}
	\textbf{\class} & \textbf{Name:} &\textit{Sergio Montoya Ramírez}\\ %Your name here instead, obviously 
	\textbf{\examnum} && \\
\textbf{\examdate} &&\\
\end{tabular*}\\
\rule[2ex]{\textwidth}{2pt}
% ---
\begin{enumerate}
%%%%%%%%%%%%%%%%%%%%%%%%%%%%%%%%%%%%%%%%%%%%%%%%%%%%%%%%%%%%%%%%%%%%%%%%%%%%%%%%%%%%%%%%%%%%%%%%%%%%%%%%%%%%%%%%%%%%%%%%%%%%%%%%%%%%%%%%%%
  \item Para este punto utilizaremos las notas sobre osciladores y preguntas realizadas al monitor.

    Para comenzar, se tomaron los operadores escalera con los cuales entonces se define
    \begin{align*}
      \hat{x} &=  \sqrt{\frac{\hbar}{2m\omega}}(a^++a)\\
      \hat{p} &= i\sqrt{\frac{m\omega\hbar}{2}}(a^+-a) \\
      \hat{p}^2 &= -\frac{m\omega\hbar}{2}(a^+a^+-a^+a-aa^++aa)\\
      \hat{x}^2 &= \frac{\hbar}{2m\omega}(a^+a^++a^+a+aa^++aa)\\
    .\end{align*}

    Una vez tenemos estos resultados, podemos aprovecharlos para calcular $\sigma_x$ y  $\sigma_p$ sin embargo para esto se da un caso curioso. Para averiguar de que se trata encontremos $<x>^2$
     \begin{align*}
       <x>^2 &= \left( \sqrt{\frac{\hbar}{2m\omega}}(a^+ + a)  \right)^2 \\
	     &= \frac{\hbar}{2m\omega}(a^++a)(a^++a) \\
	     &= \frac{\hbar}{2m\omega}(a^+a^+ + a^+a + aa^+ + aa)\\
       <x^2> &= \frac{\hbar}{2m\omega}(a^+a^+ + a^+a + aa^+ + aa)
    .\end{align*}
    Lo mismo ocurre para $<p>$ por lo tanto
     \begin{align*}
      \sigma_x &= \sqrt{<x>^2+<x^2>} \\
	 &= \sqrt{2<x>^2}\\
	 &= \sqrt{2}<x>\\
      \sigma_p &= \sqrt{2}<p> 
    .\end{align*}
    Ahora, para encontrar si se cumple el principio de incertidumbre solo hace falta multiplicar ambos lo cual nos da
    \begin{align*}
      \sigma_x \sigma_p = \sqrt{2}<x>\sqrt{2}<p> = 2\hat{x}\hat{p}  
    .\end{align*}
%%%%%%%%%%%%%%%%%%%%%%%%%%%%%%%%%%%%%%%%%%%%%%%%%%%%%%%%%%%%%%%%%%%%%%%%%%%%%%%%%%%%%%%%%%%%%%%%%%%%%%%%%%%%%%%%%%%%%%%%%%%%%%%%%%%%%%%%%%
  \item Para este punto utilizaremos las notas sobre osciladores de la complementaria.

    Para comenzar utilizaremos la definición de $\psi_n$ encontrada en la ultima parte de las notas. Con esta encontramos que  \[
      \psi_n = \frac{1}{\sqrt{n!}} (\hat{a}^+)^n\psi_0
    .\] Con esto podemos encontrar que \[
    \psi_1 = (\hat{a}^2)^n\left( \frac{m\omega}{\pi\hbar} \right)^{\frac{1}{4}}e^{\frac{-m\omega}{\hbar}\frac{x^2}{2}}
    .\]
    \begin{enumerate}
    \item Por lo tanto podemos partir desde \[\psi(x,t=0) = A \left[ 3\left(\left( \frac{m\omega}{\pi\hbar} \right)^{\frac{1}{4}}e^{-\frac{m\omega}{\hbar}\frac{x^2}{2}} \right) + 4\left( (\hat{a}^2)^n\left( \frac{m\omega}{\pi\hbar} \right)^{\frac{1}{4}}e^{\frac{-m\omega}{\hbar}\frac{x^2}{2}} \right)  \right]\] 
      lo que se puede resumir en \[
      = A\left[ \left(\left( \frac{m\omega}{\pi\hbar} \right)^{\frac{1}{4}}e^{-\frac{m\omega}{\hbar}\frac{x^2}{2}} \right)\left(3 + 4\left( (\hat{a}^2)^n \right)\right)  \right]\]

    \end{enumerate}
%%%%%%%%%%%%%%%%%%%%%%%%%%%%%%%%%%%%%%%%%%%%%%%%%%%%%%%%%%%%%%%%%%%%%%%%%%%%%%%%%%%%%%%%%%%%%%%%%%%%%%%%%%%%%%%%%%%%%%%%%%%%%%%%%%%%%%%%%%
  \item Por facilidad representaremos $|\alpha> = \bigl(\begin{smallmatrix}
i \\ -2 \\-i
\end{smallmatrix} \bigr)$ y de manera similar $|\beta> = \bigl(\begin{smallmatrix}
i\\0\\2
\end{smallmatrix}\bigr)$
\begin{enumerate}
  \item Para encontrar los bras solo debemos convertir los kets en vectores fila y conjugarlos lo que nos da
    \begin{align*}
      <\alpha| &= \begin{pmatrix} 
	-i & -2 & i
      \end{pmatrix}\\
	<\beta| &= \begin{pmatrix} 
	-i & 0 & 2
      \end{pmatrix} 
    .\end{align*}
  \item Para conseguir los resultados que nos piden solo debemos multiplicar las cosas que nos piden.
    \begin{align*}
      <\alpha|\beta> &= \begin{pmatrix} -i & -2 & i \end{pmatrix} \begin{pmatrix} i \\ 0 \\ 2 \end{pmatrix} = 1+2i\\
      <\beta|\alpha> &= \begin{pmatrix} -i & 0 & 2 \end{pmatrix} \begin{pmatrix} i \\ -2 \\ -i \end{pmatrix} = 1-2i
    .\end{align*}
    Como se pueden ver son conjugados.
  \item Definimos el operador \[\hat{A}=|\alpha><\beta| = \begin{pmatrix}
i \\ -2 \\-i
\end{pmatrix}\begin{pmatrix} 
-i & 0 & 2
\end{pmatrix} \]

    Por lo tanto, lo único que necesitamos es multiplicar ambas cosas. El resultado de esto es \[
    \begin{pmatrix} 
      1 & 0 & i2\\
      2i & 0 & -4\\
      1 & 0 & i2
    \end{pmatrix} 
    .\] 
\end{enumerate}
%%%%%%%%%%%%%%%%%%%%%%%%%%%%%%%%%%%%%%%%%%%%%%%%%%%%%%%%%%%%%%%%%%%%%%%%%%%%%%%%%%%%%%%%%%%%%%%%%%%%%%%%%%%%%%%%%%%%%%%%%%%%%%%%%%%%%%%%%%
\item Tenemos que $|1>$ y  $|2>$ son una base ortonormal. Por lo tanto, $|1><1|$ es una multiplicación de un vector columna por uno fila con sus coordenadas conjugadas. En el caso de los primeros términos 
%%%%%%%%%%%%%%%%%%%%%%%%%%%%%%%%%%%%%%%%%%%%%%%%%%%%%%%%%%%%%%%%%%%%%%%%%%%%%%%%%%%%%%%%%%%%%%%%%%%%%%%%%%%%%%%%%%%%%%%%%%%%%%%%%%%%%%%%%%
  \item 
    \begin{enumerate}
      \item Para conseguir el resultado deseado, partamos desde el Hamiltoniano y diagonal icemos para encontrar su valor propio
	\begin{align*}
	  \begin{pmatrix} 
	    1 & 0 & 0\\
	    0 & 2 & 0\\
	    0 & 0 & 2
	  \end{pmatrix} \\
	  det(H-\alpha I) &= (1-\alpha)(2-\alpha)(2-\alpha) = 0\\
	  \alpha &= 1\\
	  \alpha &= 2\\
	  \alpha &= 2
      \end{align*}
Este $\alpha$ es para el $\hat{H}$ sin constante $\lambda$ pero lo único que esto significa es que los resultados obtenidos los multipliquemos por lambda

     \item Para cada uno de los casos solo debemos aplicar.

      \begin{align*}
        A|v> = \begin{pmatrix} 
	  0 & \lambda & 0\\
	  \lambda & 0 & 0\\
	  0 & 0 & 2\lambda
	\end{pmatrix}\begin{pmatrix} 
	c_1 \\ c_2\\ c_3
	\end{pmatrix}=\begin{pmatrix} \lambda c_2 \\ \lambda c_1\\ 2\lambda c_3 \end{pmatrix}=\lambda\begin{pmatrix} c_2 \\ c_1 \\ 2c_3  \end{pmatrix}\\
      .\end{align*}

      Por otro lado para $B$ tenemos
      \begin{align*}
        B|v> = \begin{pmatrix} 
	  2\mu & 0 & 0 \\
	  0 & 0 & \mu\\
	  0 & \mu & 0
	\end{pmatrix}\begin{pmatrix} c_1\\ c_2 \\ c_3 \end{pmatrix} = \begin{pmatrix} 
	2\mu c_1 \\
	\mu c_3 \\
	\mu c_2
	\end{pmatrix} = \mu \begin{pmatrix} 2c_1 \\ c_3 \\ c_2 \end{pmatrix} 
      .\end{align*}
    \end{enumerate}
\end{enumerate}

\end{document}

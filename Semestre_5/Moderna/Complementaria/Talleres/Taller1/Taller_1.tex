\documentclass[12pt]{exam}
\usepackage{amsthm}
\usepackage{libertine}
\usepackage[utf8]{inputenc}
\usepackage[margin=1in]{geometry}
\usepackage{amsmath,amssymb}
\usepackage{multicol}
\usepackage[shortlabels]{enumitem}
\usepackage{siunitx}
\usepackage{cancel}
\usepackage{graphicx}
\usepackage{pgfplots}
\usepackage{listings}
\usepackage{tikz}


\pgfplotsset{width=10cm,compat=1.9}
\usepgfplotslibrary{external}
\tikzexternalize

\newcommand{\class}{Complementaria Moderna} % This is the name of the course 
\newcommand{\examnum}{Taller 1} % This is the name of the assignment
\newcommand{\examdate}{\today} % This is the due date
\newcommand{\timelimit}{}





\begin{document}
\pagestyle{plain}
\thispagestyle{empty}

\noindent
\begin{tabular*}{\textwidth}{l @{\extracolsep{\fill}} r @{\extracolsep{6pt}} l}
	\textbf{\class} & \textbf{Name:} & \textit{David Santigo Pachon Ballen} \\ %Your name here instead, obviously 
	\textbf{\examnum} && \textit{Sergio Montoya Ramirez} \\
\textbf{\examdate} &&\\
\end{tabular*}\\
\rule[2ex]{\textwidth}{2pt}
% ---



\begin{enumerate}
	\item \begin{enumerate}
			\item Ecuaciones de Maxwell en forma integral:
				\begin{enumerate}
					\item Ley de Gauss
						\begin{align*}
							\Phi_E = \oint \Vec{E}\cdot d\Vec{A}=0
						\end{align*}
					\item Let de Gauss para el campo mágnetico
						\begin{align*}
							& \oint \Vec{B}\cdot d\Vec{s} = 0
						\end{align*}
					\item Ley de Faraday
						\begin{align*}
							&\oint \Vec{E}\cdot d\Vec{l} = -\frac{d\Vec{B}}{dt}\\ 
						\end{align*}
					\item Ley de Ampere
						\begin{align*}
							&\oint \Vec{B}\cdot d\Vec{l} = \mu_0\varepsilon_0\frac{d\Vec{E}}{dt}
						\end{align*}
				\end{enumerate}
			\item Leyes de Maxwell vectoriales
				\begin{enumerate}
					\item Ley de Gauss:
						\begin{align*}
							& \oint \Vec{E}\cdot d\Vec{A} = \int (\nabla\cdot \Vec{E})dV  \text{Ley de Gauss}\\
							& \nabla \cdot \Vec{E} = 0\text{ Hipotesis }\Box\\
						\end{align*}
					\item Ley de Gauss para el magnetismo:
						\begin{align*}
							&\oint \Vec{B}\cdot d\Vec{s} = \int (\nabla\cdot\Vec{B})dV\text{Ley  de Gauss}\\
							&\nabla\cdot\Vec{B} = 0 \text{ Hipotesis } \Box\\
						\end{align*}
					\item Ley de Faraday:
						\begin{align*}
							&\oint \Vec{E}\cdot d\Vec{l} = \int (\nabla\times\Vec{E})d\Vec{a}\\
							&\nabla\times\Vec{E} = - \frac{d\Vec{B}}{dt}\\
						\end{align*}
					\item Ley de Ampere:
						\begin{align*}
							&\oint \Vec{B}\cdot d\Vec{l} = \int (\nabla\times\Vec{B})d\Vec{a}\\
							&\nabla\times\Vec{E} = \mu\varepsilon_0\frac{d\Vec{E}}{dt}\\
						\end{align*}
					\item Aclaración Importante:
						En este punto varias integrales fueron evitadas y retiradas como si nunca hubieran estado esto se da gracias a que la premisa nos da un espacio libre (Es decir Vacio) y por ende seria absurdo intentar Hacer una integral de volumen en el.
				\end{enumerate}
	\end{enumerate}
\item \begin{enumerate}
		\item Para mostrar esto solo debemos mostrar que la aceleración es la misma en ambos casos (Mas precisamente que difieren
			unicamente por una matriz de rotación). Para esto partimos de que la aceleración es la doble integral de la posición con
			respecto al tiempo. Mostremos primero la doble derivada de x
			\begin{align*}
				& \frac{d^2x}{d^2t} = \frac{d^2 x}{dt}
			\end{align*}
			No podemos poner algo distinto pues no sabemos que función es x, pero si sabemos que x depende de t.

			Ahora bien por otro lado tenemos que $x' = Rx + vt$ y con esto vamos a encontrar su segunda derivada.
			\begin{align*}
				&x' = Rx + vt\\
				&\frac{dx'}{dt} = R\frac{dx}{dt} + v\\
				&\frac{d^2x}{d^2t} = R\frac{d^2x}{d^2t}\\
			\end{align*}
			Como se puede ver la unica diferencia real es solamente la matriz de rotación. Dado que estos son un conjunto de
			transformaciones que mantiene la norma de los vectores entonces estas leyes son invariantes.
		\item Para llegar a este resultado debemos conseguir mostrar que
			\begin{align*}
				& \frac{1}{c^2}\frac{\partial^2 E}{\partial t^2} - \frac{\partial^2 E}{\partial x^2} \neq  \frac{1}{c^2}\frac{\partial^2 E}{\partial t^2} - \frac{\partial^2 E}{\partial x'^2} 
			\end{align*}
		Para hacer esto vamos a tomar x' como $x'=x+vt$ Ya que si tomaramos lo mismo pero con el rotacional solo seria complicarnos y no aportaria realmente nada a la demostracion. Ahora bien para la primera derivada vamos a tomar la ley de la cadena en calculo vectorial y luego utilizaremos el hecho de que $\frac{\partial}{\partial x} \frac{\partial}{\partial x} = \frac{\partial^2}{\partial x^2}$ 
		\begin{align*}
			&\frac{\partial}{\partial x'} = \frac{\partial}{\partial x}\frac{\partial x}{\partial x'} + \frac{\partial}{\partial t}\frac{\partial}{\partial x'}\\
			&\frac{\partial}{\partial x'} = \frac{\partial}{\partial x} - \frac{1}{v} \frac{\partial}{\partial t}\\
			&\frac{\partial^2}{\partial x'^2} = \frac{\partial^2}{\partial x^2} - \frac{2}{v}\frac{\partial}{\partial x \partial t} + \frac{\partial^2}{\partial t^2}
		\end{align*}
		Por lo tanto podemos concluir que $$\frac{\partial^2}{\partial x^2} \neq \frac{\partial^2}{\partial x'^2}$$
		\item Esto se debe a que las rotaciones son el unico conjunto de transformaciones que mantienen la norma de los vectores
			constante y por tanto no afecta los resultados de los calculos.
		\item Para lograr esto debemos tomar las diferencias que nos encontramos en el punto b y debemos hacer que sea igual a 0
			de la siguiente manera.
			\begin{align*}
				&\frac{\partial^2}{\partial t^2} - \frac{2}{v}\frac{\partial}{\partial x\partial t} = 0\\
				&\int (\frac{\partial^2}{\partial t^2} - \frac{2}{v}\frac{\partial}{\partial x\partial t}) dt = 0\\
				&\frac{\partial}{\partial t} - \frac{2}{v}\frac{\partial}{\partial x} = 0\\
				&\text{Note que esto es }\nabla\\
				&\frac{2}{v} (\nabla\cdot\Vec{E}) = 0\\
				&\text{y esta es la ley de Gauss que deberia ocurrrir para que se conserve.}
			\end{align*}
\end{enumerate}
\end{enumerate}


\end{document}

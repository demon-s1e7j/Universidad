  \documentclass[12pt]{exam}
\usepackage{amsthm}
\usepackage{libertine}
\usepackage[utf8]{inputenc}
\usepackage[margin=1in]{geometry}
\usepackage{amsmath,amssymb}
\usepackage{multicol}
\usepackage[shortlabels]{enumitem}
\usepackage{siunitx}
\usepackage{cancel}
\usepackage{graphicx}
\usepackage{pgfplots}
\usepackage{listings}
\usepackage{tikz}


\pgfplotsset{width=10cm,compat=1.9}
\usepgfplotslibrary{external}
\tikzexternalize

\newcommand{\class}{Moderna - Complementaria} % This is the name of the course 
\newcommand{\examnum}{Quiz 3} % This is the name of the assignment
\newcommand{\examdate}{\today} % This is the due date
\newcommand{\timelimit}{}





\begin{document}
\pagestyle{plain}
\thispagestyle{empty}

\noindent
\begin{tabular*}{\textwidth}{l @{\extracolsep{\fill}} r @{\extracolsep{6pt}} l}
	\textbf{\class} & \textbf{Name:} & \textit{Sergio Montoya}\\ %Your name here instead, obviously 
	\textbf{\examnum} &&\\
	\textbf{\examdate} &&
\end{tabular*}\\
\rule[2ex]{\textwidth}{2pt}
% ---
\begin{enumerate}
  \item 
    \begin{enumerate}
      \item En este caso nos interesa encontrar $L_z$ para el cual tenemos
        \begin{align*}
          L_z |\ell,m_\ell> &= m_\ell \hbar |\ell,m_\ell>\\
          L_z |\psi> &= A\left[L_z|1,0> + 2L_z|1,-1>\right]\\
          &= A\left[0\cdot\hbar |1,0> + 2(-1)\hbar |1,-1>\right]\\
          &= -2A\hbar |1,-1>\\
          \left<\psi\right|&= A\left[ \left<1,0\right|+2\left< 1, -1\right| \right] \\
          L_z = <\psi|L_z|\psi> &= -2A^2\hbar <1,0|1,-1> + 4A^2\hbar<1,-1|1,-1>\\
          &= 4A^2\hbar = \frac{4}{5}\hbar
        \end{align*}
      \item En este caso nos interesa $L^2$ para lo cual tenemos
        \begin{align*}
          L^2 |\ell,m_\ell> &= \hbar^2\ell(\ell+1)|\ell,m_\ell>\\ 
          L^2 |\psi> &= A\left[L^2|1,0> + 2L^2|1,-1>\right]\\
          &= A\left[\hbar^2(2)|1,0>+4\hbar^2|1,-1>\right]\\
          L^2 &= <\psi|L^2|\psi>\\
          &= A^2 \left[ 2\hbar^2 \left< 1,0 | 1,0 \right> + 4\hbar^2\left< 1, -1 | 1, 0 \right> + 4\hbar^2 \left< 1,0 | 1,/1 \right>  + 8\hbar^2\left< 1, -1 | 1, -1 \right>  \right]\\
          &= A^2 (2\hbar^2 + 0 + 0 + 8\hbar^2) = \frac{10\hbar^2}{5} = 2\hbar^2
        \end{align*}
      \item En este caso nos interesa $\left<L_x\right>$ para lo cual tenemos
        \begin{align*}
          L_\pm |\ell,m_\ell> &= \sqrt{\ell(\ell+1)-m_\ell(m_\ell\pm 1)}\hbar|\ell,m_\ell\pm 1>\\
          <L_x> &= \frac{1}{2}(<L_+>+<L_->)\\
          <L_\pm> &= <\psi|L_\pm|\psi>\\
        \end{align*}
        Ahora con esto podemos desarrollar cada lado por aparte
        \begin{align*}
          L_+|\psi> &= A[L_+|1,0>+2L_+|1,-1>]\\
          &= A[\sqrt{2}\hbar|1,1>+2\sqrt{2}\hbar|1,0>]\\
          L_+ = A^2[\sqrt{2}\hbar<1,0|1,1>+&2\sqrt{2}\hbar<1,0|1,0>+2\sqrt{2}\hbar<1,-1|1,1>+4\sqrt{2}\hbar<1,-1|1,0>]\\
          &= A^22\sqrt{2}\hbar = \frac{2\sqrt{2}}{5}\hbar
        \end{align*}
        Ahora necesitamos conseguir $L_-$
        \begin{align*}
          L_-|\psi> &= A [L_-|1,0>+2L_-|1,-1>]\\
          L_-|1,0> &= \sqrt{1(1+1)}\hbar|1,-1> = \sqrt{2}\hbar|1,-1>\\
          L_-|1,-1> &= \sqrt{2-2}\hbar|1,-2> = 0\\
          L_-|\psi> &= A\sqrt{2}\hbar|1,-1>\\
          <\psi|L_-|\psi> &= A^2[\sqrt{2}\hbar<1,0|1,-1>+2\sqrt{2}\hbar<1,-1|1,-1>]\\
          &= 2A^2\sqrt{2}\hbar = \frac{2\sqrt{2}}{5}\hbar
        \end{align*}
        Ya por ultimo solo debemos combinarlos para llegar al resultado que deseamos
        \begin{align*}
          L_x &= \frac{1}{2}\left(L_++L_-\right)\\
          L_x &= \frac{1}{2}\left(\frac{2\sqrt{2}}{5}+\frac{2\sqrt{2}}{5}\right)\hbar\\
          &= \frac{2\sqrt{2}}{5}\hbar
        \end{align*}
\item \textbf{Nota:} Es importante que para este trabajo utilizamos el valor de $A$ sin embargo no hemos desarrollado cuanto es este valor. En este caso partimos de \[
    \left|\psi\right> = A\left[ \left|1,0\right> + 2\left|1,-1\right> \right] 
.\] Por otro lado, con esto podemos conseguir que \[
\left<\psi\right|= A\left[ \left<1,0\right|+2\left< 1, -1\right| \right] 
.\] Ahora con esto podríamos conseguir \[
\left< \psi | \psi \right> = A^2 \left[ \left< 1,0 | 1,0 \right> + 2\left< 1, -1 | 1, 0 \right> + 2 \left< 1,0 | 1,/1 \right>  + 4\left< 1, -1 | 1, -1 \right>  \right] 
.\] Ahora con esto tenemos que ser conscientes de que
\begin{align*}
  \left< \psi | \psi \right> &= 1 \\
  \left< x | x \right> &= 1 \\
  \left< x | y \right> &=  0 
.\end{align*}
Con lo que nos queda \[
  1 = A^2\left[ 1 + 0 + 0 + 4 \right] 
.\] y por tanto \[
A = \sqrt{\frac{1}{5}}
.\] 
    \end{enumerate}
  \item 
    \begin{enumerate}
      \item Recordemos entonces que $n$ es el nivel de energía, $\ell$ es el orbital y $m_\ell$ ahora bien, recordemos que los valores posibles de los ultimos dos son
	\begin{align*}
	  \ell &= 0, 1, \ldots n \\
	  m_\ell &= -\ell,-\ell+1,\ldots,\ell \\
	.\end{align*}
	por lo tanto, los posibles valores de $m_\ell$ se encuentran en el conjunto  $\{x | x\in\mathbb{N}\land -3<x<3\}$ 
      \item Para este caso partimos desde un átomo de hidrógeno que con lo que nos dice el enunciado podemos describir como \[    
      .\] 
    \end{enumerate}
  \item 
    \begin{enumerate}
      \item Cuando un átomo es sometido a un campo magnético se observo experimentalmente que su espectro de emisión se dividía en lineas que parecían no estar presentes antes de eso. Existen dos casos de Efecto Zeeman, el normal y el anormal. Esta división se dio puesto que Zeeman (Que fue el encargado de explicar este efecto teóricamente) no pudo explicar el caso anómalo puesto que contaba únicamente con física clásica para solucionarlo. 
	La explicación que Zeeman dio de este fenómeno en su momento es una deducción con momentos angulares clásicos que realmente no nos interesa pues en verdad en este cumple un rol muy importante el spin. En particular lo que ocurre es que cuando un dipolo interactúa con un campo magnético este sufre un torque y se alinea (a favor o en contra) del campo magnético. Para esto requeriría una energía pero esta se bifurca.
      \item Si, de hecho el efecto zeeman ocurre cuando una partícula con Spin (todas lo tienen) interactúa con un campo magnetico se presenta efecto zeeman. Ademas, observando el espectro de emisión se pueden \textit{ver} correcciones del efecto Zeeman
      \item El átomo de hidrógeno es quizás el mas simple de los átomos estables. Un protón con un electrón. Este de hecho define un grupo de átomos que cumplen características similares y que su estudio es esencialmente el mismo que el del hidrógeno (Los Hidrogenoides). El átomo de hidrógeno es entonces una buena mezcla entre simplificación y realidad. Ademas es el átomo mas común del universo, no creo que esto aporte mucho pero es un detalle bastante interesante. 
    \end{enumerate}
\end{enumerate}


\end{document}

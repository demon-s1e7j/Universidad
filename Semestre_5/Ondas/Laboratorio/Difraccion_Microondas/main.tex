\documentclass[a4paper, amsfonts, amssymb, amsmath, reprint, showkeys, nofootinbib, twoside]{revtex4-1}
\usepackage[spanish]{babel}
\usepackage[utf8]{inputenc}
\usepackage{float}
\usepackage[colorinlistoftodos, color=green!40, prependcaption]{todonotes}
\usepackage{amsthm}
\usepackage{mathtools}
\usepackage{physics}
\usepackage{xcolor}
\usepackage{graphicx}
\usepackage[left=23mm,right=13mm,top=35mm,columnsep=15pt]{geometry} 
\usepackage{adjustbox}
\usepackage{placeins}
\usepackage[T1]{fontenc}
\usepackage{lipsum}
\usepackage{csquotes}
\usepackage[normalem]{ulem}
\useunder{\uline}{\ul}{}
\usepackage[pdftex, pdftitle={Article}, pdfauthor={Author}]{hyperref} % For hyperlinks in the PDF
%\setlength{\marginparwidth}{2.5cm}
\bibliographystyle{apsrev4-1}

\begin{document}

%El título del experimento realizado es importante.
\title{Difracción de Microondas}


\author{Sergio Montoya}

%Si necesitan poner un segundo autor, deben eliminar los porcentajes (%) iniciales.
  
\author{Carlos Devia}

\affiliation{Universidad de los Andes, Bogotá, Colombia.}

\date{\today} % Si lo dejan vacío no les saldrá fecha. La fecha que se muestra es del día en que se compila.

\maketitle

\section{Análisis Cualitativo}

\begin{enumerate}
  \item ¿Que pasa si la abertura es muy grande respecto a la longitud de onda?

    En este caso, ya no podríamos hablar de una fuente puntual y por lo tanto la interferencia observada con un punto disminuiría con la distancia.

  \item Para obtener las ecuaciones $10.1$ y  $10.5$ se hace la suposición de que la distancia hasta el receptor es mucho mayor que la distancia entre rendijas. ¿Se cumple en el montaje experimental?¿Como podría afectar las mediciones?

    En estos casos las mediciones tendrían un valor muy distinto al esperado. Esto variaría debido que las ondas no tendrían el espacio de interferir entre si generando que los resultados no se cumplieran. En nuestro montaje se cumplía este caso.

  \item ¿Que diferencia se puede ver en las mediciones de una rendija respecto a las mediciones de doble rendija?


    En las mediciones de una rendija no hay realmente interferencia. Se observan valores relativamente mas planos.
  \item Si después de la placa de la rendija sencilla o la doble rendija se encontrara un medio con un indice de refracción mayor que el del aire. ¿cómo afectaría esto a los máximos detectados de la rendija sencilla y doble rendija?¿Se verían más pegados o más separados?

    Para un objeto con indice de refracción mayor se verían mas cercanos. Esto debido a que por efecto Snell las ondas difractarían y en particular se acercarían.
\end{enumerate}

\section{Análisis Cuantitativo}

\section{Conclusiones}

Para este laboratorio se deseaba estudiar el fenómeno de difracción de las microondas para una y dos rendijas.	

\end{document}

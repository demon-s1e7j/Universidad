\documentclass[a4paper, amsfonts, amssymb, amsmath, reprint, showkeys, nofootinbib, twoside]{revtex4-1}
\usepackage[spanish]{babel}
\usepackage[utf8]{inputenc}
\usepackage{float}
\usepackage[colorinlistoftodos, color=green!40, prependcaption]{todonotes}
\usepackage{amsthm}
\usepackage{mathtools}
\usepackage{physics}
\usepackage{xcolor}
\usepackage{graphicx}
\usepackage[left=23mm,right=13mm,top=35mm,columnsep=15pt]{geometry} 
\usepackage{adjustbox}
\usepackage{placeins}
\usepackage[T1]{fontenc}
\usepackage{lipsum}
\usepackage{csquotes}
\usepackage[normalem]{ulem}
\useunder{\uline}{\ul}{}
\usepackage[pdftex, pdftitle={Article}, pdfauthor={Author}]{hyperref} % For hyperlinks in the PDF
%\setlength{\marginparwidth}{2.5cm}
\bibliographystyle{apsrev4-1}

\begin{document}

%El título del experimento realizado es importante.
\title{Ondas Mecánicas de Sonido}


\author{Carlos Devia}

%Si necesitan poner un segundo autor, deben eliminar los porcentajes (%) iniciales.
  
\author{Sergio Montoya Ramírez}
%\email{Second.Author@institution.edu}

\affiliation{Universidad de los Andes, Bogotá, Colombia.}

\date{\today} % Si lo dejan vacío no les saldrá fecha. La fecha que se muestra es del día en que se compila.


\maketitle

\section{Análisis Cualitativo}
\begin{enumerate}
  \item \textbf{Que sucede con la distancia entre nodos a medida que se tienen se tienen frecuencias mas altas del generador?}

    \textit{Solución:}  Dado que la longitud del tubo permanece igual pero la frecuencia se hace mayor entonces por la ecuación $(7.3)$ 
\end{enumerate}
\section{Análisis Cuantitativo}
\section{Conclusión}

\end{document}

\documentclass[12pt]{exam}
\usepackage{amsthm}
\usepackage{libertine}
\usepackage[utf8]{inputenc}
\usepackage[margin=1in]{geometry}
\usepackage{amsmath,amssymb}
\usepackage{multicol}
\usepackage[shortlabels]{enumitem}
\usepackage{siunitx}
\usepackage{cancel}
\usepackage{graphicx}
\usepackage{pgfplots}
\usepackage{listings}
\usepackage{tikz}


\pgfplotsset{width=10cm,compat=1.9}
\usepgfplotslibrary{external}
\tikzexternalize

\newcommand{\class}{Ondas y Fluidos - Complementaria} % This is the name of the course 
\newcommand{\examnum}{Taller 1 - Punto 1.2} % This is the name of the assignment
\newcommand{\examdate}{03/02/2023} % This is the due date
\newcommand{\timelimit}{}





\begin{document}
\pagestyle{plain}
\thispagestyle{empty}

\noindent
\begin{tabular*}{\textwidth}{l @{\extracolsep{\fill}} r @{\extracolsep{6pt}} l}
\textbf{\class} & \textbf{Name:} & \textit{Sergio Montoya Ramírez}\\ %Your name here instead, obviously 
	\textbf{\examnum} &&\textit{Yeferson Camacho}\\
	\textbf{\examdate} &&\textit{Monica Cano}\\
\end{tabular*}\\
\rule[2ex]{\textwidth}{2pt}
% ---


\section*{2}

Utilizando las representaciones vectoriales de $\sin(\theta)$ y $\cos(\theta)$.Comprobar las siguientes identidades trigonometricas.
		\begin{enumerate}
			\item $\sin^2(\theta)+\cos^2(\theta) = 1$
				\begin{align*}
					&\sin^2(\theta) + \cos^2(\theta) = \left(\frac{e^{i\theta}-e^{-i\theta}}{2i}\right)^2 + \left(\frac{e^{i\theta}+e^{-i\theta}}{2}\right)^2\\
					& =\frac{(e^{i\theta}-e^{-i\theta})^2}{-4} + \frac{(e^{i\theta}+e^{-i\theta})^2}{4}\\
					& =\frac{(e^{i\theta}+e^{-i\theta})^2-(e^{i\theta}-e^{-i\theta})^2}{4}\\
					& =\frac{(e^{2i\theta}+2e^{i\theta}e^{-i\theta}+e^{-2i\theta})-(e^{2i\theta}-2e^{i\theta}e^{-i\theta}+e^{-2i\theta})}{4}\\
					& \text{Note que: }e^{i\theta}e^{-i\theta} = e^{i\theta - i\theta}= e^{0} = 1\\
					& = \frac{e^{2i\theta}-e^{2i\theta}+2+2+e^{-2i\theta}-e^{-2i\theta}}{4}\\
					& = \frac{4}{4}\\
					& = 1 
				\end{align*}
			\item $\cos^2(\theta)-\sin^2(\theta)=\cos(2\theta)$
				\begin{align*}
					&\cos^2(\theta)-\sin^2(\theta) = \left(\left(\frac{e^{i\theta}+e^{-i\theta}}{2}\right)^2 - \left(\frac{e^{i\theta}-e^{i\theta}}{2i}\right)^2\right)\\
					&= \frac{e^{2i\theta}+2+e^{-2i\theta}}{4}+\frac{e^{2i\theta+}-2+e^{-2i\theta}}{4}\\
					&= \frac{e^{2i\theta}+e^{2i\theta}+e^{-2i\theta}+e^{-2i\theta}}{4}\\
					&= \frac{2e^{2i\theta}+2e^{-2i\theta}}{4}\\
					&= \frac{e^{i2\theta}+e^{-i2\theta}}{2}\\
					&= \cos(2\theta)
				\end{align*}
			\item $2\sin(\theta)\cos(\theta)=\sin(2\theta)$
				\begin{align*}
					&2\frac{e^{i\theta}-e^{-i\theta}}{2i}\frac{e^{i\theta}+e^{-i\theta}}{2} = 2\frac{(e^{i\theta}-e^{-i\theta})(e^{i\theta}+e^{-i\theta})}{4i}\\
					&= \frac{e^{i2\theta}-e^{-i2\theta}}{2i}\\
					&= \sin(2\theta)
				\end{align*}
			\item Analisis Dimencional: Lo trabajado aqui son equivalencias y por tanto no tiene ninguna relación real con dimenciones física.
			\item  Relación con la situación presentada: En todos los casos se demostraron las identidades trigonometricas solicitadas por los ejercicios.
			\item Conclusión: Esto fueron ejemplos de la relación que hay entre los numeros complejos y la trignometria permitiendo mostrar lo profundamente relacionados que estan estos dos temas y el por que fue una revolución en la faron las identidades trigonometricas solicitadas por los ejercicios.
			\item Conclusión: Esto fueron ejemplos de la relación que hay entre los numeros complejos y la trignometria permitiendo mostrar lo profundamente relacionados que estan estos dos temas y el por que fue una revolución en la física..
		\end{enumerate}





\end{document}

\documentclass[12pt]{exam}
\usepackage{amsthm}
\usepackage{libertine}
\usepackage[utf8]{inputenc}
\usepackage[margin=1in]{geometry}
\usepackage{amsmath,amssymb}
\usepackage{multicol}
\usepackage[shortlabels]{enumitem}
\usepackage{siunitx}
\usepackage{cancel}
\usepackage{graphicx}
\usepackage{pgfplots}
\usepackage{listings}
\usepackage{tikz}


\pgfplotsset{width=10cm,compat=1.9}
\usepgfplotslibrary{external}
\tikzexternalize

\newcommand{\class}{Ondas y Fluidos - Complementaria} % This is the name of the course 
\newcommand{\examnum}{Taller 1 - Punto 1.1} % This is the name of the assignment
\newcommand{\examdate}{03/02/2023} % This is the due date
\newcommand{\timelimit}{}





\begin{document}
\pagestyle{plain}
\thispagestyle{empty}

\noindent
\begin{tabular*}{\textwidth}{l @{\extracolsep{\fill}} r @{\extracolsep{6pt}} l}
\textbf{\class} & \textbf{Name:} & \textit{Sergio Montoya Ramírez}\\ %Your name here instead, obviously 
	\textbf{\examnum} &&\textit{Yeferson Camacho}\\
	\textbf{\examdate} &&\textit{Monica Cano}\\
\end{tabular*}\\
\rule[2ex]{\textwidth}{2pt}
% ---

\section{}


Dada la relación de Euler $e^{i\theta}=\cos(\theta)+i\sin(\theta)$, Hallar:
		\begin{enumerate}
			\item $e^{-i\theta}$

				\begin{align*}
					&e^{-i\theta} = \cos(\theta) - \sin(\theta)\\
				\end{align*}
			\item $\cos(\theta)$
				\begin{align*}
					&e^{i\theta} + e^{-i\theta} = \cos(\theta) + i\sin(\theta) + \cos(\theta) - i\sin(\theta)\\
					& = 2\cos(\theta)\\
					& \frac{e^{i\theta}+e^{-i\theta}}{2} = \cos(\theta)\\
				\end{align*}
			\item $\sin(\theta)$
				\begin{align*}
					&e^{i\theta} - e^{-i\theta} = \cos(\theta) + i\sin(\theta) - (\cos(\theta)-i\sin(\theta))\\
					&= \cos(\theta) + i\sin(\theta) - \cos(\theta) + i\sin(\theta)\\
					&= 2i\sin(\theta)\\
					&\frac{e^{i\theta}-e^{-i\theta}}{2i} = \sin(\theta)
				\end{align*}
			\item Analisis Dimensional: En este caso no contamos con dimensiones en ninguno de los casos, al menos no de manera directa. Todo lo que debia estar en este aspecto esta en la propia demostración.
			\item Relación con la situación presentada: En todos los casos se llego a la demostración de la equivalencia entre las funciones que se solicitaron.
			\item Conclusión: Gracias a la relación de Euler podemos ver que hay una equivalencia intrinceca entre los complejos y las funciones trigonometricas. Esta relación es una de las muchas razones por las cuales el analisis complejo fue una gran revolución en las matematicas y consecuentemente en la física.
		\end{enumerate}






\end{document}

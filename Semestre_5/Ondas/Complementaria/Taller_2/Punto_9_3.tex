\documentclass[12pt]{exam}
\usepackage{amsthm}
\usepackage{libertine}
\usepackage[utf8]{inputenc}
\usepackage[margin=1in]{geometry}
\usepackage{amsmath,amssymb}
\usepackage{multicol}
\usepackage[shortlabels]{enumitem}
\usepackage{siunitx}
\usepackage{cancel}
\usepackage{graphicx}
\usepackage{pgfplots}
\usepackage{listings}
\usepackage{tikz}


\pgfplotsset{width=10cm,compat=1.9}
\usepgfplotslibrary{external}
\tikzexternalize

\newcommand{\class}{Ondas y Fluidos - Complementaria} % This is the name of the course 
\newcommand{\examnum}{Taller 9 - Punto 3} % This is the name of the assignment
\newcommand{\examdate}{14/04/2023} % This is the due date
\newcommand{\timelimit}{}





\begin{document}
\pagestyle{plain}
\thispagestyle{empty}

\noindent
\begin{tabular*}{\textwidth}{l @{\extracolsep{\fill}} r @{\extracolsep{6pt}} l}
	\textbf{\class} & \textbf{Name:} & \textit{Monica Cano}\\ %Your name here instead, obviously 
	\textbf{\examnum} &&\textit{Yeferson Camacho}\\
	\textbf{\examdate} &&\textit{Sergio Montoya}\\
\end{tabular*}\\
\rule[2ex]{\textwidth}{2pt}
% ---

Si a es el numero de ondas por unidad de distancia. Entonces, por el muro tenemos 
\begin{equation}
  \label{eq:1}
  \lambda_in_i=\sin(\theta_i) ; \lambda_t n_t=\sin(\theta_t)
.\end{equation}

Ahora bien, en esta interfaz se debe cumplir que para una misma distancia haya la misma cantidad de ondas por lo tanto \[
n_i = n_t
.\] 

Por otro lado, si tomamos de \ref{eq:1} y dividimos ambas ecuaciones nos queda
\begin{align*}
  \frac{\lambda_i}{\lambda_t}=\frac{\sin(\theta_i)}{\sin(\theta_t)}
.\end{align*}

Ahora bien, dado que $\lambda\sim \frac{1}{k}\sim \frac{c}{w}\sim \frac{1}{n}$ entonces esto nos queda como
\begin{align*}
  \frac{n_t}{n_i} = \frac{\sin(\theta_i)}{\sin(\theta_t)}
.\end{align*}



\end{document}

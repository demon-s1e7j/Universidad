\documentclass[12pt]{exam}
\usepackage{amsthm}
\usepackage{libertine}
\usepackage[utf8]{inputenc}
\usepackage[margin=1in]{geometry}
\usepackage{amsmath,amssymb}
\usepackage{multicol}
\usepackage[shortlabels]{enumitem}
\usepackage{siunitx}
\usepackage{cancel}
\usepackage{graphicx}
\usepackage{pgfplots}
\usepackage{listings}
\usepackage{tikz}


\pgfplotsset{width=10cm,compat=1.9}
\usepgfplotslibrary{external}
\tikzexternalize

\newcommand{\class}{Ondas y Fluidos - Complementaria} % This is the name of the course 
\newcommand{\examnum}{Taller 9 - Punto 2} % This is the name of the assignment
\newcommand{\examdate}{14/04/2023} % This is the due date
\newcommand{\timelimit}{}





\begin{document}
\pagestyle{plain}
\thispagestyle{empty}

\noindent
\begin{tabular*}{\textwidth}{l @{\extracolsep{\fill}} r @{\extracolsep{6pt}} l}
	\textbf{\class} & \textbf{Name:} & \textit{Monica Cano}\\ %Your name here instead, obviously 
	\textbf{\examnum} &&\textit{Yeferson Camacho}\\
	\textbf{\examdate} &&\textit{Sergio Montoya}\\
\end{tabular*}\\
\rule[2ex]{\textwidth}{2pt}
% ---

Dado que tenemos el material desde el que sale el rayo y asumimos que afuera hay aire entonces tenemos esencialmente una ecuación $1.33\sin(35^{\circ})=1.00\sin(\theta)$ como se imaginaran podemos sacar el valor de $\sin(35)$ el cual es aproximadamente $0.57$ lo que al multiplicar con $1.33$ y esto resulta en $0.76$ lo que es $\sin(50)$

No se trabajo en unidades en este caso.



\end{document}

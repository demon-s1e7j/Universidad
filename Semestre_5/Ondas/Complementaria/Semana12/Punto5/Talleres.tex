  \documentclass[12pt]{exam}
\usepackage{amsthm}
\usepackage{libertine}
\usepackage[utf8]{inputenc}
\usepackage[margin=1in]{geometry}
\usepackage{amsmath,amssymb}
\usepackage{multicol}
\usepackage[shortlabels]{enumitem}
\usepackage{siunitx}
\usepackage{cancel}
\usepackage{graphicx}
\usepackage{pgfplots}
\usepackage{listings}
\usepackage{tikz}


\pgfplotsset{width=10cm,compat=1.9}
\usepgfplotslibrary{external}
\tikzexternalize

\newcommand{\class}{Ondas - Complementaria} % This is the name of the course 
\newcommand{\examnum}{Taller 12 - Punto 5} % This is the name of the assignment
\newcommand{\examdate}{\today} % This is the due date
\newcommand{\timelimit}{}





\begin{document}
\pagestyle{plain}
\thispagestyle{empty}

\noindent
\begin{tabular*}{\textwidth}{l @{\extracolsep{\fill}} r @{\extracolsep{6pt}} l}
	\textbf{\class} & \textbf{Name:} & \textit{Sergio Montoya}\\ %Your name here instead, obviously 
	\textbf{\examnum} &&\textit{Yeiferson Camacho}\\
	\textbf{\examdate} && \textit{Monica Cano}
\end{tabular*}\\
\rule[2ex]{\textwidth}{2pt}
% ---

\begin{itemize}
  \item \textbf{Enunciado: }

    Una red de difracción con unas rendijas separadas por $0.60\times 10^{-3}\ cm$ esta iluminada por luz con una longitud de onda de $500\ nm$.¿A que ángulo aparecerá el máximo de tercer orden?

  \item \textbf{Solución: }

    Para este caso, lo que debemos hacer es utilizar \[
    \alpha\sin\theta_m=m\lambda
    .\] En este caso, lo que nos interesa es $\theta_m$ por lo tanto desarrollamos como sigue.
    \begin{align*}
      \alpha\sin\theta_m &= m\lambda \\
      \sin\theta_m&= \frac{m\lambda}{\alpha} \\
    .\end{align*}
    Ahora bien, en este caso tenemos $m=3$, $\lambda=500nm$ y $\alpha=0.60\times 10^{-3}cm$. Sin embargo, en este caso necesitamos convertir las unidades. Por lo tanto nos queda
    \begin{align*}
      \lambda &= 500nm = 5\times 10^{-7}\\
      \alpha &= 0.60\times 10^{-3}cm = 6.0 \times 10^{-6}m
    .\end{align*}

    Con esto ya acomodado podemos reemitirnos a la ecuación que despejamos previamente. Por lo tanto, esto nos queda como
    \begin{align*}
      \sin\theta_m &= \frac{m\lambda}{\alpha} \\
      &= \frac{3\left( 5\times 10^{-7} \right) }{6\times 10^{-6}} \\
      \sin^{-1}\left( \sin\theta_m \right) &=\sin^{-1}\left(\frac{3\left( 5\times 10^{-7} \right) }{6\times 10^{-6}}  \right)   \\
      \theta &= \sin^{-1}\left( 0.25 \right)  \\
      \theta &\approx 14^\circ
    .\end{align*}

  \item \textbf{Revisión Unidades: }

    En este caso, unicamente dos de los valores tenian unidades y ambos eran unidades de longitud. Ademas dado que estaban en denominador y numerador esto se cancelaba. Si se desea el desarrollo este es como sigue:
    \begin{align*}
      \sin\theta_m&= \frac{\left[ L \right] }{\left[ L \right] }
    .\end{align*}
    Es obvio entonces que $\sin\theta_m$ es adimencional como era de esperarse
\end{itemize}

\end{document}

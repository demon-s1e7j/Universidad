  \documentclass[12pt]{exam}
\usepackage{amsthm}
\usepackage{libertine}
\usepackage[utf8]{inputenc}
\usepackage[margin=1in]{geometry}
\usepackage{amsmath,amssymb}
\usepackage{multicol}
\usepackage[shortlabels]{enumitem}
\usepackage{siunitx}
\usepackage{cancel}
\usepackage{graphicx}
\usepackage{pgfplots}
\usepackage{listings}
\usepackage{tikz}


\pgfplotsset{width=10cm,compat=1.9}
\usepgfplotslibrary{external}
\tikzexternalize

\newcommand{\class}{Ondas - Complementaria} % This is the name of the course 
\newcommand{\examnum}{Taller 12 - Punto 7} % This is the name of the assignment
\newcommand{\examdate}{\today} % This is the due date
\newcommand{\timelimit}{}





\begin{document}
\pagestyle{plain}
\thispagestyle{empty}

\noindent
\begin{tabular*}{\textwidth}{l @{\extracolsep{\fill}} r @{\extracolsep{6pt}} l}
	\textbf{\class} & \textbf{Name:} & \textit{Sergio Montoya}\\ %Your name here instead, obviously 
	\textbf{\examnum} &&\\
	\textbf{\examdate} &&
\end{tabular*}\\
\rule[2ex]{\textwidth}{2pt}
% ---

\begin{itemize}
  \item \textbf{Enunciado:}

    La luz emitida por una lampara de Sodio de laboratorio tiene dos componentes amarillas fuertes a $589.5923 nm$ y  $588.9953nm$. ¿A que distancia entre ellas se hallaran en el espectro de primer orden, estas dos lineas en una pantalla de $10.000$ lineas por centimetro, colocadas a 1m de la red.

  \item \textbf{Solución:}

    
\end{itemize}

\end{document}

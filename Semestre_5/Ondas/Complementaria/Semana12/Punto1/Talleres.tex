  \documentclass[12pt]{exam}
\usepackage{amsthm}
\usepackage{libertine}
\usepackage[utf8]{inputenc}
\usepackage[margin=1in]{geometry}
\usepackage{amsmath,amssymb}
\usepackage{multicol}
\usepackage[shortlabels]{enumitem}
\usepackage{siunitx}
\usepackage{cancel}
\usepackage{graphicx}
\usepackage{pgfplots}
\usepackage{listings}
\usepackage{tikz}


\pgfplotsset{width=10cm,compat=1.9}
\usepgfplotslibrary{external}
\tikzexternalize

\newcommand{\class}{Ondas - Complementaria} % This is the name of the course 
\newcommand{\examnum}{Taller 12 - Punto 1} % This is the name of the assignment
\newcommand{\examdate}{\today} % This is the due date
\newcommand{\timelimit}{}





\begin{document}
\pagestyle{plain}
\thispagestyle{empty}

\noindent
\begin{tabular*}{\textwidth}{l @{\extracolsep{\fill}} r @{\extracolsep{6pt}} l}
	\textbf{\class} & \textbf{Name:} & \textit{Sergio Montoya}\\ %Your name here instead, obviously 
  \textbf{\examnum} && \textit{Yeiferson Camacho}\\
  \textbf{\examdate} && \textit{Monica Cano}
\end{tabular*}\\
\rule[2ex]{\textwidth}{2pt}
% ---

\begin{enumerate}
  \item \textbf{Enunciado:}

    Una fuente puntual $S$ esta a una distancia perpendicular $R$ del centro de un agujero circular de radio $a$ en una pantalla opaca. Si la distancia hasta la periferia es $R + \ell$, demuestre que la difracción de Fraunhofer ocurrirá en una pantalla muy distante cuando $\lambda R \gg \frac{a^2}{2}$. ¿Cuál es el valor satisfactorio más pequeño de $R$ si el agujero tiene un radio de $1 mm$, con $\ell \le  \frac{\lambda}{10}$ y $\lambda = 500 nm$?

  \item \textbf{Solución:}

    En este caso, partimos desde 
    \begin{align*}
    \left( R+\ell \right)^2 = R^2 + a^2
    .\end{align*}

    Y desarrollamos como sigue
    \begin{align*}
      R&= \frac{\left( a^2 + \ell^2 \right)}{2\ell}\approx \frac{a^2}{2\ell} \\
      \ell R &= \frac{a^2}{2}
    .\end{align*}

    Por lo tanto, 
    \begin{align*}
      \lambda &\ll \ell \\
      \lambda R &\gg \frac{a^2}{2}
    .\end{align*}

    Por lo tanto,

    \begin{align*}
      R&= (1\times 10^{-3})^2 \frac{10}{2\lambda}=10 m
    .\end{align*}
\end{enumerate}

\end{document}

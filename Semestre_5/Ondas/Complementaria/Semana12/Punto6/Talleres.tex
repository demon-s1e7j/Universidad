  \documentclass[12pt]{exam}
\usepackage{amsthm}
\usepackage{libertine}
\usepackage[utf8]{inputenc}
\usepackage[margin=1in]{geometry}
\usepackage{amsmath,amssymb}
\usepackage{multicol}
\usepackage[shortlabels]{enumitem}
\usepackage{siunitx}
\usepackage{cancel}
\usepackage{graphicx}
\usepackage{pgfplots}
\usepackage{listings}
\usepackage{tikz}


\pgfplotsset{width=10cm,compat=1.9}
\usepgfplotslibrary{external}
\tikzexternalize

\newcommand{\class}{Ondas - Complementaria} % This is the name of the course 
\newcommand{\examnum}{Taller 12 - Punto 6} % This is the name of the assignment
\newcommand{\examdate}{\today} % This is the due date
\newcommand{\timelimit}{}





\begin{document}
\pagestyle{plain}
\thispagestyle{empty}

\noindent
\begin{tabular*}{\textwidth}{l @{\extracolsep{\fill}} r @{\extracolsep{6pt}} l}
	\textbf{\class} & \textbf{Name:} & \textit{Sergio Montoya}\\ %Your name here instead, obviously 
	\textbf{\examnum} && \textit{Yeiferson Camacho}\\
	\textbf{\examdate} && \textit{Monica Cano}
\end{tabular*}\\
\rule[2ex]{\textwidth}{2pt}
% ---

 \begin{itemize}
   \item \textbf{Enunciado:}

     Una red de difracción produce un espectro de segundo orden de luz amarilla $\left(\lambda_0 =550 nm  \right) $ a $25^{\circ}$. Calcule el espacio entre las lineas de la red. 

   \item \textbf{Solución:}

     En este caso, utilizaremos \[
     \alpha\sin\theta_m = m\lambda
     .\] Por lo tanto, podemos desarrollar como sigue
     \begin{align*}
       \alpha\sin\theta_m &= m\lambda\\
       \alpha &= \frac{m\lambda}{\sin\theta_m}
     .\end{align*}
     
     Ahora bien, en este caso sabemos que $m=2$, $\lambda = 550 nm$ y $\theta_m = 25^{\circ}$. Sin embargo, para presentar este resultado debemos convertir $\lambda$ a metros. Por lo tanto esto no queda como \[
     \lambda = 550 nm = 5.5\times 10^{-7} m
     .\] 

     Con esto entonces solamente debemos reemplazar estos valores en la ecuación que despejamos previamente y nos encontramos con:
     \begin{align*}
       \alpha &= \frac{2\left( 5.5\times 10^{-7} \right) }{\sin\left( 25^{\circ} \right) } \\
       \alpha &= 2.6\times 10^{-6}
     .\end{align*}
   \item \textbf{Analisis Unidades:}

     En este caso solo contamos con un componente con unidades (Claramente la longitud de Onda). En este caso, es un componente de longitud y dado que este se encuentra en el numerador y ademas es justo la misma unidad de $\alpha$ entonces las unidades son correctas.
 \end{itemize}

\end{document}

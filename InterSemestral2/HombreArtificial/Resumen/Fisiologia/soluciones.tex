\documentclass[12pt]{article}
\usepackage{array}
\usepackage{color}
\usepackage{amsthm}
\usepackage{eufrak}
\usepackage{lipsum}
\usepackage{pifont}
\usepackage{yfonts}
\usepackage{amsmath}
\usepackage{amssymb}
\usepackage{ccfonts}
\usepackage{comment} \usepackage{amsfonts}
\usepackage{fancyhdr}
\usepackage{graphicx}
\usepackage{listings}
\usepackage{mathrsfs}
\usepackage{setspace}
\usepackage{textcomp}
\usepackage{blindtext}
\usepackage{enumerate}
\usepackage{microtype}
\usepackage{xfakebold}
\usepackage{kantlipsum}
%\usepackage{draftwatermark}
\usepackage[spanish]{babel}
\usepackage[margin=1.5cm, top=2cm, bottom=2cm]{geometry}
\usepackage[framemethod=tikz]{mdframed}
\usepackage[colorlinks=true,citecolor=blue,linkcolor=red,urlcolor=magenta]{hyperref}

%//////////////////////////////////////////////////////
% Watermark configuration
%//////////////////////////////////////////////////////
%\SetWatermarkScale{4}
%\SetWatermarkColor{black}
%\SetWatermarkLightness{0.95}
%\SetWatermarkText{\texttt{Watermark}}

%//////////////////////////////////////////////////////
% Frame configuration
%//////////////////////////////////////////////////////
\newmdenv[tikzsetting={draw=gray,fill=white,fill opacity=0},backgroundcolor=none]{Frame}

%//////////////////////////////////////////////////////
% Font style configuration
%//////////////////////////////////////////////////////
\renewcommand{\familydefault}{\ttdefault}
\renewcommand{\rmdefault}{tt}

%//////////////////////////////////////////////////////
% Bold configuration
%//////////////////////////////////////////////////////
\newcommand{\fbseries}{\unskip\setBold\aftergroup\unsetBold\aftergroup\ignorespaces}
\makeatletter
\newcommand{\setBoldness}[1]{\def\fake@bold{#1}}
\makeatother

%//////////////////////////////////////////////////////
% Default font configuration
%//////////////////////////////////////////////////////
\DeclareFontFamily{\encodingdefault}{\ttdefault}{%
  \hyphenchar\font=\defaulthyphenchar
  \fontdimen2\font=0.33333em
  \fontdimen3\font=0.16667em
  \fontdimen4\font=0.11111em
  \fontdimen7\font=0.11111em}



\begin{document}
    %//////////////////////////////////////////////////////
% Heading Configuration
%//////////////////////////////////////////////////////
\pagestyle{fancy}
\thispagestyle{plain}
\fancyhead[RO,L]{\textbf{Geometría de Curvas y Superficies (MATE-2411)}}
\fancyhead[LO,L]{\textbf{Tarea 1}}
\setlength{\headheight}{16.0pt}

%//////////////////////////////////////////////////////
% Subsections Configuration
%//////////////////////////////////////////////////////
\renewcommand*\thesubsection{\arabic{subsection}}
\newcounter{counter}
\newlength{\palabra}
\settowidth{\palabra}{counter 999.}
\newcommand{\makeboxlabel}[1]{\fbox{#1.}\hfill}

%//////////////////////////////////////////////////////
% Personalized commands configuration
%//////////////////////////////////////////////////////
\newcommand{\N}{\mathbb{N}}
\newcommand{\Z}{\mathbb{Z}}
\newcommand{\Q}{\mathbb{Q}}
\newcommand{\R}{\mathbb{R}}
\newcommand{\C}{\mathbb{C}}
\newcommand{\re}{\operatorname{Re}}
\newcommand{\im}{\operatorname{Im}}
\newcommand{\Aut}{\operatorname{Aut}}
\newcommand{\GCD}{\operatorname{GCD}}
\newcommand{\LCD}{\operatorname{LCD}}
\linespread{1} %Line spacing

%//////////////////////////////////////////////////////
% Inline code configuration
%//////////////////////////////////////////////////////
\lstset{
gobble=5,
numbers=left,
frame=single,
framerule=1pt,
showtabs=False,
showspaces=False,
showstringspaces=False,
backgroundcolor=\color{gray}}

%//////////////////////////////////////////////////////
% Problem list configuration
%//////////////////////////////////////////////////////
\newenvironment{problems}
  {\begin{list}
     {{\fbseries Problem \arabic{counter}.}}
    {\usecounter{counter}
     \setlength{\labelsep}{1em}
     \setlength{\itemsep}{2pt}
     \setlength{\leftmargin}{2em}
     \setlength{\rightmargin}{0cm}
     \setlength{\itemindent}{1em} }}
{\end{list}}

%//////////////////////////////////////////////////////
% Appendix configuration
%//////////////////////////////////////////////////////
\newenvironment{Appendix}
  {\begin{list}
     {{\fbseries Lemma \arabic{counter}.}}
    {\usecounter{counter}
     \setlength{\labelsep}{1em}
     \setlength{\itemsep}{2pt}
     \setlength{\leftmargin}{2em}
     \setlength{\rightmargin}{0cm}
     \setlength{\itemindent}{1em} }}
{\end{list}}

%//////////////////////////////////////////////////////
% Notes configuration
%//////////////////////////////////////////////////////
\newenvironment{notes}
  {\begin{list}
     {{\fbseries Note \arabic{counter}.}}
    {\usecounter{counter}
     \setlength{\labelsep}{1em}
     \setlength{\itemsep}{2pt}
     \setlength{\leftmargin}{2em}
     \setlength{\rightmargin}{0cm}
     \setlength{\itemindent}{1em} }}
{\end{list}}

%//////////////////////////////////////////////////////
% Activity Information
%//////////////////////////////////////////////////////
\vspace*{-1cm}
\hrule width \hsize \kern 1mm \hrule width \hsize height 2pt
\begin{center}
   \parbox[c]{.32\textwidth}{
   \hspace{1cm}\\
   Sergio Montoya Ramirez\\
   202112171}
%   Luis Ernesto Tejón Rojas\\
%   202113150}
   \hspace*{\fill}
   \parbox[c]{.35\textwidth}{\centering
   Universidad de Los Andes\\
   Tarea 1\\
   Geometría de Curvas y Superficies\\
   }
   \hspace*{\fill}
   \parbox[c]{.3\textwidth}{
   \begin{flushleft}
      Bogotá D.C., Colombia\\
      \today
   \end{flushleft}}
\end{center}
\hrule width \hsize height 2pt \kern 1mm \hrule width \hsize

\bigskip


\bigskip



    \section{Fisiología}
    \subsection{Que es la fisiología}

    La fisiología es la rama de la medicina que estudia el como funciona un ser vivo.

    Es una disciplina que estudia el funcionamiento de los seres vivos. ¡Esta en estrecha relación con la anatomía!  \subsection{Que es un ser vivo?}

    Este es un tema no trivial y bastante complejo. En particular, el profesor encuentra como respuesta una definición termodinámica:

    \textit{
      Sistema químico que usa energía para alcanzar el equilibrio: Hemostasis}

      En este vamos a hablar de entropía. La entropía es en general una relación entre micro estados y macro estados. Un ser vivo estaría definido como un sistema que tiende a disminuir la entropía haciendo uso de la energía.

      \subsection{El ser Humano}

      El ser humano es, en esencia, un conjunto de seres vivos. Aproximadamente 39 mil millones de células (Que son de por si seres vivos) múltiples bacterias y varios sistemas mas. También es interesante que de hecho no es un sistema estático. Todas sus células cambian y varían y por tanto en verdad es mas parecido a una onda que a una partícula. Curioso

      \subsection{Organización de las Escalas}
      
      Tenemos muchísimas escalas y tamaño vamos en general como
       \begin{enumerate}
        \item átomo
	\item molécula
	\item célula
	\item tejido
	\item órgano
	\item sistema
	\item organismo
	\item población
      \end{enumerate}

      Y así continuaríamos.
      \section{Cardiovascular}




\end{document}

  \documentclass[12pt]{exam}
\usepackage{amsthm}
\usepackage{libertine}
\usepackage[utf8]{inputenc}
\usepackage[margin=1in]{geometry}
\usepackage{amsmath,amssymb}
\usepackage{multicol}
\usepackage[shortlabels]{enumitem}
\usepackage{siunitx}
\usepackage{cancel}
\usepackage{graphicx}
\usepackage{pgfplots}
\usepackage{listings}
\usepackage{tikz}


\pgfplotsset{width=10cm,compat=1.9}
\usepgfplotslibrary{external}
\tikzexternalize

\newcommand{\class}{Hombre Artificial} % This is the name of the course 
\newcommand{\examnum}{Resumen Organos artificiales} % This is the name of the assignment
\newcommand{\examdate}{\today} % This is the due date
\newcommand{\timelimit}{}





\begin{document}
\pagestyle{plain}
\thispagestyle{empty}

\noindent
\begin{tabular*}{\textwidth}{l @{\extracolsep{\fill}} r @{\extracolsep{6pt}} l}
	\textbf{\class} & \textbf{Name:} & \textit{Sergio Montoya}\\ %Your name here instead, obviously 
	\textbf{\examnum} &&\\
	\textbf{\examdate} &&
\end{tabular*}\\
\rule[2ex]{\textwidth}{2pt}
% ---

Esta clase comenzó antes de la presentación. En una primera instancia nos acercamos al anfiteatro donde se podían observar varios fetos conservados en lo que era un liquido transparente que no podría identificar en ese momento. Mas la curiosidad que el conocimiento me llevaba a observar con detenimiento cada uno de estos cuerpos que pintaban una imagen extraña. No paraba de recordar la primera clase del curso en donde se mostró un cuadro de unos médicos viendo una disección. Esto me recordaba al como hablábamos de la morbosidad que nos parecía que tenían. Luego de esto, llego el profesor que comenzaría a explicarnos. En una primera instancia nos hablo de los métodos usados para conservar los cuerpos. Su estilo era extraño y bastante atractivo. Nos hablo de la conservación formol-alcohol. Que consistía en retirar los líquidos y hipidos de los cuerpos (en ese caso de no natos) y así poder luego mantenerlos en suspensión con alcohol. Luego nos mostró otros fetos conservados en este caso con una técnica que permitía jugar con la refracción de los cuerpo y que en esa situación permitía marcar tejidos de interés. Posteriormente pasamos por una sección osea en donde nos explico un proceso que en esencia consistía en una fosilización acelerada con la cual los huesos podían conservarse por mas tiempo. Luego de esto pasamos a una sala de simulación de cirugía en la que nos mostró algunas conservaciones hechas a mediados del siglo pasado y que en ese momento se estaban recuperando para una exposición. Nos dijo que se habían realizado durante varios meses aplicando capas de glicerina 3 veces al día hasta que estos estuvieran conservados. Posteriormente pasamos a una sala en donde aun se estaban realizando conservaciones. Había parado en un lado de este espacian un cuerpo conservado que el medico llamo "Don Carlos" Comentando que era una persona con la que había tomado tinto cuando estaba vivo. Mostró varios animales que estaba realizando una conservación. Luego de esto pasamos a uno de los puntos que mas interesante me pareció. La morgue y no peerse por la curiosidad si no que en esta note por que lo que seria en la cultura popular relatado como un trabajo bastante morboso tenia un papel tan importante. Nos contó de una técnica de cirugía que se estaba estudiando y que presentaba de vez en cuando problemas dejando a algunos sujetos que se sometían a ella parapléjicos. Sin embargo en un estudio todos los ejemplares que se sometieron a la operación tuvieron este efecto y por tanto se deseaba saber cual era el motivo y por tanto se disecciono los cuerpos de estos conejos y se encontró que esta raza en particular tenia una arteria en la zona de la cirugía y que en consecuencia al realizarse en esencia ocasionaba un arrancamiento de esta arteria y en consecuencia era como cortar la zona. Esto marco y dejo claro que los efectos de esta investigación eran en esencia el encontrar los posibles riesgos y errores que se cometieran. Para o bien corregirlos o evitarlos. Luego de todo esto pasamos a una explicación por medio del hígado del ser humano y sus órganos como un sistema que reacciona a su entorno y se adapta. En donde los cambios mas pequeños e insignificantes parecen tener efectos inmensos en el como se desarrolla la especie.

Esta charla fue interesante y reflexiva aun cuando el profesor bombardeaba de información y datos. Se notaba en el que estaba orgulloso de lo que hacia y le parecía realmente interesante. Cosa que hacia un verdadero placer escucharlo. Sin embargo dejo claro para mi el por que esto es un proceso muy importante. Es relevante que los médicos aprendan de cuerpos. De casos reales, de dolores reales y de procesos que causaron malestar. Es importante saber cuales fueron los errores y aprender de ellos. Me recordó un poco a una enseñanza que le daba mi abuelo a mi papá que siempre me había parecido graciosa. Mi abuelo fue un medico de vieja usanza. El se graduó cuando la guerra estaba en un punto muy álgido y durante mucho tiempo tuvo que vivir en zonas rurales de Colombia atendiendo con lo que tuviera a mano y sin esperanza de obtener estudios. Ademas, en estas condiciones era muy importante para el que sus hijos supieran identificar problemas y solucionarlos. Entonces le enseñaba a mi papá y le decía:

"Quiere usted saber que esta mal en un paciente? Sienta la parte de alguien sano"

Me parecía siempre curiosa. Esto era obviamente una versión muy simplificada de lo que hicimos este día. Una manera de aproximarse bajo la respuesta a una falta de condiciones para mantener un anfiteatro en condiciones. Era saber como funciona un ser humano. Es entonces muy importante este trabajo.

Ahora bien, centrado en la clase del hombre artificial creo que una visión muy importante es saber como funciona el ser humano como un todo y como un singular. Creo que los órganos artificiales tienen que suplir a su vez muchas funciones y a la vez una tan concreta que seria imposible realizar este trabajo sin un conocimiento profundo de lo que se esta haciendo.

\end{document}

  \documentclass[12pt]{exam}
\usepackage{amsthm}
\usepackage{libertine}
\usepackage[utf8]{inputenc}
\usepackage[margin=1in]{geometry}
\usepackage{amsmath,amssymb}
\usepackage{multicol}
\usepackage[shortlabels]{enumitem}
\usepackage{siunitx}
\usepackage{cancel}
\usepackage{graphicx}
\usepackage{pgfplots}
\usepackage{listings}
\usepackage{tikz}


\pgfplotsset{width=10cm,compat=1.9}
\usepgfplotslibrary{external}
\tikzexternalize

\newcommand{\class}{Hombre Artificial} % This is the name of the course 
\newcommand{\examnum}{Resumen Modificaciones Geneticas y Células Madre} % This is the name of the assignment
\newcommand{\examdate}{\today} % This is the due date
\newcommand{\timelimit}{}





\begin{document}
\pagestyle{plain}
\thispagestyle{empty}

\noindent
\begin{tabular*}{\textwidth}{l @{\extracolsep{\fill}} r @{\extracolsep{6pt}} l}
	\textbf{\class} & \textbf{Name:} & \textit{Sergio Montoya}\\ %Your name here instead, obviously 
	\textbf{\examnum} &&\\
	\textbf{\examdate} &&
\end{tabular*}\\
\rule[2ex]{\textwidth}{2pt}
% ---

\textbf{Modificaciones Genéticas}

El ADN es la información por la que el cuerpo codifica todas las características que luego el organismo va a presentar. Por lo tanto, fue un gran hito para la humanidad el hecho de poder alterar la información que este codificaba. Para explicar esto debemos entonces hablar de genética la manera en la que el ADN codifica la información y CRISPR-CAS9 el método por el que la humanidad esta editando realizando modificaciones genéticas. 

El ADN esta compuesto por genes. Un gen es la unidad básica de expresión en el fenotipo. Esto significa que un gen. Por ejemplo, hay un gen encargado de la forma del pelo, de su color y de cada una de sus otras características así como de todas las otras partes de cuerpo y demás características. Ahora bien, los genes codifican la información por medio de las bases nitrogenadas. Las bases nitrogenadas son: Adenina, Citocina, Guanina y Timina. Estos se combinan en parejas y es su orden el que codifica la información. Expliquemos esto haciendo uso de un ejemplo. Supongamos que tenemos la cadena $AATGC$ esta cadena contiene cierta información y ademas sabemos que (en condiciones normales. Esto puede cambiar en ciertas situaciones) la cadena complementaria en el otro lado del ADN sera $TTACG$. Toda esta información se encuentra codificada en el ADN que a su ves se encuentra en el núcleo y que si seguimos creciendo va codificando cada vez mas estructuras mas complejas. Ahora bien, una característica importante es que el ADN se encuentra codificado y almacenado en cromosomas que se encuentran en el núcleo. Estos cromosomas son parejas de ADN compactificado. En total para los humanos tenemos (es la moda, algunas personas pueden variar esto) 46 cromosomas organizados en 23 parejas. Para cada una de estas parejas una viene de la madre y una del padre. 

Ahora bien, todo lo que habíamos hablado es una descripción muy básica de como funciona el ADN. Sin embargo, pueden existir cambios en el ADN de un organismo. Estos cambios se llaman mutaciones. Las mutaciones pueden ocurrir por errores en la replicaciones del ADN durante la división celular, exposición a mutagenos o a una infección viral. 

Existen varios tipos de mutaciones. Estos son:
\begin{enumerate}
  \item Punto de Mutación: Es un cambio en una sola base nitrogenada.
  \item Sustitución: Es cuando una o mas bases se reemplazan por otras teniendo el mismo numero de bases
  \item Inversión: Es cuando un segmento del cromosoma se invierte de extremo a extremo
  \item Inserción: Es cuando un segmento se reemplaza por otro con mayor numero de bases nitrogena
  \item Eliminación: Es cuando se retira una base de la secuencia
\end{enumerate}

Estas mutaciones pueden ocurrir a gran escala y son heredables aunque generalmente son recesivos.

Ahora bien, por otro lado, tenemos que hablar de la tecnica CRISPR-CAS9 que es por medio de la cual se estan realizando las ediciones geneticas. Esta consisten en esencia en una encima que altera el gADN de manera tal que al reescribir se puede cambiar el contenido de ese gen en particular o incluso desactivarlo por completo. Con esto entonces ocurre el escandalo de los CRISPR Babies los cuales fueron Lulu y Nana a las cuales les retiraron el gen del VIH para que fueran negativas siendo hijas de dos padres VIH positivo.

Otra tecnología de la ingeniería genética es la clonación por medio de la cual se pueden generar copias idénticas o casi idénticas de un organismo.

\textbf{Células Madre}

Una célula madre es una celula que se puede auto renovar y que no estan diferenciadas. Hay de muchos tipos y en general estas se dan de la siguiente manera:

Luego de la concepción se comienzan a formar celulas totipotentes. Estas son celulas que pueden diferenciarse hacia cualquier otra incluyendo tejido embrionario. Se encuentran en la Mórula. Luego de esto, en el blastocito se encuentran celulas pluripotentes las cuales pueden diferenciarse hacia cualquier tejido exceptuando el embrionario. Por ultimo en la organogenesis se encuentran celulas multipotentes. Estas células ya tienen cierta diferenciación pero son de las que salen células altamente especificas. Las celulas multipotentes tienen varios tipos:

\begin{enumerate}
  \item Hematopoyeticas
  \item Mesenquimales
\end{enumerate}

Las celulas madre se usan entonces para la regeneración de tejidos y son altamente utiles pues existe el proceso por el cual una celula ya diferenciada puede devolverse hasta ser una celula pluripotente que como dijimos previamente puede convertirse en cualquier tejido (excepto el embrionario que ya en adultos no resulta util)

En conclusión estas son dos tecnicas por las cuales se pueden abordar distintos problemas. Por medio de esto vemos como el hombre artificial se relaciona con las nuevas tecnicas que alterando o generando nuevos tejidos se pueden solucionar distintos problemas como: Enfermedades geneticas como la hemofilia, Presencia de mutaciones o incluso destrucción de tejidos.
\end{document}

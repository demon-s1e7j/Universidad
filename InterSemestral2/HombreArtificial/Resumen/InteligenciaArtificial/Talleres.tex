  \documentclass[12pt]{exam}
\usepackage{amsthm}
\usepackage{libertine}
\usepackage[utf8]{inputenc}
\usepackage[margin=1in]{geometry}
\usepackage{amsmath,amssymb}
\usepackage{multicol}
\usepackage[shortlabels]{enumitem}
\usepackage{siunitx}
\usepackage{cancel}
\usepackage{graphicx}
\usepackage{pgfplots}
\usepackage{listings}
\usepackage{tikz}


\pgfplotsset{width=10cm,compat=1.9} \usepgfplotslibrary{external}
\tikzexternalize

\newcommand{\class}{Hombre Artificial} % This is the name of the course 
\newcommand{\examnum}{Resumen Inteligencia Artificial} % This is the name of the assignment
\newcommand{\examdate}{\today} % This is the due date
\newcommand{\timelimit}{}





\begin{document}
\pagestyle{plain}
\thispagestyle{empty}

\noindent
\begin{tabular*}{\textwidth}{l @{\extracolsep{\fill}} r @{\extracolsep{6pt}} l}
	\textbf{\class} & \textbf{Name:} & \textit{Sergio Montoya}\\ %Your name here instead, obviously 
	\textbf{\examnum} &&\\
	\textbf{\examdate} &&
\end{tabular*}\\
\rule[2ex]{\textwidth}{2pt}
% ---

La inteligencia es la capacidad de razonar, integrar funciones cognitivas como la percepción, memoria, atención, lenguaje, etc. Estos tienen muchas características importantes y surge del cerebro. La relación entre el cerebro y la inteligencia es intrincada y complejo pero principalmente se relaciona con la region frontal, parietal y temporal del cerebro.

\textbf{Inteligencia Artificial}

Luego de la creación de la computación por Varios matemáticos. Una de las preguntas mas sonadas era respecto a la inteligencia de estas maquinas. En un principio para solucionar esto, Alan Turing (un matemático brillante) diseño lo que ahora conocemos como el test de Turing. Este tiene muchos detalles, sin embargo, es en esencia un juego de imitación. Si una maquina es capaz de ser indistinguible de un humano esta pasara el test de Turing.

Las aplicaciones de la inteligencia artificial son múltiples y especializadas. En particular, una de las mas frecuentes es el análisis de imagen. Esta es una cosa que para los humanos nos resulta bastante sencilla pero para una maquina es un proceso titánico en donde debe trabajar con reconocimiento de patrones para poder encontrar aquella información que le permite reconocer lo que desea o lo que esta entrenado. Siendo esto cualquier cosa para lo que se entrene. Una maquina que reconoce insectos, perros, expresiones lo que sea.

El desarrollo de estos procesos fue temprano en la computación, sin embargo, por limitaciones técnicas y falta de datos los avances que se podían realizar fueron limitados viviendo una época de pesimismo en este tema. Aun así, gracias al rápido avance de la computación y la tecnología, pronto nos encontramos con las capacidades y datos necesarios para tener un gran avance y en la época actual estamos viviendo un gran auge de las inteligencias artificiales.

\end{document}

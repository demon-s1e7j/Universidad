  \documentclass[12pt]{exam}
\usepackage{amsthm}
\usepackage{libertine}
\usepackage[utf8]{inputenc}
\usepackage[margin=1in]{geometry}
\usepackage{amsmath,amssymb}
\usepackage{multicol}
\usepackage[shortlabels]{enumitem}
\usepackage{siunitx}
\usepackage{cancel}
\usepackage{graphicx}
\usepackage{pgfplots}
\usepackage{listings}
\usepackage{tikz}


\pgfplotsset{width=10cm,compat=1.9}
\usepgfplotslibrary{external}
\tikzexternalize

\newcommand{\class}{Hombre Artificial} % This is the name of the course 
\newcommand{\examnum}{Resumen Bio Impresión} % This is the name of the assignment
\newcommand{\examdate}{\today} % This is the due date
\newcommand{\timelimit}{}





\begin{document}
\pagestyle{plain}
\thispagestyle{empty}

\noindent
\begin{tabular*}{\textwidth}{l @{\extracolsep{\fill}} r @{\extracolsep{6pt}} l}
	\textbf{\class} & \textbf{Name:} & \textit{Sergio Montoya}\\ %Your name here instead, obviously 
	\textbf{\examnum} &&\\
	\textbf{\examdate} &&
\end{tabular*}\\
\rule[2ex]{\textwidth}{2pt}
% ---

La bio impresión es una tecnología asistida por medio de la cual se puede imprimir capa por capa materiales bao funcionales y sus componentes de soporte. Con esto se pueden crear tejidos y órganos complejos con la arquitectura celular y las funciones deseadas.

En otras palabras: Es una tecnologia que usa biomateriales para crear tejidos y/o organos.

Los tipos de bio impresion son:
\begin{enumerate}
  \item Inyección de Tinta
  \item Láser
  \item Extrusión
\end{enumerate}

Y para esto los materiales que utiliza son:
\begin{enumerate}
  \item Alginato
  \item Agarosa y Quitosano
  \item Colágeno
  \item Gelatina
  \item Fibrina
  \item Hidroxiapatita
\end{enumerate}

Y sus usos son multiples. Desde la recuperación de accidentes. Por ejemplo la perdida de sensibilidad en una zona por una quemadura. El tratamiento contra traumas punzantes o cortopunzantes. Recuperacion y transplante de organos con mayor seguridad y probabilidad de éxito. En general cualquier problema que se pueda solucionar con nuevos tejidos podría ser un caso de uso interesante para la bao impresión.

Un ejemplo de uso seria poder recuperar la vista (para dar un ejemplo especialmente complejo) en casos en donde la ceguera se deba a problemas con el ojo. O para dar un ejemplo mas personal (y en una situación que no se si lo justificaría) podría generar nuevos ojos de manera tal que se pueda "Curar" el daltonismo ocasionado por el mal funcionamiento de los conos. Ambos ejemplos son especialmente complejos pero posibles. 

Por lo tanto, la bio impresion es una tecnica con mucha proyección a futuro
\end{document}

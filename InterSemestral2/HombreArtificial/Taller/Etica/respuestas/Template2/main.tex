\documentclass[a4paper, amsfonts, amssymb, amsmath, reprint, showkeys, nofootinbib, twoside]{revtex4-1}
\usepackage[spanish]{babel}
\usepackage[utf8]{inputenc}
\usepackage{float}
\usepackage[colorinlistoftodos, color=green!40, prependcaption]{todonotes}
\usepackage{amsthm}
\usepackage{mathtools}
\usepackage{physics}
\usepackage{xcolor}
\usepackage{graphicx}
\usepackage[left=23mm,right=13mm,top=35mm,columnsep=15pt]{geometry} 
\usepackage{adjustbox}
\usepackage{placeins}
\usepackage[T1]{fontenc}
\usepackage{lipsum}
\usepackage{csquotes}
\usepackage[normalem]{ulem}
\useunder{\uline}{\ul}{}
\usepackage[pdftex, pdftitle={Article}, pdfauthor={Author}]{hyperref} % For hyperlinks in the PDF
%\setlength{\marginparwidth}{2.5cm}
\bibliographystyle{apsrev4-1}

\begin{document}

%El título del experimento realizado es importante.
\title{Taller Etica}


\author{Grupo 5}

%Si necesitan poner un segundo autor, deben eliminar los porcentajes (%) iniciales.
  
%\author{Second Author}
%\email{Second.Author@institution.edu}

\affiliation{Universidad de los Andes, Bogotá, Colombia.}

\date{\today} % Si lo dejan vacío no les saldrá fecha. La fecha que se muestra es del día en que se compila.

\maketitle

\section*{Resumen}

La tuberculosis pulmonar (TB) es una enfermedad altamente contagiosa generada por los llamados bacilos de tuberculosis. En muchos casos, las personas contagiadas poseen sistemas inmunológicos que encapsulan los bacilos y hacen que la enfermedad permanezca durante largos periodos de tiempo. Además, cualquier debilitamiento en el sistema inmunológico como la presencia de VIH en el cuerpo aumenta la probabilidad de padecer TB. Para tratar la TB activa se usan medicamentos como isoniazida (INH), rifampicina y pirazinamida de forma rutinaria y para personas VIH negativas con una infección de TB (inactiva) la INH se considera útil para prevenir la enfermedad cuando hay alto riesgo de contraerla.

El uso de este tratamiento es variable según el país de acuerdo con los recursos financieros, políticas y la capacidad del sistema de salud. Así, en 1994 la American Thoracic Society estableció que en EE. UU. se debe hacer una terapia preventiva con INH a personas con VIH y TB activa. Usualmente esta terapia dura de 6 a 12 meses y se han discutido las alternativas para establecer regímenes más cortos. Con este objetivo, investigadores en EE. UU. y África estudiaron la seguridad de estos regímenes mas cortos y buscaban comprobar si la terapia de prevención era eficaz en entornos de alto riesgo de contagio.

Para lograr este objetivo se hizo un ensayo clínico aleatorizado con tres mil adultos africanos VIH positivos con alto riesgo de contraer TB, los participantes se asignaron a uno de los siguientes regímenes: Placebo, INH diaria por 6 meses, INH y rifampicina diaria por 3 meses e INH, rifampicina y pirazinamida diaria por 3 meses. El uso del placebo se justificó postulando que el tratamiento puede no ser seguro en personas VIH positivas. Cabe aclarar que todos brindaron su consentimiento informado verbal antes de ser seleccionados y el ensayo se diseñó para estudiar a los participantes por 3 años.

\section*{Princios Bioeticos}
Por un lado, encontramos que se viola el principio de \textbf{autonomía} por \textbf{ausencia de constricción} debido a que las personas en el África cuentan con unas condiciones socioeconómicas que los incentiva para aceptar hacer parte del experimento. A pesar de que ellos dan consentimiento de su participación, detrás hay incentivos de acuerdo a sus condiciones como el poder aportar a que el estudio resulte efectivo. Además, como no saben a cuál de los cuatro grupos pertenecerán, tendrán la esperanza de hacer parte del grupo exitoso. Adicionalmente, el hecho de que puedan acceder a medicamentos que resultan costosos para el mismo experimento, potencian aún más el incentivo a dar consentimiento porque en la realidad las personas del estudio, dadas sus condiciones económicas, no podrían costearlas. Más aún, al darles la libertad a las personas de que se autosuministren el medicamento, estas fácilmente pueden a su vez revenderlo de tal forma que puedan usar el dinero para cubrir sus necesidades básicas para subsistir.

Se está incumpliendo el principio de \textbf{no maleficencia} ya que se está exponiendo a población vulnerable, personas que sufren de VIH, y los encargados del estudio son conscientes de este potencial perjuicio ante esta población vulnerable. Adicionalmente, no se tiene en cuenta el principio de \textbf{justicia} ya que el experimento es realizado en África, donde hay condiciones económicas poco favorables, lo cual puede dificultar que las personas den continuidad al tratamiento.


\section*{Primera Pregunta}
\subsection*{Enunciado}
¿Se justifica el uso del placebo en este caso?

\subsection*{Respuesta}

En este caso, hay muchas variables que se deben tomar en cuenta. En particular dividiremos estas consideraciones en tres secciones distintas. Sin embargo, es importante primero tener un contexto (externo a la investigación) para poder analizarlo de manera correcta.

\subsection*{Contexto}

La asociación medica mundial constituyo en la declaración de Helsinki que en toda investigación los resultados de una técnica deben ser contrastados con la técnica mas eficaz y segura con la que se cuente en el momento. Para esto, existen dos excepciones. La primera es que no exista ningún tratamiento eficaz probado hasta ese momento en cuyo caso el uso del placebo o ninguna intervención es aceptable. El segundo caso, es que por razones metodológicas solidas y convincentes, sea necesario para determinar la eficacia y seguridad de la técnica en comparación a alguna otra. Tomando esto en cuenta y con los principios básicos de ética de investigación estudiemos cada uno de los casos por los que el uso de placebos pueda resultar cuestionable en este caso.

\subsection*{Existencia de Tratamiento}

En este caso, debemos considerar que un tratamiento para casos similares existía. De hecho, el estudio se enmarca en una temporalidad interesante. Este fue comenzado en 1993. Para este momento, los tratamientos para prevenir ya existían y se habían probado ampliamente durante varios años ya. Sin embargo, el primer momento en el que se instauro como obligatorio fue en 1994 y únicamente para estados unidos. Por lo tanto tenemos dos consideraciones importantes que hacer. Primero el estudio inicio antes de que se instaurara el tratamientos obligatorio y cuando este se instauro su alcance no incluía (legalmente) a las personas en las que se estaba realizando el estudio. Sin embargo, el tratamiento existía y se había comprobado su eficacia en varias ocasiones. Por lo tanto, no podemos decir que este no existía, simplemente no era obligatorio.

\subsection*{Metodología}

En este caso las consecuencias son interesantes. En el texto guia la justificación dada es que por la hipersensibilidad desarrollada por las personas VIH positivas estas podían ser afectadas por el medicamento en vez de resultarles beneficioso. En este caso, asumiendo que el argumento no era únicamente esta frase y en el estudio profundizaron mas sobre ello y dieron mejores argumentos para esto (Cosa que es posible y con la poca información que teneos no podemos saber con certeza) este seria un argumento valido y podría justificar el uso de placebos (es un poco complejo juzgar la validez de un argumento en función de una sola frase pero podemos asumir que esto es correcto). Sin embargo, este argumento tiene que tener en cuenta también la autonomía y el conocimiento de las personas sobre el estudio y por tanto debemos hablar también de los principios basicos de la etica de la investigación.

\subsection*{Ética de la Investigación}

Para este apartado, debemos tomar en cuenta los dos puntos anteriores. Primero, debemos ser conscientes de que el tratamiento existía. Por otro lado, existe un argumento dado por los investigadores y que asumimos que esta mucho mejor fundamentado en el estudio original de lo que dice la guia. Sin embargo, en esta situación hay una gran dependencia en el consentimiento informado y es que este fue un gran error pues se realizo de manera oral y eso no tiene validez.

\subsection*{Análisis Separado del Contexto}

Por ultimo (Aunque quizás mas importante) debemos hablar de los principios de ética de la investigación en relación al uso de placebos intentando separarnos del contexto pues como ya se hablo previamente en este contexto era inaceptable el uso de placebo con un "consentimiento oral". Sin embargo el uso de placebos tiene una encrucijada interesante.

Por un lado, es una excelente herramienta para comparar los resultados de un fármaco y saber que si es beneficioso para los usuarios. Sin embargo, es una delicada balanza entre los beneficios experimentales que tiene el uso de placebos y el daño que se le puede ocasionar a una persona por usarlo. Es cierto también que nos referimos a un estudio y como tal no se puede garantizar la eficacia del medicamento y ni siquiera aquellos que reciben el principio activo tienen garantizados que les haga bien. Esto en términos mas formales quiere decir que en esencia el placebo cumple el principio de no maleficencia. En un estudio no se garantiza el éxito y en ciertas circunstancias la información que aporta al estudio es muy importante. Por otro lado, como ya habíamos mencionado previamente existen maneras de conseguir comparaciones con otros fármacos que puedan también no solo comparar e tratamiento en relación al placebo si no en relación a otros tratamientos de los cuales se ha comprobado su eficacia y por lo tanto se actuaria cumpliendo el principio de beneficencia.

\subsection*{Conclusion Pregunta}

Con  todo lo hablado previamente podemos concluir que en principio el uso del placebo cumple un objetivo cientifico con la visión en la no maleficiencia. No se desea dañar a nadie por medio del uso de placebos. Sin embargo, para que este tenga un uso etico deben cumplirse muchas condiciones que en el estudio en cuestion no se cumplieron y en consecuencia no se justifica su uso.

\section*{Segunda pregunta}
\subsection*{Enunciado}
Dado que se espera que el régimen con múltiples medicamentos, aun cuando sea igualmente eficaz, resulte mucho más costoso que otro exclusivamente a base de
INH y, por ende, que es poco probable que sea accesible desde la perspectiva económica para la mayoría de personas con necesidades ¿debió haberse realizado
Este estudio en África? ¿Podría haberse realizado en países donde el costo de tratamiento no hubiera sido un problema?

\subsection*{Respuesta}
El estudio se realiza en África con el argumento de que es la población mayormente afectada por TB y VIH en el mundo, sin embargo, esto causa múltiples dilemas éticos, inicialmente existe un conflicto de interés pues el costo de realizar un ensayo en África es menor al de realizarlo en un país desarrollado con entes de control robustos, y la población objetivo del medicamento no es el grupo poblacional en el que se está probando pues no podrá pagarlo. Al ser una población vulnerable se viola el principio de justicia pues la realización del ensayo se aprovecha de su condición vulnerable grupal. También se considera el hecho de que el grupo de estudio se les proporcionaron suministros mensuales de los medicamentos mismos para que ellos mismos se los administren, esto genera un problema ya que el medicamento puede ser revendido y utilizado para cubrir sus necesidades básicas. El estudio pudo haber sido realizado en otro país, pero está claro que la situación de vulnerabilidad y carencia de control gubernamental fueron motivadores para realizar el estudio en África, también la toma de datos se vería drásticamente reducida debido a la falta de personas con estas enfermedades.

\section*{Conclusión}



\bibliographystyle{abbrv}
\bibliography{Referencias}
\end{document}

\documentclass[12pt]{article}
\usepackage{array}
\usepackage{color}
\usepackage{amsthm}
\usepackage{eufrak}
\usepackage{lipsum}
\usepackage{pifont}
\usepackage{yfonts}
\usepackage{amsmath}
\usepackage{amssymb}
\usepackage{ccfonts}
\usepackage{comment} \usepackage{amsfonts}
\usepackage{fancyhdr}
\usepackage{graphicx}
\usepackage{listings}
\usepackage{mathrsfs}
\usepackage{setspace}
\usepackage{textcomp}
\usepackage{blindtext}
\usepackage{enumerate}
\usepackage{microtype}
\usepackage{xfakebold}
\usepackage{kantlipsum}
%\usepackage{draftwatermark}
\usepackage[spanish]{babel}
\usepackage[margin=1.5cm, top=2cm, bottom=2cm]{geometry}
\usepackage[framemethod=tikz]{mdframed}
\usepackage[colorlinks=true,citecolor=blue,linkcolor=red,urlcolor=magenta]{hyperref}

%//////////////////////////////////////////////////////
% Watermark configuration
%//////////////////////////////////////////////////////
%\SetWatermarkScale{4}
%\SetWatermarkColor{black}
%\SetWatermarkLightness{0.95}
%\SetWatermarkText{\texttt{Watermark}}

%//////////////////////////////////////////////////////
% Frame configuration
%//////////////////////////////////////////////////////
\newmdenv[tikzsetting={draw=gray,fill=white,fill opacity=0},backgroundcolor=none]{Frame}

%//////////////////////////////////////////////////////
% Font style configuration
%//////////////////////////////////////////////////////
\renewcommand{\familydefault}{\ttdefault}
\renewcommand{\rmdefault}{tt}

%//////////////////////////////////////////////////////
% Bold configuration
%//////////////////////////////////////////////////////
\newcommand{\fbseries}{\unskip\setBold\aftergroup\unsetBold\aftergroup\ignorespaces}
\makeatletter
\newcommand{\setBoldness}[1]{\def\fake@bold{#1}}
\makeatother

%//////////////////////////////////////////////////////
% Default font configuration
%//////////////////////////////////////////////////////
\DeclareFontFamily{\encodingdefault}{\ttdefault}{%
  \hyphenchar\font=\defaulthyphenchar
  \fontdimen2\font=0.33333em
  \fontdimen3\font=0.16667em
  \fontdimen4\font=0.11111em
  \fontdimen7\font=0.11111em}



\begin{document}
    %//////////////////////////////////////////////////////
% Heading Configuration
%//////////////////////////////////////////////////////
\pagestyle{fancy}
\thispagestyle{plain}
\fancyhead[RO,L]{\textbf{Teoria de Grafos}}
\fancyhead[LO,L]{\textbf{Tarea 2}}
\setlength{\headheight}{16.0pt}

%//////////////////////////////////////////////////////
% Subsections Configuration
%//////////////////////////////////////////////////////
\renewcommand*\thesubsection{\arabic{subsection}}
\newcounter{counter}
\newlength{\palabra}
\settowidth{\palabra}{counter 999.}
\newcommand{\makeboxlabel}[1]{\fbox{#1.}\hfill}

%//////////////////////////////////////////////////////
% Personalized commands configuration
%//////////////////////////////////////////////////////
\newcommand{\N}{\mathbb{N}}
\newcommand{\Z}{\mathbb{Z}}
\newcommand{\Q}{\mathbb{Q}}
\newcommand{\R}{\mathbb{R}}
\newcommand{\C}{\mathbb{C}}
\newcommand{\re}{\operatorname{Re}}
\newcommand{\im}{\operatorname{Im}}
\newcommand{\Aut}{\operatorname{Aut}}
\newcommand{\GCD}{\operatorname{GCD}}
\newcommand{\LCD}{\operatorname{LCD}}
\linespread{1} %Line spacing

%//////////////////////////////////////////////////////
% Inline code configuration
%//////////////////////////////////////////////////////
\lstset{
gobble=5,
numbers=left,
frame=single,
framerule=1pt,
showtabs=False,
showspaces=False,
showstringspaces=False,
backgroundcolor=\color{gray}}

%//////////////////////////////////////////////////////
% Problem list configuration
%//////////////////////////////////////////////////////
\newenvironment{problems}
  {\begin{list}
     {{\fbseries Problem \arabic{counter}.}}
    {\usecounter{counter}
     \setlength{\labelsep}{1em}
     \setlength{\itemsep}{2pt}
     \setlength{\leftmargin}{2em}
     \setlength{\rightmargin}{0cm}
     \setlength{\itemindent}{1em} }}
{\end{list}}

%//////////////////////////////////////////////////////
% Appendix configuration
%//////////////////////////////////////////////////////
\newenvironment{Appendix}
  {\begin{list}
     {{\fbseries Lemma \arabic{counter}.}}
    {\usecounter{counter}
     \setlength{\labelsep}{1em}
     \setlength{\itemsep}{2pt}
     \setlength{\leftmargin}{2em}
     \setlength{\rightmargin}{0cm}
     \setlength{\itemindent}{1em} }}
{\end{list}}

%//////////////////////////////////////////////////////
% Notes configuration
%//////////////////////////////////////////////////////
\newenvironment{notes}
  {\begin{list}
     {{\fbseries Note \arabic{counter}.}}
    {\usecounter{counter}
     \setlength{\labelsep}{1em}
     \setlength{\itemsep}{2pt}
     \setlength{\leftmargin}{2em}
     \setlength{\rightmargin}{0cm}
     \setlength{\itemindent}{1em} }}
{\end{list}}

%//////////////////////////////////////////////////////
% Activity Information
%//////////////////////////////////////////////////////
\vspace*{-1cm}
\hrule width \hsize \kern 1mm \hrule width \hsize height 2pt
\begin{center}
   \parbox[c]{.32\textwidth}{
   \hspace{1cm}\\
   Sergio Montoya Ramirez\\
   202112171}
%   Luis Ernesto Tejón Rojas\\
%   202113150}
   \hspace*{\fill}
   \parbox[c]{.35\textwidth}{\centering
   Universidad de Los Andes\\
   Tarea 1\\
   Teoria de Grafos\\
   }
   \hspace*{\fill}
   \parbox[c]{.3\textwidth}{
   \begin{flushleft}
      Bogotá D.C., Colombia\\
      \today
   \end{flushleft}}
\end{center}
\hrule width \hsize height 2pt \kern 1mm \hrule width \hsize

\bigskip

\bigskip


%%%%%%%%%%%%%%%%%%%%%%%%%%%%%%%%%%%%%%% TALLER 2 %%%%%%%%%%%%%%%%%%%%%%%%%%%%%%%%%%%%%%%%%%%%%%%%%%%
    \begin{enumerate}
	%%%%%%%%%%%%%%%%%%%%%%%%%%%%%%%%% Punto 1 %%%%%%%%%%%%%%%%%%%%%%%%%%%%%%%%%%%%%%%%%%%%%%%%%%%%%%%%%%%
      \item  Solucione las siguientes ecuaciones diferenciales:
	\begin{enumerate}
	  \item $3y''+2y'+y=0;y\left( 0 \right) = 0; y'\left( 0 \right) = 1$
	  \item $y'' + 8y' + 16y = 0$
	  \item $y'' + 3y = -48x^2e^{3x}$ 
	  \item $y''-4y'+8y=e^{2x}+\sin\left( 2x \right) $
	  \item $y'''-3y''+3y'-y=x-4e^{x}$
	  \item $y''+3y'+2y=\frac{1}{e^{x}+1}$
	\end{enumerate}
	\textit{Solución:}
	\begin{enumerate}
	  \item Sea la ecuación $3y''+2y'+y=0$, Luego obtenemos la ecuación características.  $3r^2+2r+1=0$.
	    Podemos, entonces, resolver esta ecuación cuadrática factorizando tal que: \[
	    \left( r+1 \right) \left( 3r+1 \right) =0
	    .\] Esto nos da dos raices: \[
	    r_1 = -1 \land r_2 = \frac{-1}{3}
	    .\] Asi mismo, la solución general de la ecuación diferencial homogénea se puede escribir como: \[
	    y_h = c_1e^{r_1x}+c_2e^{r_2x}\implies y_h = c_1e^{-x}+c_2e^{-\frac{x}{3}x}
	    .\] Ahora bien, Dado que tenemos las condicones iniciales podemos encontrar los valores de las constantes $c_i$ en la solución general.

	    Si  $x=0$ entonces:
	     \begin{align*}
	      y_h =c_1e^{0}+c_2e^{0}=c_1+c_2=0 \land y_h' = -c_1e^{0}-\frac{c_2}{3}e^{0}=-c_1-\frac{c_2}{3}=1
	    .\end{align*}

	    Lo que significa:
	    \begin{align*}
	      c_1 = -\frac{3}{2}\\
	      c_2 = \frac{3}{2}
	    .\end{align*}
	  \item Sea la ecuación $y''+8y'+16y=0$, Luego obtenemos la ecuación caracteristicas  $r^2+8r+16=0$
	    Podemos, entonces, resolver esta ecuación cuadrática factorizando, tal que:
	    \begin{align*}
	      \left( r+4 \right) \left( r+4 \right) = 0
	    .\end{align*} Esto nos da dos raices: \[
	    r_1=r_2=-4
	    .\] Luego, \[
	    y_h = c_1e^{r_1x}+c_2xe^{r_2x}=c_1e^{-4x}+c_2xe^{-4x}
	    .\] 
	  \item Sea la ecuación $y''+3y=-48x^2e^{3x}$. Luego, decimos que: \[
	  r^2+3r=0
	  .\] Podemos, entonces, resolver esta ecuación cuadrática, tal que: \[
	  r_1 = \sqrt{3i} \land r_2=-\sqrt{3i} 
	  .\] Luego,
	  \begin{align*}
	    y_h = c_1\cos\sqrt{3i} + c_2\sin\sqrt{3i} 
	  .\end{align*}
	  Por otro lado, decimos que:
	  \begin{align*}
	    y_p = Ax^2e^{3x}+Bxe^{3x}+Ce^{3x}\\
	    y_p' = 2Axe^{3x}+3Ax^2e^{3x}+Be^{3x}+3Bxe^{3x}+3Ce^{3x}\\
	    y_p'' = 2Ae^{3x}+12Axe^{3x}+9Ax^2e^{3x}+6Be^{3x}+9Bxe^{3x}+9Ce^{3x}
	  .\end{align*}
	  Luego,
	  \begin{align*}
	    y''+3y'=12Ax^2e^{3x}+12Axe^{3x}+12Bxe^{3x}+2Ae^{3x}+6Be^{3x}+12Ce^{3x}=-48x^2e^{3x}\\
	    \implies 12Ax^2e^{3x}=-48x^2e^{3x}\land 12Axe^{3x}+12Bxe^{3x}=0\implies A+B=0 \\
	    \implies A = -4 \land B = 4
	  .\end{align*}
	  En base a esto, podemos hallar ahora C, dado que: \[
	  2Ae^{3x}+6Be^{3x}12Ce^{3x}=0 \implies 2A + 6B + 12C = 0 \implies C = \frac{-4}{3}
	  .\] Por ultimo,
	  \begin{align*}
	    y = c_1\cos\sqrt{3i} + c_2\sin\sqrt{3i} - 4x^2e^{3x} + 4xe^{3x}-\frac{4}{3}e^{3x}\\
	    y = c_1\cos\sqrt{3i} + c_2\sin\sqrt{3i} + e^{3x}\left( -4x^2+4x - \frac{4}{3} \right) 
	  .\end{align*}
	  \item Sea $y''-4y'+8y=e^{2x}+\sin\left( 2x \right) $ con este caso, lo primero que debemos hacer es encontrar la ecuación lineal homogenea correspondiente
	    \begin{align*}
	      y'' + py'+qy = 0
	    .\end{align*}
	    Para lo cual debemos hallar primero las raices de la ecuación característica que en nuestro caso tiene la forma $r^2-4r+8=0$. La cual tiene soluciones complejas y en particular da
	    \begin{align*}
	      k_1 = 2 - 2i\\
	      k_2 = 2 + 2i
	    .\end{align*}
	    Como se tienen dos raices la ecuación caracterizticas la solución tiene la forma:
	    \begin{align*}
	      y = c_1e^{x\left( 2-2i \right) }+c_2e^{x\left( 2+2i \right) }
	    .\end{align*}

	    Ahora debemos encontrar la solución heterogénea. Para comenzar debemos saber que la solución general tiene la forma
	    \begin{align*}
	      y_c = C_1 e^{x\left( 2-2i \right) }+ C_2e^{x\left( 2+2i \right) }
	    .\end{align*}

	    Con esto creamos el sistema:
	    \begin{align*}
	      0 = e^{x\left( 2-2i \right) }u_1 + e^{x\left( 2+2i \right) }u_2\\
	      \left( 2-2i \right) e^{x\left( 2-2i \right) }u_1' + \left( 2+2i \right) e^{x\left( 2+2i \right) }u_2' = e^{2x}+\sin\left( 2x \right) 
	    .\end{align*}

	    Resolviendo
	    \begin{align*}
	      u_1' = \frac{ie^{2ix}}{4}+\frac{ie^{-2x}e^{2ix}\sin\left( 2x \right) }{4}\\
	      u_2' = - \frac{ie^{-2ix}}{4}-\frac{ie^{-2x}e^{-2ix}\sin\left( 2x \right) }{4} 
	    .\end{align*}

	    Ahora con esto podemos hallar $u_1$ y $u_2$ con lo que nos quedaria
	    \begin{align*}
	      u_1 = \frac{i\left( \left( -2+2i \right) \sin\left( 2x \right) -2\cos\left( 2x \right)  \right) e^{-2x+2ix}}{16+4\left( -2+2i \right)^2}+\frac{e^{2ix}}{8}+C_1\\
	      u_2 = \frac{i\left( \left( -2-2i \right) \sin\left( 2x \right) -2\cos\left( 2x \right)  \right) e^{-2x-2ix}}{16+4\left( -2-2i \right)^2}+\frac{e^{-2ix}}{8}+C_1
	    .\end{align*}

	    Ahora si juntamos todo lo que tenemos nos queda:
	    \begin{align*}
	      y = u_1e^{x\left( 2-2i \right) } + u_2e^{x\left( 2+2i \right) }\\
	      y = C_4e^{2x}e^{2xi}+\frac{e^{2x}}{4}+C_3e^{2x}e^{-2ix}+\frac{\sin\left( 2x \right) }{20}+\frac{\cos\left( 2x \right) }{10}
	    .\end{align*}
	  \item Lo primero que debemos hacer es encontrar la solución de la ecuación auxiliar la cual es $r^{3}-3r^{2}+3r - 1 = 0$. Esta ecuación se puede factorizar como $\left( x-1 \right) \left( x^2-2x+1 \right) = \left( x-1 \right) \left( x-1 \right) \left( x-1 \right) = 0$ por lo tanto, la unica solución es el $1$ lo que nos deja con una solución general de la forma: \[
	  y_c = c_1e^{x}+c_2xe^{x}+c_3x^{2}e^{x}
	  .\] Ahora, teniendo  $y_c$ podemos pensar en una solución:
	  \begin{align*}
	    y_p = Ax + B +Cx^{3}e^{x}\\
	    y_p' = A + 3Cx^2e^{x}+Cx^{3}e^{x}\\
	    y_p'' = 6Cxe^{x}+3Cx^2e^{x}+3Cx^2e^{x}+Cx^{3}e^{x}\\
	    y_p''' = 6Ce^{x} + 6Cxe^{x}+6Cxe^{x}+3Cx^2e^{x}+6Cxe^{x}\\+3Cx^2e^{x}+3Cx^2e^{x}+Cx^{3}e^{x}
	  .\end{align*}

	  Ahora, si sustituimos los valores nos queda
	  \begin{align*}
	    6Ce^{x} + 18Cxe^{x} + 9Cx^{2}+Cx^{3}e^{x}\\
	    -3\left( 6Cxe^{x}+6Cx^{2}e^{x}+Cx^{3}e^{x} \right) \\
	    +3 \left( A + 3Cx^2e^{x}+Cx^{3}e^{x} \right) \\
	    -\left( Ax+B+Cx^{3}e^{x} \right) \\
	    = 6Ce^{x}+3A - Ax- B
	  .\end{align*}
	  Ahora con las condiciones que teniamos previamente nos queda:
	  \begin{align*}
	    6Ce^{x}=-4e^{x} \implies C = -\frac{2}{3}\\
	    -Ax = x \implies A = -1\\
	    3A-B = 0 \implies B = -3
	    y_p = -x - 3 - \frac{2}{3}x^{3}e^{x}\\
	    y = c_1e^{x}+c_2xe^{x}+c_3x^{2}e^{x}-x - 3 - \frac{2}{3}x^{3}e^{x}
	  .\end{align*}
	  \item Para comenzar debemos tomar en cuenta la ecuación auxiliar $r^2+3r+2=0$ la cual puede ser factorizada a $\left( r+1 \right) \left( r+2 \right)=0$ por lo tanto $r=\left\{ -1,-2 \right\} $ y en consecuencia las soluciones de la ecuación homogenea asociada es:
	    \begin{align*}
	      g\left( x \right) = \frac{1}{e^{x}+1}\\
	      y_1 = e^{-x}\\
	      y_2 = e^{-2x}\\
	      y_c = c_1e^{-x}+c_2e^{-2x}
	    .\end{align*}
	    Ahora entonces el Wronskiano:
	    \begin{align*}
	      \begin{bmatrix} y_1 & y_2 \\ y_1' & y_2' \end{bmatrix} \\
				  &= \begin{bmatrix} e^{-x} & e^{-2x} \\ -e^{-x} & -2e^{-2x} \end{bmatrix}  \\
				  &= -2e^{-3x} + e^{-3x} = -e^{-3x} \\
	    .\end{align*}

	    Ahora ya con esto podemos calcular $w_1$ y $w_2$
	    \begin{align*}
	      w_1 = \begin{bmatrix} 0 & y_2 \\ g\left( x \right) & y_2' \end{bmatrix} = \begin{bmatrix} 0 & e^{-2x} \\ \frac{1}{e^{x}+1} & -2e^{-2x} \end{bmatrix} = -\frac{e^{-2x}}{1+e^{x}}\\
	      w_2 = \begin{bmatrix} y_1 & 0 \\ y_1' & g\left( x \right)  \end{bmatrix} = \begin{bmatrix} e^{-x} & 0 \\ -e^{-x} & \frac{1}{1+e^{x}} \end{bmatrix} = \frac{e^{-x}}{1+e^{x}}
	    .\end{align*}

	    Ahora con esto, podemos 
	    \begin{align*}
	      u_1 = \int \frac{w_1}{w}dx = \int \frac{\frac{e^{-2x}}{1+e^{x}}}{e^{-3x}}dx\\
	      = \int \frac{1}{e^{-x}\left( 1+e^{x} \right) }dx = \int \frac{e^{x}}{1+e^{x}}dx=\ln\left( 1+e^{x} \right) \\
	      u_2 = \int \frac{w_2}{w}dx = \int \frac{\frac{e^{-x}}{1+e^{x}}}{-e^{-3x}}dx\\
	      = - \int \frac{1}{e^{-2x}\left( 1+e^{x} \right) }dx = - \int \frac{e^{2x}}{1+e^{x}}dx\\
	      p = 1+e^{x}\\
	      p-1&= e^{x} \\
	      dp = e^{x}dx\\
	      u_2=-\int \frac{p-1}{p} = -\int\left( 1-\frac{1}{p} \right) dP = -\left( p-\ln\left( p \right)  \right) = -p + \ln\left( p \right)\\
	      = -\left( 1+e^{x} \right) + \ln\left( 1 + x \right)  = -1-e^{x}+\ln\left( 1+e^{x} \right) \\
	    .\end{align*}

	    Ahora por ultimo debemos reunir todo lo que sabemos hasta ahora, lo que nos deja con:
	    \begin{align*}
	      y_p = u_1y_1 + u_2y_2\\
	      y_p = \left( \ln\left( 1+e^{x} \right)  \right) \left( e^{-x} \right) + \left( -1-e^{x} + \ln\left( 1+e^{x} \right)  \right) \left( e^{-2x} \right) \\
	      y_p = e^{-x}\ln\left( 1+e^{x} \right) + e^{-2x}\ln\left( 1+e^{x} \right) 
	    .\end{align*}
	\end{enumerate}

	Por lo tanto la solución general es $y = y_h + y_p$
	\begin{align*}
    y =  c_1e^{-x}+c_2e^{-2x} +e^{-x}\ln\left( 1+e^{x} \right) + e^{-2x}\ln\left( 1+e^{x} \right)
	.\end{align*}
	%%%%%%%%%%%%%%%%%%%%%%%%%%%%%%%%% Punto 2 %%%%%%%%%%%%%%%%%%%%%%%%%%%%%%%%%%%%%%%%%%%%%%%%%%%%%%%%%%%
      \item Un objeto de masa $100g$ estira un resorte $5cm$. La masa es puesta en movimiento hacia abajo con una velocidad $100 \frac{cm}{s}$. Halle la posición del objeto en cualquier momento del tiempo $t$, suponiendo que no existe ningún tipo de amortiguamiento. Establezca la posición del objeto en cualquier momento del tiempo $t$. Determine el tiempo donde el objeto retorna por primera vez a la posición de equilibrio.

	En este caso tenemos que partir desde la expresión general de \[
	mu\left( t \right) '' + ku\left( t \right)  = 0
	.\] En esta situación debemos saber que la solución general es de la forma:
	\begin{align*} u\left( t \right) = A\cos\left( \omega t \right) + B\sin\left( \omega t \right) 
	.\end{align*}

	En donde \[
	\omega = \sqrt{\frac{k}{m}} 
	.\] Por lo tanto nos interesa saber cual es la constante de este resorte. Para eso, vamos a tomar los dos datos iniciales y la ley de Hook. Con esto podemos obtener que: \[
	F = ke
	.\] Donde $k$ es la constante,  $e$ es la elongación y $F$ es la fuerza que en particular para este caso coincide con el peso de la masa. Por lo tanto, tenemos: \[
	O.1kg\cdot 9.8 \frac{m}{s^2} = 0.98 N = k \left( 5 cm \right) 
	.\] 
	Con esto entonces despejamos simplemente $k$ y nos queda para esa situación: \[
	\frac{0.98N}{0.05m} = k = 19.6  \frac{kg}{s^2}
	.\] 
	Ahora con esto podemos volver a la ecuación de $\omega$ con lo que nos quedaria que es
	\begin{align*}
	  w &= \sqrt{\frac{19.6 \frac{kg}{s^2}}{0.1 kg}}  \\
	  w &= \sqrt{1.96 \frac{1}{s^2}}  \\
	  w &= \frac{1.4}{s} \\
	.\end{align*}

	Con esto ya obtenido nos queda entonces unicamente reemplazar $\omega$ y encontrar los valores de $A$ y $B$ en función de las condiciones iniciales. Por lo tanto, debemos desarrollar como sigue:
	\begin{align*}
	  u &= A\cos\left( 1.4 t \right) + B\sin\left( 1.4 t\right) \\
	  5 &= A\cos\left( 0 \right) + B\sin\left( 0 \right)  \\
	  5 &= A \\
	  u' &= -1.4A\sin\left( 1.4t \right) + B 1.4 \sin\left( 1.4 t \right)  \\
	  100 &= B\cdot 1.4 \\
	  B &= 71.42 \\
	.\end{align*}

	%%%%%%%%%%%%%%%%%%%%%%%%%%%%%%%%% Punto 3 %%%%%%%%%%%%%%%%%%%%%%%%%%%%%%%%%%%%%%%%%%%%%%%%%%%%%%%%%%%
      \item En cada caso encuentre una ecuación diferencial de segundo orden para la cual $y_1$ y $y_2$ sean soluciones:
	\begin{enumerate}
	  \item $y_1\left( x \right) = e^{x};y_2\left( x \right) =e^{-x}$ 
	    En este caso, la via mas facil es encontrar una ecuación de segundo orden tal que para su solución $ar^2+br+c$ tenga como soluciones $r=\{1,-1\}$. En este caso, sabemos que la ecuación $y''-y = 0$ cumple esta condicion a la perfección. 
	  \item $y_1\left( x \right) = e^{2x};y_2\left( x \right) = xe^{2x}$

	    Sea:
	    \begin{align*}
	      y''+P\left( x \right) y' + Q\left( x \right) y = 0\\
	      y_1\left( x \right) = e^{2x};y_1'=2e^{2x};y_1''=4e^{2x}\\
	      y_2\left( x \right) = xe^{2x};y_2'=e^{2x}+2xe^{2x};y_2''=4e^{2x}\left( 1+x \right) 
	    .\end{align*}

	    Ahora sustituimos estas derivadas en la ecuación diferencial tal que 
	    \begin{align*}
	      4e^{2x}+4xe^{2x}+P\left( x \right) \left( e^{2x}+2xe^{2x} \right) + Q\left( x \right) \left( xe^{2x} \right) = 0\\
	      \Rightarrow \left( 4+P\left( x \right)  \right) e^{2x}+\left( 4+2P\left( x \right) + Q\left( x \right)  \right) xe^{2x}=0
	    .\end{align*}
	    Para que esta ecuación sea válida para cualquier $x$ los coeficientes de $e^{2x}$ y $xe^{2x}$ deben ser igual a $0$. Por lo tanto:
	    \begin{align*}
	      4+P\left( x \right) = 0\\
	      4+2P\left( x \right) + Q\left( x \right) = 0\\
	      P\left( x \right) = -4\\
	      Q\left( x \right) = 4
	    .\end{align*}

	    Lo que queda como:
	    \begin{align*}
	      y'' - 4y' + 4y = 0
	    .\end{align*}
	\end{enumerate}
	%%%%%%%%%%%%%%%%%%%%%%%%%%%%%%%%% Punto 4 %%%%%%%%%%%%%%%%%%%%%%%%%%%%%%%%%%%%%%%%%%%%%%%%%%%%%%%%%%%
      \item Halle la solución general de la ecuación diferencial dada utilizando el método de reducción de orden y la solución de la ecuación homogénea $\left( 1-x^2 \right) y''-2xy'+6y = 0; y_1=3x^2-1$ 

	Para iniciar partimos de la ecuación original: \[
	  (1-x^2)y''-2xy'+6y=0
	.\] y ademas tenemos la siguiente información que nos resultara de interes
	\begin{table}[htpb]
	  \centering
	  \begin{tabular}{|c|c|}
	    \hline
	    $y_1 = 3x^2-1$ & $y = vy_1$ \\
	    $y_1' = 6x$ & $y' = v'y_1 + vy_1'$\\
	    $y_1'' = 6$ & $y'' = v''y_1 + 2v'y_1' + vy_1''$\\
	    \hline
	  \end{tabular}
	\end{table}

	Ahora con esto, si reemplazamos los valores que conocemos en la ecuación original desarrollamos como sigue
	\begin{align*}
	  \left( 1-x^2 \right) \left( v''y_1+2v'y_1'+vy_1'' \right) -2x\left( v'y_1 + vy_1' \right) + 6y_1v = 0 \\
	  \left( 1-x^2 \right) \left( v''y_1 + 2v'y_1' \right) - 2xv'y_1 + \left[ \left( 1-x^2 \right) y_1'' - 2xy_1' + 6y_1 \right]v = 0\\
	  \left( 1-x^2 \right) v''y_1 + \left[ \left( 1-x^2 \right) y_1' - xy_1\right]2v' = 0 
	.\end{align*}

	Ahora que tenemos la ecuación de una manera mucho mas comprensible podemos separar las variables lo que nos quedaria como:
	\begin{align*}
	  \left( 1-x^2 \right) v''y_1 = 2v'\left[ xy_1 - \left( 1-x^2 \right) y_1' \right] \\
	  \frac{v''}{v'} = \frac{2\left[ xy_1 - \left( 1-x^2 \right) y_1'\right] }{\left( 1-x^2 \right) y_1}
	.\end{align*}

	Ahora si hacemos una reducción de variable tal que $w = v' \land w'=v''$ con lo que esto nos queda como
	\begin{align*}
	  \frac{dw}{w} = \frac{2\left[ xy_1 - \left( 1-x^2 \right) y_1'\right] }{\left( 1-x^2 \right) y_1} dx\\
	  \frac{dw}{w} = 2\left[ \frac{x}{\left( 1-x^2 \right) } - \frac{y_1'}{y_1} \right]dx\\
	  \int \frac{dw}{w} = 2\int \left(\frac{x}{1-x^2} - \frac{y_1'}{y_1}\right)dx\\
	  \ln\left( w \right) = 2 \left( -\frac{1}{2}\ln\left( 1-x^2 \right) - \ln\left( y_1 \right)  \right)\\
	  w = e^{-\ln\left( 1-x^2 \right) - 2\ln\left( y_1 \right) }\\
	  w = \left( 1-x^2 \right)^{-1} \left( 3x^2-1 \right)^{-2}
	.\end{align*}

	Entonces, con esto podemos volver a la definición de $w$ y por lo tanto sabemos que
	\begin{align*}
	  v = \int \frac{1}{\left( 1-x^2 \right) \left( 3x^2-1 \right)^2}\\
	  v = \frac{1}{8}\ln\left( x+1 \right) - \frac{1}{8}\ln\left( x-1 \right) - \frac{\sqrt{3} }{8\left( \sqrt{3} x+1 \right) }-\frac{\sqrt{3} }{8\left( \sqrt{3} x-1 \right) }\\
	  y = \frac{1}{8}\ln\left( x+1 \right) - \frac{1}{8}\ln\left( x-1 \right) - \frac{\sqrt{3} }{8\left( \sqrt{3} x+1 \right) }-\frac{\sqrt{3} }{8\left( \sqrt{3} x-1 \right) }\left( 3x^2-1 \right) \\
	  y = \frac{12x^2\ln\left( \sqrt[8]{\frac{x+1}{x-1}}  \right) -3x-4\ln\left( \sqrt[8]{\frac{x+1}{x-1}}  \right) }{4}
	.\end{align*}
	%%%%%%%%%%%%%%%%%%%%%%%%%%%%%%%%% Punto 5 %%%%%%%%%%%%%%%%%%%%%%%%%%%%%%%%%%%%%%%%%%%%%%%%%%%%%%%%%%%
      \item Encuentre una solución particular de la ecuación $x^2y''-x\left( x+2 \right)y'+\left( x+2 \right)y=2x^3$ conociendo que $y_1\left( x \right) = x$ y $y_2\left( x \right) = xe^{x}$ son soluciones independientes de la ecuación homogénea asociada.

	En este caso vale la pena iniciar por despejar $y''$ con lo que nos quedaria
	\begin{align*}
	  y'' - \frac{\left( x+2 \right) }{x} y' + \frac{\left( x+2 \right) }{x^2}y=2x
	.\end{align*}
	Con esto entonces podemos desarrollar
	\begin{align*}
	  y_h &= c_1y_1\left( x \right) + c_2y_2\left( x \right) \\
	  0 &=  u_1'y_1+u_2'y_2\\
	  y_p &= u_1y_1+u_2y_2 \\
	  0 &= u_1y_1+u_2y_2 \\
	  2x &= u_1'y_1+u_{2}'y_2=2x
	.\end{align*}

	Con esto por tanto nos queda 
	\begin{align*}
	  W &= \begin{bmatrix} x & xe^{x}\\ 1 & e^{x}+xe^{x} \end{bmatrix} = \left( xe^{x}+x^2e^{x} \right) - xe^{x}= x^2e^{x}\\
	  u_1' &= \begin{bmatrix} 0 & xe^{x} \\ 2x & e^{x}+xe^{x} \end{bmatrix} = \frac{-2x^2e^{x}}{w}=-2\\
	  u_2' &= \begin{bmatrix} x & 0 \\ 1 & 2x \end{bmatrix} = \frac{2x^2}{w}=2e^{-x} \\
	  u_1 &= -2x \\
	  u_2 &= -2e^{-x}
	.\end{align*}

	Por lo tanto $y$ quedaria como
	\begin{align*}
	  y = y_1 + y_2 + \left( -2x^2 \right) + \left( -2x \right) 
	.\end{align*}
    \end{enumerate}
\end{document}

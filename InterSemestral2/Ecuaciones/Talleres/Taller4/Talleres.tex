  \documentclass[12pt]{exam}
\usepackage{amsthm}
\usepackage{libertine}
\usepackage[utf8]{inputenc}
\usepackage[margin=1in]{geometry}
\usepackage{amsmath,amssymb}
\usepackage{multicol}
\usepackage[shortlabels]{enumitem}
\usepackage{siunitx}
\usepackage{cancel}
\usepackage{graphicx}
\usepackage{pgfplots}
\usepackage{listings} \usepackage{tikz}


\pgfplotsset{width=10cm,compat=1.9}
\usepgfplotslibrary{external}
\tikzexternalize

\newcommand{\class}{Ecuaciones Diferenciales} % This is the name of the course 
\newcommand{\examnum}{Taller 4} % This is the name of the assignment
\newcommand{\examdate}{\today} % This is the due date
\newcommand{\timelimit}{}





\begin{document}
\pagestyle{plain}
\thispagestyle{empty}

\noindent
\begin{tabular*}{\textwidth}{l @{\extracolsep{\fill}} r @{\extracolsep{6pt}} l}
	\textbf{\class} & \textbf{Name:} & \textit{Sergio Montoya}\\ %Your name here instead, obviously 
	\textbf{\examnum} &&\\
	\textbf{\examdate} &&
\end{tabular*}\\
\rule[2ex]{\textwidth}{2pt}
% ---

\begin{enumerate}
  \item Halle la solución general de los siguientes sistemas de ecuaciones diferenciales:
    \begin{enumerate}
      \item $X'=\begin{pmatrix}
	  1 & 1 & 2 \\
	  1 & 2 & 1 \\
	  2 & 1 & 1 
	\end{pmatrix} X$

    Ahora bien, para comenzar tenemos que encontrar el polinomio caracteristico.
    En particular, esto significa que debemos hallar $det\left( A - \lambda I \right) $ y en consecuencia nos interesa
    \begin{align*}
        det\left( \begin{pmatrix} 
                1 - \lambda & 1 & 2 \\
                1 & 2 - \lambda & 1 \\
                2 & 1 & 1 - \lambda
        \end{pmatrix}  \right) &= - \left( \left( 1-\lambda \right) - 2 \right) + (2-\lambda)\left( \left( 1-\lambda \right)^{2} - 4 \right)   \\
         & - \left( \left( 1-\lambda \right) - 2 \right)  \\
         &= - \left( -1-\lambda \right) + \left( 2 - \lambda \right) \left( 1 - 2\lambda + \lambda^2 - 4 \right) - \left( -1-\lambda \right)  \\
         &= 2 + 2\lambda + \left( 2 - 4\lambda + 2\lambda^2-8-\lambda + 2\lambda^2 - \lambda^3 + 4\lambda \right) \\
         &= -\lambda^3 + 4\lambda^2 + \lambda - 4 \\
    .\end{align*}
    Ahora con esto debemos encontrar los valores propios lo cual significa que debemos encontrar los puntos en donde
    este polinomio vale 0. Esto lo podemos encontrar por medio de la regla de Ruffini y encontramos que sus 
    raice son: $1, -1, 4$. Que como se podra notar son  $3$ tal cual como esperabamos y por consecuencia no hay repetición.

    Por lo tanto, podemos ahora encontrar los vectores propios de esta matriz:
    \begin{align*}
        \begin{pmatrix} 
            1 - \lambda & 1 & 2 \\
            1 & 2 - \lambda & 1 \\
            2 & 1 & 1 - \lambda \\
        \end{pmatrix} \alpha_n = \begin{pmatrix} 0 \\ 0 \\ 0 \end{pmatrix} \\
        \begin{pmatrix} 
            1 - 1 & 1 & 2 \\
            1 & 2 - 1 & 1 \\
            2 & 1 & 1 - 1\\
        \end{pmatrix} =  \begin{pmatrix} 
            0 & 1 & 2 \\
            1 & 1 & 1 \\
            2 & 1 & 0 \\
        \end{pmatrix} \Rightarrow \lambda_1 = \begin{pmatrix} 1 \\ -2 \\ 1 \end{pmatrix} \\
        \begin{pmatrix} 
            1 + 1 & 1 & 2 \\
            1 & 2 + 1 & 1 \\
            2 & 1 & 1 + 1\\
        \end{pmatrix} =  \begin{pmatrix} 
            2 & 1 & 2 \\
            1 & 3 & 1 \\
            2 & 1 & 2 \\
            \end{pmatrix} \Rightarrow \lambda_{-1} = \begin{pmatrix} 1 \\ 0 \\ -1 \end{pmatrix} \\
        \begin{pmatrix} 
            1 - 4 & 1 & 2 \\
            1 & 2 - 4 & 1 \\
            2 & 1 & 1 - 4\\
        \end{pmatrix} =  \begin{pmatrix} 
            -3 & 1 & 2 \\
            1 & -2 & 1 \\
            2 & 1 & -3 \\
        \end{pmatrix} \Rightarrow \lambda_4 = \begin{pmatrix} 1 \\ 1 \\ 1 \end{pmatrix} \\
    .\end{align*}

    Con esto entonces la solución es:
    \begin{align*}
        X = c_1 \begin{pmatrix} 1 \\ -2 \\1 \end{pmatrix} e^t + c_2 \begin{pmatrix} 1 \\ 0 \\ -1 \end{pmatrix} e^{-t} +  c_3 \begin{pmatrix} 1 \\ 1 \\ 1 \end{pmatrix} e^{4t}
    .\end{align*}
      \item $X' = \begin{pmatrix} 
	  4 & 1 & 0 \\
	  0 & 4 & 1 \\
	  0 & 0 & 3
	\end{pmatrix} X$

    Lo primero que debemos calcular es el polinomio caracteriztico. Por lo tanto nos interesa
    $det\left( A - \lambda I \right) $:
    \begin{align*}
        det\left( \begin{pmatrix} 
                4 - \lambda & 1 & 0 \\
                0 & 4 - \lambda & 1 \\
                0 & 0 & 3 - \lambda 
        \end{pmatrix}  \right) = - \left( 3-\lambda \right) \left( 4-\lambda \right)^2
    .\end{align*}

    Con esto podemos entonces saber que los valores propios de esta matriz son: $3, 4, 4$
    
    Por lo tanto, ahora podemos entonces encontrar los vectores propios para cada uno de estos casos. Lo que nos
    da:
    \begin{align*}
        \begin{pmatrix} 
            4 - \lambda & 1 & 0 \\
            0 & 4 - \lambda & 1 \\
            0 & 0 & 3 - \lambda 
        \end{pmatrix} \alpha_\lambda = \begin{pmatrix} 0 \\ 0 \\ 0 \end{pmatrix} \\
        \begin{pmatrix} 
            4 - 3 & 1 & 0 \\
            0 & 4 - 3 & 1 \\
            0 & 0 & 3 - 3 
            \end{pmatrix} = \begin{pmatrix}
            1 & 1 & 0 \\
            0 & 1 & 1 \\
            0 & 0 & 0 \\
        \end{pmatrix} \Rightarrow \alpha_3 = \begin{pmatrix} 1 \\ -1 \\ 1 \end{pmatrix} \\
        \begin{pmatrix} 
            4 - 4 & 1 & 0 \\
            0 & 4 - 4 & 1 \\
            0 & 0 & 3 - 4 
            \end{pmatrix} = \begin{pmatrix}
            0 & 1 & 0 \\
            0 & 0 & 1 \\
            0 & 0 & -1 \\
        \end{pmatrix} \Rightarrow \alpha_4 = \begin{pmatrix} 1 \\ 0 \\ 0 \end{pmatrix} \\
    .\end{align*}


    Ahora bien, si bien estos son los vectores propios. Necesitamos otro vector mas. En particular usaremos que
    \begin{align*}
        \begin{pmatrix} 0 & 1 & 0 \\ 0 & 0 & 1 \\ 0 & 0 & -1 \end{pmatrix} \alpha_{4_2} = \begin{pmatrix} 1 \\ 0 \\ 0 \end{pmatrix} \Rightarrow \alpha_{4_2} = \begin{pmatrix} 0 \\ 1 \\ 0 \end{pmatrix}  
    .\end{align*}

    Ahora si, con esto si podemos plantear una solución
    \begin{align*}
        X = c_1 \begin{pmatrix} 1 \\ -1 \\ 1 \end{pmatrix} e^{3t} + c_2 \begin{pmatrix} 1 \\ 0 \\ 0 \end{pmatrix} e^{4t} + c_3 \left(\begin{pmatrix} 0 \\ 1 \\ 0 \end{pmatrix}  e^{4t} + \begin{pmatrix} 1 \\ 0 \\ 0 \end{pmatrix} t e^{4t}\right)
    .\end{align*}
      \item $X' = \begin{pmatrix} 
	  1 & -12 & -14 \\
	  1 & 2 & -3 \\
	  1 & 1 & -2
	\end{pmatrix} X$ donde $X\left( 0 \right) =\begin{pmatrix} 4 \\ 6 \\ -7 \end{pmatrix} X $

    lo primero que debemos hacer es calcular el polinomio caracteristico. Por lo tanto, nos interes
    $det\left( A - \lambda I \right) $ 
    \begin{align*}
        det \left( \begin{pmatrix} 
                1 - \lambda & -12 & -14 \\
                1 & 2 - \lambda & -3 \\
                1 & 1 & - 2 - \lambda
        \end{pmatrix}  \right) &= -\left( 36 + 28 - 14\lambda \right) + \left( -3 + 3\lambda + 14 \right) \\
        &+ \left( 2 + \lambda \right) \left( \left( 1-\lambda \right) \left( 2 - \lambda \right) + 12 \right) \\
        &= \left( -53 + 17\lambda \right) + \left( 2 + \lambda \right) \left( 2-\lambda - 2\lambda + \lambda^2 + 12 \right) \\
        &= \left( -53 + 17\lambda \right) + \\
        & \left( 4 - 2\lambda - 4\lambda + 2\lambda^2 + 24 + 2\lambda - \lambda^2 - 2\lambda^2 + \lambda^3 + 12\lambda \right) \\
        &= \left( -53 + 17\lambda \right) + \left( 4 + 8\lambda - \lambda^2 + 24 \right)  \\
        &= \lambda^3 - \lambda^2 + 25\lambda - 25
    .\end{align*}

    Con esto entonces ahora si podemos encontrar los valores propios que en este caso son $1,5i,-5i$

    Por lo tanto, ahora podemos entonces encontrar los vectores propios para cada uno de estos casos. Lo que nos
    da:
    \begin{align*}
        \begin{pmatrix} 
            1 - \lambda & -12 & -14 \\
            1 & 2 - \lambda & -3 \\
            1 & 1 & - 2 - \lambda\\
        \end{pmatrix} \alpha_\lambda = \begin{pmatrix} 0\\0\\0 \end{pmatrix} \\
        \begin{pmatrix} 
            0 & -12 & -14 \\
            1 & 1 & -3 \\
            1 & 1 & -3
        \end{pmatrix} \Rightarrow \lambda_1 = \begin{pmatrix} 25 \\ -7 \\ 6 \end{pmatrix} \\
        \begin{pmatrix} 
            1 - 5i & -12 & -14 \\
            1 & 2 - 5i & -3 \\
            1 & 1 & -2 - 5i
        \end{pmatrix} \Rightarrow \lambda_{5i} = \begin{pmatrix} 1 + 5i \\ 1 \\ 1 \end{pmatrix} = \begin{pmatrix}  1 \\ 1 \\ 1 \end{pmatrix} + i\begin{pmatrix} 5 \\ 0 \\ 0 \end{pmatrix}  \\
        \begin{pmatrix} 
            1 + 5i & -12 & -14 \\
            1 & 2 + 5i & -3 \\
            1 & 1 & -2 + 5i
        \end{pmatrix} \Rightarrow \lambda_{-5i} = \begin{pmatrix} 1-5i \\ 1 \\ 1 \end{pmatrix} = \begin{pmatrix} 1 \\ 1 \\ 1 \end{pmatrix} +i \begin{pmatrix} -5 \\ 0 \\ 0 \end{pmatrix} 
    .\end{align*}

    con esto entonces podemos encontrar la solución la cual es:
    \begin{align*}
        X = c_1 \begin{pmatrix} 25 \\ -7 \\6 \end{pmatrix} + c_2 \left[ \begin{pmatrix} 1\\1\\1 \end{pmatrix} \cos\left( 5t \right) + \begin{pmatrix} 5\\0\\0 \end{pmatrix} \sin\left( 5t \right)  \right] e^{t} \\
        +c_3 \left[ \begin{pmatrix} 1\\1\\1 \end{pmatrix} \cos\left( -5t \right) + \begin{pmatrix} -5\\0\\0 \end{pmatrix} \sin\left( -5t \right)  \right] e^{t}
    .\end{align*}
    \end{enumerate}

  \item Calcule los coeficientes de Fourier $a_0,a_n,b_n$ de $f\left( x \right) = \left\{  
      \begin{array}{lcc}
	\pi^2 & -\pi\le x\le 0\\
	\left( x-\pi \right)^2 & 0 < x <\pi
      \end{array}
    \right. $, con $f\left( x + 2\pi \right) = f\left( x \right) $. Empleando los coeficientes, exprese la serie de Fourier de $f$ y muestre con ella que  \[
      \sum_{n=1}^{\infty} \frac{1}{n^2}=\frac{\pi^2}{6}
    .\] 

    Lo primero que nos piden son los coeficientes de fourier. Para esto, vamos a utilizar las definiciones:
    \begin{align*}
      a_0 &= \frac{1}{L}\int_{-L}^{L}f\left( x \right) dx \\
      a_n &= \frac{1}{L}\int_{-L}^{L}f\left( x \right) \cos\left( \frac{n\pi x}{L} \right) dx \\
      b_n &= \frac{1}{L}\int_{-L}^{L}f\left( x \right) \sin\left( \frac{n\pi x}{L} \right) dx \\
    .\end{align*}

    Ahora si, teniendo estas definiciones apliquemoslas
    \begin{align*}
      a_0 &= \frac{1}{\pi}\left[ \int_{-\pi}^{0}\pi^{2}dx + \int_{0}^{\pi}\left( x-\pi \right)^2 dx \right] \\
      &= \frac{1}{\pi}\left[ \pi^{3} + \frac{\pi^{3}}{3} \right] = \frac{4\pi^2}{3}\\
      a_n &= \frac{1}{\pi} \left[ \int_{-\pi}^{0}\pi^2\cos\left( nx \right) dx + \int_{0}^{\pi}\left( x-\pi \right)^2 \cos\left( nx \right) dx  \right]  \\
      &= \frac{1}{\pi}\left[ \frac{2\pi\left( -1 \right)^{n}-2\pi\left( \left( -1 \right)^{n}-1 \right) }{n^{2}} \right]=\frac{2}{n^2}  \\
      b_n &= \frac{1}{\pi}\left[ \int_{-\pi}^{0}\pi^2\sin\left( nx \right) dx + \int_{0}^{\pi}\left( x-\pi \right)^2\sin\left( nx \right)  \right] \\
      &= \frac{1}{\pi}\left[ \frac{\pi^2\left( -1 \right)^{n}n^2+2\left( \left( -1 \right)^{n}-2 \right) }{n^{3}} \right] 
    .\end{align*}

    Por lo tanto: 
    \begin{align*}
      f\left( x \right) &= \frac{a_0}{2} + \sum_{n=1}^{\infty} a_n\cos\left( \frac{n\pi x}{L} \right) + \sum_{n=1}^{\infty} b_n\sin\left( \frac{n\pi x}{L} \right)  \\
      f\left( x \right) &= \frac{2\pi^2}{3} + \sum_{n=1}^{\infty} \frac{2 \cos\left( nx \right) }{n^2} + \sum_{n=1}^{\infty} \frac{1}{\pi}\left[ \frac{\pi^2\left( -1 \right)^{n}n^2+2\left( \left( -1 \right)^{n}-2 \right) }{n^{3}} \right]\sin\left( nx \right) 
    .\end{align*}

    Ahora bien, si $x=0$ entonces:
     \begin{align*}
      f\left( 0 \right) &= \frac{2\pi^2}{3} + \sum_{n=1}^{\infty} \frac{2 \cos\left( n0 \right) }{n^2} + \sum_{n=1}^{\infty} \frac{1}{\pi}\left[ \frac{\pi^2\left( -1 \right)^{n}n^2+2\left( \left( -1 \right)^{n}-2 \right) }{n^{3}} \right]\sin\left( n0 \right) \\
      f\left( 0 \right) &= \frac{2\pi^2}{3}+\sum_{n=1}^{\infty} \frac{2}{n^2} + 0 = \pi^2 \\
      2 \sum_{n=1}^{\infty} \frac{1}{n^2} &= \pi^2 - \frac{2\pi^2}{3} \\
      \sum_{n=1}^{\infty} \frac{1}{n^2} &= \frac{\pi^2}{6}
    .\end{align*}
  \item Halle la solución del problema: \[
  \left\{ 
    \begin{array}{lcc}
      u_{xx}=4u_{tt}, & 0<x<L,&t>0\\
      u\left( 0,t \right) = 0,& u\left( L,t \right) = 0,& t>0\\
      u\left( x,0 \right) = 3\sin\left( 2x \right) , & 0<x\le L
    \end{array}
    \right.
  .\] 

  En este caso para comenzar debemos saber que tras la conversión nos queda
  \begin{align*}
    a = \frac{1}{2}\\
    L = L
  .\end{align*}

  Ahora bien, por otro lado, tenemos que
  \begin{align*}
    u\left( x,0 \right) = \sum_{n=1}^{\infty} a_n\sin\left( \frac{n\pi x}{L} \right) 
  .\end{align*}
  Que en este caso no podemos saber con cual $n$ contamos puesto que la longitud no esta definida. Pero asumiendo que es $L=\pi$ tendriamos que:
  $a_2 = 3$

  Ahora bien, por otro lado tambien necesitamos $u_t\left( x,0 \right) $ que en particular seria:
  \begin{align*}
    u_t\left( x,0 \right) = \sum_{n=1}^{\infty} n\cdot a\cdot b_n \sin\left( \frac{n\pi x}{L} \right) 
  .\end{align*}

  Por ultimo en esas circunstancias nos quedaría que
  \begin{align*}
  u\left( x,t \right) = \sum_{n=1}^{\infty} \left[ a_n \cos\left( \frac{n\pi a}{L}t \right) + b_n\sin\left( \frac{n\pi a }{L}t \right)  \right] \sin\left( \frac{n\pi x}{L} \right) 
  .\end{align*}
\end{enumerate}
\end{document}

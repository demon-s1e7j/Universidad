\documentclass{report}

\documentclass[12pt]{article}
\usepackage{array}
\usepackage{color}
\usepackage{amsthm}
\usepackage{eufrak}
\usepackage{lipsum}
\usepackage{pifont}
\usepackage{yfonts}
\usepackage{amsmath}
\usepackage{amssymb}
\usepackage{ccfonts}
\usepackage{comment} \usepackage{amsfonts}
\usepackage{fancyhdr}
\usepackage{graphicx}
\usepackage{listings}
\usepackage{mathrsfs}
\usepackage{setspace}
\usepackage{textcomp}
\usepackage{blindtext}
\usepackage{enumerate}
\usepackage{microtype}
\usepackage{xfakebold}
\usepackage{kantlipsum}
%\usepackage{draftwatermark}
\usepackage[spanish]{babel}
\usepackage[margin=1.5cm, top=2cm, bottom=2cm]{geometry}
\usepackage[framemethod=tikz]{mdframed}
\usepackage[colorlinks=true,citecolor=blue,linkcolor=red,urlcolor=magenta]{hyperref}

%//////////////////////////////////////////////////////
% Watermark configuration
%//////////////////////////////////////////////////////
%\SetWatermarkScale{4}
%\SetWatermarkColor{black}
%\SetWatermarkLightness{0.95}
%\SetWatermarkText{\texttt{Watermark}}

%//////////////////////////////////////////////////////
% Frame configuration
%//////////////////////////////////////////////////////
\newmdenv[tikzsetting={draw=gray,fill=white,fill opacity=0},backgroundcolor=none]{Frame}

%//////////////////////////////////////////////////////
% Font style configuration
%//////////////////////////////////////////////////////
\renewcommand{\familydefault}{\ttdefault}
\renewcommand{\rmdefault}{tt}

%//////////////////////////////////////////////////////
% Bold configuration
%//////////////////////////////////////////////////////
\newcommand{\fbseries}{\unskip\setBold\aftergroup\unsetBold\aftergroup\ignorespaces}
\makeatletter
\newcommand{\setBoldness}[1]{\def\fake@bold{#1}}
\makeatother

%//////////////////////////////////////////////////////
% Default font configuration
%//////////////////////////////////////////////////////
\DeclareFontFamily{\encodingdefault}{\ttdefault}{%
  \hyphenchar\font=\defaulthyphenchar
  \fontdimen2\font=0.33333em
  \fontdimen3\font=0.16667em
  \fontdimen4\font=0.11111em
  \fontdimen7\font=0.11111em}


\input{macros}
\input{letterfonts}

\title{\Huge{Ecuaciones Diferenciales}\\Apuntes}
\author{\huge{SergiOS}}
\date{}

\begin{document}

\maketitle
\newpage% or \cleardoublepage
% \pdfbookmark[<level>]{<title>}{<dest>}
\pdfbookmark[section]{\contentsname}{toc}
\tableofcontents
\pagebreak

\chapter{Apuntes Clase 05/06/2023}
\section{Apuntes Importantes}
\begin{enumerate}
  \item El profe es cubano, no se de que sirve pero el lo dijo xD
  \item Los exámenes ahora serán los jueves.
  \item Vale la pena leer el libro antes.
\end{enumerate}
\section{Clase Como tal}
Vamos a hablar de funciones pares e impares y sus desarrollos en funciones trigonométricas. En particular se puede notar que si una función es par su desarrollo seria únicamente con cosenos y con senos para los impares.
\section{Preliminares}
\begin{enumerate}
  \item Casi todos los reales son \textbf{IRRACIONALES}
  \item El producto de dos irracionales no ha de ser irracional.
  \item Sistema de Medidas y Equivalencias (Nos van a pedir Sistema Imperial)
  \item Técnicas Algebraicas (Productos Notables)
  \item Exponentes y Radicales
\end{enumerate}
\subsection{Teoría y Operaciones con funciones}
Esta cosa es extraña literalmente el colegio. \textbf{CHIDO}
\chapter{Previa Clase 05/06/2023}
El objetivo de estos capitulos es resolver la ecuación diferencial lineal. Esta ecuación es de la forma 
\begin{equation}
  \label{eq:Primer_Orden}
  \frac{dy}{dt}=f(y,t)
.\end{equation}
Esta ecuación no tiene un metodo de solución general si no que depende de los casos y en consecuencia hace que los distintos metodos deban ser considerados
\section{Metodo de la integración de Factores}
En el caso de que la función $f$ en la ecuación \ref{eq:Primer_Orden} dependa de manera lineal de y a esta ecuación se le llamaria una ecuación lineal de primer orden. Esta puede ser expresada como
\begin{equation}
  \label{eq:Primer_Orden_Lineal}
  \frac{dy}{dt}=-ay+b
.\end{equation}

Donde $a$ y $b$ son funciones de t. Sin embargo cuando estos no son constantes es mas comun enunciar estas ecuaciones con la forma:
\begin{equation}
  \label{eq:Primer_Orden_Lineal_General}
  \frac{dy}{dt}+p(t)y=g(t)
.\end{equation}

Ahora bien, estas dos ecuaciones aunque \ref{eq:Primer_Orden_Lineal} es un caso particular de \ref{eq:Primer_Orden_Lineal_General} este tiene una solución muy distinta. En particular su solución es:
\begin{align*}
  \frac{\frac{dy}{dt}}{y-\left( \frac{b}{a} \right) }&= -a \\
  \ln\left| y -\left( \frac{b}{a} \right)  \right| &= -at + C \\
  y &=  \left( \frac{b}{a} \right) + Ce^{-at}
.\end{align*}

Sin embargo, esta solución no funciona para el caso general. En particular, el metodo general que se utiliza es heredado de Leibniz; Este requiere multiplicar la ecuación diferencial por una función llamada $\mu\left( t \right) $ la cual se le denomina como \textit{Factor de Integración}. El problema principal de este metodo es justamente encontrar este factor.


\ex{}{
  \label{ex:Ejemplo1}
  Resuelva la ecuación diferencial \[
\frac{dy}{dt}+\frac{1}{2}y = \frac{1}{2}e^{\frac{t}{3}}
.\] 

Para lograr esta solución lo primero que debemos hacer es multiplicar ambos lados de la ecución por un  $\mu\left( t \right) $ que aun permanece indeterminado \[
\mu\left( t \right) \frac{dy}{dt}+\mu\left( t \right) \frac{1}{2}y = \frac{1}{2}\mu\left( t \right) e^{\frac{t}{3}}
.\] 

Ahora, lo que nos interesa es encontrar una $\mu\left( t \right) $ tal que el lado izquierdo de la ecuación tenga la forma de una derivada reconocible. En este caso, quizás sea claro que tiene una forma parecida a la multiplicación de funciones. Esto seria algo del estilo \[
\frac{d\left[ \mu\left( t \right) y \right] }{dt} = \mu\left( t \right) \frac{dy}{dt} + \frac{d\mu\left( t \right) }{dt}y
.\] Observamos entonces que en ese caso, coincidiría si \[
\frac{d\mu\left( t \right) }{dt} = \frac{1}{2}\mu\left( t \right) 
.\] 
Pero esta ecuación es igual a la ecuación \ref{eq:Primer_Orden_Lineal} cuando $a=-\frac{1}{2}$ y $b=0$ por lo que podemos desarrollar como habíamos hecho antes y  en consecuencia ya tendríamos una solución como habíamos desarrollado antes. Esta solución es: \[
\mu\left( t \right) = Ce^{\frac{t}{2}}
.\] En este caso, dado que no nos interesa el factor de integración mas general asumiremos $C=1$.

Ahora bien, en nuestra ecuación original esto entonces haría que se convirtiera en  \[
\frac{d}{dt}\left( e^{\frac{t}{2}}y \right) = \frac{1}{2}e^{\frac{5t}{6}}
.\] 

Ahora con esto si integramos a ambos lados obtenemos \[
e^{\frac{t}{2}}y = \frac{3}{5}e^{\frac{5t}{6}}+C
.\] 
Con lo cual solo necesitaríamos despejar y para lo que obtenemos
\begin{align*}
  y = \frac{3}{5}e^{\frac{t}{3}} + Ce^{-\frac{t}{2}}
.\end{align*}
el cual es el resultado general. En particular si queremos que sea la solución que pasa por un punto especifico $\left( 0,1 \right) $ debemos sustituir y los valores y encontrar la y que satisfaga la ecuación.
}
Con el ejemplo anterior en mente podemos ahora generalizar para las ecuaciones de la forma
\begin{equation}
  \label{eq:Primer_Orden_Lineal_Particular1}
  \frac{dy}{dt}+at = g\left( t \right) 
.\end{equation}

Para este caso debemos proceder de manera muy similar a lo que hicimos en el ejemplo \ref{ex:Ejemplo1}. Sin embargo se separa del ejemplo en la medida que $\mu\left( t \right) $ cambia y debe satisfacer \[
\frac{d\mu}{dt}=a\mu
.\] mas allá de esto el factor de integración queda entonces como $\mu\left( t \right) = e^{at}$. Por lo tanto, si multiplicamos para la ecuación \ref{eq:Primer_Orden_Lineal_Particular1} nos queda
\begin{align*}
  e^{at}\frac{dy}{dt}+ae^{at}y &= e^{at}g\left( t \right) \\
  \frac{d}{dt}\left( e^{at}y \right) &= e^{at}g\left( t \right)  \\
  e^{at}y = \int e^{at}g\left( t \right) dt + c
.\end{align*}

Por lo tanto, la familia de soluciones para las ecuaciones del tipo \ref{eq:Primer_Orden_Lineal_Particular1} son de la forma
\begin{equation}
  \label{eq:Solucion_Primer_Orden_Lineal_Particular1}
  y = e^{-at}\int_{t_0}^{t}  e^{as}g\left( s \right) ds + ce^{-at}
.\end{equation}

Donde para la ecuación \ref{eq:Solucion_Primer_Orden_Lineal_Particular1} utilizamos la $s$ para denotar la variable de integración para distinguirla de la variable independiente $t$ y se escogió un $t_0$ como el limite de integración mínimo.

\ex{}{
  \label{ex:Ejemplo2}
  Resuelva la ecuación diferencial \[
  \frac{dy}{dt}-2y=4-t
.\] Discuta el comportamiento de las soluciones a medida que $t\to \infty$. En este caso, este ejemplo tiene la forma de la ecuación \ref{eq:Primer_Orden_Lineal_Particular1} con $a=-2$ ; por lo tanto, el factor de integración es $\mu\left( t \right) =e^{-2t}$. Multiplicando entonces nos queda 
\begin{align*}
  e^{-2t}\frac{dy}{dt}-2e^{-2t}y &= 4e^{-2t}-te^{-2t} \\
  \frac{d}{dy}\left( e^{-2t}y \right) &= 4e^{-2t}-te^{-2t}
.\end{align*}
Ahora, si integramos a ambos lados de la ecuación tenemos que \[
e^{-2t}y = -2e^{-2t}+\frac{1}{2}te^{-2t}+\frac{1}{4}e^{-2t}+c
.\] 
Por lo tanto podemos entonces despejar $y$ con lo que nos queda \[
y = -\frac{7}{4} + \frac{1}{2}t + Ce^{-2t}
.\] 
}

Ahora, si retornamos a la ecuación general de primer orden, es decir la ecuación \ref{eq:Primer_Orden_Lineal_General} para encontrar el factor de integración debemos multiplicar por este aun indeterminado $\mu\left( t \right) $ y por lo tanto nos queda \[
\mu\left( t \right) \frac{dy}{dt} + p\left( t \right) \mu\left( t \right) y = \mu\left( t \right) g\left( t \right) 
.\] Ahora bien, si seguimos una linea de pensamiento similar a la del ejemplo \ref{ex:Ejemplo1} vemos que el lado izquierdo es la derivada del producto  $\mu\left( t \right) y$ siempre y cuando \[
\frac{d\mu\left( t \right) }{dt} = p\left( t \right) \mu\left( t \right) 
.\] Si temporalmente asumimos que $\mu\left( t \right) $ es positiva entonces tenemos \[
\frac{\frac{d\mu\left( t \right) }{dt}}{\mu\left( t \right) } = p\left( t \right) 
.\]  y en consecuencia podemos desarrollar 
\begin{align*}
  \ln\mu\left( t \right) &= \int p\left( t \right) dt + k \\
  \mu\left( t \right) &= \exp\left( \int p\left( t \right) dt \right)
.\end{align*}

Note que $\mu\left( t \right) $ es positivo para todo $t$, como asumimos. Ahora bien, si nos devolvemos a cuando recién habíamos multiplicado $\mu\left( t \right) $ ahora nos quedaría \[
\frac{d}{dt}\left[ \mu\left( t \right) y \right] = \mu\left( t \right) g\left( t \right)
.\] Por lo tanto podemos desarrollar
\begin{align*}
  \mu\left( t \right) y &= \int \mu\left( t \right) g\left( t \right) dt + C \\
  y &= \frac{1}{\mu\left( t \right) } \left[ \int_{t_0}^{t}\mu\left( s \right) g\left( s \right) ds + C \right] 
.\end{align*}
\ex{}{
  \label{ex:Ejemplo 3}
  Resuelva el problema del valor inicial
  \begin{align*}
    ty' + 2y &= 4t^2 \\
    y\left( 1 \right) &= 2
  .\end{align*}

  Para determinar $p\left( t \right) $ y $g\left( t \right) $ de manera correctamente, debemos primero re escribir la primera ecuación. En la forma de la ecuación \ref{eq:Primer_Orden_Lineal_General} por lo que tendríamos \[
  y' + \left( \frac{2}{t} \right) y = 4t
  .\] Por lo tanto, $p\left( t \right) = \frac{2}{t}$ y $g\left( t \right) = 4t$. Para resolver  esta ecuación primero tenemos que computar el factor de integración
  \begin{align*}
    \mu\left( t \right) &= \exp\left( \int \frac{2}{t}dt \right) = e^{2\ln\left| t \right| }= t^2 \\
    t^2y' + 2ty &= \left( t^2y \right)' = 4t^{3} \\
    t^2y &= t^{4} + C \\
    y &= t^2 + \frac{c}{t^2} \\
  .\end{align*}
}
\chapter{Apuntes Clase 06/06/2023}

\section{Clasificación}

Las educaciones diferenciales son de 2 tipos principalmente. Ecuaciones de ordinarias y derivadas parciales. Ahora vamos a desarrollar algunos puntos de cada una:
\subsection{Ecuaciones Ordinarias}
En este caso la ecuación diferencial tiene una función que depende de una variable independiente. Estas pueden ser de primer orden o de orden $n$ y tienen muchas formas. Ejemplos de ello fueron discutidos ampliamente en el capitulo anterior.
\section{Ejercicios}
\begin{align*}
  y' + \frac{2}{t}y &= 4t \\
.\end{align*}
\chapter{Estudio}

\end{document}

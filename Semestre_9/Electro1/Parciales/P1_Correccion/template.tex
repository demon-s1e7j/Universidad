\documentclass{report}

\documentclass[12pt]{article}
\usepackage{array}
\usepackage{color}
\usepackage{amsthm}
\usepackage{eufrak}
\usepackage{lipsum}
\usepackage{pifont}
\usepackage{yfonts}
\usepackage{amsmath}
\usepackage{amssymb}
\usepackage{ccfonts}
\usepackage{comment} \usepackage{amsfonts}
\usepackage{fancyhdr}
\usepackage{graphicx}
\usepackage{listings}
\usepackage{mathrsfs}
\usepackage{setspace}
\usepackage{textcomp}
\usepackage{blindtext}
\usepackage{enumerate}
\usepackage{microtype}
\usepackage{xfakebold}
\usepackage{kantlipsum}
%\usepackage{draftwatermark}
\usepackage[spanish]{babel}
\usepackage[margin=1.5cm, top=2cm, bottom=2cm]{geometry}
\usepackage[framemethod=tikz]{mdframed}
\usepackage[colorlinks=true,citecolor=blue,linkcolor=red,urlcolor=magenta]{hyperref}

%//////////////////////////////////////////////////////
% Watermark configuration
%//////////////////////////////////////////////////////
%\SetWatermarkScale{4}
%\SetWatermarkColor{black}
%\SetWatermarkLightness{0.95}
%\SetWatermarkText{\texttt{Watermark}}

%//////////////////////////////////////////////////////
% Frame configuration
%//////////////////////////////////////////////////////
\newmdenv[tikzsetting={draw=gray,fill=white,fill opacity=0},backgroundcolor=none]{Frame}

%//////////////////////////////////////////////////////
% Font style configuration
%//////////////////////////////////////////////////////
\renewcommand{\familydefault}{\ttdefault}
\renewcommand{\rmdefault}{tt}

%//////////////////////////////////////////////////////
% Bold configuration
%//////////////////////////////////////////////////////
\newcommand{\fbseries}{\unskip\setBold\aftergroup\unsetBold\aftergroup\ignorespaces}
\makeatletter
\newcommand{\setBoldness}[1]{\def\fake@bold{#1}}
\makeatother

%//////////////////////////////////////////////////////
% Default font configuration
%//////////////////////////////////////////////////////
\DeclareFontFamily{\encodingdefault}{\ttdefault}{%
  \hyphenchar\font=\defaulthyphenchar
  \fontdimen2\font=0.33333em
  \fontdimen3\font=0.16667em
  \fontdimen4\font=0.11111em
  \fontdimen7\font=0.11111em}


\input{macros}
\input{letterfonts}

\title{\Huge{Electromagnetismo 1}\\Corrección Parcial 1}
\author{\huge{Sergio Montoya Ramirez}\\202112171}
\date{}

\begin{document}

\maketitle
\newpage% or \cleardoublepage
% \pdfbookmark[<level>]{<title>}{<dest>}
\pdfbookmark[section]{\contentsname}{toc}
\tableofcontents
\pagebreak

\chapter{}

En este caso podemos dividir el problema en 3 subproblemas cada uno mas simple. Estos serian

\begin{itemize}
	\item \textbf{Tramo Recto Vertical:} esto seria una linea que va de $y = R$ hasta $y = 2R$. Para un segmento recto
		se cumple que

		$$ dE = \frac{1}{4\pi\varepsilon_0}\frac{dq}{\left|r\right|^2} \hat{r} $$


		Sabemos que $dq = \lambda dy$, ademas, por la ubicación podemos saber que $\left|r\right|^2 = y^2$ y $\hat{r} = - \hat{y}$

		En este caso, sabemos que este campo apunta para $-\hat{y}$. Ahora, para saber lo que vale tenemos:

		\begin{align*}
			E_y^{\left(v\right)} &= \frac{\lambda}{4\pi\varepsilon_0}\int_R^{2R} \frac{dy}{y^2}(-\hat{y})\\
			E_y^{\left(v\right)} &= - \frac{\lambda}{4\pi\varepsilon_0} \left[-\frac{1}{y}\right]_{R}^{2R}\hat{y}\\
			E_y^{\left(v\right)} &= - \frac{\lambda}{4\pi\varepsilon_0} \left[- \frac{1}{2R} + \frac{1}{R}\right]\hat{y}\\
			E_y^{\left(v\right)} &= - \frac{\lambda}{4\pi\varepsilon_0} \left[\frac{2 - 1}{2R}\right]\hat{y}\\
			E_y^{\left(v\right)} &= - \frac{\lambda}{4\pi\varepsilon_0} \left[\frac{1}{2R}\right]\hat{y}\\
		\end{align*}
	\item \textbf{Tramo Recto Horizontal:} Del mismo modo que en el punto anterior podemos ver que 
		$dq = \lambda dx$, ademas, por la ubicación podemos saber que $\left|r\right|^2 = x^2$ y $\hat{r} = - \hat{x}$ con esto entonces nos queda en esencia la misma integral que antes pero con $\hat{x}$
		\begin{align*}
			E_x^{\left(h\right)} &= \frac{\lambda}{4\pi\varepsilon_0}\int_R^{2R} \frac{dx}{x^2}(-\hat{x})\\
			E_x^{\left(h\right)} &= - \frac{\lambda}{4\pi\varepsilon_0} \left[-\frac{1}{x}\right]_{R}^{2R}\hat{x}\\
			E_x^{\left(h\right)} &= - \frac{\lambda}{4\pi\varepsilon_0} \left[- \frac{1}{2R} + \frac{1}{R}\right]\hat{x}\\
			E_x^{\left(h\right)} &= - \frac{\lambda}{4\pi\varepsilon_0} \left[\frac{2 - 1}{2R}\right]\hat{x}\\
			E_x^{\left(h\right)} &= - \frac{\lambda}{4\pi\varepsilon_0} \left[\frac{1}{2R}\right]\hat{x}\\
		\end{align*}

	\item \textbf{Tramo de Arco de Cuarto de Circulo:} De manera similar al punto anterior
		\begin{align*}
			dE &= \frac{1}{4\pi\varepsilon_0} \frac{\lambda d\ell}{R^2} \left(-\hat{r}\right)\\
		\end{align*}

		donde $d\ell$ es el diferencia de longitud para el arco que depende del angulo $\theta$. Por lo tanto esto seria
		\begin{align*}
			dE &= - \frac{\lambda}{4\pi\varepsilon_0} \frac{R d\theta}{R^2} \hat{r}\\
			E &= \int_{\frac{\pi}{2}}^0 - \frac{\lambda}{4\pi\varepsilon_0} \frac{d\theta}{R} \hat{r}\\
			E &= - \frac{\lambda}{4\pi\varepsilon_0 R} \int_{\frac{\pi}{2}}^0 d\theta \hat{r}\\
			E &= - \frac{\lambda}{4\pi\varepsilon_0 R} \left(\int \cos\theta d\theta  \hat{x} + \int \sin\theta d\theta \hat{y}\right)\\
			E &= - \frac{\lambda}{4\pi\varepsilon_0 R} - \left( \hat{x} + \hat{y}\right)\\
			E &= \frac{\lambda}{4\pi\varepsilon_0 R} \left( \hat{x} + \hat{y}\right)\\
		\end{align*}
\end{itemize}

Ahora si sumamos todo nos da:
\begin{align*}
	E(P) &= E^a + E^v + E^h\\
	&= \frac{\lambda}{4\pi\varepsilon_0 R} \left( \hat{x} + \hat{y}\right) - \frac{\lambda}{4\pi\varepsilon_0} \left[\frac{1}{2R}\right]\hat{x} - \frac{\lambda}{4\pi\varepsilon_0} \left[\frac{1}{2R}\right]\hat{y}\\
	&= \frac{\lambda}{4\pi\varepsilon_0 R} \left( \hat{x} + \hat{y}\right) - \frac{\lambda}{4\pi\varepsilon_0} \left[\frac{1}{2R}\right]\left(\hat{x} + \hat{y}\right)\\
	&= \frac{2\lambda - \lambda}{2\cdot4\pi\varepsilon_0 R} \left( \hat{x} + \hat{y}\right)\\
	&= \frac{\lambda}{8\pi\varepsilon_0 R} \left( \hat{x} + \hat{y}\right)\\
\end{align*}

\chapter{}

\section{}

Para este punto vamos a usar la ley de Gauss

\subsection{$r < a$}

No se tiene carga encerrada por lo tanto $E = 0$

\subsection{$a < r < b$}

En este caso la carga encerrada es:
\begin{align*}
	Q_{enc} &= \int \rho(r) dV\\
	Q_{enc} &= \int \rho(r) rdrd\phi dz\\
	Q_{enc} &= \int_a^r \frac{k}{r^3} rdr \int_0^{2\pi} d\phi \int_0^L dz\\
	Q_{enc} &= \int_a^r \frac{k}{r^3} rdr 2\pi L\\
	Q_{enc} &= 2\pi L k \int_a^r \frac{1}{r^2} dr\\
	Q_{enc} &= 2\pi L k \left(\frac{1}{a} - \frac{1}{r}\right)\\
\end{align*}

Ahora por Gauss tenemos:
\begin{align*}
	E 2 \pi r L &= \frac{Q_{enc}}{\varepsilon_0}\\
	E &= \frac{1}{\varepsilon_0 2\pi r L}2\pi L k \left(\frac{1}{a} - \frac{1}{r}\right)\\
	E &= \frac{1}{\varepsilon_0 \cancel{2\pi} r \cancel{L}}\cancel{2\pi} \cancel{L} k \left(\frac{1}{a} - \frac{1}{r}\right)\\
	E &= \frac{k}{\varepsilon_0 r} \left(\frac{1}{a} - \frac{1}{r}\right)\\
	E &= \frac{k}{\varepsilon_0} \left(\frac{1}{ar} - \frac{1}{r^2}\right)
\end{align*}

\subsection{$r > b$}

Para este caso ahora tenemos siempre la misma carga encerrada. El calculo es esencialmente el mismo que en el punto anterior pero cambiando los limites de integración de $r$ a $b$. Repitamos el proceso:

\begin{align*}
	Q_{enc} &= \int \rho(r) dV\\
	Q_{enc} &= \int \rho(r) rdrd\phi dz\\
	Q_{enc} &= \int_a^b \frac{k}{r^3} rdr \int_0^{2\pi} d\phi \int_0^L dz\\
	Q_{enc} &= \int_a^b \frac{k}{r^3} rdr 2\pi L\\
	Q_{enc} &= 2\pi L k \int_a^b \frac{1}{r^2} dr\\
	Q_{enc} &= 2\pi L k \left(\frac{1}{a} - \frac{1}{b}\right)\\
\end{align*}

Ahora poniendolo en Gauss tenemos:
\begin{align*}
	E 2 \pi r L &= \frac{Q_{enc}}{\varepsilon_0}\\
	E &= \frac{1}{\varepsilon_0 2\pi r L}2\pi L k \left(\frac{1}{a} - \frac{1}{b}\right)\\
	E &= \frac{1}{\varepsilon_0 \cancel{2\pi} r \cancel{L}}\cancel{2\pi} \cancel{L} k \left(\frac{1}{a} - \frac{1}{b}\right)\\
	E &= \frac{k}{\varepsilon_0 r} \left(\frac{1}{a} - \frac{1}{b}\right)\\
\end{align*}

Con esto encontramos el campo para todo $r$ y si lo escribimos completo seria

\begin{align*}
	E(r) &= \begin{cases}0 & r < a\\
		\frac{k}{\varepsilon_0} \left(\frac{1}{ar} - \frac{1}{r^2}\right) & a < r < b\\
		\frac{k}{\varepsilon_0 r} \left(\frac{1}{a} - \frac{1}{b}\right) & r > b\\
	\end{cases}
\end{align*}

\section{}

Para este caso partimos de:

\[
	V(r_f) - V(r_i) = - \int_{r_i}^{r_f} E(r) dr
\]

Con esto entonces es importante notar que esta integral para nuestro caso debe ser dividida en tres secciones para que coincida con las partes planeadas:

\begin{align*}
	V(2b) - V\left(\frac{a}{2}\right) &= - \int_{\frac{a}{2}}^{2b} E(r) dr\\
	&= - \left(\int_{\frac{a}{2}}^{a} E(r) dr + \int_{a}^{b} E(r) dr + \int_{b}^{2b} E(r) dr\right)\\
\end{align*}

Para la primera integral se da que:
$$
	\int_{\frac{a}{2}}^{a} E(r) dr = \int_{\frac{a}{2}}^{a} 0 dr = 0
$$

Para la segunda integral tenemos:
\begin{align*}
	\int_{a}^{b} E(r) dr &= \int_a^b \frac{k}{\varepsilon_0} \left(\frac{1}{ar} - \frac{1}{r^2}\right) dr\\
	&= \frac{k}{\varepsilon_0} \int_a^b  \left(\frac{1}{ar} - \frac{1}{r^2}\right) dr\\
	&= \frac{k}{\varepsilon_0} \left(\int_a^b  \frac{1}{ar} dr - \int_a^b \frac{1}{r^2} dr\right)\\
	&= \frac{k}{\varepsilon_0} \left(a^{-1}\int_a^b  \frac{1}{r} dr - \int_a^b \frac{1}{r^2} dr\right)\\
	&= \frac{k}{\varepsilon_0} \left(a^{-1}\left[\ln(r) \right]_a^{b} - \left[ - \frac{1}{r} \right]_a^b \right)\\
	&= \frac{k}{\varepsilon_0} \left(a^{-1}\left[\ln(b) - \ln(a)\right] - \left[ - \frac{1}{b} + \frac{1}{a}\right]\right)\\
	&= \frac{k}{\varepsilon_0} \left(\frac{1}{a}\ln\left(\frac{b}{a}\right) + \frac{1}{b} - \frac{1}{a}\right)
\end{align*}

Para la tercera integral tenemos:

\begin{align*}
	\int_{b}^{2b} E(r) dr &= \int_{b}^{2b} \frac{k}{\varepsilon_0 r} \left(\frac{1}{a} - \frac{1}{b}\right) dr\\
	\int_{b}^{2b} E(r) dr &= \frac{k}{\varepsilon_0} \left(\frac{1}{a} - \frac{1}{b}\right) \int_{b}^{2b} \frac{1}{r}  dr\\
	\int_{b}^{2b} E(r) dr &= \frac{k}{\varepsilon_0} \left(\frac{1}{a} - \frac{1}{b}\right) \left[\ln (r) \right]_{b}^{2R} \\
	\int_{b}^{2b} E(r) dr &= \frac{k}{\varepsilon_0} \left(\frac{1}{a} - \frac{1}{b}\right) \left[\ln(2b) - \ln(b) \right] \\
	\int_{b}^{2b} E(r) dr &= \frac{k}{\varepsilon_0} \left(\frac{1}{a} - \frac{1}{b}\right) \left[\ln\left(\frac{2b}{b}\right)\right] \\
	\int_{b}^{2b} E(r) dr &= \frac{k}{\varepsilon_0} \left(\frac{1}{a} - \frac{1}{b}\right)\ln(2)\\
\end{align*}

Ahora reemplazando en el termino original tenemos:

\begin{align*}
	V(2b) - V\left(\frac{a}{2}\right) &= - \left(\int_{\frac{a}{2}}^{a} E(r) dr + \int_{a}^{b} E(r) dr + \int_{b}^{2b} E(r) dr\right)\\
	&= - \left(0 + \frac{k}{\varepsilon_0} \left(\frac{1}{a}\ln\left(\frac{b}{a}\right) + \frac{1}{b} - \frac{1}{a}\right) + \frac{k}{\varepsilon_0} \left(\frac{1}{a} - \frac{1}{b}\right)\ln(2)\right)\\
	&= - \frac{k}{\varepsilon_0} \left(\frac{1}{a}\ln\left(\frac{b}{a}\right) + \frac{1}{b} - \frac{1}{a} + \left(\frac{1}{a} - \frac{1}{b}\right)\ln(2)\right)\\
\end{align*}

\chapter{}

\section{}

Las condiciones de frontera son:
\begin{enumerate}
	\item \textbf{Continuidad} en la superficie debe cumplirse que:
		\[
			V_{r < R}(R, \theta) = V_{r > R}(R, \theta) = V_0(\theta) = k P_3(\cos\theta)
		\]
	\item \textbf{Discontinuidad de la Derivada Radial:} debido a la carga superficial
		\[
			-\varepsilon_0 \left.\frac{\partial V_{r > R}}{\partial r}\right|_{r=R} + \varepsilon_0 \left.\frac{\partial V_{r < R}}{\partial r}\right|_{r=R} = \sigma(\theta)
		\]
	\item $V_{r < R}$ debe ser finito en $r = 0$
	\item $r \to \infty \implies V_{r > R} \to 0$ 
\end{enumerate}

\section{}

Para el caso de $r < R$ tenemos que
\begin{align*}
	V_{r < R}(R) &= V_0\\
	\sum_{\ell = 0}^\infty A_\ell R^\ell P_\ell(\cos\theta) &= k P_3(\cos\theta)\\
	\int_{-1}^{1}\left[\sum_{\ell = 0}^\infty A_\ell R^\ell P_\ell(\cos\theta)\right]P_m(\cos\theta)d\cos\theta &= \int_{-1}^1 k P_3(\cos\theta) P_m(\cos\theta) d\cos\theta\\
	\sum_{\ell = 0}^\infty A_\ell R^\ell \int_{-1}^{1} P_\ell(\cos\theta)P_m(\cos\theta)d\cos\theta &= k \int_{-1}^1 P_3(\cos\theta) P_m(\cos\theta) d\cos\theta\\
	\int_{-1}^1 P_3(\cos\theta) P_m(\cos\theta) d\cos\theta &= 0\ \forall m \neq 3\\
	A_\ell R^\ell \int_{-1}^{1} P_\ell(\cos\theta)P_m(\cos\theta)d\cos\theta &= 0\ \forall \ell \neq m \\
\end{align*}

Con esto entonces podemos notar que el unico termino que sobrevive es $\ell = 3$ podriamos continuar por esta ruta pero la verdad es que esta integral ya nos aporto lo que debia y el termino $\frac{2}{2\ell + 1}$ se cancelaria para ambos casos. Con esto entonces tenemos:

\begin{align*}
	V_{r < R}(R) &= V_0\\
	A_3 R^3 P_3(\cos\theta) &= k P_3(\cos\theta)\\
	A_3 &= \frac{k}{R^3} 
\end{align*}

Entonces en general esto seria:
\[
	V_{r < R}(R) = \frac{k}{R^3} r^3 P_3(\cos\theta)
\]

\section{}

Este funciona de manera muy similar a la sección anterior partimos de:
\begin{align*}
	V_{r > R}(R) &= V_0\\
	\sum_{\ell = 0}^{\infty} \frac{B_\ell}{R^{\ell + 1}} P_\ell(\cos\theta) &= k P_3(\cos\theta)\\
	\int_{-1}^{1}\left[\sum_{\ell = 0}^{\infty} \frac{B_\ell}{R^{\ell + 1}} P_\ell(\cos\theta)\right]P_m(\cos\theta)d\cos\theta &= \int_{-1}^{1} k P_3(\cos\theta)P_m(\cos\theta) d\cos\theta\\
	\sum_{\ell = 0}^{\infty} \frac{B_\ell}{R^{\ell + 1}} \int_{-1}^{1} P_\ell(\cos\theta)P_m(\cos\theta)d\cos\theta &= k \int_{-1}^{1} P_3(\cos\theta)P_m(\cos\theta)d\cos\theta\\
	\int_{-1}^1 P_3(\cos\theta) P_m(\cos\theta) d\cos\theta &= 0\ \forall m \neq 3\\
	\frac{B_\ell}{R^{\ell + 2}} \int_{-1}^{1} P_\ell(\cos\theta)P_m(\cos\theta)d\cos\theta &= 0\ \forall \ell \neq m \\
\end{align*}

Como en el caso anterior volvemos a quedar en el caso $\ell = 3$ y volvemos a ignorar lo que sigue pues de nuevo simplemente se cancelaria.

\begin{align*}
	V_{r > R}(R) &= V_0\\
	\frac{B_3}{R^{3 + 1}} P_3(\cos\theta) &= k P_3(\cos\theta)\\
	\frac{B_3}{R^{4}} &= k \\
	B_3 &= k R^4
\end{align*}

Lo que nos deja con el termino general:
\[
	V_{r > R} = k \frac{R^4}{r^4} P_3(\cos\theta)
\]

\section{}

tenemos
\[
	-\varepsilon_0 \left.\frac{\partial V_{r > R}}{\partial r}\right|_{r=R} + \varepsilon_0 \left.\frac{\partial V_{r < R}}{\partial r}\right|_{r=R} = \sigma(\theta)
\]

Reemplazando tenemos:
\begin{align*}
	\sigma(\theta) &= -\varepsilon_0 \left.\frac{\partial V_{r > R}}{\partial r}\right|_{r=R} + \varepsilon_0 \left.\frac{\partial V_{r < R}}{\partial r}\right|_{r=R}\\
	&= -\varepsilon_0 \left.\frac{\partial k\frac{R^4}{r^4}P_3(\cos\theta)}{\partial r}\right|_{r=R} + \varepsilon_0 \left.\frac{\partial \frac{k}{R^3}r^3 P_3(\cos\theta)}{\partial r}\right|_{r=R} \\
	\varepsilon_0 \left.\frac{\partial k\frac{R^4}{r^4}P_3(\cos\theta)}{\partial r}\right|_{r=R} &= - \left.4 k \frac{R^4}{r^5} P_3(\cos\theta)\right|_{r = R}\\
	&= -4\frac{k}{R}P_3(\cos\theta)\\
	\left.\frac{\partial \frac{k}{R^3}r^3 P_3(\cos\theta)}{\partial r}\right|_{r=R} &= \left. 3\frac{k}{R^3}r^2 P_3(\cos\theta) \right|_{r=R}\\
	&= 3 \frac{k}{R}P_3(\cos\theta)\\
	\sigma(\theta) &= -\varepsilon_0 \left.\frac{\partial V_{r > R}}{\partial r}\right|_{r=R} + \varepsilon_0 \left.\frac{\partial V_{r < R}}{\partial r}\right|_{r=R}\\
	\sigma(\theta) &= \varepsilon_0 4\frac{k}{R}P_3(\cos\theta) + \varepsilon_0 3 \frac{k}{R}P_3(\cos\theta)\\
	\sigma(\theta) &= \varepsilon_0 7\frac{k}{R}P_3(\cos\theta)
\end{align*}

\chapter{}

\section{}

Dado por el metodo de imagenes tendriamos

\begin{center}
\begin{tikzpicture}[
    charge/.style={circle, minimum size=6mm, thick, draw=black},
    real/.style={fill=red!30},
    image/.style={fill=blue!30, opacity=0.8},
    plane/.style={draw=black!50, thick, fill=gray!10}]
    
    % Eje z
    \draw[-Latex] (0,-3) -- (0,3) node[above] {$z$};
    
    % Plano semiconductor (X-Y)
    \draw[plane] (-3,0) -- (3,0) node[right] {Plano Semiconductor (X-Y)};
    \fill[gray!20, opacity=0.4] (-3,0) rectangle (3,0.2);
    
    % Cargas reales
    \node[charge, real] (q1) at (0,1.5) {$-2q$};
    \node[charge, real] (q2) at (0,3) {$+q$};
    
    % Cargas imagen
    \node[charge, image] (iq1) at (0,-1.5) {$+2q$};
    \node[charge, image] (iq2) at (0,-3) {$-q$};
    
    % Líneas de posición
    \draw[dashed] (-0.5,1.5) -- ++(-1,0) node[left] {$z = d$};
    \draw[dashed] (-0.5,3) -- ++(-1,0) node[left] {$z = 2d$};
    \draw[dashed] (-0.5,-1.5) -- ++(-1,0) node[left] {$z = -d$};
    \draw[dashed] (-0.5,-3) -- ++(-1,0) node[left] {$z = -2d$};
\end{tikzpicture}
\end{center}

En donde las cargas azules son imagenes que nos permiten hacer este calculo. 

para calcular el potencial electrico en un punto con \( z > 0 \) debemos sumar los potenciales de las 4 cargas.

\[
V(x, y, z) = \frac{1}{4\pi\epsilon_0} \left[
\frac{-2q}{\sqrt{x^2 + y^2 + (z - d)^2}} +
\frac{+q}{\sqrt{x^2 + y^2 + (z - 2d)^2}} +
\frac{+2q}{\sqrt{x^2 + y^2 + (z + d)^2}} +
\frac{-q}{\sqrt{x^2 + y^2 + (z + 2d)^2}}
\right]
\]

Si usamos coordenadas cilíndricas (\( \rho = \sqrt{x^2 + y^2} \), \( \phi \), \( z \)), el potencial depende solo de \( \rho \) y \( z \) (simetría axial):

\[
V(\rho, z) = \frac{1}{4\pi\epsilon_0} \left[
\frac{-2q}{\sqrt{\rho^2 + (z - d)^2}} +
\frac{+q}{\sqrt{\rho^2 + (z - 2d)^2}} +
\frac{+2q}{\sqrt{\rho^2 + (z + d)^2}} +
\frac{-q}{\sqrt{\rho^2 + (z + 2d)^2}}
\right]
\]

\section{}

La solución planteada en el punto anterior utiliza el metodo de las imagenes como base para su desarrollo. Este metodo se fundamente principalmente en el teorema de la unicidad. Esto basicamente nos dice que si una solución para la ecuación de Poisson satisface las condiciones de frontera entonces esta solución debe ser la unica valida. Nosotros podemos ver que esto se cumple pues para $z = 0$ se da que:

\[
V(x, y, 0) = \frac{1}{4\pi\epsilon_0} \left[
\frac{-2q}{\sqrt{x^2 + y^2 + d^2}} +
\frac{+q}{\sqrt{x^2 + y^2 + (2d)^2}} +
\frac{+2q}{\sqrt{x^2 + y^2 + d^2}} +
\frac{-q}{\sqrt{x^2 + y^2 + (2d)^2}}
\right] = 0.
\]
Que es exactamente lo que esperabamos
\[
V(x, y, 0) = 0
\]

Ademas podemos ver que $d\to \infty \implies V \to 0$ que tambien corresponde con la otra condición de frontera. En esencia haciendo uso del teorema de unicidad se creo el metodo de las imagenes que como vimos esta solución cumple las condiciones y por tanto es unica.

\section{}

La fuerza que experimenta $-2q$ es la suma de las fuerzas que ejercen las otras cargas sobre esta. Por lo tanto todo lo que debemos hacer es plantear:

\begin{align*}
  F_i &= \frac{1}{4\pi\epsilon_0} \frac{(q_i)(q_i)}{\left| r \right|^2} \hat{r}\\
  F_{\text{net}} &= F_1 + F_2 + F_3\\
  F_1 &= \frac{1}{4\pi\epsilon_0} \frac{(-2q)(+q)}{d^2} - \hat{z}\\
  &= \frac{2q^2}{4\pi\epsilon_0 d^2} \hat{z}\\
  F_2 &= \frac{1}{4\pi\epsilon_0} \frac{(-2q)(+2q)}{(2d)^2} \hat{z}\\ 
  &= -\frac{4q^2}{16\pi\epsilon_0 d^2} \hat{z}\\ 
  &= -\frac{q^2}{4\pi\epsilon_0 d^2} \hat{z}\\
  F_3 &=  \frac{1}{4\pi\epsilon_0} \frac{(-2q)(-q)}{(3d)^2} \hat{z}\\
  &= \frac{2q^2}{36\pi\epsilon_0 d^2} \hat{z}\\
  &= \frac{q^2}{18\pi\epsilon_0 d^2} \hat{z}\\
  F_{\text{net}} &= F_1 + F_2 + F_3\\
  &= \left( \frac{2q^2}{4\pi\epsilon_0 d^2} - \frac{q^2}{4\pi\epsilon_0 d^2} + \frac{q^2}{18\pi\epsilon_0 d^2} \right) \hat{z}\\
  &= \frac{q^2}{4\pi\epsilon_0 d^2} \left( \frac{2}{1} - \frac{1}{1} + \frac{1}{4.5} \right) \hat{z}\\
  &= \frac{11}{36} \frac{q^2}{\pi\epsilon_0 d^2} \hat{z}.
\end{align*}

\section{}

El campo electrico para una carga $Q$ en la componente vertical con $z=0$ es:
\[
E_z = -\frac{Q}{4\pi\epsilon_0} \frac{z_i}{\left(\rho^2 + z_i^2\right)^{3/2}},
\]

Con lo cual podemos pasarlo para cada una de las cargas de modo tal que:
\begin{align*}
  E_{z1} &= -\frac{(-2q)}{4\pi\epsilon_0} \frac{d}{\left(\rho^2 + d^2\right)^{3/2}}\\ &= \frac{2qd}{4\pi\epsilon_0 \left(\rho^2 + d^2\right)^{3/2}}\\
  E_{z2} &= -\frac{(+q)}{4\pi\epsilon_0} \frac{2d}{\left(\rho^2 + (2d)^2\right)^{3/2}}\\ &= -\frac{2qd}{4\pi\epsilon_0 \left(\rho^2 + 4d^2\right)^{3/2}}\\
  E_{z3} &= -\frac{(+2q)}{4\pi\epsilon_0} \frac{(-d)}{\left(\rho^2 + d^2\right)^{3/2}}\\ &= \frac{2qd}{4\pi\epsilon_0 \left(\rho^2 + d^2\right)^{3/2}}\\
  E_{z4} &= -\frac{(-q)}{4\pi\epsilon_0} \frac{(-2d)}{\left(\rho^2 + (2d)^2\right)^{3/2}}\\ &= -\frac{2qd}{4\pi\epsilon_0 \left(\rho^2 + 4d^2\right)^{3/2}}
\end{align*}

Ahora sumando todo esto nos da:
\begin{align*}
  E_z &= \frac{2qd}{4\pi\epsilon_0 \left(\rho^2 + d^2\right)^{3/2}} + \frac{2qd}{4\pi\epsilon_0 \left(\rho^2 + d^2\right)^{3/2}} - \frac{2qd}{4\pi\epsilon_0 \left(\rho^2 + 4d^2\right)^{3/2}} - \frac{2qd}{4\pi\epsilon_0 \left(\rho^2 + 4d^2\right)^{3/2}}\\
  &= \frac{qd}{\pi\epsilon_0} \left[ \frac{1}{\left(\rho^2 + d^2\right)^{3/2}} - \frac{1}{\left(\rho^2 + 4d^2\right)^{3/2}} \right]
\end{align*}

Con lo cual la densidad de carga es:

\begin{align*}
  \sigma(\rho) &= \epsilon_0 E_z\\
  &= \frac{qd}{\pi} \left[ \frac{1}{\left(\rho^2 + d^2\right)^{3/2}} - \frac{1}{\left(\rho^2 + 4d^2\right)^{3/2}} \right]
\end{align*}

\end{document}

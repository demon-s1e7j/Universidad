\documentclass{report}

\documentclass[12pt]{article}
\usepackage{array}
\usepackage{color}
\usepackage{amsthm}
\usepackage{eufrak}
\usepackage{lipsum}
\usepackage{pifont}
\usepackage{yfonts}
\usepackage{amsmath}
\usepackage{amssymb}
\usepackage{ccfonts}
\usepackage{comment} \usepackage{amsfonts}
\usepackage{fancyhdr}
\usepackage{graphicx}
\usepackage{listings}
\usepackage{mathrsfs}
\usepackage{setspace}
\usepackage{textcomp}
\usepackage{blindtext}
\usepackage{enumerate}
\usepackage{microtype}
\usepackage{xfakebold}
\usepackage{kantlipsum}
%\usepackage{draftwatermark}
\usepackage[spanish]{babel}
\usepackage[margin=1.5cm, top=2cm, bottom=2cm]{geometry}
\usepackage[framemethod=tikz]{mdframed}
\usepackage[colorlinks=true,citecolor=blue,linkcolor=red,urlcolor=magenta]{hyperref}

%//////////////////////////////////////////////////////
% Watermark configuration
%//////////////////////////////////////////////////////
%\SetWatermarkScale{4}
%\SetWatermarkColor{black}
%\SetWatermarkLightness{0.95}
%\SetWatermarkText{\texttt{Watermark}}

%//////////////////////////////////////////////////////
% Frame configuration
%//////////////////////////////////////////////////////
\newmdenv[tikzsetting={draw=gray,fill=white,fill opacity=0},backgroundcolor=none]{Frame}

%//////////////////////////////////////////////////////
% Font style configuration
%//////////////////////////////////////////////////////
\renewcommand{\familydefault}{\ttdefault}
\renewcommand{\rmdefault}{tt}

%//////////////////////////////////////////////////////
% Bold configuration
%//////////////////////////////////////////////////////
\newcommand{\fbseries}{\unskip\setBold\aftergroup\unsetBold\aftergroup\ignorespaces}
\makeatletter
\newcommand{\setBoldness}[1]{\def\fake@bold{#1}}
\makeatother

%//////////////////////////////////////////////////////
% Default font configuration
%//////////////////////////////////////////////////////
\DeclareFontFamily{\encodingdefault}{\ttdefault}{%
  \hyphenchar\font=\defaulthyphenchar
  \fontdimen2\font=0.33333em
  \fontdimen3\font=0.16667em
  \fontdimen4\font=0.11111em
  \fontdimen7\font=0.11111em}


%From M275 "Topology" at SJSU
\newcommand{\id}{\mathrm{id}}
\newcommand{\taking}[1]{\xrightarrow{#1}}
\newcommand{\inv}{^{-1}}

%From M170 "Introduction to Graph Theory" at SJSU
\DeclareMathOperator{\diam}{diam}
\DeclareMathOperator{\ord}{ord}
\newcommand{\defeq}{\overset{\mathrm{def}}{=}}

%From the USAMO .tex files
\newcommand{\ts}{\textsuperscript}
\newcommand{\dg}{^\circ}
\newcommand{\ii}{\item}

% % From Math 55 and Math 145 at Harvard
% \newenvironment{subproof}[1][Proof]{%
% \begin{proof}[#1] \renewcommand{\qedsymbol}{$\blacksquare$}}%
% {\end{proof}}

\newcommand{\liff}{\leftrightarrow}
\newcommand{\lthen}{\rightarrow}
\newcommand{\opname}{\operatorname}
\newcommand{\surjto}{\twoheadrightarrow}
\newcommand{\injto}{\hookrightarrow}
\newcommand{\On}{\mathrm{On}} % ordinals
\DeclareMathOperator{\img}{im} % Image
\DeclareMathOperator{\Img}{Im} % Image
\DeclareMathOperator{\coker}{coker} % Cokernel
\DeclareMathOperator{\Coker}{Coker} % Cokernel
\DeclareMathOperator{\Ker}{Ker} % Kernel
\DeclareMathOperator{\rank}{rank}
\DeclareMathOperator{\Spec}{Spec} % spectrum
\DeclareMathOperator{\Tr}{Tr} % trace
\DeclareMathOperator{\pr}{pr} % projection
\DeclareMathOperator{\ext}{ext} % extension
\DeclareMathOperator{\pred}{pred} % predecessor
\DeclareMathOperator{\dom}{dom} % domain
\DeclareMathOperator{\ran}{ran} % range
\DeclareMathOperator{\Hom}{Hom} % homomorphism
\DeclareMathOperator{\Mor}{Mor} % morphisms
\DeclareMathOperator{\End}{End} % endomorphism

\newcommand{\eps}{\epsilon}
\newcommand{\veps}{\varepsilon}
\newcommand{\ol}{\overline}
\newcommand{\ul}{\underline}
\newcommand{\wt}{\widetilde}
\newcommand{\wh}{\widehat}
\newcommand{\vocab}[1]{\textbf{\color{blue} #1}}
\providecommand{\half}{\frac{1}{2}}
\newcommand{\dang}{\measuredangle} %% Directed angle
\newcommand{\ray}[1]{\overrightarrow{#1}}
\newcommand{\seg}[1]{\overline{#1}}
\newcommand{\arc}[1]{\wideparen{#1}}
\DeclareMathOperator{\cis}{cis}
\DeclareMathOperator*{\lcm}{lcm}
\DeclareMathOperator*{\argmin}{arg min}
\DeclareMathOperator*{\argmax}{arg max}
\newcommand{\cycsum}{\sum_{\mathrm{cyc}}}
\newcommand{\symsum}{\sum_{\mathrm{sym}}}
\newcommand{\cycprod}{\prod_{\mathrm{cyc}}}
\newcommand{\symprod}{\prod_{\mathrm{sym}}}
\newcommand{\Qed}{\begin{flushright}\qed\end{flushright}}
\newcommand{\parinn}{\setlength{\parindent}{1cm}}
\newcommand{\parinf}{\setlength{\parindent}{0cm}}
% \newcommand{\norm}{\|\cdot\|}
\newcommand{\inorm}{\norm_{\infty}}
\newcommand{\opensets}{\{V_{\alpha}\}_{\alpha\in I}}
\newcommand{\oset}{V_{\alpha}}
\newcommand{\opset}[1]{V_{\alpha_{#1}}}
\newcommand{\lub}{\text{lub}}
\newcommand{\del}[2]{\frac{\partial #1}{\partial #2}}
\newcommand{\Del}[3]{\frac{\partial^{#1} #2}{\partial^{#1} #3}}
\newcommand{\deld}[2]{\dfrac{\partial #1}{\partial #2}}
\newcommand{\Deld}[3]{\dfrac{\partial^{#1} #2}{\partial^{#1} #3}}
\newcommand{\lm}{\lambda}
\newcommand{\uin}{\mathbin{\rotatebox[origin=c]{90}{$\in$}}}
\newcommand{\usubset}{\mathbin{\rotatebox[origin=c]{90}{$\subset$}}}
\newcommand{\lt}{\left}
\newcommand{\rt}{\right}
\newcommand{\paren}[1]{\left(#1\right)}
\newcommand{\bs}[1]{\boldsymbol{#1}}
\newcommand{\exs}{\exists}
\newcommand{\st}{\strut}
\newcommand{\dps}[1]{\displaystyle{#1}}

\newcommand{\sol}{\setlength{\parindent}{0cm}\textbf{\textit{Solution:}}\setlength{\parindent}{1cm} }
\newcommand{\solve}[1]{\setlength{\parindent}{0cm}\textbf{\textit{Solution: }}\setlength{\parindent}{1cm}#1 \Qed}

% Things Lie
\newcommand{\kb}{\mathfrak b}
\newcommand{\kg}{\mathfrak g}
\newcommand{\kh}{\mathfrak h}
\newcommand{\kn}{\mathfrak n}
\newcommand{\ku}{\mathfrak u}
\newcommand{\kz}{\mathfrak z}
\DeclareMathOperator{\Ext}{Ext} % Ext functor
\DeclareMathOperator{\Tor}{Tor} % Tor functor
\newcommand{\gl}{\opname{\mathfrak{gl}}} % frak gl group
\renewcommand{\sl}{\opname{\mathfrak{sl}}} % frak sl group chktex 6

% More script letters etc.
\newcommand{\SA}{\mathcal A}
\newcommand{\SB}{\mathcal B}
\newcommand{\SC}{\mathcal C}
\newcommand{\SF}{\mathcal F}
\newcommand{\SG}{\mathcal G}
\newcommand{\SH}{\mathcal H}
\newcommand{\OO}{\mathcal O}

\newcommand{\SCA}{\mathscr A}
\newcommand{\SCB}{\mathscr B}
\newcommand{\SCC}{\mathscr C}
\newcommand{\SCD}{\mathscr D}
\newcommand{\SCE}{\mathscr E}
\newcommand{\SCF}{\mathscr F}
\newcommand{\SCG}{\mathscr G}
\newcommand{\SCH}{\mathscr H}

% Mathfrak primes
\newcommand{\km}{\mathfrak m}
\newcommand{\kp}{\mathfrak p}
\newcommand{\kq}{\mathfrak q}

% number sets
\newcommand{\RR}[1][]{\ensuremath{\ifstrempty{#1}{\mathbb{R}}{\mathbb{R}^{#1}}}}
\newcommand{\NN}[1][]{\ensuremath{\ifstrempty{#1}{\mathbb{N}}{\mathbb{N}^{#1}}}}
\newcommand{\ZZ}[1][]{\ensuremath{\ifstrempty{#1}{\mathbb{Z}}{\mathbb{Z}^{#1}}}}
\newcommand{\QQ}[1][]{\ensuremath{\ifstrempty{#1}{\mathbb{Q}}{\mathbb{Q}^{#1}}}}
\newcommand{\CC}[1][]{\ensuremath{\ifstrempty{#1}{\mathbb{C}}{\mathbb{C}^{#1}}}}
\newcommand{\PP}[1][]{\ensuremath{\ifstrempty{#1}{\mathbb{P}}{\mathbb{P}^{#1}}}}
\newcommand{\HH}[1][]{\ensuremath{\ifstrempty{#1}{\mathbb{H}}{\mathbb{H}^{#1}}}}
\newcommand{\FF}[1][]{\ensuremath{\ifstrempty{#1}{\mathbb{F}}{\mathbb{F}^{#1}}}}
% expected value
\newcommand{\EE}{\ensuremath{\mathbb{E}}}
\newcommand{\charin}{\text{ char }}
\DeclareMathOperator{\sign}{sign}
\DeclareMathOperator{\Aut}{Aut}
\DeclareMathOperator{\Inn}{Inn}
\DeclareMathOperator{\Syl}{Syl}
\DeclareMathOperator{\Gal}{Gal}
\DeclareMathOperator{\GL}{GL} % General linear group
\DeclareMathOperator{\SL}{SL} % Special linear group

%---------------------------------------
% BlackBoard Math Fonts :-
%---------------------------------------

%Captital Letters
\newcommand{\bbA}{\mathbb{A}}	\newcommand{\bbB}{\mathbb{B}}
\newcommand{\bbC}{\mathbb{C}}	\newcommand{\bbD}{\mathbb{D}}
\newcommand{\bbE}{\mathbb{E}}	\newcommand{\bbF}{\mathbb{F}}
\newcommand{\bbG}{\mathbb{G}}	\newcommand{\bbH}{\mathbb{H}}
\newcommand{\bbI}{\mathbb{I}}	\newcommand{\bbJ}{\mathbb{J}}
\newcommand{\bbK}{\mathbb{K}}	\newcommand{\bbL}{\mathbb{L}}
\newcommand{\bbM}{\mathbb{M}}	\newcommand{\bbN}{\mathbb{N}}
\newcommand{\bbO}{\mathbb{O}}	\newcommand{\bbP}{\mathbb{P}}
\newcommand{\bbQ}{\mathbb{Q}}	\newcommand{\bbR}{\mathbb{R}}
\newcommand{\bbS}{\mathbb{S}}	\newcommand{\bbT}{\mathbb{T}}
\newcommand{\bbU}{\mathbb{U}}	\newcommand{\bbV}{\mathbb{V}}
\newcommand{\bbW}{\mathbb{W}}	\newcommand{\bbX}{\mathbb{X}}
\newcommand{\bbY}{\mathbb{Y}}	\newcommand{\bbZ}{\mathbb{Z}}

%---------------------------------------
% MathCal Fonts :-
%---------------------------------------

%Captital Letters
\newcommand{\mcA}{\mathcal{A}}	\newcommand{\mcB}{\mathcal{B}}
\newcommand{\mcC}{\mathcal{C}}	\newcommand{\mcD}{\mathcal{D}}
\newcommand{\mcE}{\mathcal{E}}	\newcommand{\mcF}{\mathcal{F}}
\newcommand{\mcG}{\mathcal{G}}	\newcommand{\mcH}{\mathcal{H}}
\newcommand{\mcI}{\mathcal{I}}	\newcommand{\mcJ}{\mathcal{J}}
\newcommand{\mcK}{\mathcal{K}}	\newcommand{\mcL}{\mathcal{L}}
\newcommand{\mcM}{\mathcal{M}}	\newcommand{\mcN}{\mathcal{N}}
\newcommand{\mcO}{\mathcal{O}}	\newcommand{\mcP}{\mathcal{P}}
\newcommand{\mcQ}{\mathcal{Q}}	\newcommand{\mcR}{\mathcal{R}}
\newcommand{\mcS}{\mathcal{S}}	\newcommand{\mcT}{\mathcal{T}}
\newcommand{\mcU}{\mathcal{U}}	\newcommand{\mcV}{\mathcal{V}}
\newcommand{\mcW}{\mathcal{W}}	\newcommand{\mcX}{\mathcal{X}}
\newcommand{\mcY}{\mathcal{Y}}	\newcommand{\mcZ}{\mathcal{Z}}


%---------------------------------------
% Bold Math Fonts :-
%---------------------------------------

%Captital Letters
\newcommand{\bmA}{\boldsymbol{A}}	\newcommand{\bmB}{\boldsymbol{B}}
\newcommand{\bmC}{\boldsymbol{C}}	\newcommand{\bmD}{\boldsymbol{D}}
\newcommand{\bmE}{\boldsymbol{E}}	\newcommand{\bmF}{\boldsymbol{F}}
\newcommand{\bmG}{\boldsymbol{G}}	\newcommand{\bmH}{\boldsymbol{H}}
\newcommand{\bmI}{\boldsymbol{I}}	\newcommand{\bmJ}{\boldsymbol{J}}
\newcommand{\bmK}{\boldsymbol{K}}	\newcommand{\bmL}{\boldsymbol{L}}
\newcommand{\bmM}{\boldsymbol{M}}	\newcommand{\bmN}{\boldsymbol{N}}
\newcommand{\bmO}{\boldsymbol{O}}	\newcommand{\bmP}{\boldsymbol{P}}
\newcommand{\bmQ}{\boldsymbol{Q}}	\newcommand{\bmR}{\boldsymbol{R}}
\newcommand{\bmS}{\boldsymbol{S}}	\newcommand{\bmT}{\boldsymbol{T}}
\newcommand{\bmU}{\boldsymbol{U}}	\newcommand{\bmV}{\boldsymbol{V}}
\newcommand{\bmW}{\boldsymbol{W}}	\newcommand{\bmX}{\boldsymbol{X}}
\newcommand{\bmY}{\boldsymbol{Y}}	\newcommand{\bmZ}{\boldsymbol{Z}}
%Small Letters
\newcommand{\bma}{\boldsymbol{a}}	\newcommand{\bmb}{\boldsymbol{b}}
\newcommand{\bmc}{\boldsymbol{c}}	\newcommand{\bmd}{\boldsymbol{d}}
\newcommand{\bme}{\boldsymbol{e}}	\newcommand{\bmf}{\boldsymbol{f}}
\newcommand{\bmg}{\boldsymbol{g}}	\newcommand{\bmh}{\boldsymbol{h}}
\newcommand{\bmi}{\boldsymbol{i}}	\newcommand{\bmj}{\boldsymbol{j}}
\newcommand{\bmk}{\boldsymbol{k}}	\newcommand{\bml}{\boldsymbol{l}}
\newcommand{\bmm}{\boldsymbol{m}}	\newcommand{\bmn}{\boldsymbol{n}}
\newcommand{\bmo}{\boldsymbol{o}}	\newcommand{\bmp}{\boldsymbol{p}}
\newcommand{\bmq}{\boldsymbol{q}}	\newcommand{\bmr}{\boldsymbol{r}}
\newcommand{\bms}{\boldsymbol{s}}	\newcommand{\bmt}{\boldsymbol{t}}
\newcommand{\bmu}{\boldsymbol{u}}	\newcommand{\bmv}{\boldsymbol{v}}
\newcommand{\bmw}{\boldsymbol{w}}	\newcommand{\bmx}{\boldsymbol{x}}
\newcommand{\bmy}{\boldsymbol{y}}	\newcommand{\bmz}{\boldsymbol{z}}

%---------------------------------------
% Scr Math Fonts :-
%---------------------------------------

\newcommand{\sA}{{\mathscr{A}}}   \newcommand{\sB}{{\mathscr{B}}}
\newcommand{\sC}{{\mathscr{C}}}   \newcommand{\sD}{{\mathscr{D}}}
\newcommand{\sE}{{\mathscr{E}}}   \newcommand{\sF}{{\mathscr{F}}}
\newcommand{\sG}{{\mathscr{G}}}   \newcommand{\sH}{{\mathscr{H}}}
\newcommand{\sI}{{\mathscr{I}}}   \newcommand{\sJ}{{\mathscr{J}}}
\newcommand{\sK}{{\mathscr{K}}}   \newcommand{\sL}{{\mathscr{L}}}
\newcommand{\sM}{{\mathscr{M}}}   \newcommand{\sN}{{\mathscr{N}}}
\newcommand{\sO}{{\mathscr{O}}}   \newcommand{\sP}{{\mathscr{P}}}
\newcommand{\sQ}{{\mathscr{Q}}}   \newcommand{\sR}{{\mathscr{R}}}
\newcommand{\sS}{{\mathscr{S}}}   \newcommand{\sT}{{\mathscr{T}}}
\newcommand{\sU}{{\mathscr{U}}}   \newcommand{\sV}{{\mathscr{V}}}
\newcommand{\sW}{{\mathscr{W}}}   \newcommand{\sX}{{\mathscr{X}}}
\newcommand{\sY}{{\mathscr{Y}}}   \newcommand{\sZ}{{\mathscr{Z}}}


%---------------------------------------
% Math Fraktur Font
%---------------------------------------

%Captital Letters
\newcommand{\mfA}{\mathfrak{A}}	\newcommand{\mfB}{\mathfrak{B}}
\newcommand{\mfC}{\mathfrak{C}}	\newcommand{\mfD}{\mathfrak{D}}
\newcommand{\mfE}{\mathfrak{E}}	\newcommand{\mfF}{\mathfrak{F}}
\newcommand{\mfG}{\mathfrak{G}}	\newcommand{\mfH}{\mathfrak{H}}
\newcommand{\mfI}{\mathfrak{I}}	\newcommand{\mfJ}{\mathfrak{J}}
\newcommand{\mfK}{\mathfrak{K}}	\newcommand{\mfL}{\mathfrak{L}}
\newcommand{\mfM}{\mathfrak{M}}	\newcommand{\mfN}{\mathfrak{N}}
\newcommand{\mfO}{\mathfrak{O}}	\newcommand{\mfP}{\mathfrak{P}}
\newcommand{\mfQ}{\mathfrak{Q}}	\newcommand{\mfR}{\mathfrak{R}}
\newcommand{\mfS}{\mathfrak{S}}	\newcommand{\mfT}{\mathfrak{T}}
\newcommand{\mfU}{\mathfrak{U}}	\newcommand{\mfV}{\mathfrak{V}}
\newcommand{\mfW}{\mathfrak{W}}	\newcommand{\mfX}{\mathfrak{X}}
\newcommand{\mfY}{\mathfrak{Y}}	\newcommand{\mfZ}{\mathfrak{Z}}
%Small Letters
\newcommand{\mfa}{\mathfrak{a}}	\newcommand{\mfb}{\mathfrak{b}}
\newcommand{\mfc}{\mathfrak{c}}	\newcommand{\mfd}{\mathfrak{d}}
\newcommand{\mfe}{\mathfrak{e}}	\newcommand{\mff}{\mathfrak{f}}
\newcommand{\mfg}{\mathfrak{g}}	\newcommand{\mfh}{\mathfrak{h}}
\newcommand{\mfi}{\mathfrak{i}}	\newcommand{\mfj}{\mathfrak{j}}
\newcommand{\mfk}{\mathfrak{k}}	\newcommand{\mfl}{\mathfrak{l}}
\newcommand{\mfm}{\mathfrak{m}}	\newcommand{\mfn}{\mathfrak{n}}
\newcommand{\mfo}{\mathfrak{o}}	\newcommand{\mfp}{\mathfrak{p}}
\newcommand{\mfq}{\mathfrak{q}}	\newcommand{\mfr}{\mathfrak{r}}
\newcommand{\mfs}{\mathfrak{s}}	\newcommand{\mft}{\mathfrak{t}}
\newcommand{\mfu}{\mathfrak{u}}	\newcommand{\mfv}{\mathfrak{v}}
\newcommand{\mfw}{\mathfrak{w}}	\newcommand{\mfx}{\mathfrak{x}}
\newcommand{\mfy}{\mathfrak{y}}	\newcommand{\mfz}{\mathfrak{z}}


\title{\Huge{Electromagnetismo 1}\\Tarea 1}
\author{\huge{Sergio Montoya Ramírez}}
\date{202112171}

\begin{document}

\maketitle
\newpage% or \cleardoublepage
% \pdfbookmark[<level>]{<title>}{<dest>}
\pdfbookmark[section]{\contentsname}{toc}
\tableofcontents
\pagebreak

\chapter{}
Partiendo de que: \[
  \vec{E} = \frac{1}{4\pi\varepsilon_0} \int \frac{dq}{r^2} \hat{r}
.\] podemos notar que en este caso por simetría las componentes $x$ y $y$ no aportan dado que por cada punto con carga existe un opuesto con carga igual (dado que $\rho$ es constante) pero a $180^{\circ}$ lo que hace que sus componentes en estos ejes se cancelen.

Ahora bien, si expresamos esto en esféricas nos quedaríamos entonces con que la componente $z$ del campo seria \[
d E_z = dE \cdot \cos\left( \theta \right) = \frac{1}{4\pi\varepsilon_0} \frac{\rho dV}{r^2} \cdot \cos\left( \theta \right) 
.\] Ahora bien, en este caso $dV$ lo vamos a expresar en coordinadas esféricas (pues dada la geometría del sistema es lo mas simple). Por lo tanto, este diferencial quedaría como: \[
dV = \left( dr \right) \left( r d\theta \right) \left( r \sin\theta d\phi \right) = r^2\sin\theta dr d\theta d\phi
.\] Por lo tanto esto nos quedaría como:
\begin{align*}
  E = \frac{1}{4\pi\varepsilon_0}\int\int \int \frac{\rho}{r^2}\cos\theta r^2 \sin\theta dr d\theta d\phi
.\end{align*}

Ahora bien, estas integrales están definidas. En este caso necesitamos que tengan los limites de integración que expresen de manera correcta la forma de la figura. En este caso esto seria así:
\begin{enumerate}
  \item $r$: Dado que es un solido de $0$ a $R$.
  \item  $\theta$ :  Dado que solo tenemos el hemisferio inferior seria $\frac{\pi}{2}$ a $\pi$
  \item $\phi$ Dado que es toda la semiesfera seria de $0$ a $2\pi$
\end{enumerate}

Con lo cual podemos ahora si ver la integral:
\begin{align*}
  \vec{E} &= \frac{1}{4\pi\varepsilon_0}\int_{0}^{2\pi} \int_{\frac{\pi}{2}}^{\pi} \int_{0}^{R} \frac{\rho}{r^2}\cos\theta r^2 \sin\theta dr d\theta d\phi\\
  \vec{E} &= \frac{\rho}{4\pi\varepsilon_0}\int_{0}^{2\pi} \int_{\frac{\pi}{2}}^{\pi} \int_{0}^{R} \cos\theta \sin\theta dr d\theta d\phi\\
  &= \frac{\rho}{4\pi\varepsilon_0} \int_{0}^{2\pi} \int_{\frac{\pi}{2}}^{\pi} \cos\theta \sin\theta \left[ r \right]_{0}^{R} d\theta d\phi \\
  &= \frac{\rho R}{4\pi\varepsilon_0} \int_{0}^{2\pi} \int_{\frac{\pi}{2}}^{\pi} \cos\theta \sin\theta d\theta d\phi \\
  u &= \sin\theta \implies du = \cos\theta d\theta \\
  &= \frac{\rho R}{4\pi\varepsilon_0} \int_{0}^{2\pi} \int_{1}^{0} u du d\phi \\
  &= \frac{\rho R}{4\pi\varepsilon_0} \int_{0}^{2\pi} \left[ \frac{u^2}{2} \right]_{1}^{0} d\phi \\
  &= -\frac{\rho R}{8\pi\varepsilon_0}\int_{0}^{2\pi} d\phi \\
  &= -\frac{\rho R}{8\pi\varepsilon_0}\left[ \phi \right]_{0}^{2\pi} \\
  &= -\frac{\rho R \cancel{2}\cancel{\pi}}{\cancel{8}\cancel{\pi}\varepsilon_0} \\
  &= - \frac{\rho R}{4\varepsilon_0} \square
.\end{align*}

\chapter{}

En este caso vamos a partir de: \[
V\left( r \right) = \frac{1}{4\pi\varepsilon_0}\int \frac{dq}{\varkappa}
.\] Ademas, en este caso dado que sabemos que es una superficie el disco sabemos que \[
dq = \sigma da
.\] que en este caso para coordenadas cilíndricas seria: \[
dq = \sigma dr r d\theta
.\] sustituyendo en la ecuación anterior tenemos: \[
V\left( r \right) = \frac{\sigma}{4\pi\varepsilon_0}\int \frac{r dr d\theta}{\varkappa}
.\] Ahora bien para encontrar $\varkappa$ vamos a realizar el siguiente desarrollo
 \begin{align*}
   \vec{\varkappa} &= \vec{r} - \vec{r'}\\
   &= \left( r \hat{r} + 0 \hat{\theta} + 0 \hat{z} \right) - \left( 0 \hat{r} + 0 \theta + z \hat{z} \right)  \\
   &= r \hat{r} + 0 \hat{\theta} - z\hat{z} \\
   \varkappa^2 &= \varkappa \cdot  \varkappa = r^2 + 0^2 + \left( -z \right) ^2 \\
   \varkappa^2 &= r^2 + z^2 \\
   \varkappa &= \sqrt{r^2 + z^2}
.\end{align*}

Ahora bien, ya sustituyendo con todo nos da: \[
V\left( r \right) = \frac{\sigma}{4\pi\varepsilon_0}\int_{0}^{2\pi}d\theta \int_{0}^{R} \frac{r dr}{\sqrt{r^2 + z^2} }
.\] Ahora, para desarrollar la integral realmente compleja veamos la siguiente opción
\begin{align*}
  u &= r^2 + z^2 \\
  du &= 2r dr \\
  \frac{du}{2} &= r dr \\
  \int_{0}^{R} \frac{r dr}{\sqrt{r^2 + z^2} } &= \frac{1}{2}\int_{z^2}^{R^2 + z^2} \frac{du}{\sqrt{u} } \\
  &= \frac{1}{2}\left[ 2\sqrt{u}  \right]_{z^2}^{R^2 + z^2} \\
  &= \left( \sqrt{R^2 + z^2} - z \right)  \\
.\end{align*}

Que ahora  sustituyendo todo queda:
\begin{align*}
  V\left( r \right) &= \frac{\sigma}{4\pi\varepsilon_0} 2\pi \left( \sqrt{R^2 + z^2} - z  \right) \\
  V\left( r \right) &= \frac{\sigma}{2\varepsilon_0}\left( \sqrt{R^2 + z^2} - z \right) 
.\end{align*}

Ahora, cuando nos piden el campo debemos recordar que \[
E = \nabla V
.\] pero como en este caso solo tenemos componente $z$ esto nos queda como \[
E = - \frac{d V}{dz}
.\] Para lo cual podemos desarrollar:
\begin{align*}
  E\left( z \right) &= -\frac{\sigma}{2\varepsilon_0}\left( \frac{1}{\sqrt{R^2 - z^2} }z - 1 \right)  \\
  E\left( z \right) &= \frac{\sigma}{2\varepsilon_0}\left(1 - \frac{1}{ \sqrt{R^2 - z^2} }z \right)
.\end{align*}

Ahora, para este caso nos preguntan que ocurre cuando $R \to \infty$. En cuyo caso el valor de $R$ en la expresión $\sqrt{R^2 - z^2}$ se haría muy grande y por tanto $\sqrt{R^2 + z^2} \approx R $ y con lo cual al $R \to  \infty$ queda el campo como
\begin{align*}
  E\left( z \right) &= \frac{\sigma}{2\varepsilon_0} \left( 1 - 0 \right)  \\
  E\left( z \right) &= \frac{\sigma}{2\varepsilon}
.\end{align*}

Lo cual lo podemos identificar como un plano infinito (lo que tiene sentido pues un disco de radio infinito es un plano infinito).

Por otro lado, para los casos en los que: $z \gg R$ podemos hacer el siguiente desarrollo:
\begin{align*}
  \sqrt{R^2 + z^2} &= z\sqrt{1 + \frac{R^2}{z^2}}  \\
  \frac{R^2}{z^2} &\ll 1\\
  x &= \frac{R^2}{z^2} \ll 1 \\
  \sqrt{1 + x} &\approx \left( 1 + \frac{x}{2} \right) \\
  z\sqrt{1 + \frac{R^2}{z^2}} &\approx z\left( 1 + \frac{R^2}{2z^2} \right) 
.\end{align*}

Con lo cual podemos sustituir y nos da:
\begin{align*}
  E &= \frac{\sigma}{2\varepsilon_0} \left( 1 - \frac{z}{z\left( 1 + \frac{R^2}{2z^2} \right) } \right) \\
  E &= \frac{\sigma}{2\varepsilon_0} \left( 1 - \frac{1}{\left( 1 + \frac{R^2}{2z^2} \right) } \right) \\
  \frac{1}{1 + x} &\approx 1 - x; x \ll 1\\
  E &= \frac{\sigma}{2\varepsilon_0}\left( 1 - 1 + \frac{R^2}{2z^2} \right)  \\
  E &= \frac{\sigma R^2}{4\varepsilon z^2}
.\end{align*}

Que ahora bien, tomando en cuenta que el disco esta cargado como $\sigma sup$ donde  $sup$ es la superficie de un disco ( o lo que es lo mismo $2\pi R^2$ entonces podemos ver que aquí podríamos sustituir y nos quedaría: \[
E = \frac{Q}{4\pi\varepsilon_0} \frac{1}{z^2}
.\]  Que es la formula de una carga puntual $\square$.

\chapter{}

En este caso, podemos modelar este sistema esencialmente con la regla de la superposición como si en vez de quitar un pedazo pusiéramos otra circunferencia pero con densidad de carga $-\rho$ de modo tal que en este punto se cancelen.

Por lo tanto, iniciemos por el valor de la esfera completa. Esta es simplemente:

\chapter{}

Para este caso, vamos a partir de un punto muy similar a si este fuera un plano infinito. Sin embargo, hay una diferencia relevante. La diferencia es que en este caso la cantidad de carga es  variable dependiendo de $y$.

\chapter{}

\section{}

Para calcular la carga superficial debemos encontrar la carga que tendría la superficie y dividirla por el área. En este caso sabemos que todas estas son esferas por lo tanto tienen un área de $4\pi r^2$ donde $r$ es el radio concreto de cada esfera. Ademas, dado que sabemos que la esfera esta hecha de un semiconductor entonces la carga de la cara interna de cada una de esas cargas (Es decir $\sigma_a$ y $\sigma_b$) van a tener una carga opuesta a la carga que lo contiene y a su vez la esfera completa va a tener la suma de las dos cargas. Por lo tanto esto se calcularía como:
\begin{align*}
  \sigma_a &= - \frac{q_a}{4\pi a^2} \\
  \sigma_b &= - \frac{q_b}{4\pi b^2} \\
  \sigma_R &= \frac{q_a + q_b}{4\pi R^2}
.\end{align*}

\section{}

Dado que ya conocemos la carga que tiene esta superficie y que esta es una esfera podemos aplicar la ley de Gauss sin preocuparnos por sus detalles lo que nos da:
\begin{align*}
  E \cdot \oint_{C} d\vec{A} &= \frac{q_{en}}{\varepsilon_0} \\ 
  E \cdot 4 \pi r^2 &= \frac{\left( q_a + q_b \right) + q_a + q_b - q_a - q_b}{\varepsilon_0} \\
  E &= \frac{q_a + q_b}{4\pi \varepsilon_0 r^2} \\
  \vec{E} &= \frac{q_a + q_b}{4\pi \varepsilon_0 r^2} \hat{r}
.\end{align*}

\section{}

En este punto podríamos usar de nuevo ley de Gauss. Sin embargo, dado que sabemos que la única carga encerrada dentro de estas cavidades es la puntual podemos simplemente utilizar la formula del campo eléctrico para una carga puntual. Con lo cual nos queda:
\begin{align*}
  \vec{E_a} &= \frac{1}{4\pi\varepsilon_0}\frac{q_a}{r_a^2}\hat{r_a} \\
  \vec{E_b} &= \frac{1}{4\pi\varepsilon_0}\frac{q_b}{r_b^2}\hat{r_b}
.\end{align*}

\section{}

En este caso vamos a trabajar en un abstracto. Dadas las condiciones de un semiconductor sabemos que las únicas cargas que podrían hacer que el punto $q_x$ sufriera una fuerza son las de la superficie del semiconductor. Sin embargo, si intentamos encontrar el campo eléctrico dentro de esta cavidad por ley de Gauss (Cosa que es completamente valida pues seguimos utilizando una curva cerrada) podemos notar que este no contiene ninguna carga en su interior. Por lo tanto el campo causado por la carga en la superficie es 0. Esto en verdad no es si no una manera de reescribir lo que se a mostrado en clase que para una esfera vacía el campo que reside en su interior es 0.

\section{}

En este caso dado que cada una de las cargas esta esencialmente dividido y aislado por el conductor que los separa entonces no habría diferencia para la fuerza que experimenta $q_a$ ni $q_b$ ni tampoco para el campo eléctrico que hay dentro de cada cavidad. Tampoco habría diferencia para el caso de la carga superficial de $\sigma_a$ o $\sigma_b$. Lo que si cambiaría es todo lo relacionado a la parte externa pues tendría que tener en consideración la nueva carga. Es decir, el campo eléctrico fuera del conductor así como su carga superficial cambiaría.

\chapter{}

Para este caso vamos a partir del campo eléctrico en un cilindro que lo podemos conseguir con la ley de Gauss.
\begin{align*}
  \oint_{C} \vec{E} d\vec{A} &= \frac{Q_{enc}}{\varepsilon_0} \\
  E \cdot 2\pi r \cdot L &= \frac{Q_{enc}}{\varepsilon_0} \\
  E &= \frac{Q_{enc}}{2\pi r \cdot L \cdot \varepsilon_0} \\
  \vec{E} &= \frac{Q_{enc}}{2\pi\cdot L \cdot \varepsilon_0 \cdot r} \hat{r}
.\end{align*}

Ahora en este caso podemos calcular la diferencia de potencial como:
\begin{align*}
  V\left( a \right) - V\left( b \right) &= \int_{a}^{b} E\cdot dl \\
  &= \int_{a}^{b} \frac{Q_{enc}}{2\pi\cdot L \cdot \varepsilon_0 \cdot r} dr\\
  &= \frac{Q_{enc}}{2\pi\cdot L \cdot \varepsilon_0}\int_{a}^{b} \frac{1}{r} dr\\
  &= \frac{Q_{enc}}{2\pi\cdot L \cdot \varepsilon_0} \left[ \ln(x) \right]_{a}^{b}\\
  &= \frac{Q_{enc}}{2\pi\cdot L \cdot \varepsilon_0} \ln(b) - \ln(a)\\
  &= \frac{Q_{enc}}{2\pi\cdot L \cdot \varepsilon_0} \ln\left( \frac{b}{a} \right)
.\end{align*}

Ahora con esto la capacitancia es:
\begin{align*}
  C &= \frac{Q}{V} \\
  &= \frac{Q}{\frac{Q}{2\pi\cdot L \cdot \varepsilon_0} \ln\left( \frac{b}{a} \right)} \\
  &= \frac{Q \cdot 2\pi\cdot L \cdot \varepsilon_0}{Q \ln\left( \frac{b}{a} \right)} \\
  &= \frac{2\pi\cdot L \cdot \varepsilon_0}{\ln\left( \frac{b}{a} \right)}
.\end{align*}

Ahora esta es la capacitancia total. Si no la piden por unidad de longitud seria dividir entre $L$ lo que nos daría:
 \begin{align*}
  \frac{C}{L} &= \frac{\frac{2\pi\cdot L \cdot \varepsilon_0}{\ln\left( \frac{b}{a} \right)}}{L}  \\
   &= \frac{2\pi \cdot \varepsilon_0}{\ln\left( \frac{b}{a} \right)} 
.\end{align*}


\end{document}

\documentclass{report}

\documentclass[12pt]{article}
\usepackage{array}
\usepackage{color}
\usepackage{amsthm}
\usepackage{eufrak}
\usepackage{lipsum}
\usepackage{pifont}
\usepackage{yfonts}
\usepackage{amsmath}
\usepackage{amssymb}
\usepackage{ccfonts}
\usepackage{comment} \usepackage{amsfonts}
\usepackage{fancyhdr}
\usepackage{graphicx}
\usepackage{listings}
\usepackage{mathrsfs}
\usepackage{setspace}
\usepackage{textcomp}
\usepackage{blindtext}
\usepackage{enumerate}
\usepackage{microtype}
\usepackage{xfakebold}
\usepackage{kantlipsum}
%\usepackage{draftwatermark}
\usepackage[spanish]{babel}
\usepackage[margin=1.5cm, top=2cm, bottom=2cm]{geometry}
\usepackage[framemethod=tikz]{mdframed}
\usepackage[colorlinks=true,citecolor=blue,linkcolor=red,urlcolor=magenta]{hyperref}

%//////////////////////////////////////////////////////
% Watermark configuration
%//////////////////////////////////////////////////////
%\SetWatermarkScale{4}
%\SetWatermarkColor{black}
%\SetWatermarkLightness{0.95}
%\SetWatermarkText{\texttt{Watermark}}

%//////////////////////////////////////////////////////
% Frame configuration
%//////////////////////////////////////////////////////
\newmdenv[tikzsetting={draw=gray,fill=white,fill opacity=0},backgroundcolor=none]{Frame}

%//////////////////////////////////////////////////////
% Font style configuration
%//////////////////////////////////////////////////////
\renewcommand{\familydefault}{\ttdefault}
\renewcommand{\rmdefault}{tt}

%//////////////////////////////////////////////////////
% Bold configuration
%//////////////////////////////////////////////////////
\newcommand{\fbseries}{\unskip\setBold\aftergroup\unsetBold\aftergroup\ignorespaces}
\makeatletter
\newcommand{\setBoldness}[1]{\def\fake@bold{#1}}
\makeatother

%//////////////////////////////////////////////////////
% Default font configuration
%//////////////////////////////////////////////////////
\DeclareFontFamily{\encodingdefault}{\ttdefault}{%
  \hyphenchar\font=\defaulthyphenchar
  \fontdimen2\font=0.33333em
  \fontdimen3\font=0.16667em
  \fontdimen4\font=0.11111em
  \fontdimen7\font=0.11111em}


\input{macros}
\input{letterfonts}

\usepackage{float}
\usetikzlibrary{decorations.markings, calc}
\usetikzlibrary{arrows.meta}

\title{\Huge{Electro 1}\\Tarea 4}
\author{\huge{Sergio Montoya Ramirez}} \date{}

\begin{document}

\maketitle
\newpage% or \cleardoublepage
% \pdfbookmark[<level>]{<title>}{<dest>}
\pdfbookmark[section]{\contentsname}{toc}
\tableofcontents
\pagebreak

\chapter{Punto 2}

\dfn{Fuerza de Lorentz}{
	\[ \vec{F} = q \left( \vec{E} + \vec{v} \times \vec{B} \right) \]
}

\section{}

\begin{align*}
	\vec{v} &= 0 \hat{x} + \dot{y} \hat{y} + z \hat{z}\\
	\vec{B} &= B_0 \hat{x} + 0 \hat{y} + 0 \hat{z}\\
	\vec{v} \times \vec{B} &= \begin{bmatrix}
		\hat{x} & \hat{y} & \hat{z}\\
		0 & \dot{y} & \dot{z} \\
		B_0 & 0 & 0
	\end{bmatrix}\\
	&= 0 \hat{x} + \dot{z}B_0 \hat{y} - \dot{y}B_0 \hat{z}\\
	&= \dot{z}B_0 \hat{y} - \dot{y}B_0 \hat{z}\\
	\vec{F} &= Q \left( \vec{E} + \vec{v} \times \vec{B} \right)\\
	\vec{F} &= Q \left( E_0 \hat{z} + \dot{z}B_0 \hat{y} - \dot{y}B_0 \hat{z} \right)\\
	\vec{F_x} &= 0\\
	\vec{F_y} &= Q \dot{z}B_0 \hat{y}\\
	\vec{F_z} &= Q \left( E_0 - \dot{y}B_0 \right) \hat{z}\\
	m \ddot{x} &= 0\\
	m \ddot{y} &= Q \dot{z}B_0 \hat{y}\\
	m \ddot{z} &= Q \left( E_0 - \dot{y}B_0 \right) \hat{z}
\end{align*}

Con esto entonces tenemos:
\begin{align*}
	\ddot{z} + \frac{Q B_0}{m}\dot{y} - \frac{Q E_0}{m} &= 0\\
	\omega &= \frac{Q B_0}{m}\\
	\ddot{z} + \omega \dot{y} - \omega \frac{E_0}{B_0} &= 0\\
	\ddot{z} + \omega \left(\dot{y} - \frac{E_0}{B_0} \right) &= 0\\
\end{align*}

Dado que la particula sale del reposo, entonces $\vec{v_y} = 0$ por lo tanto queda como:
\begin{align*}
	m\vec{\dot{v_x}} &= q \left(E - v_z B\right) \hat{x} \\
  \implies\vec{v_x} &= \frac{E}{B} \sin \left( \omega t\right)\\
	m\vec{\dot{v_z}} &= q \left( v_x B \right) \hat{z} \\
  \implies \vec{v_z} &= \frac{E}{B} \left[ 1 - \cos\left( \omega t\right)\right]
\end{align*}

Con estas ecuaciones podemos ver que la particula se mueve en forma de una cicloide.
Y tambien:
\begin{align*}
	\ddot{y} - \frac{Q B_0}{m} \dot{z} &= 0\\
	\omega &= \frac{Q B_0}{m}\\
	\ddot{y} - \omega \dot{z} &= 0\\
	\ddot{y} &= \omega \dot{z}\\
\end{align*}

Juntando todo:
\begin{align*}
	\dddot{y} &= \omega \ddot{z} \\
	\iff \ddot{z} &= \frac{\dddot{y}}{\omega}\\
	\ddot{z} &= \frac{\dddot{y}}{\omega}\\
	&= \omega \left( \frac{E_0}{B_0} - \dot{y} \right)\\
	\dddot{y} &= \omega^2 \left( \frac{E_0}{B_0} - \dot{y} \right)
\end{align*}

Con esto entonces:
\begin{align*}
	y(t) &= C_1 \cos \left(\omega t \right) + C_2 \sin\left( \omega t \right) + \frac{E_0}{B_0} t + C_3\\
	z(t) &= C_2 \cos \left(\omega t \right) - C_1 \sin\left( \omega t \right) + \frac{E_0}{B_0} + C_3\\
\end{align*}

Ahora lo que nos hace falta es encontrar los valores dadas las condiciones iniciales.
\begin{align*}
	\dot{y} (0) &= \dot{z} (0) = 0\\
	y(0) &= z(0) = 0
\end{align*}

Con esto entonces:
\begin{align*}
	y(0) &= C_1 \cos \left(\omega 0 \right) + C_2 \sin\left( \omega 0 \right) + \frac{E_0}{B_0} 0 + C_3\\
	y(0) &= C_1 + C_3\\
	z(0) &= C_2 \cos \left(\omega 0 \right) - C_1 \sin\left( \omega 0 \right) + C_4\\
	z(0) &= C_2 + C_4\\
	C_1 + C_3 &= C_2 + C_4\\
	\dot{y}(t) &= - C_1\omega \sin \left(\omega t \right) + C_2 \omega\cos\left( \omega t \right) + \frac{E_0}{B_0}\\
	\dot{z}(t) &= - C_2\omega \sin \left(\omega t \right) - C_1 \omega\cos\left( \omega t \right)\\
	C_2 \omega + \frac{E_0}{B_0} &= - C_1 \omega\\
	\leftrightarrow C_2 &= \frac{1}{\omega} \left( - \frac{E_0}{B_0} - C_1 \omega\right)\\
	&= - \frac{E_0}{B_0 \omega} - C_1\\
\end{align*}

Con esto entonces:
\begin{align*}
	C_1 + C_3 &= 0 \iff C_1 = - C_3\\
	C_2 \omega + \frac{E_0}{B_0} &= 0 \iff C_2 = - \frac{E_0}{\omega B_0}\\
	C_2 + C_4 &= 0\\
	C_2 &= - C_4\\
	C_4 &= \frac{E_0}{\omega B_0}
\end{align*}

Con lo cual finalmente queda:
\begin{align*}
	y(t) &= - \frac{E_0}{\omega B_0} \sin \left(\omega t\right) + \frac{E_0}{B_0} t\\
	&= \frac{E_0}{B_0 \omega} \left( \omega t - \sin \left(\omega t \right)\right)\\
	z(t) &= - \frac{E_0}{\omega B_0} \cos\left(\omega t\right) + \frac{E_0}{\omega B_0}\\
	z(t) &= \frac{E_0}{\omega B_0} \left( 1 - \cos\left(\omega t\right)\right)
\end{align*}

\section{}

Ahora con esto para encontrar el Radio tomemos
\begin{align*}
	R &= \frac{E_0}{\omega B_0}\\
	y(t) &= \frac{E_0}{B_0 \omega} \left(\omega t - \sin\left(\omega t\right)\right)\\
	- \sin \left(\omega t\right) &= \frac{B_0 \omega}{E_0} \left(y(t) - \frac{E_0 t}{B_0}\right)\\
	Z(t) &= \frac{E_0}{\omega B_0} \left(1 - \cos\left(\omega t \right)\right)\\
	- \cos\left(\omega t\right) &= \frac{\omega B_0}{E_0}\left(z(t) - \frac{E_0}{\omega B_0}\right)\\
	- \cos\left(\omega t\right) &= \frac{1}{R} \left( z(t) - R\right)\\
	\cos^2\left(\omega t\right) + \sin^2\left(\omega t\right) &= 1\\
	a^2 &= -a^2\\
	\sin^2\left(\omega t\right) + \cos^2\left(\omega t\right) &= \left(- \sin\left(\omega t\right)\right)^2 + \left(- \cos\left(\omega t\right)\right)^2\\
	\left(\frac{1}{R^2} \left(y(t) - R \omega t\right)^2\right) + \left(\frac{1}{R^2} \left(z(t) - R \right)^2\right) &= 1\\
	\left(y(t) - R\omega t\right)^2 + \left(z(t) - R\right)^2 &= R^2
\end{align*}

Con esto encontramos un circulo con radio $R$

\section{}

Este problema es en esencia el ejemplo $5.2$ de la cuarta edición del Griffiths. Por lo tanto reutilizare su dibujo para no complicarme con ello.

\begin{figure}[H]
	\begin{center}
		\includegraphics[width=0.45\textwidth]{img/book_5_7.png}
	\end{center}
	\caption{Dibujo del movimiento.}\label{fig:book_5_7}
\end{figure}

\chapter{Punto 3}

\section{}

El campo de una corriente es:
\dfn{Biot-Savart}{
	\[
		\vec{B}\left(\vec{r}\right) = \frac{\mu_0}{4\pi} \int \frac{\vec{I}\times \hat{r}}{r^2}dl' = \frac{\mu_0}{4\pi} I \int \frac{d\ell' \times \hat{r}}{r^2}
	\]
}

Vamos a tomar entonces este ejercicio como el de una espira cuadrada con lado $2R$

\begin{center}
	\begin{tikzpicture}[
    current direction/.style={
        decoration={
            markings,
            mark=at position 0.5 with {\arrow{latex}}
        },
        postaction={decorate}
    },
    medida/.style={|<->|, red, thick, shorten >=1mm, shorten <=1mm}
]

\def\R{1.5} % Tamaño base del sistema

% Ejes coordenados
\draw[->] (-2.5*\R,0) -- (2.5*\R,0) node[right] {$z$};
\draw[->] (0,-2.5*\R) -- (0,2.5*\R) node[above] {$y$};

% Espira cuadrada con corriente (lado izquierdo en rojo)
\draw[thick, current direction] (\R,\R) -- (-\R,\R) node[midway, above] {$I$}; % Superior
\draw[thick, red, current direction] (-\R,\R) -- (-\R,-\R); % Izquierdo (rojo)
\draw[thick, current direction] (-\R,-\R) -- (\R,-\R); % Inferior
\draw[thick, current direction] (\R,-\R) -- (\R,\R); % Derecho

% Punto central
\node[fill=red, circle, inner sep=2pt] at (0,0) {};

% Punto en lado izquierdo (a mitad de altura)
\coordinate (punto) at (-\R,0.5*\R);
\fill[red] (punto) circle (2pt);

% Línea entre centro y punto
\draw[dashed] (0,0) -- (punto);

% Triángulo rectángulo
\draw[help lines] (0,0) -- (-\R,0) -- (punto); % Lados del triángulo
\draw[medida] (-\R,0) -- node[right] {$y$} (punto); % Medida vertical
\draw[medida] (0,0) -- node[below] {$R$} (-\R,0); % Medida horizontal

% Marcas de ángulo recto
\draw[red] (-\R,0) ++(0.15,0.15) -- ++(-0.15,0) -- ++(0,-0.15);

% Etiqueta distancia

\end{tikzpicture}
\end{center}

Visto desde arriba y si lo miramos por un lado seria
\begin{center}
	\begin{tikzpicture}[scale=0.5]
  % Eje horizontal (corriente I)
  \draw[thick, ->] (-2,0) -- (5,0) node[right] {$I$};

  % Elemento de corriente dl'
  \draw[thick, ->] (3,0) -- (4,0) node[midway, above] {$d\vec{l}'$};

  % Punto P
  \fill (3,3) circle (3pt) node[above] {$P$};

  % Vector r
  \draw[thick, ->] (4,0) -- (3,3) node[midway, above right] {$\vec{r}$};

  % Línea vertical desde dl' hasta la altura de P
  \draw[dashed] (4,0) -- (4,3);

  % Ángulo theta
  \draw (4,3) -- (3.8,3) arc (90:135:0.2) node[midway, left] {$\theta$};

  % Ángulo alpha
  \draw (4,0) -- (4.2,0) arc (0:45:0.2) node[midway, right] {$\alpha$};

  % Distancia r
  \draw[dashed] (3,3) -- (3,0) node[midway, left] {$r$};

  % Distancia l'
  \draw[thick, -] (0,-0.5) -- (4,-0.5);
  \draw[thick, -|] (0,-0.7) -- (0,-0.3);
  \draw[thick, |-] (4,-0.7) -- (4,-0.3);
  \node[below] at (2,-0.5) {$l'$};

\end{tikzpicture}
\end{center}

En estos diagramas se ve que $d\ell' \times \hat{r}$ apunta hacia afuera y tiene magnitud \[
	\left|dl'\right|\left|\hat{r}\right| \sin\alpha = d\ell' \cos\theta 
\]

Ademas, podemos notar que:
\begin{align*}
	\tan\theta &= \frac{\ell'}{R} \iff \ell' = \tan\theta R\\
	\frac{d\ell'}{d\theta} &= R\frac{1}{\cos^2\theta}\\
	d\ell' &= \frac{R}{\cos^2\theta} d\theta\\
	\cos\theta &= \frac{R}{\left|\vec{r}\right|} \iff R = \left|\vec{r}\right| \cos\theta\\
	R &= r \cos\theta\\
	r &= \frac{R}{\cos\theta}\\
	\frac{1}{r^2} = \frac{\cos^2\theta}{R^2}\\
\end{align*}

Con esto
\begin{align*}
	\left|\vec{B}\left(\vec{r}\right)\right| &= \frac{\mu_0}{4\pi} \int \frac{\vec{I}\times \hat{r}}{r^2}dl' = \frac{\mu_0}{4\pi} I \int \frac{d\ell' \times \hat{r}}{r^2}\\
	&= \frac{\mu_0 I}{4\pi} \int \frac{d\ell' \cos\theta}{r^2}\\
	&= \frac{\mu_0 I}{4\pi} \int \frac{\cos^2\theta}{R^2} \cos\theta \frac{R}{\cos^2\theta} d\theta\\
	&= \frac{\mu_0 I}{4\pi} \int \frac{\cos\theta}{R} d\theta\\
	&= \frac{\mu_0 I}{4R\pi} \int \cos\theta d\theta\\
	&= \frac{\mu_0 I}{4R\pi} \left(\sin\theta_2 - \sin\theta_1\right)\\
\end{align*}

Ahora partimos la espira en cuatro segmentos y nos da:
\begin{align*}
	&= \frac{\mu_0 I}{4R\pi} \left(\sin\theta_2 - \sin\theta_1\right)\\
	&= \frac{\mu_0 I}{4R\pi} \left(\sin\frac{\pi}{4} - \sin-\frac{\pi}{4}\right)\\
	&= \frac{\mu_0 I}{4R\pi} \left(\frac{1}{\sqrt{2}} + \frac{1}{\sqrt{2}}\right)\\
	&= \frac{\mu_0 I\sqrt{2}}{4R\pi}
\end{align*}

Y en total tenemos 4 de estas secciones

\section{}

\begin{center}
	\begin{tikzpicture}[>=Stealth]
  % Punto P
  \coordinate (P) at (0,0);
  \node at (P) [below left] {$P$};

  % Radios a y b
  \draw[dashed] (P) -- (2,0) node[midway, below] {$a$};
  \draw[dashed] (P) -- ({2*cos(60)},{2*sin(60)}) node[midway, above left] {$b$};

  % Arco de radio a
  \draw[thick, ->] (2,0) arc (0:60:2);

  % Arco de radio b
  \draw[thick, ->] ({2.5*cos(60)},{2.5*sin(60)}) arc (60:0:2.5);

  % Segmento vertical izquierdo
  \draw[thick, ->] ({2.5*cos(60)},{2.5*sin(60)}) -- ({2*cos(60)},{2*sin(60)}) node[midway, left] {$I$};

  % Segmento horizontal derecho
  \draw[thick, ->] (2.5,0) -- (2,0) node[midway, below] {$I$};

  % Segmento vertical superior
  \draw[thick, ->] (0,{2.5*sin(60)}) -- (0,{2*sin(60)}) node[midway, left] {$I$};

  % Ángulo
  \draw (0.4,0) arc (0:60:0.4);

\end{tikzpicture}
\end{center}

Con esto entonces:
\begin{align*}
	\left|d\ell' \times \hat{r} \right| &= \left|d\ell'\right|\left|\hat{r}\right| \sin \left(90^{\circ}\right)\\
	\left| \vec{r}\right| &= r = R\\
	\left|\vec{B}\right| &= \frac{\mu_0 I}{4\pi} \int_{\frac{\pi}{2}}^0 \frac{\left|\vec{d\ell}'\times \hat{r}\right|}{R^2}\\
	\left|\vec{B}\right| &= \frac{\mu_0 I}{4\pi R} \int_{\theta_1}^{\theta_2} d\theta
\end{align*}

Con lo que

\begin{align*}
	\left|\vec{B}\right| &= \frac{\mu_0 I}{4\pi R} \int_{\theta_1}^{\theta_2} d\theta\\
	\left|\vec{B}\right| &= \frac{\mu_0 I}{4\pi R} \frac{\pi}{2}\\
	\left|\vec{B}\right| &= \frac{\mu_0 I}{8 R}\\
\end{align*}

\section{}

Este punto es basicamente una combinación de los anteriores.

Tenemos para las secciones rectas:
\begin{align*}
	\left| \vec{B} \right| &= \frac{\mu_0 I}{4\pi R} \left(\sin\theta_2 - \sin\theta_1\right)
\end{align*}

Con la regla de la mano derecha vemos que entra el vector
\begin{align*}
	\left| \vec{B} \right| &= \frac{\mu_0 I}{4\pi R}\left(\sin\pi - \sin 0\right)\\
	&= \frac{\mu_0 I}{4\pi R}\\
\end{align*}

En el caso del semicirculo 

\begin{align*}
	\left| \vec{B}\right| &= \frac{\mu_0 I}{4\pi R} \int_{\theta_1}^{\theta_2} d\theta\\
	&= \frac{\mu_0 I}{4 R}\left( \frac{3}{2} - \frac{1}{2}\right)\\
	&= \frac{\mu I}{4 R}
\end{align*}

Ahora sumamos todo y nos queda:
\begin{align*}
	B &= \frac{\mu_0 I}{4\pi R} + \frac{\mu_0 I}{4\pi R} + \frac{\mu I}{4 R}\\
	&= \frac{\mu_0 I}{4R} \left(\frac{2}{\pi} + 1 \right)
\end{align*}



\chapter{Punto 7}

Si pensamos en los componentes particularmente en la circunferencia tenemos que esta seria constante alrededor de un loop de amper concentrico en el solenoide. Por lo tanto:
\begin{align*}
	\oint \vec{B} \cdot d\ell &= B_\phi 2 \pi R &= \mu_0 I = 0
\end{align*}

Lo que implica que este loop no encierra corriente. Por lo tanto solo nos queda un componente que es un vector en el eje vertical que por la regla de la mano derecha sabemos que sube.

Con esto entonces tenemos:

\begin{align*}
	\vec{B} &= \begin{cases}
		\mu_0 n I \hat{z}, & r < R\\
		0, & r > R\\
	\end{cases}\\
	\vec{B_1} &= \begin{cases}
		\mu_0 n_1 I \hat{z}, & r < a\\
		0, & r > a\\
	\end{cases}\\
	\vec{B} &= \begin{cases}
		-\mu_0 n_2 I \hat{z}, & r < b\\
		0, & r > b\\
	\end{cases}
\end{align*}

Con este desarrollo entonces:

\begin{enumerate}
	\item 
		\begin{align*}
			\vec{B} &= B_1 + B_2\\
			&= \mu_0 I \left(n1 - n2\right) \hat{z}
		\end{align*}

	\item
		\begin{align*}
			\vec{B} &= - \mu_0 n_2 I \hat{z}
		\end{align*}
	
	\item
		\begin{align*}
			\vec{B} &= 0
		\end{align*}
\end{enumerate}


\chapter{Punto 8}

Tomemos un punto $\vec{r}' = \left(0, y, 0\right)$ y una fuente en $\left(x', y', z'\right)$ tenemos entonces que
\begin{align*}
	\mathbf{r} &= -x' \hat{\mathbf{x}} + (y - y') \hat{\mathbf{y}} - z' \hat{\mathbf{z}}, \quad d\mathbf{l}' = dx' \hat{\mathbf{x}} + dy' \hat{\mathbf{y}};\\
	d\mathbf{l}' \times \mathbf{r} &=
	\begin{vmatrix}
	\hat{\mathbf{x}} & \hat{\mathbf{y}} & \hat{\mathbf{z}} \\
	dx' & dy' & 0 \\
	-x' & (y - y') & -z'
	\end{vmatrix}
	= (-z' dy') \hat{\mathbf{x}} + (z' dx') \hat{\mathbf{y}} + [(y - y') dx' + x' dy'] \hat{\mathbf{z}}.\\
	d\mathbf{B}_1 &= \frac{\mu_0 I}{4\pi} \frac{d\mathbf{l}' \times \mathbf{r}}{r^3} = \frac{\mu_0 I}{4\pi} \frac{(-z' dy') \hat{\mathbf{x}} + (z' dx') \hat{\mathbf{y}} + [(y - y') dx' + x' dy'] \hat{\mathbf{z}}}{[(x')^2 + (y - y')^2 + (z')^2]^{3/2}}.
\end{align*}

Ahora tomando en cuenta que el solenoide es infinito lo que hace que exista un elemento fuente puesto en $\left(x', y', -z'\right)$

\begin{align*}
	\mathbf{r} = -x' \hat{\mathbf{x}} + (y - y') \hat{\mathbf{y}} - (-z') \hat{\mathbf{z}} = -x' \hat{\mathbf{x}} + (y - y') \hat{\mathbf{y}} + z' \hat{\mathbf{z}}, \quad d\mathbf{l}' = dx' \hat{\mathbf{x}} + dy' \hat{\mathbf{y}}\\
	d\mathbf{l}' \times \mathbf{r} =
\begin{vmatrix}
\hat{\mathbf{x}} & \hat{\mathbf{y}} & \hat{\mathbf{z}} \\
dx' & dy' & 0 \\
-x' & (y - y') & z'
\end{vmatrix}
= (z' dy') \hat{\mathbf{x}} + (-z' dx') \hat{\mathbf{y}} + [(y - y') dx' + x' dy'] \hat{\mathbf{z}}.\\
d\mathbf{B}_1 = \frac{\mu_0 I}{4\pi} \frac{d\mathbf{l}' \times \mathbf{r}}{r^3} = \frac{\mu_0 I}{4\pi} \frac{(z' dy') \hat{\mathbf{x}} + (-z' dx') \hat{\mathbf{y}} + [(y - y') dx' + x' dy'] \hat{\mathbf{z}}}{[(x')^2 + (y - y')^2 + (z')^2]^{3/2}}.
\end{align*}

Esto era evidente que seria simetrico en todos los otros componentes excepto en uno que era lo que nos pedian demostrar.

Para terminar este punto es importante notar que con los mismos argumentos que en el punto anterior se sigue:
\begin{align*}
	\vec{B} &= \begin{cases}
		\mu_0 n I \hat{z}, & r < R\\
		0, & r > R\\
	\end{cases}\\
\end{align*}

Ahora bien, para el caso de un toroide tenemos:
\begin{align*}
	\vec{B} &= \begin{cases}
		\frac{\mu_0 N I}{2\pi r} \hat{\phi}, & \text{dentro}\\
		0, & \text{fuera}\\
	\end{cases}\\
\end{align*}

Ahora, si el radio es muy grande entonces:
\begin{align*}
	\frac{N}{2\pi r} &= n\\
	\vec{B} &= \begin{cases}
		\mu_0 n I \hat{z}, & r < R\\
		0, & r > R\\
	\end{cases}\\
\end{align*}
	

\chapter{Punto 11}

Para este fonografo dado que rota a una velocidad angular $\omega$ tenemos
\begin{align*}
	I &= \sigma v dr\\
	&= \sigma \omega r dr
\end{align*}

Y podemos ver esto como la suma infinitesimal de pequeños arcos como:
\begin{align*}
	\mu &= \int_0^R \pi r^2 \sigma \omega r dr\\
	\mu &= \pi \sigma \omega \int_0^R r^3 dr\\
	\mu &= \pi \sigma \omega \left[ \frac{r^4}{4} \right]_0^R\\
	\mu &= \pi \sigma \omega \frac{R^4}{4} \square
\end{align*}

\chapter{Punto 12}

\section{}

Lo primero que debemos hacer para este punto es encontrar el campo magnético para una espira. Para ello, vamos a usar Biot-Savart.

\dfn{Biot-Savart}{
	\[ \vec{B} = \frac{\mu_0}{4\pi} \int \frac{I \vec{ds} \times \hat{r}}{r^2} \]
}

Ademas, este es el ejemplo $5.6$ de la sexta edición del Griffiths. Tomemos entonces su grafica para ubicarnos mejor:

\begin{figure}[h]
	\begin{center}
		\includegraphics[width=0.25\textwidth]{img/book_5_21.png}
	\end{center}
	\caption{Figura de Representación para el problema de una espira.}\label{fig:Fig_5_21}
\end{figure}

Con esto entonces puede notar que los componentes de $d I'$ y de $d B$ se cancelan en todos los ejes excepto en el vertical en donde se suman. Por lo tanto:
\begin{align*}
	B &= \frac{\mu_0}{4\pi} \int \frac{I \vec{d s} \times \hat{r}}{r^2}\\
	B &\propto \int d B_y = \int dB \cos\theta\\
	B &= \frac{\mu_0}{4\pi} \int \frac{I \vec{d s} \sin\left(90^{\circ}\right)}{r^2}\cos\theta\\
	B &= \frac{\mu_0 I}{4\pi} \int \frac{\vec{d s}}{r^2}\cos\theta\\
	\cos\theta &= \frac{R}{\sqrt{z^2 + R^2}}\\
	r &= \sqrt{z^2 + R^2}\\
	B &= \frac{\mu_0 I}{4\pi} \int \frac{\vec{d s}}{\left(z^2 + R^2\right)}\frac{R}{\sqrt{z^2 + R^2}}\\
	B &= \frac{\mu_0 I R}{4\pi \left[ z^2 + R^2\right]^{\frac{3}{2}}} \int \vec{d s}\\
	B &= \frac{\mu_0 I R}{4\pi \left[ z^2 + R^2\right]^{\frac{3}{2}}} \left( 2\pi R\right)\\
	B &= \frac{\mu_0 I R^2}{2\left[ z^2 + R^2\right]^{\frac{3}{2}}}\\
\end{align*}

Ahora bien dado que tenemos dos espiras por superposición podemos poner:

\begin{enumerate}
	\item $$z = \frac{d}{2} + z$$
		\begin{align*}
			B_+ &= \frac{\mu_0 I R^2}{2\left[ z^2 + R^2\right]^{\frac{3}{2}}}\\
			B_+ &= \frac{\mu_0 I R^2}{2\left[ \left(\frac{d}{2} + z\right)^2 + R^2\right]^{\frac{3}{2}}}\\
		\end{align*}
	\item $$z = \frac{d}{2} - z$$
		\begin{align*}
			B_- &= \frac{\mu_0 I R^2}{2\left[ z^2 + R^2\right]^{\frac{3}{2}}}\\
			B_- &= \frac{\mu_0 I R^2}{2\left[ \left(\frac{d}{2} - z\right)^2 + R^2\right]^{\frac{3}{2}}}\\
		\end{align*}
\end{enumerate}

Con lo cual el resultado total es:
\begin{align*}
	B &= B_+ + B_-\\
	& = \frac{\mu_0 I R^2}{2\left[ \left(\frac{d}{2} + z\right)^2 + R^2\right]^{\frac{3}{2}}} + \frac{\mu_0 I R^2}{2\left[ \left(\frac{d}{2} - z\right)^2 + R^2\right]^{\frac{3}{2}}}\\
	& = \frac{\mu_0 I R^2}{2} \left[\frac{1}{\left[ \left(\frac{d}{2} + z\right)^2 + R^2\right]^{\frac{3}{2}}} + \frac{1}{\left[ \left(\frac{d}{2} - z\right)^2 + R^2\right]^{\frac{3}{2}}}\right]\\
\end{align*}

\section{}

Podemos simplemente poner este resultado con python como:
\lstinputlisting[language=Python]{./code/punto_12_b.py}

Con lo cual recibimos la siguiente grafica:

\begin{figure}[H]
	\begin{center}
		\includegraphics[width=0.75\textwidth]{img/punto_12_b.png}
	\end{center}
	\caption{Campo magnetico a lo largo del eje de dos bobinas de Helmholtz (d = R)}\label{fig:Punto_12_b}
\end{figure}

\section{}

Este punto lo podemos mirar basicamente como si esta corriente no varie mucho. Para hacer esto en esencia lo
que nos interesa es encontrar que $\frac{d^2B}{dz^2}(0) = \frac{dB}{dz}(0) = 0$ para algun $d$. Por simetria ya sabemos
que $\frac{dB}{dz}(0) = 0$. Por lo tanto solo nos queda encontrar una $d$ en la que se cumpla lo primero.

Para esto vamos a ponerlo en Sympy:
\lstinputlisting[language=Python]{./code/punto_12_c.py}

Con esto entonces podemos saber que para $B$ tenemos:
\begin{align*}
	\frac{dB}{d z}&= - \frac{24 I R^{2} \mu_{0} \left(\left(4 R^{2} + \left(d - 2 z\right)^{2}\right)^{\frac{5}{2}} \left(d + 2 z\right) - \left(4 R^{2} + \left(d + 2 z\right)^{2}\right)^{\frac{5}{2}} \left(d - 2 z\right)\right)}{\left(4 R^{2} + \left(d - 2 z\right)^{2}\right)^{\frac{5}{2}} \left(4 R^{2} + \left(d + 2 z\right)^{2}\right)^{\frac{5}{2}}}\\
	\frac{d^2 B}{d z^2} &= - \frac{48 I R^{2} \mu_{0}}{\left(4 R^{2} + \left(d + 2 z\right)^{2}\right)^{\frac{5}{2}}} + \frac{240 I R^{2} \mu_{0} \left(d + 2 z\right)^{2}}{\left(4 R^{2} + \left(d + 2 z\right)^{2}\right)^{\frac{7}{2}}} - \frac{48 I R^{2} \mu_{0}}{\left(4 R^{2} + \left(d - 2 z\right)^{2}\right)^{\frac{5}{2}}} + \frac{240 I R^{2} \mu_{0} \left(d - 2 z\right)^{2}}{\left(4 R^{2} + \left(d - 2 z\right)^{2}\right)^{\frac{7}{2}}}
\end{align*}

Ahora para solucionar podemos simplemente reemplazar $z = 0$ que nos queda como:
\[
	\frac{d^2B}{dz^2}(0) = \frac{384 I R^{2} \mu_{0} \left(- R^{2} + d^{2}\right)}{\left(4 R^{2} + d^{2}\right)^{\frac{7}{2}}}
\]

Y por ultimo bucamos un valor de $d$ para el cual
\[
	\frac{d^2B}{dz^2}(0) = 0
\]

Y nos da como resultado:
\[
	\left[ \left\{ d : R\right\}\right]
\]

Y con esto queda solucionado. Ahora veamos esta, esta bobina de Helmholtz se hace mucho mas estable cuando $d = R$ cosa que explica el por que trabajamos con ello en el punto anterior.

\section{}

Este punto es esencialmente equivalente al $A$ por lo tanto no volveremos a mirar como solucionar el campo para una espira y simplemente partiremos de antes:
\begin{align*}
	B &= B_+ + B_-\\
	& = \frac{\mu_0 I R^2}{2\left[ \left(\frac{d}{2} + z\right)^2 + R^2\right]^{\frac{3}{2}}} + \frac{\mu_0 - I R^2}{2\left[ \left(\frac{d}{2} - z\right)^2 + R^2\right]^{\frac{3}{2}}}\\
	& = \frac{\mu_0 I R^2}{2} \left[\frac{1}{\left[ \left(\frac{d}{2} + z\right)^2 + R^2\right]^{\frac{3}{2}}} - \frac{1}{\left[ \left(\frac{d}{2} - z\right)^2 + R^2\right]^{\frac{3}{2}}}\right]\\
\end{align*}

En esencia es evidente lo que estoy poniendo pues es simplemente decir que cuando las corrientes son inversas no se contribuyen si no que se restan.

\section{}

Para graficar esto vamos a reutilizar el codigo de antes simplemente cambiando un signo:

\lstinputlisting[language=Python]{./code/punto_12_e.py}

Con lo que nos queda el siguiente resultado:
\begin{figure}[H]
	\begin{center}
		\includegraphics[width=0.95\textwidth]{img/punto_12_e.png}
	\end{center}
	\caption{Campo magnetico a lo largo del eje de Anti-Helmholtz (d = R)}\label{fig:}
\end{figure}

\section{}

Ahora vamos a buscar que en el centro $ B = - B_T$ que recordemos es aproximadamente $B_T = 50 \mu T$. Lo primero es notar que este va a ser una bobina de Helmholtz y no una antibobina pues nos interesa que en el centro sea el mayor valor y no $0$. Por lo tanto tomaremos los ejemplos anteriores.

Lo que haremos en esencia sera coger el termino anterior y reemplazarle $z = 0$ y $d = R$. Esto nos dara el resultado para $B(0)$ de una bobina de Helmholtz que queda como:
\begin{align*}
	B & = \frac{\mu_0 I R^2}{2} \left[\frac{1}{\left[ \left(\frac{d}{2} + z\right)^2 + R^2\right]^{\frac{3}{2}}} + \frac{1}{\left[ \left(\frac{d}{2} - z\right)^2 + R^2\right]^{\frac{3}{2}}}\right]\\
	B & = \frac{\mu_0 I R^2}{2} \left[\frac{1}{\left[ \left(\frac{R}{2} + 0\right)^2 + R^2\right]^{\frac{3}{2}}} + \frac{1}{\left[ \left(\frac{R}{2} - 0\right)^2 + R^2\right]^{\frac{3}{2}}}\right]\\
	B(0) &= \frac{8 \sqrt{5} I \mu_{0}}{25 R}
\end{align*}

Ahora con esto lo que nos interesa es ver cuando $B(0) = B_{tierra}$ lo cual nos permitiria despejar para la corriente y simplemente con eso ya tendriamos dado $R$ cual deberia ser la corriente que pase para que en el centro el campo terrestre se anule:

\begin{align*}
	B(0) &= \frac{8 \sqrt{5} I \mu_{0}}{25 R} = B_{tierra}\\
	B(0) &= \frac{5 \sqrt{5} B_{tierra} R}{8 \mu_{0}}
\end{align*}

Ahora ya apartir de esto podemos hacerlo tan arbitrario como  querramos.

Para mostrar esto tambien lo hice con sympy y obtuve los mismos resultados:
\lstinputlisting[language=Python]{./code/punto_12_f.py}


\end{document}

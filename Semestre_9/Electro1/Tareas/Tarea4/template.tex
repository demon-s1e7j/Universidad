\documentclass{report}

\documentclass[12pt]{article}
\usepackage{array}
\usepackage{color}
\usepackage{amsthm}
\usepackage{eufrak}
\usepackage{lipsum}
\usepackage{pifont}
\usepackage{yfonts}
\usepackage{amsmath}
\usepackage{amssymb}
\usepackage{ccfonts}
\usepackage{comment} \usepackage{amsfonts}
\usepackage{fancyhdr}
\usepackage{graphicx}
\usepackage{listings}
\usepackage{mathrsfs}
\usepackage{setspace}
\usepackage{textcomp}
\usepackage{blindtext}
\usepackage{enumerate}
\usepackage{microtype}
\usepackage{xfakebold}
\usepackage{kantlipsum}
%\usepackage{draftwatermark}
\usepackage[spanish]{babel}
\usepackage[margin=1.5cm, top=2cm, bottom=2cm]{geometry}
\usepackage[framemethod=tikz]{mdframed}
\usepackage[colorlinks=true,citecolor=blue,linkcolor=red,urlcolor=magenta]{hyperref}

%//////////////////////////////////////////////////////
% Watermark configuration
%//////////////////////////////////////////////////////
%\SetWatermarkScale{4}
%\SetWatermarkColor{black}
%\SetWatermarkLightness{0.95}
%\SetWatermarkText{\texttt{Watermark}}

%//////////////////////////////////////////////////////
% Frame configuration
%//////////////////////////////////////////////////////
\newmdenv[tikzsetting={draw=gray,fill=white,fill opacity=0},backgroundcolor=none]{Frame}

%//////////////////////////////////////////////////////
% Font style configuration
%//////////////////////////////////////////////////////
\renewcommand{\familydefault}{\ttdefault}
\renewcommand{\rmdefault}{tt}

%//////////////////////////////////////////////////////
% Bold configuration
%//////////////////////////////////////////////////////
\newcommand{\fbseries}{\unskip\setBold\aftergroup\unsetBold\aftergroup\ignorespaces}
\makeatletter
\newcommand{\setBoldness}[1]{\def\fake@bold{#1}}
\makeatother

%//////////////////////////////////////////////////////
% Default font configuration
%//////////////////////////////////////////////////////
\DeclareFontFamily{\encodingdefault}{\ttdefault}{%
  \hyphenchar\font=\defaulthyphenchar
  \fontdimen2\font=0.33333em
  \fontdimen3\font=0.16667em
  \fontdimen4\font=0.11111em
  \fontdimen7\font=0.11111em}


\input{macros}
\input{letterfonts}

\usepackage{float}

\title{\Huge{Electro 1}\\Tarea 4}
\author{\huge{Sergio Montoya Ramirez}}
\date{}

\begin{document}

\maketitle
\newpage% or \cleardoublepage
% \pdfbookmark[<level>]{<title>}{<dest>}
\pdfbookmark[section]{\contentsname}{toc}
\tableofcontents
\pagebreak

\chapter{Punto 2}

\section{}

\dfn{Fuerza de Lorentz}{
	\[ \vec{F} = q \left( \vec{E} + \vec{v} \times \vec{B} \right) \]
}

Miremos entonces:
\begin{align*}
	\vec{E} &= E\hat{x}\\
	\vec{B} &= B\hat{y}\\
	\vec{v} &= v_x \hat{x}+ v_y \hat{y}+ v_z \hat{z} \\
	\vec{v} \times \vec{B} &= \begin{bmatrix}
		\hat{x} & \hat{y} & \hat{z} \\
		v_x & v_y & v_z \\
		0 & B & 0
	\end{bmatrix}\\
	&= \hat{x} \left( v_y 0 - v_z B \right) - \hat{y} \left( v_x 0 - v_z 0 \right) + \hat{z} \left( v_x B - v_y 0 \right) \\
	&=  - v_z B \hat{x}+  v_x B \hat{z}\\
	\vec{F} &= q \left( E\hat{x} - v_z B \hat{x} + v_x B \hat{z} \right) \\
	\vec{F} &= q \left( \left(E - v_z B\right) \hat{x} + v_x B \hat{z} \right) \\
\end{align*}

Con esta fuerza entonces podemos calcular con la ecuación de movimiento:
\begin{align*}
	m\vec{\dot{v_x}} &= q \left(E - v_z B\right) \hat{x} \\
	m\vec{\dot{v_y}} &= q \left( 0 \right) \hat{y} \\
	\implies \vec{\dot{v_y}} &= 0 \\
	m\vec{\dot{v_z}} &= q \left( v_x B \right) \hat{z} \\
\end{align*}

Dado que la particula sale del reposo, entonces $\vec{v_y} = 0$

\section{}

\section{}

\chapter{Punto 3}

\chapter{Punto 7}

\chapter{Punto 8}

\chapter{Punto 9}

\chapter{Punto 11}

\chapter{Punto 12}

\section{}

Lo primero que debemos hacer para este punto es encontrar el campo magnético para una espira. Para ello, vamos a usar Biot-Savart.

\dfn{Biot-Savart}{
	\[ \vec{B} = \frac{\mu_0}{4\pi} \int \frac{I \vec{ds} \times \hat{r}}{r^2} \]
}

Ademas, este es el ejemplo $5.6$ de la sexta edición del Griffiths. Tomemos entonces su grafica para ubicarnos mejor:

\begin{figure}[h]
	\begin{center}
		\includegraphics[width=0.25\textwidth]{img/book_5_21.png}
	\end{center}
	\caption{Figura de Representación para el problema de una espira.}\label{fig:Fig_5_21}
\end{figure}

Con esto entonces puede notar que los componentes de $d I'$ y de $d B$ se cancelan en todos los ejes excepto en el vertical en donde se suman. Por lo tanto:
\begin{align*}
	B &= \frac{\mu_0}{4\pi} \int \frac{I \vec{d s} \times \hat{r}}{r^2}\\
	B &\propto \int d B_y = \int dB \cos\theta\\
	B &= \frac{\mu_0}{4\pi} \int \frac{I \vec{d s} \sin\left(90^{\circ}\right)}{r^2}\cos\theta\\
	B &= \frac{\mu_0 I}{4\pi} \int \frac{\vec{d s}}{r^2}\cos\theta\\
	\cos\theta &= \frac{R}{\sqrt{z^2 + R^2}}\\
	r &= \sqrt{z^2 + R^2}\\
	B &= \frac{\mu_0 I}{4\pi} \int \frac{\vec{d s}}{\left(z^2 + R^2\right)}\frac{R}{\sqrt{z^2 + R^2}}\\
	B &= \frac{\mu_0 I R}{4\pi \left[ z^2 + R^2\right]^{\frac{3}{2}}} \int \vec{d s}\\
	B &= \frac{\mu_0 I R}{4\pi \left[ z^2 + R^2\right]^{\frac{3}{2}}} \left( 2\pi R\right)\\
	B &= \frac{\mu_0 I R^2}{2\left[ z^2 + R^2\right]^{\frac{3}{2}}}\\
\end{align*}

Ahora bien dado que tenemos dos espiras por superposición podemos poner:

\begin{enumerate}
	\item $$z = \frac{d}{2} + z$$
		\begin{align*}
			B_+ &= \frac{\mu_0 I R^2}{2\left[ z^2 + R^2\right]^{\frac{3}{2}}}\\
			B_+ &= \frac{\mu_0 I R^2}{2\left[ \left(\frac{d}{2} + z\right)^2 + R^2\right]^{\frac{3}{2}}}\\
		\end{align*}
	\item $$z = \frac{d}{2} - z$$
		\begin{align*}
			B_- &= \frac{\mu_0 I R^2}{2\left[ z^2 + R^2\right]^{\frac{3}{2}}}\\
			B_- &= \frac{\mu_0 I R^2}{2\left[ \left(\frac{d}{2} - z\right)^2 + R^2\right]^{\frac{3}{2}}}\\
		\end{align*}
\end{enumerate}

Con lo cual el resultado total es:
\begin{align*}
	B &= B_+ + B_-\\
	& = \frac{\mu_0 I R^2}{2\left[ \left(\frac{d}{2} + z\right)^2 + R^2\right]^{\frac{3}{2}}} + \frac{\mu_0 I R^2}{2\left[ \left(\frac{d}{2} - z\right)^2 + R^2\right]^{\frac{3}{2}}}\\
	& = \frac{\mu_0 I R^2}{2} \left[\frac{1}{\left[ \left(\frac{d}{2} + z\right)^2 + R^2\right]^{\frac{3}{2}}} + \frac{1}{\left[ \left(\frac{d}{2} - z\right)^2 + R^2\right]^{\frac{3}{2}}}\right]\\
\end{align*}

\section{}

Podemos simplemente poner este resultado con python como:
\lstinputlisting[language=Python]{./code/punto_12_b.py}

Con lo cual recibimos la siguiente grafica:

\begin{figure}[H]
	\begin{center}
		\includegraphics[width=0.75\textwidth]{img/punto_12_b.png}
	\end{center}
	\caption{Campo magnetico a lo largo del eje de dos bobinas de Helmholtz (d = R)}\label{fig:Punto_12_b}
\end{figure}

\section{}

Este punto lo podemos mirar basicamente como si esta corriente no varie mucho. Para hacer esto en esencia lo
que nos interesa es encontrar que $\frac{d^2B}{dz^2}(0) = \frac{dB}{dz}(0) = 0$ para algun $d$. Por simetria ya sabemos
que $\frac{dB}{dz}(0) = 0$. Por lo tanto solo nos queda encontrar una $d$ en la que se cumpla lo primero.

Para esto vamos a ponerlo en Sympy:
\lstinputlisting[language=Python]{./code/punto_12_c.py}

Con esto entonces podemos saber que para $B$ tenemos:
\begin{align*}
	\frac{dB}{d z}&= - \frac{24 I R^{2} \mu_{0} \left(\left(4 R^{2} + \left(d - 2 z\right)^{2}\right)^{\frac{5}{2}} \left(d + 2 z\right) - \left(4 R^{2} + \left(d + 2 z\right)^{2}\right)^{\frac{5}{2}} \left(d - 2 z\right)\right)}{\left(4 R^{2} + \left(d - 2 z\right)^{2}\right)^{\frac{5}{2}} \left(4 R^{2} + \left(d + 2 z\right)^{2}\right)^{\frac{5}{2}}}\\
	\frac{d^2 B}{d z^2} &= - \frac{48 I R^{2} \mu_{0}}{\left(4 R^{2} + \left(d + 2 z\right)^{2}\right)^{\frac{5}{2}}} + \frac{240 I R^{2} \mu_{0} \left(d + 2 z\right)^{2}}{\left(4 R^{2} + \left(d + 2 z\right)^{2}\right)^{\frac{7}{2}}} - \frac{48 I R^{2} \mu_{0}}{\left(4 R^{2} + \left(d - 2 z\right)^{2}\right)^{\frac{5}{2}}} + \frac{240 I R^{2} \mu_{0} \left(d - 2 z\right)^{2}}{\left(4 R^{2} + \left(d - 2 z\right)^{2}\right)^{\frac{7}{2}}}
\end{align*}

Ahora para solucionar podemos simplemente reemplazar $z = 0$ que nos queda como:
\[
	\frac{d^2B}{dz^2}(0) = \frac{384 I R^{2} \mu_{0} \left(- R^{2} + d^{2}\right)}{\left(4 R^{2} + d^{2}\right)^{\frac{7}{2}}}
\]

Y por ultimo bucamos un valor de $d$ para el cual
\[
	\frac{d^2B}{dz^2}(0) = 0
\]

Y nos da como resultado:
\[
	\left[ \left\{ d : R\right\}\right]
\]

Y con esto queda solucionado. Ahora veamos esta, esta bobina de Helmholtz se hace mucho mas estable cuando $d = R$ cosa que explica el por que trabajamos con ello en el punto anterior.

\section{}

Este punto es esencialmente equivalente al $A$ por lo tanto no volveremos a mirar como solucionar el campo para una espira y simplemente partiremos de antes:
\begin{align*}
	B &= B_+ + B_-\\
	& = \frac{\mu_0 I R^2}{2\left[ \left(\frac{d}{2} + z\right)^2 + R^2\right]^{\frac{3}{2}}} + \frac{\mu_0 - I R^2}{2\left[ \left(\frac{d}{2} - z\right)^2 + R^2\right]^{\frac{3}{2}}}\\
	& = \frac{\mu_0 I R^2}{2} \left[\frac{1}{\left[ \left(\frac{d}{2} + z\right)^2 + R^2\right]^{\frac{3}{2}}} - \frac{1}{\left[ \left(\frac{d}{2} - z\right)^2 + R^2\right]^{\frac{3}{2}}}\right]\\
\end{align*}

En esencia es evidente lo que estoy poniendo pues es simplemente decir que cuando las corrientes son inversas no se contribuyen si no que se restan.

\section{}

Para graficar esto vamos a reutilizar el codigo de antes simplemente cambiando un signo:

\lstinputlisting[language=Python]{./code/punto_12_e.py}

Con lo que nos queda el siguiente resultado:
\begin{figure}[H]
	\begin{center}
		\includegraphics[width=0.95\textwidth]{img/punto_12_e.png}
	\end{center}
	\caption{Campo magnetico a lo largo del eje de Anti-Helmholtz (d = R)}\label{fig:}
\end{figure}

\section{}

Ahora vamos a buscar que en el centro $ B = - B_T$ que recordemos es aproximadamente $B_T = 50 \mu T$. Lo primero es notar que este va a ser una bobina de Helmholtz y no una antibobina pues nos interesa que en el centro sea el mayor valor y no $0$. Por lo tanto tomaremos los ejemplos anteriores.

Lo que haremos en esencia sera coger el termino anterior y reemplazarle $z = 0$ y $d = R$. Esto nos dara el resultado para $B(0)$ de una bobina de Helmholtz que queda como:
\begin{align*}
	B & = \frac{\mu_0 I R^2}{2} \left[\frac{1}{\left[ \left(\frac{d}{2} + z\right)^2 + R^2\right]^{\frac{3}{2}}} + \frac{1}{\left[ \left(\frac{d}{2} - z\right)^2 + R^2\right]^{\frac{3}{2}}}\right]\\
	B & = \frac{\mu_0 I R^2}{2} \left[\frac{1}{\left[ \left(\frac{R}{2} + 0\right)^2 + R^2\right]^{\frac{3}{2}}} + \frac{1}{\left[ \left(\frac{R}{2} - 0\right)^2 + R^2\right]^{\frac{3}{2}}}\right]\\
	B(0) &= \frac{8 \sqrt{5} I \mu_{0}}{25 R}
\end{align*}

Ahora con esto lo que nos interesa es ver cuando $B(0) = B_{tierra}$ lo cual nos permitiria despejar para la corriente y simplemente con eso ya tendriamos dado $R$ cual deberia ser la corriente que pase para que en el centro el campo terrestre se anule:

\begin{align*}
	B(0) &= \frac{8 \sqrt{5} I \mu_{0}}{25 R} = B_{tierra}\\
	B(0) &= \frac{5 \sqrt{5} B_{tierra} R}{8 \mu_{0}}
\end{align*}

Ahora ya apartir de esto podemos hacerlo tan arbitrario como  querramos.

Para mostrar esto tambien lo hice con sympy y obtuve los mismos resultados:
\lstinputlisting[language=Python]{./code/punto_12_f.py}


\end{document}

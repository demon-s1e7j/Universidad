\documentclass{report}

\documentclass[12pt]{article}
\usepackage{array}
\usepackage{color}
\usepackage{amsthm}
\usepackage{eufrak}
\usepackage{lipsum}
\usepackage{pifont}
\usepackage{yfonts}
\usepackage{amsmath}
\usepackage{amssymb}
\usepackage{ccfonts}
\usepackage{comment} \usepackage{amsfonts}
\usepackage{fancyhdr}
\usepackage{graphicx}
\usepackage{listings}
\usepackage{mathrsfs}
\usepackage{setspace}
\usepackage{textcomp}
\usepackage{blindtext}
\usepackage{enumerate}
\usepackage{microtype}
\usepackage{xfakebold}
\usepackage{kantlipsum}
%\usepackage{draftwatermark}
\usepackage[spanish]{babel}
\usepackage[margin=1.5cm, top=2cm, bottom=2cm]{geometry}
\usepackage[framemethod=tikz]{mdframed}
\usepackage[colorlinks=true,citecolor=blue,linkcolor=red,urlcolor=magenta]{hyperref}

%//////////////////////////////////////////////////////
% Watermark configuration
%//////////////////////////////////////////////////////
%\SetWatermarkScale{4}
%\SetWatermarkColor{black}
%\SetWatermarkLightness{0.95}
%\SetWatermarkText{\texttt{Watermark}}

%//////////////////////////////////////////////////////
% Frame configuration
%//////////////////////////////////////////////////////
\newmdenv[tikzsetting={draw=gray,fill=white,fill opacity=0},backgroundcolor=none]{Frame}

%//////////////////////////////////////////////////////
% Font style configuration
%//////////////////////////////////////////////////////
\renewcommand{\familydefault}{\ttdefault}
\renewcommand{\rmdefault}{tt}

%//////////////////////////////////////////////////////
% Bold configuration
%//////////////////////////////////////////////////////
\newcommand{\fbseries}{\unskip\setBold\aftergroup\unsetBold\aftergroup\ignorespaces}
\makeatletter
\newcommand{\setBoldness}[1]{\def\fake@bold{#1}}
\makeatother

%//////////////////////////////////////////////////////
% Default font configuration
%//////////////////////////////////////////////////////
\DeclareFontFamily{\encodingdefault}{\ttdefault}{%
  \hyphenchar\font=\defaulthyphenchar
  \fontdimen2\font=0.33333em
  \fontdimen3\font=0.16667em
  \fontdimen4\font=0.11111em
  \fontdimen7\font=0.11111em}


\input{macros}
\input{letterfonts}
\usepackage{amsmath}
\usetikzlibrary{arrows.meta, shapes.geometric}

\renewcommand{\qed}{\hfill\square}

\title{\Huge{Electromagnetismo 1}\\Tarea 5}
\author{\huge{Sergio Montoya Ramírez}}
\date{}

\begin{document}

\maketitle
\newpage% or \cleardoublepage
% \pdfbookmark[<level>]{<title>}{<dest>}
\pdfbookmark[section]{\contentsname}{toc}
\tableofcontents
\pagebreak

\chapter{Punto 7.10}

Para comenzar debemos tomar en consideración que la $FEM$ seria: \[
  \mathcal{E}\left( t \right) = - \frac{d\Phi}{dt}
.\] 

Por lo tanto, debemos iniciar por encontrar el flujo magnético atreves de la $FEM$. Para esto entonces usemos \[
\Phi = BS\cos\theta
.\] Tenemos cuanto es el área (que corresponde con $a^2$ dado que es una espira cuadrada). Ademas, dado que esta rotando entonces $\theta$ varia con el tiempo respecto a  $\omega t$ Ademas de tener un campo  $B$ lo que seria entonces.
 \begin{align*}
   \Phi &= B\cdot S\\
	&= B a^2 \cos\left( \omega t \right)\\
   \mathcal{E}\left( t \right) &= - \frac{d\Phi}{dt}\\
			       &= - \frac{d\left( B a^2 \cos\left( \omega t \right)  \right) }{dt}\\
			       &= - B a^2 \left( - \sin\left( \omega t \right)  \right) \omega\\
			       &= B \omega a^2 \sin\left( \omega t \right) \qed
.\end{align*}

\chapter{Punto 7.12}

Para este caso, volvemos a iniciar por calcular el flujo magnético. Para esto necesitamos saber cuanto es el área que dado que es una espira circular seria $\pi r^2 = \pi \left( \frac{a}{2} \right)^2 = \frac{\pi a^2}{4}$. Ademas de que ya tenemos el campo magnético entonces esto seria \[
  \Phi = B(t)\cdot A = \frac{\pi a^2}{4}B_0 \cos\left( \omega t \right) 
.\]

Ahora con esto podemos de nuevo encontrar la $FEM$ que seria
 \begin{align*}
   \mathcal{E} &= - \frac{d\Phi}{dt}\\
	       &= - \frac{\pi a^2}{4}B_0 \frac{d}{dt}\left[ \cos\left( \omega t \right)  \right] \\
	       &= - \frac{\pi a^2}{4}B_0 \left[ -\sin\left( \omega t \right)  \right]\omega \\
	       &= \frac{\pi a^2}{4}\omega B_0 \sin\left( \omega t \right)
.\end{align*}

Con todo esto entonces podemos terminar calculando la corriente inducida lo que quedaría como
\begin{align*}
  I\left( t \right) &= \frac{\mathcal{E}}{R}\\
  &= \frac{\frac{\pi a^2 \omega B_0}{4}\sin\left( \omega t \right) }{R} \\
  &= \frac{\pi a^2 \omega B_0}{4R}\sin\left( \omega t \right)\qed
.\end{align*}

\chapter{Punto 7.16}

\section{Parte B}

Para esto vamos a usar un loop amperiano. 

\begin{tikzpicture}[
    cyl/.style={
        cylinder, 
        shape aspect=4, 
        %rotate=90,  % Orientación horizontal
        minimum height=4cm, 
        minimum width=1.2cm, 
        cylinder body fill=blue!10, 
        cylinder end fill=blue!5,
        draw=black!80
    },
    loop/.style={
        thick, 
        red, 
        -{Stealth[scale=1.2]},
        fill opacity=0
    },
    wire/.style={thick, black!60},
    axis/.style={dashed, black!50},
    campo/.style={blue, -{Stealth}}
]

% Cilindro coaxial (horizontal, eje z)
\node[cyl] (tubo) at (0,0) {};

% Cable central (horizontal)
\draw[wire] (-3,0) -- (3,0) node[midway, above=2mm] {$I(t)$};

% Eje z (horizontal)
\draw[axis, -{Stealth}] (-4,0) -- (4,0) node[right] {$z$};

% Loop amperiano (rectángulo alrededor del cilindro)
\draw[loop] (-1.5,0.5) rectangle (1.5,-0.5)
    node[midway, right=5mm, black] {Loop amperiano};
    
% Longitud del loop (l)
\draw[loop, {Stealth}-{Stealth}] (1.5,0.5) -- (1.5,-0.5)
    node[midway, right] {$l$};

% Flechas de campo magnético (circunferencial)
\foreach \x in {-1,0,1} {
    \draw[campo] (\x, 0.7) arc (90:270:0.3 and 0.7);
    \draw[campo] (\x, -0.7) arc (270:450:0.3 and 0.7);
}

% Etiquetas
\node[blue] at (0,1.3) {$\mathbf{B}_\phi(s,t)$};
\node[red] at (2.5,0) {$\mathbf{E}_z(s,t)$};
\node[black!60] at (-3.5,0) {Cable central};

\end{tikzpicture}


\end{document}

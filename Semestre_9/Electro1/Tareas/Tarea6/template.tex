\documentclass{report}

\documentclass[12pt]{article}
\usepackage{array}
\usepackage{color}
\usepackage{amsthm}
\usepackage{eufrak}
\usepackage{lipsum}
\usepackage{pifont}
\usepackage{yfonts}
\usepackage{amsmath}
\usepackage{amssymb}
\usepackage{ccfonts}
\usepackage{comment} \usepackage{amsfonts}
\usepackage{fancyhdr}
\usepackage{graphicx}
\usepackage{listings}
\usepackage{mathrsfs}
\usepackage{setspace}
\usepackage{textcomp}
\usepackage{blindtext}
\usepackage{enumerate}
\usepackage{microtype}
\usepackage{xfakebold}
\usepackage{kantlipsum}
%\usepackage{draftwatermark}
\usepackage[spanish]{babel}
\usepackage[margin=1.5cm, top=2cm, bottom=2cm]{geometry}
\usepackage[framemethod=tikz]{mdframed}
\usepackage[colorlinks=true,citecolor=blue,linkcolor=red,urlcolor=magenta]{hyperref}

%//////////////////////////////////////////////////////
% Watermark configuration
%//////////////////////////////////////////////////////
%\SetWatermarkScale{4}
%\SetWatermarkColor{black}
%\SetWatermarkLightness{0.95}
%\SetWatermarkText{\texttt{Watermark}}

%//////////////////////////////////////////////////////
% Frame configuration
%//////////////////////////////////////////////////////
\newmdenv[tikzsetting={draw=gray,fill=white,fill opacity=0},backgroundcolor=none]{Frame}

%//////////////////////////////////////////////////////
% Font style configuration
%//////////////////////////////////////////////////////
\renewcommand{\familydefault}{\ttdefault}
\renewcommand{\rmdefault}{tt}

%//////////////////////////////////////////////////////
% Bold configuration
%//////////////////////////////////////////////////////
\newcommand{\fbseries}{\unskip\setBold\aftergroup\unsetBold\aftergroup\ignorespaces}
\makeatletter
\newcommand{\setBoldness}[1]{\def\fake@bold{#1}}
\makeatother

%//////////////////////////////////////////////////////
% Default font configuration
%//////////////////////////////////////////////////////
\DeclareFontFamily{\encodingdefault}{\ttdefault}{%
  \hyphenchar\font=\defaulthyphenchar
  \fontdimen2\font=0.33333em
  \fontdimen3\font=0.16667em
  \fontdimen4\font=0.11111em
  \fontdimen7\font=0.11111em}


%From M275 "Topology" at SJSU
\newcommand{\id}{\mathrm{id}}
\newcommand{\taking}[1]{\xrightarrow{#1}}
\newcommand{\inv}{^{-1}}

%From M170 "Introduction to Graph Theory" at SJSU
\DeclareMathOperator{\diam}{diam}
\DeclareMathOperator{\ord}{ord}
\newcommand{\defeq}{\overset{\mathrm{def}}{=}}

%From the USAMO .tex files
\newcommand{\ts}{\textsuperscript}
\newcommand{\dg}{^\circ}
\newcommand{\ii}{\item}

% % From Math 55 and Math 145 at Harvard
% \newenvironment{subproof}[1][Proof]{%
% \begin{proof}[#1] \renewcommand{\qedsymbol}{$\blacksquare$}}%
% {\end{proof}}

\newcommand{\liff}{\leftrightarrow}
\newcommand{\lthen}{\rightarrow}
\newcommand{\opname}{\operatorname}
\newcommand{\surjto}{\twoheadrightarrow}
\newcommand{\injto}{\hookrightarrow}
\newcommand{\On}{\mathrm{On}} % ordinals
\DeclareMathOperator{\img}{im} % Image
\DeclareMathOperator{\Img}{Im} % Image
\DeclareMathOperator{\coker}{coker} % Cokernel
\DeclareMathOperator{\Coker}{Coker} % Cokernel
\DeclareMathOperator{\Ker}{Ker} % Kernel
\DeclareMathOperator{\rank}{rank}
\DeclareMathOperator{\Spec}{Spec} % spectrum
\DeclareMathOperator{\Tr}{Tr} % trace
\DeclareMathOperator{\pr}{pr} % projection
\DeclareMathOperator{\ext}{ext} % extension
\DeclareMathOperator{\pred}{pred} % predecessor
\DeclareMathOperator{\dom}{dom} % domain
\DeclareMathOperator{\ran}{ran} % range
\DeclareMathOperator{\Hom}{Hom} % homomorphism
\DeclareMathOperator{\Mor}{Mor} % morphisms
\DeclareMathOperator{\End}{End} % endomorphism

\newcommand{\eps}{\epsilon}
\newcommand{\veps}{\varepsilon}
\newcommand{\ol}{\overline}
\newcommand{\ul}{\underline}
\newcommand{\wt}{\widetilde}
\newcommand{\wh}{\widehat}
\newcommand{\vocab}[1]{\textbf{\color{blue} #1}}
\providecommand{\half}{\frac{1}{2}}
\newcommand{\dang}{\measuredangle} %% Directed angle
\newcommand{\ray}[1]{\overrightarrow{#1}}
\newcommand{\seg}[1]{\overline{#1}}
\newcommand{\arc}[1]{\wideparen{#1}}
\DeclareMathOperator{\cis}{cis}
\DeclareMathOperator*{\lcm}{lcm}
\DeclareMathOperator*{\argmin}{arg min}
\DeclareMathOperator*{\argmax}{arg max}
\newcommand{\cycsum}{\sum_{\mathrm{cyc}}}
\newcommand{\symsum}{\sum_{\mathrm{sym}}}
\newcommand{\cycprod}{\prod_{\mathrm{cyc}}}
\newcommand{\symprod}{\prod_{\mathrm{sym}}}
\newcommand{\Qed}{\begin{flushright}\qed\end{flushright}}
\newcommand{\parinn}{\setlength{\parindent}{1cm}}
\newcommand{\parinf}{\setlength{\parindent}{0cm}}
% \newcommand{\norm}{\|\cdot\|}
\newcommand{\inorm}{\norm_{\infty}}
\newcommand{\opensets}{\{V_{\alpha}\}_{\alpha\in I}}
\newcommand{\oset}{V_{\alpha}}
\newcommand{\opset}[1]{V_{\alpha_{#1}}}
\newcommand{\lub}{\text{lub}}
\newcommand{\del}[2]{\frac{\partial #1}{\partial #2}}
\newcommand{\Del}[3]{\frac{\partial^{#1} #2}{\partial^{#1} #3}}
\newcommand{\deld}[2]{\dfrac{\partial #1}{\partial #2}}
\newcommand{\Deld}[3]{\dfrac{\partial^{#1} #2}{\partial^{#1} #3}}
\newcommand{\lm}{\lambda}
\newcommand{\uin}{\mathbin{\rotatebox[origin=c]{90}{$\in$}}}
\newcommand{\usubset}{\mathbin{\rotatebox[origin=c]{90}{$\subset$}}}
\newcommand{\lt}{\left}
\newcommand{\rt}{\right}
\newcommand{\paren}[1]{\left(#1\right)}
\newcommand{\bs}[1]{\boldsymbol{#1}}
\newcommand{\exs}{\exists}
\newcommand{\st}{\strut}
\newcommand{\dps}[1]{\displaystyle{#1}}

\newcommand{\sol}{\setlength{\parindent}{0cm}\textbf{\textit{Solution:}}\setlength{\parindent}{1cm} }
\newcommand{\solve}[1]{\setlength{\parindent}{0cm}\textbf{\textit{Solution: }}\setlength{\parindent}{1cm}#1 \Qed}

% Things Lie
\newcommand{\kb}{\mathfrak b}
\newcommand{\kg}{\mathfrak g}
\newcommand{\kh}{\mathfrak h}
\newcommand{\kn}{\mathfrak n}
\newcommand{\ku}{\mathfrak u}
\newcommand{\kz}{\mathfrak z}
\DeclareMathOperator{\Ext}{Ext} % Ext functor
\DeclareMathOperator{\Tor}{Tor} % Tor functor
\newcommand{\gl}{\opname{\mathfrak{gl}}} % frak gl group
\renewcommand{\sl}{\opname{\mathfrak{sl}}} % frak sl group chktex 6

% More script letters etc.
\newcommand{\SA}{\mathcal A}
\newcommand{\SB}{\mathcal B}
\newcommand{\SC}{\mathcal C}
\newcommand{\SF}{\mathcal F}
\newcommand{\SG}{\mathcal G}
\newcommand{\SH}{\mathcal H}
\newcommand{\OO}{\mathcal O}

\newcommand{\SCA}{\mathscr A}
\newcommand{\SCB}{\mathscr B}
\newcommand{\SCC}{\mathscr C}
\newcommand{\SCD}{\mathscr D}
\newcommand{\SCE}{\mathscr E}
\newcommand{\SCF}{\mathscr F}
\newcommand{\SCG}{\mathscr G}
\newcommand{\SCH}{\mathscr H}

% Mathfrak primes
\newcommand{\km}{\mathfrak m}
\newcommand{\kp}{\mathfrak p}
\newcommand{\kq}{\mathfrak q}

% number sets
\newcommand{\RR}[1][]{\ensuremath{\ifstrempty{#1}{\mathbb{R}}{\mathbb{R}^{#1}}}}
\newcommand{\NN}[1][]{\ensuremath{\ifstrempty{#1}{\mathbb{N}}{\mathbb{N}^{#1}}}}
\newcommand{\ZZ}[1][]{\ensuremath{\ifstrempty{#1}{\mathbb{Z}}{\mathbb{Z}^{#1}}}}
\newcommand{\QQ}[1][]{\ensuremath{\ifstrempty{#1}{\mathbb{Q}}{\mathbb{Q}^{#1}}}}
\newcommand{\CC}[1][]{\ensuremath{\ifstrempty{#1}{\mathbb{C}}{\mathbb{C}^{#1}}}}
\newcommand{\PP}[1][]{\ensuremath{\ifstrempty{#1}{\mathbb{P}}{\mathbb{P}^{#1}}}}
\newcommand{\HH}[1][]{\ensuremath{\ifstrempty{#1}{\mathbb{H}}{\mathbb{H}^{#1}}}}
\newcommand{\FF}[1][]{\ensuremath{\ifstrempty{#1}{\mathbb{F}}{\mathbb{F}^{#1}}}}
% expected value
\newcommand{\EE}{\ensuremath{\mathbb{E}}}
\newcommand{\charin}{\text{ char }}
\DeclareMathOperator{\sign}{sign}
\DeclareMathOperator{\Aut}{Aut}
\DeclareMathOperator{\Inn}{Inn}
\DeclareMathOperator{\Syl}{Syl}
\DeclareMathOperator{\Gal}{Gal}
\DeclareMathOperator{\GL}{GL} % General linear group
\DeclareMathOperator{\SL}{SL} % Special linear group

%---------------------------------------
% BlackBoard Math Fonts :-
%---------------------------------------

%Captital Letters
\newcommand{\bbA}{\mathbb{A}}	\newcommand{\bbB}{\mathbb{B}}
\newcommand{\bbC}{\mathbb{C}}	\newcommand{\bbD}{\mathbb{D}}
\newcommand{\bbE}{\mathbb{E}}	\newcommand{\bbF}{\mathbb{F}}
\newcommand{\bbG}{\mathbb{G}}	\newcommand{\bbH}{\mathbb{H}}
\newcommand{\bbI}{\mathbb{I}}	\newcommand{\bbJ}{\mathbb{J}}
\newcommand{\bbK}{\mathbb{K}}	\newcommand{\bbL}{\mathbb{L}}
\newcommand{\bbM}{\mathbb{M}}	\newcommand{\bbN}{\mathbb{N}}
\newcommand{\bbO}{\mathbb{O}}	\newcommand{\bbP}{\mathbb{P}}
\newcommand{\bbQ}{\mathbb{Q}}	\newcommand{\bbR}{\mathbb{R}}
\newcommand{\bbS}{\mathbb{S}}	\newcommand{\bbT}{\mathbb{T}}
\newcommand{\bbU}{\mathbb{U}}	\newcommand{\bbV}{\mathbb{V}}
\newcommand{\bbW}{\mathbb{W}}	\newcommand{\bbX}{\mathbb{X}}
\newcommand{\bbY}{\mathbb{Y}}	\newcommand{\bbZ}{\mathbb{Z}}

%---------------------------------------
% MathCal Fonts :-
%---------------------------------------

%Captital Letters
\newcommand{\mcA}{\mathcal{A}}	\newcommand{\mcB}{\mathcal{B}}
\newcommand{\mcC}{\mathcal{C}}	\newcommand{\mcD}{\mathcal{D}}
\newcommand{\mcE}{\mathcal{E}}	\newcommand{\mcF}{\mathcal{F}}
\newcommand{\mcG}{\mathcal{G}}	\newcommand{\mcH}{\mathcal{H}}
\newcommand{\mcI}{\mathcal{I}}	\newcommand{\mcJ}{\mathcal{J}}
\newcommand{\mcK}{\mathcal{K}}	\newcommand{\mcL}{\mathcal{L}}
\newcommand{\mcM}{\mathcal{M}}	\newcommand{\mcN}{\mathcal{N}}
\newcommand{\mcO}{\mathcal{O}}	\newcommand{\mcP}{\mathcal{P}}
\newcommand{\mcQ}{\mathcal{Q}}	\newcommand{\mcR}{\mathcal{R}}
\newcommand{\mcS}{\mathcal{S}}	\newcommand{\mcT}{\mathcal{T}}
\newcommand{\mcU}{\mathcal{U}}	\newcommand{\mcV}{\mathcal{V}}
\newcommand{\mcW}{\mathcal{W}}	\newcommand{\mcX}{\mathcal{X}}
\newcommand{\mcY}{\mathcal{Y}}	\newcommand{\mcZ}{\mathcal{Z}}


%---------------------------------------
% Bold Math Fonts :-
%---------------------------------------

%Captital Letters
\newcommand{\bmA}{\boldsymbol{A}}	\newcommand{\bmB}{\boldsymbol{B}}
\newcommand{\bmC}{\boldsymbol{C}}	\newcommand{\bmD}{\boldsymbol{D}}
\newcommand{\bmE}{\boldsymbol{E}}	\newcommand{\bmF}{\boldsymbol{F}}
\newcommand{\bmG}{\boldsymbol{G}}	\newcommand{\bmH}{\boldsymbol{H}}
\newcommand{\bmI}{\boldsymbol{I}}	\newcommand{\bmJ}{\boldsymbol{J}}
\newcommand{\bmK}{\boldsymbol{K}}	\newcommand{\bmL}{\boldsymbol{L}}
\newcommand{\bmM}{\boldsymbol{M}}	\newcommand{\bmN}{\boldsymbol{N}}
\newcommand{\bmO}{\boldsymbol{O}}	\newcommand{\bmP}{\boldsymbol{P}}
\newcommand{\bmQ}{\boldsymbol{Q}}	\newcommand{\bmR}{\boldsymbol{R}}
\newcommand{\bmS}{\boldsymbol{S}}	\newcommand{\bmT}{\boldsymbol{T}}
\newcommand{\bmU}{\boldsymbol{U}}	\newcommand{\bmV}{\boldsymbol{V}}
\newcommand{\bmW}{\boldsymbol{W}}	\newcommand{\bmX}{\boldsymbol{X}}
\newcommand{\bmY}{\boldsymbol{Y}}	\newcommand{\bmZ}{\boldsymbol{Z}}
%Small Letters
\newcommand{\bma}{\boldsymbol{a}}	\newcommand{\bmb}{\boldsymbol{b}}
\newcommand{\bmc}{\boldsymbol{c}}	\newcommand{\bmd}{\boldsymbol{d}}
\newcommand{\bme}{\boldsymbol{e}}	\newcommand{\bmf}{\boldsymbol{f}}
\newcommand{\bmg}{\boldsymbol{g}}	\newcommand{\bmh}{\boldsymbol{h}}
\newcommand{\bmi}{\boldsymbol{i}}	\newcommand{\bmj}{\boldsymbol{j}}
\newcommand{\bmk}{\boldsymbol{k}}	\newcommand{\bml}{\boldsymbol{l}}
\newcommand{\bmm}{\boldsymbol{m}}	\newcommand{\bmn}{\boldsymbol{n}}
\newcommand{\bmo}{\boldsymbol{o}}	\newcommand{\bmp}{\boldsymbol{p}}
\newcommand{\bmq}{\boldsymbol{q}}	\newcommand{\bmr}{\boldsymbol{r}}
\newcommand{\bms}{\boldsymbol{s}}	\newcommand{\bmt}{\boldsymbol{t}}
\newcommand{\bmu}{\boldsymbol{u}}	\newcommand{\bmv}{\boldsymbol{v}}
\newcommand{\bmw}{\boldsymbol{w}}	\newcommand{\bmx}{\boldsymbol{x}}
\newcommand{\bmy}{\boldsymbol{y}}	\newcommand{\bmz}{\boldsymbol{z}}

%---------------------------------------
% Scr Math Fonts :-
%---------------------------------------

\newcommand{\sA}{{\mathscr{A}}}   \newcommand{\sB}{{\mathscr{B}}}
\newcommand{\sC}{{\mathscr{C}}}   \newcommand{\sD}{{\mathscr{D}}}
\newcommand{\sE}{{\mathscr{E}}}   \newcommand{\sF}{{\mathscr{F}}}
\newcommand{\sG}{{\mathscr{G}}}   \newcommand{\sH}{{\mathscr{H}}}
\newcommand{\sI}{{\mathscr{I}}}   \newcommand{\sJ}{{\mathscr{J}}}
\newcommand{\sK}{{\mathscr{K}}}   \newcommand{\sL}{{\mathscr{L}}}
\newcommand{\sM}{{\mathscr{M}}}   \newcommand{\sN}{{\mathscr{N}}}
\newcommand{\sO}{{\mathscr{O}}}   \newcommand{\sP}{{\mathscr{P}}}
\newcommand{\sQ}{{\mathscr{Q}}}   \newcommand{\sR}{{\mathscr{R}}}
\newcommand{\sS}{{\mathscr{S}}}   \newcommand{\sT}{{\mathscr{T}}}
\newcommand{\sU}{{\mathscr{U}}}   \newcommand{\sV}{{\mathscr{V}}}
\newcommand{\sW}{{\mathscr{W}}}   \newcommand{\sX}{{\mathscr{X}}}
\newcommand{\sY}{{\mathscr{Y}}}   \newcommand{\sZ}{{\mathscr{Z}}}


%---------------------------------------
% Math Fraktur Font
%---------------------------------------

%Captital Letters
\newcommand{\mfA}{\mathfrak{A}}	\newcommand{\mfB}{\mathfrak{B}}
\newcommand{\mfC}{\mathfrak{C}}	\newcommand{\mfD}{\mathfrak{D}}
\newcommand{\mfE}{\mathfrak{E}}	\newcommand{\mfF}{\mathfrak{F}}
\newcommand{\mfG}{\mathfrak{G}}	\newcommand{\mfH}{\mathfrak{H}}
\newcommand{\mfI}{\mathfrak{I}}	\newcommand{\mfJ}{\mathfrak{J}}
\newcommand{\mfK}{\mathfrak{K}}	\newcommand{\mfL}{\mathfrak{L}}
\newcommand{\mfM}{\mathfrak{M}}	\newcommand{\mfN}{\mathfrak{N}}
\newcommand{\mfO}{\mathfrak{O}}	\newcommand{\mfP}{\mathfrak{P}}
\newcommand{\mfQ}{\mathfrak{Q}}	\newcommand{\mfR}{\mathfrak{R}}
\newcommand{\mfS}{\mathfrak{S}}	\newcommand{\mfT}{\mathfrak{T}}
\newcommand{\mfU}{\mathfrak{U}}	\newcommand{\mfV}{\mathfrak{V}}
\newcommand{\mfW}{\mathfrak{W}}	\newcommand{\mfX}{\mathfrak{X}}
\newcommand{\mfY}{\mathfrak{Y}}	\newcommand{\mfZ}{\mathfrak{Z}}
%Small Letters
\newcommand{\mfa}{\mathfrak{a}}	\newcommand{\mfb}{\mathfrak{b}}
\newcommand{\mfc}{\mathfrak{c}}	\newcommand{\mfd}{\mathfrak{d}}
\newcommand{\mfe}{\mathfrak{e}}	\newcommand{\mff}{\mathfrak{f}}
\newcommand{\mfg}{\mathfrak{g}}	\newcommand{\mfh}{\mathfrak{h}}
\newcommand{\mfi}{\mathfrak{i}}	\newcommand{\mfj}{\mathfrak{j}}
\newcommand{\mfk}{\mathfrak{k}}	\newcommand{\mfl}{\mathfrak{l}}
\newcommand{\mfm}{\mathfrak{m}}	\newcommand{\mfn}{\mathfrak{n}}
\newcommand{\mfo}{\mathfrak{o}}	\newcommand{\mfp}{\mathfrak{p}}
\newcommand{\mfq}{\mathfrak{q}}	\newcommand{\mfr}{\mathfrak{r}}
\newcommand{\mfs}{\mathfrak{s}}	\newcommand{\mft}{\mathfrak{t}}
\newcommand{\mfu}{\mathfrak{u}}	\newcommand{\mfv}{\mathfrak{v}}
\newcommand{\mfw}{\mathfrak{w}}	\newcommand{\mfx}{\mathfrak{x}}
\newcommand{\mfy}{\mathfrak{y}}	\newcommand{\mfz}{\mathfrak{z}}

\usepackage{amsmath}
\usetikzlibrary{arrows.meta, shapes.geometric}

\renewcommand{\qed}{\hfill\square}

\title{\Huge{Electromagnetismo 1}\\Tarea 5}
\author{\huge{Sergio Montoya Ramírez}}
\date{}

\begin{document}

\maketitle
\newpage% or \cleardoublepage
% \pdfbookmark[<level>]{<title>}{<dest>}
\pdfbookmark[section]{\contentsname}{toc}
\tableofcontents
\pagebreak

\chapter{Punto 7.10}

Para comenzar debemos tomar en consideración que la $FEM$ seria: \[
  \mathcal{E}\left( t \right) = - \frac{d\Phi}{dt}
.\] 

Por lo tanto, debemos iniciar por encontrar el flujo magnético atreves de la $FEM$. Para esto entonces usemos \[
\Phi = BS\cos\theta
.\] Tenemos cuanto es el área (que corresponde con $a^2$ dado que es una espira cuadrada). Ademas, dado que esta rotando entonces $\theta$ varia con el tiempo respecto a  $\omega t$ Ademas de tener un campo  $B$ lo que seria entonces.
 \begin{align*}
   \Phi &= B\cdot S\\
	&= B a^2 \cos\left( \omega t \right)\\
   \mathcal{E}\left( t \right) &= - \frac{d\Phi}{dt}\\
			       &= - \frac{d\left( B a^2 \cos\left( \omega t \right)  \right) }{dt}\\
			       &= - B a^2 \left( - \sin\left( \omega t \right)  \right) \omega\\
			       &= B \omega a^2 \sin\left( \omega t \right) \qed
.\end{align*}

\chapter{Punto 7.12}

Para este caso, volvemos a iniciar por calcular el flujo magnético. Para esto necesitamos saber cuanto es el área que dado que es una espira circular seria $\pi r^2 = \pi \left( \frac{a}{2} \right)^2 = \frac{\pi a^2}{4}$. Ademas de que ya tenemos el campo magnético entonces esto seria \[
  \Phi = B(t)\cdot A = \frac{\pi a^2}{4}B_0 \cos\left( \omega t \right) 
.\]

Ahora con esto podemos de nuevo encontrar la $FEM$ que seria
 \begin{align*}
   \mathcal{E} &= - \frac{d\Phi}{dt}\\
	       &= - \frac{\pi a^2}{4}B_0 \frac{d}{dt}\left[ \cos\left( \omega t \right)  \right] \\
	       &= - \frac{\pi a^2}{4}B_0 \left[ -\sin\left( \omega t \right)  \right]\omega \\
	       &= \frac{\pi a^2}{4}\omega B_0 \sin\left( \omega t \right)
.\end{align*}

Con todo esto entonces podemos terminar calculando la corriente inducida lo que quedaría como
\begin{align*}
  I\left( t \right) &= \frac{\mathcal{E}}{R}\\
  &= \frac{\frac{\pi a^2 \omega B_0}{4}\sin\left( \omega t \right) }{R} \\
  &= \frac{\pi a^2 \omega B_0}{4R}\sin\left( \omega t \right)\qed
.\end{align*}

\chapter{Punto 7.16}

\section{Parte A}

Note que para el campo magnetico que genera el cilindro se cumple que por simetria se cancela de manera interna. Por lo tanto, solo queda el campo inducido por el cable que va saliendo de la pagina y por lo tanto induce una corriente de manera longitudinal.

\section{Parte B}

Para esto vamos a usar un loop amperiano. que sea un rectangulo que salga del cilindro y tomando en cuenta que este se cancela por fuera entonces esto daria $B=0$ afuera. Con esto entonces
\begin{align*}
	\oint E \cdot dl &= - \frac{d\Phi}{dt}\\
	\oint E \cdot dl &= El\\
	- \frac{d\Phi}{dt} &= -\frac{d}{dt} \int B \cdot da\\
	&= - \frac{d}{dt}\int_s^a \frac{\mu_0 I}{2\pi s'}l ds'\\
	El &= - \frac{dI}{dt} \frac{\mu_0}{2\pi}\ln\left(\frac{a}{s}\right)l\\
	E &= - \frac{dI}{dt} \frac{\mu_0}{2\pi}\ln\left(\frac{a}{s}\right)\\
	\frac{dI}{dt} &= - I_0 \omega \sin \omega t\\
	E &= - \frac{dI}{dt} \frac{\mu_0I_0 \omega}{2\pi}\sin(\omega t)\ln\left(\frac{a}{s}\right)\\
\end{align*}

\chapter{Punto 7.20}

\section{}
El campo en el eje es \[B = \frac{\mu_0 I}{2}\frac{b^2}{(b^2 + z^2)^{\frac{3}{2}}}\hat{z}\]

por lo tanto el flujo por el area es:
\begin{align*}
	\Phi &= \int B \cdot da\\
	&= B \pi a^2\\
	&= \frac{\mu_0 \pi I a^2 b^2}{2(b^2 + z^2)}^{\frac{3}{2}}
\end{align*}

\section{}

El campo es \[B = \frac{\mu_0}{4\pi}\frac{m}{r^3}(2\cos\theta \hat{r} + \sin\theta\hat{\theta})\], donde $m = I\pi a^2$ Ahora si integramos
\begin{align*}
	\Phi &= \int B \cdot da\\
	&= \frac{\mu_0}{4\pi}\frac{I\pi a^2}{r^3} \int (2\cos\theta)(r^2\sin\theta d\theta d\phi)\\
	&= \frac{\mu_0 I a^2}{2r} 2\pi \int_0^\theta \cos\theta\sin\theta d\theta\\
	&= \frac{\mu_0 I \pi a^2}{r}\left.\frac{\sin^2\theta}{2}\right|_0^\theta\\
	&= \frac{\mu_0 \pi I a^2 b^2}{2(b^2 + z^2)^{\frac{3}{2}}}
\end{align*}

\section{}

Partimos desde
\begin{align*}
	\phi_1 &= M_{12} I_2\\
	\phi_2 &= M_{21} I_1\\
	M_{12}&= \frac{\frac{\mu_0 \pi I_1 a^2 b^2}{2(b^2 + z^2)}^{\frac{3}{2}}}{I_1}\\
	M_{21}&= \frac{\frac{\mu_0 \pi I_2 a^2 b^2}{2(b^2 + z^2)}^{\frac{3}{2}}}{I_2}\\
	M_{12} &= M_{21} = \frac{\mu_0 \pi a^2 b^2}{2(b^2 + z^2)^{\frac{3}{2}}}
\end{align*}

\chapter{Punto 7.45}

Para este caso tenemos
\begin{align*}
	f &= v \times B\\
	v &= \omega a\sin\theta\hat{\phi}\\
	f &= \omega a B_0 \sin\theta (\hat{\phi}\times \hat{z})\\
	\mathcal{E} &= \int f \cdot dl\\
	dl &= a d\theta \hat{\theta}\\
	\mathcal{E} &= \omega a^2 B_0 \int_0^{\frac{\pi}{2}} \sin\theta (\hat{\phi} \times \hat{z})\cdot \hat{\theta}d\theta\\
	\hat{\theta}\cdot(\hat{\phi}\times \hat{z} &= \hat{z}\cdot(\hat{\theta}\times \hat{\phi})\\
	&= \hat{z} \cdot \hat{r}\\
	&= \cos \theta\\
	\mathcal{E} &= \omega a^2 B_0 \int_{0}^{\frac{\pi}{2}}\sin\theta \cos\theta d\theta\\
	&= \omega a^2 B_0 \frac{\sin^2\theta}{2}|_0^{\frac{\pi}{2}}\\
	&= \frac{1}{2}\omega a^2 B_0
\end{align*}

\chapter{Punto 7.54}

\section{}

Suponga una corriente $I_1$ y $I_2$ entonces (Si $\Phi$ es el flujo atravez de un turno):
\begin{align*}
	\Phi_1 &= I_1L_1 + MI_2 = N_1 \Phi\\
	\Phi_2 &= I_2L_2 + MI_1 = N_2 \Phi\\
	\Phi &= I_1 \frac{L_1}{N_1} + I_2\frac{M}{N_1} = I_2\frac{L_2}{N_2} + I_1 \frac{M}{N_2}\\
\end{align*}

En el caso de $I_1 = 0$ tenemos $\frac{M}{N_1} = \frac{L_2}{N_2}$. Por el otro lado, en el caso de $I_2 = 0$ tendremos $\frac{L_1}{N_1} = \frac{M}{N_2}$. Ahora tenemos:

\begin{align*}
	\frac{\frac{M}{N_1}}{\frac{L_1}{N_1}} &= \frac{\frac{L_2}{N_2}}{\frac{M}{N_2}}\\
	\frac{M}{L_1} &= \frac{L_2}{M}\\
	L_1L_2 &= M^2
\end{align*}

\section{}

\begin{align*}
	-\mathcal{E}_1 &= \frac{d\Phi_1}{dt}\\
	&= L_1 \frac{dI_1}{dt} + M\frac{dI_2}{dt}\\
	&= V_1\cos(\omega t)\\
	-\mathcal{E}_2 &= \frac{d\Phi_2}{dt}\\
	&= L_2 \frac{dI_2}{dt} + M\frac{dI_1}{dt}\\
	&= -I_2R
\end{align*}

\section{}

\begin{align*}
	L_1 \frac{dI_1}{dt} + M \frac{dI_2}{dt} &= V_1 \cos(\omega t)\\
	L_2 \frac{dI_2}{dt} + M \frac{dI_1}{dt} &= -I_2 R\\
	L_1 L_2 \frac{dI_1}{dt} + M L_2 \frac{dI_2}{dt} &= L_2 V_1 \cos(\omega t)\\
	L_1 L_2 \frac{dI_1}{dt} + M(-I_2 R - M \frac{dI_1}{dt}) &= L_2 V_1 \cos(\omega t)
\end{align*}

Usando \(L_1 L_2 = M^2\), los términos con \(\frac{dI_1}{dt}\) se cancelan:  

\begin{equation*}
-M R I_2 = L_2 V_1 \cos(\omega t) \implies I_2(t) = -\frac{L_2 V_1}{M R} \cos(\omega t)
\end{equation*}

\begin{align*}
	\frac{dI_2}{dt} &= \frac{L_2 V_1 \omega}{M R} \sin(\omega t)\\
	L_1 \frac{dI_1}{dt} + M \left(\frac{L_2 V_1 \omega}{M R} \sin(\omega t)\right) &= V_1 \cos(\omega t)\\
	\frac{dI_1}{dt} &= \frac{V_1}{L_1} \cos(\omega t) - \frac{L_2 \omega V_1}{L_1 R} \sin(\omega t)\\
	I_1(t) &= \frac{V_1}{L_1} \left(\frac{1}{\omega} \sin(\omega t) + \frac{L_2}{R} \cos(\omega t)\right)\\
	I_2(t) &= -\frac{L_2 V_1}{M R} \cos(\omega t)
\end{align*}

\section{}

\begin{align*}
	\frac{V_{out}}{V_{in}} &= \frac{I^2 R}{V_1\cos\omega t}\\
	&= \frac{-\frac{L_2V_1}{MR}\cos\omega t R}{V_1\cos\omega t}\\
	&= - \frac{L_2}{M}\\
	&= - \frac{N_2}{N_1}
\end{align*}

\section{}

Para $P_in$

\begin{align*}
	P_{in} &= V_{in}I_1\\
	&= (V_1 \cos\omega t)\left(\frac{V_1}{L_1}\right)\left(\frac{1}{\omega}\sin\omega t + \frac{L_2}{R}\cos\omega t\right)\\
	&= \frac{V_1^2}{L_1} \left(\frac{1}{\omega}\sin\omega t\cos\omega t + \frac{L_2}{R}\cos^2\omega t\right)
\end{align*}

Para $P_{out}$

\begin{align*}
	P_{out} &= V_{out} I_2\\
	&= I_2^2 R\\
	&= \frac{(L_2 V_1)^2}{M^2 R}\cos^2\omega t
\end{align*}

Para los valores promedio se tiene

\begin{align*}
	\cos^2(\omega t) &= \frac{1 + \cos(2\omega t)}{2}\\
	\langle \cos^2(\omega t) \rangle &= \frac{1}{T} \int_0^T \frac{1 + \cos(2\omega t)}{2} \, dt = \frac{1}{2}.
\end{align*}

   \begin{align*}
	   \sin(\omega t)\cos(\omega t) &= \frac{\sin(2\omega t)}{2}\\
	   \langle \sin(\omega t)\cos(\omega t) \rangle &= 0.
   \end{align*}

Con lo cual podemos reemplazar en las ecuaciones encontrada anteriormente para que nos de:
\begin{align*}
	\langle P_{in}\rangle &= \frac{V_1^2}{L_1} \left(\frac{1}{\omega}\sin\omega t\cos\omega t + \frac{L_2}{R}\cos^2\omega t\right)\\
	\langle P_{in}\rangle &= \frac{V_1^2}{L_1} \left(\frac{L_2}{2R}\right)\\
	\langle P_{in}\rangle &= \frac{V_1^2L_2}{2RL_1}
\end{align*}

Y por otro lado
\begin{align*}
	\langle P_{out}\rangle &= \frac{(L_2 V_1)^2}{M^2 R}\cos^2\omega t\\
	\langle P_{out}\rangle &= \frac{(L_2 V_1)^2}{2M^2 R}\\
	\langle P_{out}\rangle &= \frac{(L_2^2 V_1)^2}{2L_1L_2 R}\\
	\langle P_{out}\rangle &= \frac{V_1^2L_2}{2RL_1}
\end{align*}

Lo que muestra lo que queriamos $\langle P_{in} \rangle = \langle P_{out} \rangle$

\end{document}

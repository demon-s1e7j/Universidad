\documentclass{report}

\documentclass[12pt]{article}
\usepackage{array}
\usepackage{color}
\usepackage{amsthm}
\usepackage{eufrak}
\usepackage{lipsum}
\usepackage{pifont}
\usepackage{yfonts}
\usepackage{amsmath}
\usepackage{amssymb}
\usepackage{ccfonts}
\usepackage{comment} \usepackage{amsfonts}
\usepackage{fancyhdr}
\usepackage{graphicx}
\usepackage{listings}
\usepackage{mathrsfs}
\usepackage{setspace}
\usepackage{textcomp}
\usepackage{blindtext}
\usepackage{enumerate}
\usepackage{microtype}
\usepackage{xfakebold}
\usepackage{kantlipsum}
%\usepackage{draftwatermark}
\usepackage[spanish]{babel}
\usepackage[margin=1.5cm, top=2cm, bottom=2cm]{geometry}
\usepackage[framemethod=tikz]{mdframed}
\usepackage[colorlinks=true,citecolor=blue,linkcolor=red,urlcolor=magenta]{hyperref}

%//////////////////////////////////////////////////////
% Watermark configuration
%//////////////////////////////////////////////////////
%\SetWatermarkScale{4}
%\SetWatermarkColor{black}
%\SetWatermarkLightness{0.95}
%\SetWatermarkText{\texttt{Watermark}}

%//////////////////////////////////////////////////////
% Frame configuration
%//////////////////////////////////////////////////////
\newmdenv[tikzsetting={draw=gray,fill=white,fill opacity=0},backgroundcolor=none]{Frame}

%//////////////////////////////////////////////////////
% Font style configuration
%//////////////////////////////////////////////////////
\renewcommand{\familydefault}{\ttdefault}
\renewcommand{\rmdefault}{tt}

%//////////////////////////////////////////////////////
% Bold configuration
%//////////////////////////////////////////////////////
\newcommand{\fbseries}{\unskip\setBold\aftergroup\unsetBold\aftergroup\ignorespaces}
\makeatletter
\newcommand{\setBoldness}[1]{\def\fake@bold{#1}}
\makeatother

%//////////////////////////////////////////////////////
% Default font configuration
%//////////////////////////////////////////////////////
\DeclareFontFamily{\encodingdefault}{\ttdefault}{%
  \hyphenchar\font=\defaulthyphenchar
  \fontdimen2\font=0.33333em
  \fontdimen3\font=0.16667em
  \fontdimen4\font=0.11111em
  \fontdimen7\font=0.11111em}


\input{macros}
\input{letterfonts}

\usepackage{listings}
\usepackage{xcolor}

\definecolor{codegreen}{rgb}{0,0.6,0}
\definecolor{codegray}{rgb}{0.5,0.5,0.5}
\definecolor{codepurple}{rgb}{0.58,0,0.82}
\definecolor{backcolour}{rgb}{0.95,0.95,0.92}

\lstdefinestyle{mystyle}{
    backgroundcolor=\color{backcolour},   
    commentstyle=\color{codegreen},
    keywordstyle=\color{magenta},
    numberstyle=\tiny\color{codegray},
    stringstyle=\color{codepurple},
    basicstyle=\ttfamily\footnotesize,
    breakatwhitespace=false,         
    breaklines=true,                 
    captionpos=b,                    
    keepspaces=true,                 
    numbers=left,                    
    numbersep=5pt,                  
    showspaces=false,                
    showstringspaces=false,
    showtabs=false,                  
    tabsize=2
}

\lstset{style=mystyle}

\title{\Huge{Electromagnetismo 1}\\Tarea 5}
\author{\huge{Sergio Montoya Ramirez}}
\date{}

\begin{document}

\maketitle
\newpage% or \cleardoublepage
% \pdfbookmark[<level>]{<title>}{<dest>}
\pdfbookmark[section]{\contentsname}{toc}
\tableofcontents
\pagebreak

\chapter{Punto 1}

\section{}

Tenemos que:
\[
  F = 2\pi I R B\cos\theta
\]

Pero ademas
\begin{align*}
  B &= \frac{\mu_0}{4\pi} \frac{\left[ 3 (m_1\cdot \hat{r})\hat{r} - m_1 \right]}{r^3}\\
  B \cos\theta &= B \cdot \hat{y}\\
  B\cos\theta &=\frac{\mu_0}{4\pi} \frac{1}{r^3} \left[ 3 (m_1\cdot \hat{r})(\hat{r} \cdot \hat{y}) - (m_1\cdot \hat{y}) \right]\\
  m_1 \cdot \hat{y} &= 0\\
  \hat{r}\cdot\hat{y} &= \sin\phi\\
  m_1 \cdot \hat{r} &= m_1 \cos\theta\\
  B\cos\theta &= \frac{\mu_0}{4\pi} \frac{1}{r^3} \left[ 3 m_1\sin\phi \cos\phi \right]
\end{align*}

Ahora bien, si reemplazamos en la ecuación previa:
\begin{align*}
  F &= 2 \pi I R \frac{\mu_0}{4\pi} \frac{1}{r^3} \left[ 3 m_1\sin\phi \cos\phi \right]\\
  \sin\phi &= \frac{R}{r}\\
  \cos\phi &= \frac{\sqrt{r^2 - R^2}}{r}\\
  F &= 3 \frac{\mu_0}{2} m_1 I R^2 \frac{\sqrt{r^2 - R^2}}{r^5}\\
  I R^2 \pi &= m_2\\
  F &= 3 \frac{\mu_0}{2\pi} m_1m_2\frac{\sqrt{r^2 - R^2}}{r^5}
\end{align*}

Ahora bien, en el caso de que $R \ll r$

\begin{align*}
  F &= 3 \frac{\mu_0}{2\pi} m_1m_2\frac{\sqrt{r^2}}{r^5}\\
  F &= 3 \frac{\mu_0}{2\pi} m_1m_2\frac{r}{r^5}\\
  F &= 3 \frac{\mu_0}{2\pi} \frac{m_1m_2}{r^4}
\end{align*}

\section{}

Tenemos que
\begin{align*}
  F &= \nabla (m_2 \cdot B)\\
  &= (m_2 \cdot \nabla)B\\
  &= \left(m_2 \frac{d}{dz} \right)\left[ \frac{\mu_0}{4\pi} \frac{1}{z^3} (3(m_1\cdot\hat{z})\hat{z} - m_1)\right]\\
  2 m_1 &= (3(m_1\cdot\hat{z})\hat{z} - m_1)\\
  &= \left(m_2 \frac{d}{dz} \right)\left[ \frac{\mu_0}{4\pi} \frac{1}{z^3} 2m_1\right]\\
  &= \left(m_2 \frac{d}{dz} \right)\left[ \frac{\mu_0}{2\pi} \frac{1}{z^3} m_1\right]\\
  &= \frac{\mu_0}{2\pi} m_1 m_2 \frac{d}{dz} \left[ \frac{1}{z^3} \right]\\
  \frac{d}{dz} \left[ \frac{1}{z^3} \right] &= - 3 \frac{1}{z^4}\\
  F &= \frac{\mu_0}{2\pi} m_1 m_2 \left( - 3 \frac{1}{z^4} \right)\\
  F &= 3 \frac{\mu_0}{2\pi} \frac{m_1m_2}{r^4}
\end{align*}

\chapter{Punto 3}

\section{Esfera Magnetizada}

Tenemos $K_M = M \times \hat{n}$ que teniendo un vector normal $\hat{n} = \hat{r}$ nos quedamos con:
\[
  K_M = M \times \hat{r}
\]

\section{Cascaron}

Ahora, teniendo un cascaron con carga superficial tenemos
\begin{align*}
  K_\sigma &= \sigma v\\
  &= \sigma (\omega \times r)\\
  &= \sigma (\omega \times R\hat{r})\\
  &= R\sigma (\omega \times \hat{r})
\end{align*}

\section{Igualdad}

Tenemos que mostrar:
\begin{align*}
  K_M &= K_\sigma\\
  M \times \hat{r} &= \sigma R (\omega \times \hat{r})\\
  M &= \sigma R \omega
\end{align*}

Con esto entonces mostramos que si se cumple esta condición estos tendran una densidad de corriente igual.

\chapter{Punto 5}

Tenemos
\begin{align*}
  \nabla \times M &= J_b\\
  &= \frac{1}{s}\frac{\partial}{\partial s}(s k s^2)\hat{z}\\
  &= \frac{1}{s}(3ks^2)\hat{z}\\
  &= 3ks\hat{z}
\end{align*}

Ademas, tenemos
\begin{align*}
  K_b &= M \times \hat{n}\\
  &= k s^2 (\hat{\phi} \times \hat{s})\\
  &= -kR^2\hat{z}
\end{align*}

Por lo tanto, la corriente fluye encima del cilindro y luego vuelve a bajar a la superficie. Ahora, dado que las corrientes son simetricas en la simetria  podemos conseguir el campo con la ley de Ampere.

\begin{align*}
  B \cdot 2\pi s &= \mu_0 I_{enc}\\
  &= \mu_0 \int_0^s J_b da\\
  &= 2\pi k \mu_0 s^3\\
  B &= \mu_0 k s^2 \hat{\phi}\\
  &= \mu_0 M
\end{align*}

Fuera del cilindro $I_{enc} = 0 \implies B = 0$

\chapter{Punto 8}

Iniciamos con
\begin{align*}
  \oint H \cdot d l &= I_{f_{enc}}\\
  &= I\\
  \implies H &= \frac{I}{2\pi s}\hat{\phi}\\
  B &= \mu_0 (1 + \chi_m)H\\
  &= \mu_0 (1 + \chi_m) \frac{I}{2\pi s}\hat{\phi}\\
  M &= \chi_m H\\
  &= \frac{\chi_m I}{2\pi s}\hat{\phi}\\
  J_b &= \nabla \times M\\
  &= \frac{1}{s}\frac{\partial}{\partial s}\left(s \frac{\chi_m I}{2\pi s} \right)\hat{z}\\
  &= 0\\
  K_b &= M \times \hat{n}\\
  &= \begin{cases}
    \frac{\chi_m I}{2\pi a}\hat{z} & s = a\\
    -\frac{\chi_m I}{2\pi b}\hat{z} & r = b
  \end{cases}
\end{align*}

La corriente total de un loop entre los cilindros:

\begin{align*}
  I + \frac{\chi_m I}{2\pi a} 2\pi a &= (1 + \chi_m)I\\
  \oint B \cdot dl &= \mu_0 I_{enc}\\
  &= \mu_0 (1 + \chi_m) I\\
  B &= \frac{\mu_0(1 + \chi_m)I}{2\pi s}\hat{\phi}
\end{align*}

\chapter{Punto 9}

Tenemos
\begin{align*}
  \oint H \cdot dl &= H (2\pi s)\\
  &= I_{f_{enc}}\\
  &= \begin{cases}
    I \left(\frac{s^2}{a^2}\right), & (s < a)\\
    I, & (s > a)
  \end{cases}\\
  H &= \begin{cases}
    \frac{Is}{2\pi a^2}, & (s < a)\\
    \frac{I}{2\pi s}, & (s > a)
  \end{cases}
\end{align*}

Ahora por lo tanto
\begin{align*}
  B &= \mu H\\
  &= \begin{cases}
    \frac{\mu_0(1 + \chi_m)Is}{2\pi a^2}, & (s < a)\\
    \frac{\mu_0 I}{2\pi s}, & (s > a)
  \end{cases}
\end{align*}

Para el $J_b$, $K_b$ y $I_b$

\begin{align*}
  J_b &= \chi_m J_f\\
  J_f &= \frac{I}{\pi a^2}\\
  J_b &= \frac{\chi_m I}{\pi a^2}\\
  K_b &= M \times \hat{n} = \chi_m H \times \hat{n}\\
  &= \frac{\chi_m I}{2\pi a}\\
  I_b &= J_b(\pi a^2) + K_b(2\pi a)\\
  &= \chi_m I - \chi_m I\\
  &= 0
\end{align*}

\chapter{Punto 11}

\section{}

\begin{align*}
  B_1 &= \frac{\mu_0}{4\pi} \frac{2m}{z^3}\hat{z}\\
  m_2\cdot B_1 &= - \frac{\mu_0}{2\pi}\frac{m^2}{z^3}\\
  F &= \nabla (m\cdot B)\\
  &= \frac{\partial}{\partial z}\left[ - \frac{\mu_0}{2\pi}\frac{m^2}{z^3} \right] \hat{z}\\
  &= \frac{3\mu_0 m^2}{2\pi z^4}\hat{z}
\end{align*}

Esta fuerza tiene que ser igual a la fuerza gravitacional para cancelarce por lo tanto
\begin{align*}
  \frac{3\mu_0 m^2}{2\pi z^4} - m_d g &= 0\\
  \frac{3\mu_0 m^2}{2\pi z^4} &= m_d g\\
  \frac{1}{2\pi z^4} &= \frac{m_d g}{3\mu_0 m^2}\\
  \frac{1}{z^4} &= \frac{2\pi m_d g}{3\mu_0 m^2}\\
  z^4 &= \frac{3\mu_0 m^2}{2\pi m_d g}\\
  z &= \left(\frac{3\mu_0 m^2}{2\pi m_d g}\right)^{\frac{1}{4}}\\
\end{align*}

\section{}

Agregando un iman entonces el iman en el medio siente dos fuerzas, una hacia arriba y una hacia abajo quedando
\begin{align*}
  \frac{3\mu_0 m^2}{2\pi x^4} -\frac{3\mu_0 m^2}{2\pi y^4} - m_d g &= 0\\
\end{align*}

Ahora bien, el iman de arriba es repelido por el iman del medio y atraido por el de abajo lo que queda:

\begin{align*}
  \frac{3\mu_0 m^2}{2\pi y^4} -\frac{3\mu_0 m^2}{2\pi (x + y)^4} - m_d g &= 0\\
\end{align*}

Ahora si sustraemos todo tenemos
\begin{align*}
  \frac{3\mu_0 m^2}{2\pi x^4} -\frac{3\mu_0 m^2}{2\pi y^4} - m_d g  - \left(\frac{3\mu_0 m^2}{2\pi y^4} -\frac{3\mu_0 m^2}{2\pi (x + y)^4} - m_d g\right) &= 0\\
  \frac{3\mu_0 m^2}{2\pi x^4} -\frac{3\mu_0 m^2}{2\pi y^4} - m_d g  - \frac{3\mu_0 m^2}{2\pi y^4} + \frac{3\mu_0 m^2}{2\pi (x + y)^4} + m_d g &= 0\\
  \left(\frac{3\mu_0 m^2}{2\pi x^4} -\frac{3\mu_0 m^2}{2\pi y^4}   - \frac{3\mu_0 m^2}{2\pi y^4} + \frac{3\mu_0 m^2}{2\pi (x + y)^4}\right) + m_d g - m_d g &= 0\\
  \frac{3\mu_0 m^2}{2\pi} \left(\frac{1}{x^4} -\frac{1}{y^4}   - \frac{1}{y^4} + \frac{1}{(x + y)^4}\right) &= 0\\
  \frac{1}{x^4} - \frac{2}{y^4} + \frac{1}{(x + y)^4} &= 0\\
  \left(\frac{1}{x^4} - \frac{2}{y^4} + \frac{1}{(x + y)^4}\right)y^{4} &= 0\\
  \frac{1}{\left(\frac{x}{y}\right)^4} - 2 + \frac{1}{\left(\frac{x}{y} + 1\right)^4} &= 0\\
  \frac{1}{\left(\frac{x}{y}\right)^4} + \frac{1}{\left(\frac{x}{y} + 1\right)^4} &= 2\\
  a &= \frac{x}{y}\\
  \frac{1}{a^4} + \frac{1}{\left(a + 1\right)^4} &= 2
\end{align*}

Ahora podemos meter esto en sympy de la siguiente manera

\lstinputlisting[language=Python]{./code/punto_11.py}

Esto nos da como resultado:

\begin{align*}
  a &= \\
  &-1.85011497953762\\
  &0.850114979537622\\
  &-1.01324463844346 - 0.809817345066314 i\\
  &-1.01324463844346 + 0.809817345066314 i\\
  &-0.5 - 0.195044382162859 i\\
  &-0.5 + 0.195044382162859 i\\
  &0.0132446384434638 - 0.809817345066314 i\\
  &0.0132446384434638 + 0.809817345066314 i
\end{align*}

En donde el unico que nos sirve es $$\frac{x}{y} = 0.850114979537622$$

\end{document}

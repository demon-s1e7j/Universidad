\documentclass{report}

\documentclass[12pt]{article}
\usepackage{array}
\usepackage{color}
\usepackage{amsthm}
\usepackage{eufrak}
\usepackage{lipsum}
\usepackage{pifont}
\usepackage{yfonts}
\usepackage{amsmath}
\usepackage{amssymb}
\usepackage{ccfonts}
\usepackage{comment} \usepackage{amsfonts}
\usepackage{fancyhdr}
\usepackage{graphicx}
\usepackage{listings}
\usepackage{mathrsfs}
\usepackage{setspace}
\usepackage{textcomp}
\usepackage{blindtext}
\usepackage{enumerate}
\usepackage{microtype}
\usepackage{xfakebold}
\usepackage{kantlipsum}
%\usepackage{draftwatermark}
\usepackage[spanish]{babel}
\usepackage[margin=1.5cm, top=2cm, bottom=2cm]{geometry}
\usepackage[framemethod=tikz]{mdframed}
\usepackage[colorlinks=true,citecolor=blue,linkcolor=red,urlcolor=magenta]{hyperref}

%//////////////////////////////////////////////////////
% Watermark configuration
%//////////////////////////////////////////////////////
%\SetWatermarkScale{4}
%\SetWatermarkColor{black}
%\SetWatermarkLightness{0.95}
%\SetWatermarkText{\texttt{Watermark}}

%//////////////////////////////////////////////////////
% Frame configuration
%//////////////////////////////////////////////////////
\newmdenv[tikzsetting={draw=gray,fill=white,fill opacity=0},backgroundcolor=none]{Frame}

%//////////////////////////////////////////////////////
% Font style configuration
%//////////////////////////////////////////////////////
\renewcommand{\familydefault}{\ttdefault}
\renewcommand{\rmdefault}{tt}

%//////////////////////////////////////////////////////
% Bold configuration
%//////////////////////////////////////////////////////
\newcommand{\fbseries}{\unskip\setBold\aftergroup\unsetBold\aftergroup\ignorespaces}
\makeatletter
\newcommand{\setBoldness}[1]{\def\fake@bold{#1}}
\makeatother

%//////////////////////////////////////////////////////
% Default font configuration
%//////////////////////////////////////////////////////
\DeclareFontFamily{\encodingdefault}{\ttdefault}{%
  \hyphenchar\font=\defaulthyphenchar
  \fontdimen2\font=0.33333em
  \fontdimen3\font=0.16667em
  \fontdimen4\font=0.11111em
  \fontdimen7\font=0.11111em}


%From M275 "Topology" at SJSU
\newcommand{\id}{\mathrm{id}}
\newcommand{\taking}[1]{\xrightarrow{#1}}
\newcommand{\inv}{^{-1}}

%From M170 "Introduction to Graph Theory" at SJSU
\DeclareMathOperator{\diam}{diam}
\DeclareMathOperator{\ord}{ord}
\newcommand{\defeq}{\overset{\mathrm{def}}{=}}

%From the USAMO .tex files
\newcommand{\ts}{\textsuperscript}
\newcommand{\dg}{^\circ}
\newcommand{\ii}{\item}

% % From Math 55 and Math 145 at Harvard
% \newenvironment{subproof}[1][Proof]{%
% \begin{proof}[#1] \renewcommand{\qedsymbol}{$\blacksquare$}}%
% {\end{proof}}

\newcommand{\liff}{\leftrightarrow}
\newcommand{\lthen}{\rightarrow}
\newcommand{\opname}{\operatorname}
\newcommand{\surjto}{\twoheadrightarrow}
\newcommand{\injto}{\hookrightarrow}
\newcommand{\On}{\mathrm{On}} % ordinals
\DeclareMathOperator{\img}{im} % Image
\DeclareMathOperator{\Img}{Im} % Image
\DeclareMathOperator{\coker}{coker} % Cokernel
\DeclareMathOperator{\Coker}{Coker} % Cokernel
\DeclareMathOperator{\Ker}{Ker} % Kernel
\DeclareMathOperator{\rank}{rank}
\DeclareMathOperator{\Spec}{Spec} % spectrum
\DeclareMathOperator{\Tr}{Tr} % trace
\DeclareMathOperator{\pr}{pr} % projection
\DeclareMathOperator{\ext}{ext} % extension
\DeclareMathOperator{\pred}{pred} % predecessor
\DeclareMathOperator{\dom}{dom} % domain
\DeclareMathOperator{\ran}{ran} % range
\DeclareMathOperator{\Hom}{Hom} % homomorphism
\DeclareMathOperator{\Mor}{Mor} % morphisms
\DeclareMathOperator{\End}{End} % endomorphism

\newcommand{\eps}{\epsilon}
\newcommand{\veps}{\varepsilon}
\newcommand{\ol}{\overline}
\newcommand{\ul}{\underline}
\newcommand{\wt}{\widetilde}
\newcommand{\wh}{\widehat}
\newcommand{\vocab}[1]{\textbf{\color{blue} #1}}
\providecommand{\half}{\frac{1}{2}}
\newcommand{\dang}{\measuredangle} %% Directed angle
\newcommand{\ray}[1]{\overrightarrow{#1}}
\newcommand{\seg}[1]{\overline{#1}}
\newcommand{\arc}[1]{\wideparen{#1}}
\DeclareMathOperator{\cis}{cis}
\DeclareMathOperator*{\lcm}{lcm}
\DeclareMathOperator*{\argmin}{arg min}
\DeclareMathOperator*{\argmax}{arg max}
\newcommand{\cycsum}{\sum_{\mathrm{cyc}}}
\newcommand{\symsum}{\sum_{\mathrm{sym}}}
\newcommand{\cycprod}{\prod_{\mathrm{cyc}}}
\newcommand{\symprod}{\prod_{\mathrm{sym}}}
\newcommand{\Qed}{\begin{flushright}\qed\end{flushright}}
\newcommand{\parinn}{\setlength{\parindent}{1cm}}
\newcommand{\parinf}{\setlength{\parindent}{0cm}}
% \newcommand{\norm}{\|\cdot\|}
\newcommand{\inorm}{\norm_{\infty}}
\newcommand{\opensets}{\{V_{\alpha}\}_{\alpha\in I}}
\newcommand{\oset}{V_{\alpha}}
\newcommand{\opset}[1]{V_{\alpha_{#1}}}
\newcommand{\lub}{\text{lub}}
\newcommand{\del}[2]{\frac{\partial #1}{\partial #2}}
\newcommand{\Del}[3]{\frac{\partial^{#1} #2}{\partial^{#1} #3}}
\newcommand{\deld}[2]{\dfrac{\partial #1}{\partial #2}}
\newcommand{\Deld}[3]{\dfrac{\partial^{#1} #2}{\partial^{#1} #3}}
\newcommand{\lm}{\lambda}
\newcommand{\uin}{\mathbin{\rotatebox[origin=c]{90}{$\in$}}}
\newcommand{\usubset}{\mathbin{\rotatebox[origin=c]{90}{$\subset$}}}
\newcommand{\lt}{\left}
\newcommand{\rt}{\right}
\newcommand{\paren}[1]{\left(#1\right)}
\newcommand{\bs}[1]{\boldsymbol{#1}}
\newcommand{\exs}{\exists}
\newcommand{\st}{\strut}
\newcommand{\dps}[1]{\displaystyle{#1}}

\newcommand{\sol}{\setlength{\parindent}{0cm}\textbf{\textit{Solution:}}\setlength{\parindent}{1cm} }
\newcommand{\solve}[1]{\setlength{\parindent}{0cm}\textbf{\textit{Solution: }}\setlength{\parindent}{1cm}#1 \Qed}

% Things Lie
\newcommand{\kb}{\mathfrak b}
\newcommand{\kg}{\mathfrak g}
\newcommand{\kh}{\mathfrak h}
\newcommand{\kn}{\mathfrak n}
\newcommand{\ku}{\mathfrak u}
\newcommand{\kz}{\mathfrak z}
\DeclareMathOperator{\Ext}{Ext} % Ext functor
\DeclareMathOperator{\Tor}{Tor} % Tor functor
\newcommand{\gl}{\opname{\mathfrak{gl}}} % frak gl group
\renewcommand{\sl}{\opname{\mathfrak{sl}}} % frak sl group chktex 6

% More script letters etc.
\newcommand{\SA}{\mathcal A}
\newcommand{\SB}{\mathcal B}
\newcommand{\SC}{\mathcal C}
\newcommand{\SF}{\mathcal F}
\newcommand{\SG}{\mathcal G}
\newcommand{\SH}{\mathcal H}
\newcommand{\OO}{\mathcal O}

\newcommand{\SCA}{\mathscr A}
\newcommand{\SCB}{\mathscr B}
\newcommand{\SCC}{\mathscr C}
\newcommand{\SCD}{\mathscr D}
\newcommand{\SCE}{\mathscr E}
\newcommand{\SCF}{\mathscr F}
\newcommand{\SCG}{\mathscr G}
\newcommand{\SCH}{\mathscr H}

% Mathfrak primes
\newcommand{\km}{\mathfrak m}
\newcommand{\kp}{\mathfrak p}
\newcommand{\kq}{\mathfrak q}

% number sets
\newcommand{\RR}[1][]{\ensuremath{\ifstrempty{#1}{\mathbb{R}}{\mathbb{R}^{#1}}}}
\newcommand{\NN}[1][]{\ensuremath{\ifstrempty{#1}{\mathbb{N}}{\mathbb{N}^{#1}}}}
\newcommand{\ZZ}[1][]{\ensuremath{\ifstrempty{#1}{\mathbb{Z}}{\mathbb{Z}^{#1}}}}
\newcommand{\QQ}[1][]{\ensuremath{\ifstrempty{#1}{\mathbb{Q}}{\mathbb{Q}^{#1}}}}
\newcommand{\CC}[1][]{\ensuremath{\ifstrempty{#1}{\mathbb{C}}{\mathbb{C}^{#1}}}}
\newcommand{\PP}[1][]{\ensuremath{\ifstrempty{#1}{\mathbb{P}}{\mathbb{P}^{#1}}}}
\newcommand{\HH}[1][]{\ensuremath{\ifstrempty{#1}{\mathbb{H}}{\mathbb{H}^{#1}}}}
\newcommand{\FF}[1][]{\ensuremath{\ifstrempty{#1}{\mathbb{F}}{\mathbb{F}^{#1}}}}
% expected value
\newcommand{\EE}{\ensuremath{\mathbb{E}}}
\newcommand{\charin}{\text{ char }}
\DeclareMathOperator{\sign}{sign}
\DeclareMathOperator{\Aut}{Aut}
\DeclareMathOperator{\Inn}{Inn}
\DeclareMathOperator{\Syl}{Syl}
\DeclareMathOperator{\Gal}{Gal}
\DeclareMathOperator{\GL}{GL} % General linear group
\DeclareMathOperator{\SL}{SL} % Special linear group

%---------------------------------------
% BlackBoard Math Fonts :-
%---------------------------------------

%Captital Letters
\newcommand{\bbA}{\mathbb{A}}	\newcommand{\bbB}{\mathbb{B}}
\newcommand{\bbC}{\mathbb{C}}	\newcommand{\bbD}{\mathbb{D}}
\newcommand{\bbE}{\mathbb{E}}	\newcommand{\bbF}{\mathbb{F}}
\newcommand{\bbG}{\mathbb{G}}	\newcommand{\bbH}{\mathbb{H}}
\newcommand{\bbI}{\mathbb{I}}	\newcommand{\bbJ}{\mathbb{J}}
\newcommand{\bbK}{\mathbb{K}}	\newcommand{\bbL}{\mathbb{L}}
\newcommand{\bbM}{\mathbb{M}}	\newcommand{\bbN}{\mathbb{N}}
\newcommand{\bbO}{\mathbb{O}}	\newcommand{\bbP}{\mathbb{P}}
\newcommand{\bbQ}{\mathbb{Q}}	\newcommand{\bbR}{\mathbb{R}}
\newcommand{\bbS}{\mathbb{S}}	\newcommand{\bbT}{\mathbb{T}}
\newcommand{\bbU}{\mathbb{U}}	\newcommand{\bbV}{\mathbb{V}}
\newcommand{\bbW}{\mathbb{W}}	\newcommand{\bbX}{\mathbb{X}}
\newcommand{\bbY}{\mathbb{Y}}	\newcommand{\bbZ}{\mathbb{Z}}

%---------------------------------------
% MathCal Fonts :-
%---------------------------------------

%Captital Letters
\newcommand{\mcA}{\mathcal{A}}	\newcommand{\mcB}{\mathcal{B}}
\newcommand{\mcC}{\mathcal{C}}	\newcommand{\mcD}{\mathcal{D}}
\newcommand{\mcE}{\mathcal{E}}	\newcommand{\mcF}{\mathcal{F}}
\newcommand{\mcG}{\mathcal{G}}	\newcommand{\mcH}{\mathcal{H}}
\newcommand{\mcI}{\mathcal{I}}	\newcommand{\mcJ}{\mathcal{J}}
\newcommand{\mcK}{\mathcal{K}}	\newcommand{\mcL}{\mathcal{L}}
\newcommand{\mcM}{\mathcal{M}}	\newcommand{\mcN}{\mathcal{N}}
\newcommand{\mcO}{\mathcal{O}}	\newcommand{\mcP}{\mathcal{P}}
\newcommand{\mcQ}{\mathcal{Q}}	\newcommand{\mcR}{\mathcal{R}}
\newcommand{\mcS}{\mathcal{S}}	\newcommand{\mcT}{\mathcal{T}}
\newcommand{\mcU}{\mathcal{U}}	\newcommand{\mcV}{\mathcal{V}}
\newcommand{\mcW}{\mathcal{W}}	\newcommand{\mcX}{\mathcal{X}}
\newcommand{\mcY}{\mathcal{Y}}	\newcommand{\mcZ}{\mathcal{Z}}


%---------------------------------------
% Bold Math Fonts :-
%---------------------------------------

%Captital Letters
\newcommand{\bmA}{\boldsymbol{A}}	\newcommand{\bmB}{\boldsymbol{B}}
\newcommand{\bmC}{\boldsymbol{C}}	\newcommand{\bmD}{\boldsymbol{D}}
\newcommand{\bmE}{\boldsymbol{E}}	\newcommand{\bmF}{\boldsymbol{F}}
\newcommand{\bmG}{\boldsymbol{G}}	\newcommand{\bmH}{\boldsymbol{H}}
\newcommand{\bmI}{\boldsymbol{I}}	\newcommand{\bmJ}{\boldsymbol{J}}
\newcommand{\bmK}{\boldsymbol{K}}	\newcommand{\bmL}{\boldsymbol{L}}
\newcommand{\bmM}{\boldsymbol{M}}	\newcommand{\bmN}{\boldsymbol{N}}
\newcommand{\bmO}{\boldsymbol{O}}	\newcommand{\bmP}{\boldsymbol{P}}
\newcommand{\bmQ}{\boldsymbol{Q}}	\newcommand{\bmR}{\boldsymbol{R}}
\newcommand{\bmS}{\boldsymbol{S}}	\newcommand{\bmT}{\boldsymbol{T}}
\newcommand{\bmU}{\boldsymbol{U}}	\newcommand{\bmV}{\boldsymbol{V}}
\newcommand{\bmW}{\boldsymbol{W}}	\newcommand{\bmX}{\boldsymbol{X}}
\newcommand{\bmY}{\boldsymbol{Y}}	\newcommand{\bmZ}{\boldsymbol{Z}}
%Small Letters
\newcommand{\bma}{\boldsymbol{a}}	\newcommand{\bmb}{\boldsymbol{b}}
\newcommand{\bmc}{\boldsymbol{c}}	\newcommand{\bmd}{\boldsymbol{d}}
\newcommand{\bme}{\boldsymbol{e}}	\newcommand{\bmf}{\boldsymbol{f}}
\newcommand{\bmg}{\boldsymbol{g}}	\newcommand{\bmh}{\boldsymbol{h}}
\newcommand{\bmi}{\boldsymbol{i}}	\newcommand{\bmj}{\boldsymbol{j}}
\newcommand{\bmk}{\boldsymbol{k}}	\newcommand{\bml}{\boldsymbol{l}}
\newcommand{\bmm}{\boldsymbol{m}}	\newcommand{\bmn}{\boldsymbol{n}}
\newcommand{\bmo}{\boldsymbol{o}}	\newcommand{\bmp}{\boldsymbol{p}}
\newcommand{\bmq}{\boldsymbol{q}}	\newcommand{\bmr}{\boldsymbol{r}}
\newcommand{\bms}{\boldsymbol{s}}	\newcommand{\bmt}{\boldsymbol{t}}
\newcommand{\bmu}{\boldsymbol{u}}	\newcommand{\bmv}{\boldsymbol{v}}
\newcommand{\bmw}{\boldsymbol{w}}	\newcommand{\bmx}{\boldsymbol{x}}
\newcommand{\bmy}{\boldsymbol{y}}	\newcommand{\bmz}{\boldsymbol{z}}

%---------------------------------------
% Scr Math Fonts :-
%---------------------------------------

\newcommand{\sA}{{\mathscr{A}}}   \newcommand{\sB}{{\mathscr{B}}}
\newcommand{\sC}{{\mathscr{C}}}   \newcommand{\sD}{{\mathscr{D}}}
\newcommand{\sE}{{\mathscr{E}}}   \newcommand{\sF}{{\mathscr{F}}}
\newcommand{\sG}{{\mathscr{G}}}   \newcommand{\sH}{{\mathscr{H}}}
\newcommand{\sI}{{\mathscr{I}}}   \newcommand{\sJ}{{\mathscr{J}}}
\newcommand{\sK}{{\mathscr{K}}}   \newcommand{\sL}{{\mathscr{L}}}
\newcommand{\sM}{{\mathscr{M}}}   \newcommand{\sN}{{\mathscr{N}}}
\newcommand{\sO}{{\mathscr{O}}}   \newcommand{\sP}{{\mathscr{P}}}
\newcommand{\sQ}{{\mathscr{Q}}}   \newcommand{\sR}{{\mathscr{R}}}
\newcommand{\sS}{{\mathscr{S}}}   \newcommand{\sT}{{\mathscr{T}}}
\newcommand{\sU}{{\mathscr{U}}}   \newcommand{\sV}{{\mathscr{V}}}
\newcommand{\sW}{{\mathscr{W}}}   \newcommand{\sX}{{\mathscr{X}}}
\newcommand{\sY}{{\mathscr{Y}}}   \newcommand{\sZ}{{\mathscr{Z}}}


%---------------------------------------
% Math Fraktur Font
%---------------------------------------

%Captital Letters
\newcommand{\mfA}{\mathfrak{A}}	\newcommand{\mfB}{\mathfrak{B}}
\newcommand{\mfC}{\mathfrak{C}}	\newcommand{\mfD}{\mathfrak{D}}
\newcommand{\mfE}{\mathfrak{E}}	\newcommand{\mfF}{\mathfrak{F}}
\newcommand{\mfG}{\mathfrak{G}}	\newcommand{\mfH}{\mathfrak{H}}
\newcommand{\mfI}{\mathfrak{I}}	\newcommand{\mfJ}{\mathfrak{J}}
\newcommand{\mfK}{\mathfrak{K}}	\newcommand{\mfL}{\mathfrak{L}}
\newcommand{\mfM}{\mathfrak{M}}	\newcommand{\mfN}{\mathfrak{N}}
\newcommand{\mfO}{\mathfrak{O}}	\newcommand{\mfP}{\mathfrak{P}}
\newcommand{\mfQ}{\mathfrak{Q}}	\newcommand{\mfR}{\mathfrak{R}}
\newcommand{\mfS}{\mathfrak{S}}	\newcommand{\mfT}{\mathfrak{T}}
\newcommand{\mfU}{\mathfrak{U}}	\newcommand{\mfV}{\mathfrak{V}}
\newcommand{\mfW}{\mathfrak{W}}	\newcommand{\mfX}{\mathfrak{X}}
\newcommand{\mfY}{\mathfrak{Y}}	\newcommand{\mfZ}{\mathfrak{Z}}
%Small Letters
\newcommand{\mfa}{\mathfrak{a}}	\newcommand{\mfb}{\mathfrak{b}}
\newcommand{\mfc}{\mathfrak{c}}	\newcommand{\mfd}{\mathfrak{d}}
\newcommand{\mfe}{\mathfrak{e}}	\newcommand{\mff}{\mathfrak{f}}
\newcommand{\mfg}{\mathfrak{g}}	\newcommand{\mfh}{\mathfrak{h}}
\newcommand{\mfi}{\mathfrak{i}}	\newcommand{\mfj}{\mathfrak{j}}
\newcommand{\mfk}{\mathfrak{k}}	\newcommand{\mfl}{\mathfrak{l}}
\newcommand{\mfm}{\mathfrak{m}}	\newcommand{\mfn}{\mathfrak{n}}
\newcommand{\mfo}{\mathfrak{o}}	\newcommand{\mfp}{\mathfrak{p}}
\newcommand{\mfq}{\mathfrak{q}}	\newcommand{\mfr}{\mathfrak{r}}
\newcommand{\mfs}{\mathfrak{s}}	\newcommand{\mft}{\mathfrak{t}}
\newcommand{\mfu}{\mathfrak{u}}	\newcommand{\mfv}{\mathfrak{v}}
\newcommand{\mfw}{\mathfrak{w}}	\newcommand{\mfx}{\mathfrak{x}}
\newcommand{\mfy}{\mathfrak{y}}	\newcommand{\mfz}{\mathfrak{z}}


\usepackage{listings}
\usepackage{xcolor}

\definecolor{codegreen}{rgb}{0,0.6,0}
\definecolor{codegray}{rgb}{0.5,0.5,0.5}
\definecolor{codepurple}{rgb}{0.58,0,0.82}
\definecolor{backcolour}{rgb}{0.95,0.95,0.92}

\lstdefinestyle{mystyle}{
    backgroundcolor=\color{backcolour},   
    commentstyle=\color{codegreen},
    keywordstyle=\color{magenta},
    numberstyle=\tiny\color{codegray},
    stringstyle=\color{codepurple},
    basicstyle=\ttfamily\footnotesize,
    breakatwhitespace=false,         
    breaklines=true,                 
    captionpos=b,                    
    keepspaces=true,                 
    numbers=left,                    
    numbersep=5pt,                  
    showspaces=false,                
    showstringspaces=false,
    showtabs=false,                  
    tabsize=2
}

\lstset{style=mystyle}

\title{\Huge{Some Class}\\Random Examples}
\author{\huge{Your Name}}
\date{}

\begin{document}

\maketitle
\newpage% or \cleardoublepage
% \pdfbookmark[<level>]{<title>}{<dest>}
\pdfbookmark[section]{\contentsname}{toc}
\tableofcontents
\pagebreak

\chapter{Punto 1}

\subsection{}

Tenemos que:
\[
  F = 2\pi I R B\cos\theta
\]

Pero ademas
\begin{align*}
  B &= \frac{\mu_0}{4\pi} \frac{\left[ 3 (m_1\cdot \hat{r})\hat{r} - m_1 \right]}{r^3}\\
  B \cos\theta &= B \cdot \hat{y}\\
  B\cos\theta &=\frac{\mu_0}{4\pi} \frac{1}{r^3} \left[ 3 (m_1\cdot \hat{r})(\hat{r} \cdot \hat{y}) - (m_1\cdot \hat{y}) \right]\\
  m_1 \cdot \hat{y} &= 0\\
  \hat{r}\cdot\hat{y} &= \sin\phi\\
  m_1 \cdot \hat{r} &= m_1 \cos\theta\\
  B\cos\theta &= \frac{\mu_0}{4\pi} \frac{1}{r^3} \left[ 3 m_1\sin\phi \cos\phi \right]\\
\end{align*}

Ahora bien, si reemplazamos en la ecuación previa:
\begin{align*}
  F &= 2 \pi I R \frac{\mu_0}{4\pi} \frac{1}{r^3} \left[ 3 m_1\sin\phi \cos\phi \right]\\
  \sin\phi &= \frac{R}{r}\\
  \cos\phi &= \frac{\sqrt{r^2 - R^2}}{r}\\
  F &= 3 \frac{\mu_0}{2} m_1 I R^2 \frac{\sqrt{r^2 - R^2}}{r^5}\\
  I R^2 \pi &= m_2\\
  F &= 3 \frac{\mu_0}{2\pi} m_1m_2\frac{\sqrt{r^2 - R^2}}{r^5}
\end{align*}

Ahora bien, en el caso de que $R \ll r$

\begin{align*}
  F &= 3 \frac{\mu_0}{2\pi} m_1m_2\frac{\sqrt{r^2}}{r^5}\\ F &= 3 \frac{\mu_0}{2\pi} m_1m_2\frac{r}{r^5}\\
  F &= 3 \frac{\mu_0}{2\pi} \frac{m_1m_2}{r^4}
\end{align*}

\subsection{}

Tenemos que
\begin{align*}
  F &= \nabla (m_2 \cdot B)\\
  &= (m_2 \cdot \nabla)B\\
  &= \left(m_2 \frac{d}{dz} \right)\left[ \frac{\mu_0}{4\pi} \frac{1}{z^3} (3(m_1\cdot\hat{z})\hat{z} - m_1)\right]\\
  2 m_1 &= (3(m_1\cdot\hat{z})\hat{z} - m_1)\\
  &= \left(m_2 \frac{d}{dz} \right)\left[ \frac{\mu_0}{4\pi} \frac{1}{z^3} 2m_1\right]\\
  &= \left(m_2 \frac{d}{dz} \right)\left[ \frac{\mu_0}{2\pi} \frac{1}{z^3} m_1\right]\\
  &= \frac{\mu_0}{2\pi} m_1 m_2 \frac{d}{dz} \left[ \frac{1}{z^3} \right]\\
  \frac{d}{dz} \left[ \frac{1}{z^3} \right] &= - 3 \frac{1}{z^4}\\
  F &= \frac{\mu_0}{2\pi} m_1 m_2 \left( - 3 \frac{1}{z^4} \right)\\
  F &= 3 \frac{\mu_0}{2\pi} \frac{m_1m_2}{r^4}
\end{align*}

\chapter{Punto 3}

\subsection{Esfera Magnetizada}

Tenemos $K_M = M \times \hat{n}$ que teniendo un vector normal $\hat{n} = \hat{r}$ nos quedamos con:
\[
  K_M = M \times \hat{r}
\]

\section{Cascaron}

Ahora, teniendo un cascaron con carga superficial tenemos
\begin{align*}
  K_\sigma &= \sigma v\\
  &= \sigma (\omega \times r)
  &= \sigma (\omega \times R\hat{r})
  &= R\sigma (\omega \times \hat{r})
\end{align*}

\section{Igualdad}

Tenemos que mostrar:
\begin{align*}
  K_M &= K_\sigma\\
  M \times \hat{r} &= \sigma R (\omega \times \hat{r})\\
  M &= \sigma R \omega
\end{align*}

Con esto entonces mostramos que si se cumple esta condición estos tendran una densidad de corriente igual.

\chapter{Punto 5}

Tenemos
\begin{align*}
  \nabla \times M &= J_b\\
  &= \frac{1}{s}\frac{\partial}{\partial s}(s k s^2)\hat{z}\\
  &= \frac{1}{s}(3ks^2)\hat{z}\\
  &= 3ks\hat{z}
\end{align*}

Ademas, tenemos
\begin{align*}
  K_b &= M \times \hat{n}\\
  &= k s^2 (\hat{\phi} \times \hat{s})\\
  &= -kR^2\hat{z}
\end{align*}

Por lo tanto, la corriente fluye encima del cilindro y luego vuelve a bajar a la superficie. Ahora, dado que las corrientes son simetricas en la simetria  podemos conseguir el campo con la ley de Ampere.

\begin{align*}
  B \cdot 2\pi s &= \mu_0 I_{enc}\\
  &= \mu_0 \int_0^s J_b da\\
  &= 2\pi k \mu_0 s^3\\
  B &= \mu_0 k s^2 \hat{\phi}\\
  &= \mu_0 M
\end{align*}

Fuera del cilindro $I_{enc} = 0 \implies B = 0$

\chapter{Punto 8}

Iniciamos con
\begin{align*}
  \oint H \cdot d l &= I_{f_{enc}}\\
  &= I\\
  \implies H &= \frac{I}{2\pi s}\hat{\phi}\\
  B &= \mu_0 (1 + \chi_m)H\\
  &= \mu_0 (1 + \chi_m) \frac{I}{2\pi s}\hat{\phi}\\
  M &= \chi_m H\\
  &= \frac{\chi_m I}{2\pi s}\hat{\phi}\\
  J_b &= \nabla \times M\\
  &= \frac{1}{s}\frac{\partial}{\partial s}\left(s \frac{\chi_m I}{2\pi s} \right)\hat{z}\\
  &= 0\\
  K_b &= M \times \hat{n}\\
  &= \begin{cases}
    \frac{\chi_m I}{2\pi a}\hat{z} & s = a\\
    -\frac{\chi_m I}{2\pi b}\hat{z} & r = b
  \end{cases}
\end{align*}

La corriente total de un loop entre los cilindros:

\begin{align*}
  I + \frac{\chi_m I}{2\pi a} 2\pi a &= (1 + \chi_m)I\\
  \oint B \cdot dl &= \mu_0 I_{enc}\\
  &= \mu_0 (1 + \chi_m) I\\
  B &= \frac{\mu_0(1 + \chi_m)I}{2\pi s}\hat{\phi}
\end{align*}

\chapter{Punto 9}

Tenemos
\begin{align*}
  \oint H \cdot dl &= H (2\pi s)\\
  &= I_{f_{enc}}\\
  &= \begin{cases}
    I \left(\frac{s^2}{a^2}\right), & (s < a)\\
    I, & (s > a)
  \end{cases}\\
  H &= \begin{cases}
    \frac{Is}{2\pi a^2}, & (s < a)\\
    \frac{I}{2\pi s}, & (s > a)
  \end{cases}
\end{align*}

Ahora por lo tanto
\begin{align*}
  B &= \mu H\\
  &= \begin{cases}
    \frac{\mu_0(1 + \chi_m)Is}{2\pi a^2}, & (s < a)\\
    \frac{\mu_0 I}{2\pi s}, & (s > a)
  \end{cases}
\end{align*}

Para el $J_b$, $K_b$ y $I_b$

\begin{align*}
  J_b &= \chi_m J_f\\
  J_f &= \frac{I}{\pi a^2}\\
  J_b &= \frac{\chi_m I}{\pi a^2}\\
  K_b &= M \times \hat{n} = \chi_m H \times \hat{n}\\
  &= \frac{\chi_m I}{2\pi a}\\
  I_b &= J_b(\pi a^2) + K_b(2\pi a)\\
  &= \chi_m I - \chi_m I\\
  &= 0
\end{align*}

\chapter{Punto 11}

\section{}

\begin{align*}
  B_1 &= \frac{\mu_0}{4\pi} \frac{2m}{z^3}\hat{z}\\
  m_2\cdot B_1 &= - \frac{\mu_0}{2\pi}\frac{m^2}{z^3}\\
  F &= \nabla (m\cdot B)\\
  &= \frac{\partial}{\partial z}\left[ - \frac{\mu_0}{2\pi}\frac{m^2}{z^3} \right] \hat{z}\\
  &= \frac{3\mu_0 m^2}{2\pi z^4}\hat{z}
\end{align*}

Esta fuerza tiene que ser igual a la fuerza gravitacional para cancelarce por lo tanto
\begin{align*}
  \frac{3\mu_0 m^2}{2\pi z^4} - m_d g &= 0\\
  \frac{3\mu_0 m^2}{2\pi z^4} &= m_d g\\
  \frac{1}{2\pi z^4} &= \frac{m_d g}{3\mu_0 m^2}\\
  \frac{1}{z^4} &= \frac{2\pi m_d g}{3\mu_0 m^2}\\
  z^4 &= \frac{3\mu_0 m^2}{2\pi m_d g}\\
  z &= \left(\frac{3\mu_0 m^2}{2\pi m_d g}\right)^{\frac{1}{4}}\\
\end{align*}

\section{}

Agregando un iman entonces el iman en el medio siente dos fuerzas, una hacia arriba y una hacia abajo quedando
\begin{align*}
  \frac{3\mu_0 m^2}{2\pi x^4} -\frac{3\mu_0 m^2}{2\pi y^4} - m_d g &= 0\\
\end{align*}

Ahora bien, el iman de arriba es repelido por el iman del medio y atraido por el de abajo lo que queda:

\begin{align*}
  \frac{3\mu_0 m^2}{2\pi y^4} -\frac{3\mu_0 m^2}{2\pi (x + y)^4} - m_d g &= 0\\
\end{align*}

Ahora si sustraemos todo tenemos
\begin{align*}
  \frac{3\mu_0 m^2}{2\pi x^4} -\frac{3\mu_0 m^2}{2\pi y^4} - m_d g  - \left(\frac{3\mu_0 m^2}{2\pi y^4} -\frac{3\mu_0 m^2}{2\pi (x + y)^4} - m_d g\right) &= 0\\
  \frac{3\mu_0 m^2}{2\pi x^4} -\frac{3\mu_0 m^2}{2\pi y^4} - m_d g  - \frac{3\mu_0 m^2}{2\pi y^4} + \frac{3\mu_0 m^2}{2\pi (x + y)^4} + m_d g &= 0\\
  \left(\frac{3\mu_0 m^2}{2\pi x^4} -\frac{3\mu_0 m^2}{2\pi y^4}   - \frac{3\mu_0 m^2}{2\pi y^4} + \frac{3\mu_0 m^2}{2\pi (x + y)^4}\right) + m_d g - m_d g &= 0\\
  \frac{3\mu_0 m^2}{2\pi} \left(\frac{1}{x^4} -\frac{1}{y^4}   - \frac{1}{y^4} + \frac{1}{(x + y)^4}\right) &= 0\\
  \frac{1}{x^4} - \frac{2}{y^4} + \frac{1}{(x + y)^4} &= 0\\
  \left(\frac{1}{x^4} - \frac{2}{y^4} + \frac{1}{(x + y)^4}\right)y^{4} &= 0\\
  \frac{1}{\left(\frac{x}{y}\right)^4} - 2 + \frac{1}{\left(\frac{x}{y} + 1\right)^4} &= 0\\
  \frac{1}{\left(\frac{x}{y}\right)^4} + \frac{1}{\left(\frac{x}{y} + 1\right)^4} &= 2\\
  a &= \frac{x}{y}\\
  \frac{1}{a^4} + \frac{1}{\left(a + 1\right)^4} &= 2\\
\end{align*}

Ahora podemos meter esto en sympy de la siguiente manera

\lstinputlisting[language=Python]{./code/punto_11.py}

Esto nos da como resultado:

\begin{align*}
  a &= \\
  &-1.85011497953762\\
  &0.850114979537622\\
  &-1.01324463844346 - 0.809817345066314 i\\
  &-1.01324463844346 + 0.809817345066314 i\\
  &-0.5 - 0.195044382162859 i\\
  &-0.5 + 0.195044382162859 i\\
  &0.0132446384434638 - 0.809817345066314 i\\
  &0.0132446384434638 + 0.809817345066314 i\\
\end{align*}

En donde el unico que nos sirve es $$\frac{x}{y} = 0.850114979537622$$

\end{document}

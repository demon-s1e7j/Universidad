\documentclass{report}

\usepackage[spanish]{babel}

\NewDocumentCommand{\paren}{m}{\left(#1\right)}
\NewDocumentCommand{\inangle}{m}{\left<#1\right>}

\documentclass[12pt]{article}
\usepackage{array}
\usepackage{color}
\usepackage{amsthm}
\usepackage{eufrak}
\usepackage{lipsum}
\usepackage{pifont}
\usepackage{yfonts}
\usepackage{amsmath}
\usepackage{amssymb}
\usepackage{ccfonts}
\usepackage{comment} \usepackage{amsfonts}
\usepackage{fancyhdr}
\usepackage{graphicx}
\usepackage{listings}
\usepackage{mathrsfs}
\usepackage{setspace}
\usepackage{textcomp}
\usepackage{blindtext}
\usepackage{enumerate}
\usepackage{microtype}
\usepackage{xfakebold}
\usepackage{kantlipsum}
%\usepackage{draftwatermark}
\usepackage[spanish]{babel}
\usepackage[margin=1.5cm, top=2cm, bottom=2cm]{geometry}
\usepackage[framemethod=tikz]{mdframed}
\usepackage[colorlinks=true,citecolor=blue,linkcolor=red,urlcolor=magenta]{hyperref}

%//////////////////////////////////////////////////////
% Watermark configuration
%//////////////////////////////////////////////////////
%\SetWatermarkScale{4}
%\SetWatermarkColor{black}
%\SetWatermarkLightness{0.95}
%\SetWatermarkText{\texttt{Watermark}}

%//////////////////////////////////////////////////////
% Frame configuration
%//////////////////////////////////////////////////////
\newmdenv[tikzsetting={draw=gray,fill=white,fill opacity=0},backgroundcolor=none]{Frame}

%//////////////////////////////////////////////////////
% Font style configuration
%//////////////////////////////////////////////////////
\renewcommand{\familydefault}{\ttdefault}
\renewcommand{\rmdefault}{tt}

%//////////////////////////////////////////////////////
% Bold configuration
%//////////////////////////////////////////////////////
\newcommand{\fbseries}{\unskip\setBold\aftergroup\unsetBold\aftergroup\ignorespaces}
\makeatletter
\newcommand{\setBoldness}[1]{\def\fake@bold{#1}}
\makeatother

%//////////////////////////////////////////////////////
% Default font configuration
%//////////////////////////////////////////////////////
\DeclareFontFamily{\encodingdefault}{\ttdefault}{%
  \hyphenchar\font=\defaulthyphenchar
  \fontdimen2\font=0.33333em
  \fontdimen3\font=0.16667em
  \fontdimen4\font=0.11111em
  \fontdimen7\font=0.11111em}


%From M275 "Topology" at SJSU
\newcommand{\id}{\mathrm{id}}
\newcommand{\taking}[1]{\xrightarrow{#1}}
\newcommand{\inv}{^{-1}}

%From M170 "Introduction to Graph Theory" at SJSU
\DeclareMathOperator{\diam}{diam}
\DeclareMathOperator{\ord}{ord}
\newcommand{\defeq}{\overset{\mathrm{def}}{=}}

%From the USAMO .tex files
\newcommand{\ts}{\textsuperscript}
\newcommand{\dg}{^\circ}
\newcommand{\ii}{\item}

% % From Math 55 and Math 145 at Harvard
% \newenvironment{subproof}[1][Proof]{%
% \begin{proof}[#1] \renewcommand{\qedsymbol}{$\blacksquare$}}%
% {\end{proof}}

\newcommand{\liff}{\leftrightarrow}
\newcommand{\lthen}{\rightarrow}
\newcommand{\opname}{\operatorname}
\newcommand{\surjto}{\twoheadrightarrow}
\newcommand{\injto}{\hookrightarrow}
\newcommand{\On}{\mathrm{On}} % ordinals
\DeclareMathOperator{\img}{im} % Image
\DeclareMathOperator{\Img}{Im} % Image
\DeclareMathOperator{\coker}{coker} % Cokernel
\DeclareMathOperator{\Coker}{Coker} % Cokernel
\DeclareMathOperator{\Ker}{Ker} % Kernel
\DeclareMathOperator{\rank}{rank}
\DeclareMathOperator{\Spec}{Spec} % spectrum
\DeclareMathOperator{\Tr}{Tr} % trace
\DeclareMathOperator{\pr}{pr} % projection
\DeclareMathOperator{\ext}{ext} % extension
\DeclareMathOperator{\pred}{pred} % predecessor
\DeclareMathOperator{\dom}{dom} % domain
\DeclareMathOperator{\ran}{ran} % range
\DeclareMathOperator{\Hom}{Hom} % homomorphism
\DeclareMathOperator{\Mor}{Mor} % morphisms
\DeclareMathOperator{\End}{End} % endomorphism

\newcommand{\eps}{\epsilon}
\newcommand{\veps}{\varepsilon}
\newcommand{\ol}{\overline}
\newcommand{\ul}{\underline}
\newcommand{\wt}{\widetilde}
\newcommand{\wh}{\widehat}
\newcommand{\vocab}[1]{\textbf{\color{blue} #1}}
\providecommand{\half}{\frac{1}{2}}
\newcommand{\dang}{\measuredangle} %% Directed angle
\newcommand{\ray}[1]{\overrightarrow{#1}}
\newcommand{\seg}[1]{\overline{#1}}
\newcommand{\arc}[1]{\wideparen{#1}}
\DeclareMathOperator{\cis}{cis}
\DeclareMathOperator*{\lcm}{lcm}
\DeclareMathOperator*{\argmin}{arg min}
\DeclareMathOperator*{\argmax}{arg max}
\newcommand{\cycsum}{\sum_{\mathrm{cyc}}}
\newcommand{\symsum}{\sum_{\mathrm{sym}}}
\newcommand{\cycprod}{\prod_{\mathrm{cyc}}}
\newcommand{\symprod}{\prod_{\mathrm{sym}}}
\newcommand{\Qed}{\begin{flushright}\qed\end{flushright}}
\newcommand{\parinn}{\setlength{\parindent}{1cm}}
\newcommand{\parinf}{\setlength{\parindent}{0cm}}
% \newcommand{\norm}{\|\cdot\|}
\newcommand{\inorm}{\norm_{\infty}}
\newcommand{\opensets}{\{V_{\alpha}\}_{\alpha\in I}}
\newcommand{\oset}{V_{\alpha}}
\newcommand{\opset}[1]{V_{\alpha_{#1}}}
\newcommand{\lub}{\text{lub}}
\newcommand{\del}[2]{\frac{\partial #1}{\partial #2}}
\newcommand{\Del}[3]{\frac{\partial^{#1} #2}{\partial^{#1} #3}}
\newcommand{\deld}[2]{\dfrac{\partial #1}{\partial #2}}
\newcommand{\Deld}[3]{\dfrac{\partial^{#1} #2}{\partial^{#1} #3}}
\newcommand{\lm}{\lambda}
\newcommand{\uin}{\mathbin{\rotatebox[origin=c]{90}{$\in$}}}
\newcommand{\usubset}{\mathbin{\rotatebox[origin=c]{90}{$\subset$}}}
\newcommand{\lt}{\left}
\newcommand{\rt}{\right}
\newcommand{\paren}[1]{\left(#1\right)}
\newcommand{\bs}[1]{\boldsymbol{#1}}
\newcommand{\exs}{\exists}
\newcommand{\st}{\strut}
\newcommand{\dps}[1]{\displaystyle{#1}}

\newcommand{\sol}{\setlength{\parindent}{0cm}\textbf{\textit{Solution:}}\setlength{\parindent}{1cm} }
\newcommand{\solve}[1]{\setlength{\parindent}{0cm}\textbf{\textit{Solution: }}\setlength{\parindent}{1cm}#1 \Qed}

% Things Lie
\newcommand{\kb}{\mathfrak b}
\newcommand{\kg}{\mathfrak g}
\newcommand{\kh}{\mathfrak h}
\newcommand{\kn}{\mathfrak n}
\newcommand{\ku}{\mathfrak u}
\newcommand{\kz}{\mathfrak z}
\DeclareMathOperator{\Ext}{Ext} % Ext functor
\DeclareMathOperator{\Tor}{Tor} % Tor functor
\newcommand{\gl}{\opname{\mathfrak{gl}}} % frak gl group
\renewcommand{\sl}{\opname{\mathfrak{sl}}} % frak sl group chktex 6

% More script letters etc.
\newcommand{\SA}{\mathcal A}
\newcommand{\SB}{\mathcal B}
\newcommand{\SC}{\mathcal C}
\newcommand{\SF}{\mathcal F}
\newcommand{\SG}{\mathcal G}
\newcommand{\SH}{\mathcal H}
\newcommand{\OO}{\mathcal O}

\newcommand{\SCA}{\mathscr A}
\newcommand{\SCB}{\mathscr B}
\newcommand{\SCC}{\mathscr C}
\newcommand{\SCD}{\mathscr D}
\newcommand{\SCE}{\mathscr E}
\newcommand{\SCF}{\mathscr F}
\newcommand{\SCG}{\mathscr G}
\newcommand{\SCH}{\mathscr H}

% Mathfrak primes
\newcommand{\km}{\mathfrak m}
\newcommand{\kp}{\mathfrak p}
\newcommand{\kq}{\mathfrak q}

% number sets
\newcommand{\RR}[1][]{\ensuremath{\ifstrempty{#1}{\mathbb{R}}{\mathbb{R}^{#1}}}}
\newcommand{\NN}[1][]{\ensuremath{\ifstrempty{#1}{\mathbb{N}}{\mathbb{N}^{#1}}}}
\newcommand{\ZZ}[1][]{\ensuremath{\ifstrempty{#1}{\mathbb{Z}}{\mathbb{Z}^{#1}}}}
\newcommand{\QQ}[1][]{\ensuremath{\ifstrempty{#1}{\mathbb{Q}}{\mathbb{Q}^{#1}}}}
\newcommand{\CC}[1][]{\ensuremath{\ifstrempty{#1}{\mathbb{C}}{\mathbb{C}^{#1}}}}
\newcommand{\PP}[1][]{\ensuremath{\ifstrempty{#1}{\mathbb{P}}{\mathbb{P}^{#1}}}}
\newcommand{\HH}[1][]{\ensuremath{\ifstrempty{#1}{\mathbb{H}}{\mathbb{H}^{#1}}}}
\newcommand{\FF}[1][]{\ensuremath{\ifstrempty{#1}{\mathbb{F}}{\mathbb{F}^{#1}}}}
% expected value
\newcommand{\EE}{\ensuremath{\mathbb{E}}}
\newcommand{\charin}{\text{ char }}
\DeclareMathOperator{\sign}{sign}
\DeclareMathOperator{\Aut}{Aut}
\DeclareMathOperator{\Inn}{Inn}
\DeclareMathOperator{\Syl}{Syl}
\DeclareMathOperator{\Gal}{Gal}
\DeclareMathOperator{\GL}{GL} % General linear group
\DeclareMathOperator{\SL}{SL} % Special linear group

%---------------------------------------
% BlackBoard Math Fonts :-
%---------------------------------------

%Captital Letters
\newcommand{\bbA}{\mathbb{A}}	\newcommand{\bbB}{\mathbb{B}}
\newcommand{\bbC}{\mathbb{C}}	\newcommand{\bbD}{\mathbb{D}}
\newcommand{\bbE}{\mathbb{E}}	\newcommand{\bbF}{\mathbb{F}}
\newcommand{\bbG}{\mathbb{G}}	\newcommand{\bbH}{\mathbb{H}}
\newcommand{\bbI}{\mathbb{I}}	\newcommand{\bbJ}{\mathbb{J}}
\newcommand{\bbK}{\mathbb{K}}	\newcommand{\bbL}{\mathbb{L}}
\newcommand{\bbM}{\mathbb{M}}	\newcommand{\bbN}{\mathbb{N}}
\newcommand{\bbO}{\mathbb{O}}	\newcommand{\bbP}{\mathbb{P}}
\newcommand{\bbQ}{\mathbb{Q}}	\newcommand{\bbR}{\mathbb{R}}
\newcommand{\bbS}{\mathbb{S}}	\newcommand{\bbT}{\mathbb{T}}
\newcommand{\bbU}{\mathbb{U}}	\newcommand{\bbV}{\mathbb{V}}
\newcommand{\bbW}{\mathbb{W}}	\newcommand{\bbX}{\mathbb{X}}
\newcommand{\bbY}{\mathbb{Y}}	\newcommand{\bbZ}{\mathbb{Z}}

%---------------------------------------
% MathCal Fonts :-
%---------------------------------------

%Captital Letters
\newcommand{\mcA}{\mathcal{A}}	\newcommand{\mcB}{\mathcal{B}}
\newcommand{\mcC}{\mathcal{C}}	\newcommand{\mcD}{\mathcal{D}}
\newcommand{\mcE}{\mathcal{E}}	\newcommand{\mcF}{\mathcal{F}}
\newcommand{\mcG}{\mathcal{G}}	\newcommand{\mcH}{\mathcal{H}}
\newcommand{\mcI}{\mathcal{I}}	\newcommand{\mcJ}{\mathcal{J}}
\newcommand{\mcK}{\mathcal{K}}	\newcommand{\mcL}{\mathcal{L}}
\newcommand{\mcM}{\mathcal{M}}	\newcommand{\mcN}{\mathcal{N}}
\newcommand{\mcO}{\mathcal{O}}	\newcommand{\mcP}{\mathcal{P}}
\newcommand{\mcQ}{\mathcal{Q}}	\newcommand{\mcR}{\mathcal{R}}
\newcommand{\mcS}{\mathcal{S}}	\newcommand{\mcT}{\mathcal{T}}
\newcommand{\mcU}{\mathcal{U}}	\newcommand{\mcV}{\mathcal{V}}
\newcommand{\mcW}{\mathcal{W}}	\newcommand{\mcX}{\mathcal{X}}
\newcommand{\mcY}{\mathcal{Y}}	\newcommand{\mcZ}{\mathcal{Z}}


%---------------------------------------
% Bold Math Fonts :-
%---------------------------------------

%Captital Letters
\newcommand{\bmA}{\boldsymbol{A}}	\newcommand{\bmB}{\boldsymbol{B}}
\newcommand{\bmC}{\boldsymbol{C}}	\newcommand{\bmD}{\boldsymbol{D}}
\newcommand{\bmE}{\boldsymbol{E}}	\newcommand{\bmF}{\boldsymbol{F}}
\newcommand{\bmG}{\boldsymbol{G}}	\newcommand{\bmH}{\boldsymbol{H}}
\newcommand{\bmI}{\boldsymbol{I}}	\newcommand{\bmJ}{\boldsymbol{J}}
\newcommand{\bmK}{\boldsymbol{K}}	\newcommand{\bmL}{\boldsymbol{L}}
\newcommand{\bmM}{\boldsymbol{M}}	\newcommand{\bmN}{\boldsymbol{N}}
\newcommand{\bmO}{\boldsymbol{O}}	\newcommand{\bmP}{\boldsymbol{P}}
\newcommand{\bmQ}{\boldsymbol{Q}}	\newcommand{\bmR}{\boldsymbol{R}}
\newcommand{\bmS}{\boldsymbol{S}}	\newcommand{\bmT}{\boldsymbol{T}}
\newcommand{\bmU}{\boldsymbol{U}}	\newcommand{\bmV}{\boldsymbol{V}}
\newcommand{\bmW}{\boldsymbol{W}}	\newcommand{\bmX}{\boldsymbol{X}}
\newcommand{\bmY}{\boldsymbol{Y}}	\newcommand{\bmZ}{\boldsymbol{Z}}
%Small Letters
\newcommand{\bma}{\boldsymbol{a}}	\newcommand{\bmb}{\boldsymbol{b}}
\newcommand{\bmc}{\boldsymbol{c}}	\newcommand{\bmd}{\boldsymbol{d}}
\newcommand{\bme}{\boldsymbol{e}}	\newcommand{\bmf}{\boldsymbol{f}}
\newcommand{\bmg}{\boldsymbol{g}}	\newcommand{\bmh}{\boldsymbol{h}}
\newcommand{\bmi}{\boldsymbol{i}}	\newcommand{\bmj}{\boldsymbol{j}}
\newcommand{\bmk}{\boldsymbol{k}}	\newcommand{\bml}{\boldsymbol{l}}
\newcommand{\bmm}{\boldsymbol{m}}	\newcommand{\bmn}{\boldsymbol{n}}
\newcommand{\bmo}{\boldsymbol{o}}	\newcommand{\bmp}{\boldsymbol{p}}
\newcommand{\bmq}{\boldsymbol{q}}	\newcommand{\bmr}{\boldsymbol{r}}
\newcommand{\bms}{\boldsymbol{s}}	\newcommand{\bmt}{\boldsymbol{t}}
\newcommand{\bmu}{\boldsymbol{u}}	\newcommand{\bmv}{\boldsymbol{v}}
\newcommand{\bmw}{\boldsymbol{w}}	\newcommand{\bmx}{\boldsymbol{x}}
\newcommand{\bmy}{\boldsymbol{y}}	\newcommand{\bmz}{\boldsymbol{z}}

%---------------------------------------
% Scr Math Fonts :-
%---------------------------------------

\newcommand{\sA}{{\mathscr{A}}}   \newcommand{\sB}{{\mathscr{B}}}
\newcommand{\sC}{{\mathscr{C}}}   \newcommand{\sD}{{\mathscr{D}}}
\newcommand{\sE}{{\mathscr{E}}}   \newcommand{\sF}{{\mathscr{F}}}
\newcommand{\sG}{{\mathscr{G}}}   \newcommand{\sH}{{\mathscr{H}}}
\newcommand{\sI}{{\mathscr{I}}}   \newcommand{\sJ}{{\mathscr{J}}}
\newcommand{\sK}{{\mathscr{K}}}   \newcommand{\sL}{{\mathscr{L}}}
\newcommand{\sM}{{\mathscr{M}}}   \newcommand{\sN}{{\mathscr{N}}}
\newcommand{\sO}{{\mathscr{O}}}   \newcommand{\sP}{{\mathscr{P}}}
\newcommand{\sQ}{{\mathscr{Q}}}   \newcommand{\sR}{{\mathscr{R}}}
\newcommand{\sS}{{\mathscr{S}}}   \newcommand{\sT}{{\mathscr{T}}}
\newcommand{\sU}{{\mathscr{U}}}   \newcommand{\sV}{{\mathscr{V}}}
\newcommand{\sW}{{\mathscr{W}}}   \newcommand{\sX}{{\mathscr{X}}}
\newcommand{\sY}{{\mathscr{Y}}}   \newcommand{\sZ}{{\mathscr{Z}}}


%---------------------------------------
% Math Fraktur Font
%---------------------------------------

%Captital Letters
\newcommand{\mfA}{\mathfrak{A}}	\newcommand{\mfB}{\mathfrak{B}}
\newcommand{\mfC}{\mathfrak{C}}	\newcommand{\mfD}{\mathfrak{D}}
\newcommand{\mfE}{\mathfrak{E}}	\newcommand{\mfF}{\mathfrak{F}}
\newcommand{\mfG}{\mathfrak{G}}	\newcommand{\mfH}{\mathfrak{H}}
\newcommand{\mfI}{\mathfrak{I}}	\newcommand{\mfJ}{\mathfrak{J}}
\newcommand{\mfK}{\mathfrak{K}}	\newcommand{\mfL}{\mathfrak{L}}
\newcommand{\mfM}{\mathfrak{M}}	\newcommand{\mfN}{\mathfrak{N}}
\newcommand{\mfO}{\mathfrak{O}}	\newcommand{\mfP}{\mathfrak{P}}
\newcommand{\mfQ}{\mathfrak{Q}}	\newcommand{\mfR}{\mathfrak{R}}
\newcommand{\mfS}{\mathfrak{S}}	\newcommand{\mfT}{\mathfrak{T}}
\newcommand{\mfU}{\mathfrak{U}}	\newcommand{\mfV}{\mathfrak{V}}
\newcommand{\mfW}{\mathfrak{W}}	\newcommand{\mfX}{\mathfrak{X}}
\newcommand{\mfY}{\mathfrak{Y}}	\newcommand{\mfZ}{\mathfrak{Z}}
%Small Letters
\newcommand{\mfa}{\mathfrak{a}}	\newcommand{\mfb}{\mathfrak{b}}
\newcommand{\mfc}{\mathfrak{c}}	\newcommand{\mfd}{\mathfrak{d}}
\newcommand{\mfe}{\mathfrak{e}}	\newcommand{\mff}{\mathfrak{f}}
\newcommand{\mfg}{\mathfrak{g}}	\newcommand{\mfh}{\mathfrak{h}}
\newcommand{\mfi}{\mathfrak{i}}	\newcommand{\mfj}{\mathfrak{j}}
\newcommand{\mfk}{\mathfrak{k}}	\newcommand{\mfl}{\mathfrak{l}}
\newcommand{\mfm}{\mathfrak{m}}	\newcommand{\mfn}{\mathfrak{n}}
\newcommand{\mfo}{\mathfrak{o}}	\newcommand{\mfp}{\mathfrak{p}}
\newcommand{\mfq}{\mathfrak{q}}	\newcommand{\mfr}{\mathfrak{r}}
\newcommand{\mfs}{\mathfrak{s}}	\newcommand{\mft}{\mathfrak{t}}
\newcommand{\mfu}{\mathfrak{u}}	\newcommand{\mfv}{\mathfrak{v}}
\newcommand{\mfw}{\mathfrak{w}}	\newcommand{\mfx}{\mathfrak{x}}
\newcommand{\mfy}{\mathfrak{y}}	\newcommand{\mfz}{\mathfrak{z}}


\title{\Huge{Física Estadistica}\\Tarea 5}
\author{\huge{Sergio Montoya Ramirez}\\ 202112171}
\date{\today}

\begin{document}

\maketitle
\newpage% or \cleardoublepage
% \pdfbookmark[<level>]{<title>}{<dest>}
\pdfbookmark[section]{\contentsname}{toc}
\tableofcontents
\pagebreak

%%% PUNTO 1 %%%%%%%%%%%%%%%%%%%%%%%%%%%%%%%%%%%%%%%%%%%%%%%%%%%%
\chapter{}

\section{}

En las secciones 6.1 y 6.2 del librio Pathria se llego a

\begin{align}
  \frac{PV}{kT} &= \sum_{\varepsilon}\ln\paren{1 + ze^{-\beta\varepsilon}}\label{eq:8-1}\\
  N &= \sum_{\varepsilon} \frac{1}{z^{-1}e^{\beta\varepsilon} + 1}\label{eq:8-2}
\end{align}

Sin embargo
\[
  \sum_{\varepsilon} \to \int_0^\infty g(\varepsilon) d\varepsilon
\]

donde
\[
  g(\varepsilon) d\varepsilon = \frac{V g \sqrt{\varepsilon}}{2\pi^2 \hbar^3} (2m)^{3/2} d\varepsilon,
\] 

Ademas usaremos:
\[
  f_{n}\paren{z} = \frac{1}{\Gamma\paren{n}}\int_0^\infty \frac{x^{n - 1}}{z^{-1}e^{x} + 1}
\]

por lo tanto aplicando en \ref{eq:8-1} y \ref{eq:8-2} tenemos
\begin{enumerate}
  \item Para \ref{eq:8-1}
    \begin{align*}
      \frac{PV}{kT} &= \sum_{\varepsilon}\ln\paren{1 + ze^{-\beta\varepsilon}}\\
      \frac{PV}{kT} &= \int_0^\infty \ln\paren{1 + ze^{-\beta\varepsilon}}\frac{V g \sqrt{\varepsilon}}{2\pi^2 \hbar^3} (2m)^{3/2} d\varepsilon\\
      \frac{PV}{kT} &= \frac{V g }{2\pi^2 \hbar^3} (2m)^{3/2}\int_0^\infty \ln\paren{1 + ze^{-\beta\varepsilon}} \sqrt{\varepsilon}d\varepsilon\\
      x &= \beta x\\
      \varepsilon &= kTx\\
      d\varepsilon &= kTdx\\
      \frac{PV}{kT} &= \frac{V g }{2\pi^2 \hbar^3} (2m)^{3/2}\int_0^\infty \ln\paren{1 + ze^{-x}} \sqrt{kTx}kTdx\\
      \frac{PV}{kT}&=\frac{V g }{2\pi^2 \hbar^3} (2m)^{3/2} \paren{kT}^{\frac{3}{2}}\int_0^\infty \ln\paren{1 + ze^{-x}}\sqrt{x}dx
    \end{align*}

    Ahora para solucionar la integral podemos hacerla por partes de la siguiente manera

    \begin{align*}
      u &= \ln\paren{1 + ze^{-x}}\\
      du &= \frac{-ze^{-x}}{1 + ze^{-x}}dx\\
      dv &= \sqrt{x}dx\\
      v &= \frac{2}{3}x^{\frac{3}{2}}\\
      \int udv &= uv - \int vdu\\
      \int_0^\infty \ln\paren{1 + ze^{-x}}\sqrt{x}dx &= \left[\ln\paren{1 + ze^{-x}}\frac{2}{3}x^{\frac{3}{2}}\right]_{0}^\infty - \int_0^\infty \frac{2}{3}x^{\frac{3}{2}}\frac{-ze^{-x}}{1 + ze^{-x}}dx\\
      &=  \frac{2}{3}\int_0^\infty \frac{x^{\frac{3}{2}}ze^{-x}}{1 + ze^{-x}}dx\\
      &=  \frac{2}{3}\Gamma\paren{\frac{5}{2}}f_{\frac{5}{2}}\paren{z}\\
      \Gamma\paren{\frac{5}{2}} &= \frac{3}{4}\sqrt{\pi}\\
      &=  \frac{\sqrt{\pi}}{2}f_{\frac{5}{2}}\paren{z}
    \end{align*}

    Con esto entonces
    \begin{align*}
      \frac{PV}{kT}&=\frac{V g }{2\pi^2 \hbar^3} (2m)^{3/2} \paren{kT}^{\frac{3}{2}}\frac{\sqrt{\pi}}{2}f_{\frac{5}{2}}\paren{z}\\
      \frac{PV}{kT}&=\frac{V g }{2\pi^2 \hbar^3} (2mkT)^{3/2} \frac{\sqrt{\pi}}{2}f_{\frac{5}{2}}\paren{z}\\
      \frac{PV}{kT}&=\frac{V g }{2\pi^2 \frac{h^3}{8\pi^3}} (2mkT)^{3/2} \frac{\sqrt{\pi}}{2}f_{\frac{5}{2}}\paren{z}\\
      \frac{PV}{kT}&=\frac{V g }{\frac{h^3}{2\pi}} (2mkT)^{3/2} \frac{\sqrt{\pi}}{2}f_{\frac{5}{2}}\paren{z}\\
      \frac{PV}{kT}&=2\pi\frac{V g }{h^3} (2mkT)^{3/2} \frac{\sqrt{\pi}}{2}f_{\frac{5}{2}}\paren{z}\\
      \frac{PV}{kT}&=\frac{V g }{h^3} (2\pi mkT)^{3/2} f_{\frac{5}{2}}\paren{z}\\
      \lambda &= \frac{h}{\sqrt{2\pi mkT}}\\
      \lambda^3 &= \frac{h^3}{\paren{2\pi mkT}^{\frac{3}{2}}}\\
      \frac{1}{\lambda^3} &= \frac{\paren{2\pi mkT}^{\frac{3}{2}}}{h^3}\\
      \frac{PV}{kT}&=\frac{V g }{\lambda^3} f_{\frac{5}{2}}\paren{z}\\
      \frac{P}{kT}&=\frac{g }{\lambda^3} f_{\frac{5}{2}}\paren{z}\\
    \end{align*}
    
  \item Para \ref{eq:8-2}
    \begin{align*}
      N &= \sum_{\varepsilon} \frac{1}{z^{-1}e^{\beta\varepsilon} + 1}\\
      &= \int_0^{\infty} \frac{1}{z^{-1}e^{\beta\varepsilon} + 1} g(\varepsilon) d\varepsilon\\
      &= \int_0^{\infty} \frac{1}{z^{-1}e^{\beta\varepsilon} + 1} \frac{V g \sqrt{\varepsilon}}{2\pi^2 \hbar^3} (2m)^{3/2} d\varepsilon\\
      &= \frac{V g }{2\pi^2 \hbar^3} (2m)^{3/2} \int_0^{\infty} \frac{1}{z^{-1}e^{\beta\varepsilon} + 1} \sqrt{\varepsilon} d\varepsilon\\
      &= \frac{V g }{2\pi^2 \hbar^3} (2m)^{3/2} \int_0^{\infty} \frac{\varepsilon^{1/2}}{z^{-1}e^{\beta\varepsilon} + 1}  d\varepsilon\\
      &= \frac{V g }{2\pi^2 \hbar^3} (2m)^{3/2} \int_0^{\infty} \frac{\varepsilon^{1/2}}{z^{-1}e^{\beta\varepsilon} + 1}  d\varepsilon\\
      x &= \beta\varepsilon\\
      kTx &= \varepsilon\\
      d\varepsilon &= kTdx\\
      &= \frac{V g }{2\pi^2 \hbar^3} (2m)^{3/2} \int_0^{\infty} \frac{\paren{kTx}^{1/2}}{z^{-1}e^{x} + 1}  kTdx\\
      &= \frac{V g }{2\pi^2 \hbar^3} (2mkT)^{3/2} \int_0^{\infty} \frac{\paren{x}^{1/2}}{z^{-1}e^{x} + 1}  dx\\
      &= \frac{V g }{2\pi^2 \hbar^3} (2mkT)^{3/2} \Gamma\paren{\frac{3}{2}} f_{\frac{3}{2}}\paren{z}\\
      &= \frac{V g }{2\pi^2 \hbar^3} (2mkT)^{3/2} \frac{\sqrt{\pi}}{2} f_{\frac{3}{2}}\paren{z}\\
      &= \frac{V g }{\frac{h^3}{2\pi}} (2mkT)^{3/2} \frac{\sqrt{\pi}}{2} f_{\frac{3}{2}}\paren{z}\\
      &= 2\pi\frac{V g }{h^3} (2mkT)^{3/2} \frac{\sqrt{\pi}}{2} f_{\frac{3}{2}}\paren{z}\\
      &= \frac{V g }{h^3} (2\pi mkT)^{3/2} f_{\frac{3}{2}}\paren{z}\\
      &= \frac{V g }{\lambda^3} f_{\frac{3}{2}}\paren{z}\\
      N &= \frac{V g }{\lambda^3} f_{\frac{3}{2}}\paren{z}\\
      \frac{N}{V} &= \frac{g}{\lambda^3} f_{\frac{3}{2}}\paren{z}\\
    \end{align*}
\end{enumerate}

\section{}

Tenemos
\begin{align*}
  U &= kT^2 \paren{\frac{\partial}{\partial T}\frac{PV}{kT}}\\
  U &= kT^2 \paren{\frac{\partial}{\partial T}\frac{Vg}{\lambda^3}f_{\frac{5}{2}}\paren{z}}\\
  U &= kT^2Vg \paren{\frac{\partial}{\partial T}\frac{1}{\lambda^3}f_{\frac{5}{2}}\paren{z}}\\
  U &= kT^2Vg \paren{\frac{\partial}{\partial T}\frac{1}{\lambda^3}f_{\frac{5}{2}}\paren{z} + \frac{1}{\lambda^3}\frac{\partial}{\partial T}f_{\frac{5}{2}}\paren{z}}\\
  U &= kT^2Vg \paren{\frac{3}{2\lambda^3T}f_{\frac{5}{2}}\paren{z} + \frac{1}{\lambda^3}0}\\
  U &= kT^2Vg \frac{3}{2\lambda^3T}f_{\frac{5}{2}}\paren{z}\\
  U &= \frac{3kT^2Vg}{2\lambda^3T}f_{\frac{5}{2}}\paren{z}\\
  \frac{N}{V} &= \frac{g}{\lambda^3} f_{\frac{3}{2}}\paren{z}\\
  \frac{N}{f_{\frac{3}{2}}\paren{z}} &= \frac{gV}{\lambda^3}\\
  U &= \frac{3kT N}{2f_{\frac{3}{2}}\paren{z}}f_{\frac{5}{2}}\paren{z}\\
  U &= \frac{3}{2}kT N\frac{f_{\frac{5}{2}}\paren{z}}{f_{\frac{3}{2}}\paren{z}}\\
\end{align*}

\section{}

Para esto usaremos
\[
  C_V = \paren{\frac{\partial U}{\partial T}}_{V}
\]

Con lo cual:
\begin{align*}
  U &= \frac{3}{2}kT N\frac{f_{\frac{5}{2}}\paren{z}}{f_{\frac{3}{2}}\paren{z}}\\
  C_V &= \paren{\frac{\partial}{\partial T} \frac{3}{2}kT N\frac{f_{\frac{5}{2}}\paren{z}}{f_{\frac{3}{2}}\paren{z}} }_{V}\\
  C_V &= \frac{3}{2}Nk\paren{\frac{\partial}{\partial T} T \frac{f_{\frac{5}{2}}\paren{z}}{f_{\frac{3}{2}}\paren{z}} }_{V}\\
  C_V &= \frac{3}{2}Nk\paren{ \frac{f_{\frac{5}{2}}\paren{z}}{f_{\frac{3}{2}}\paren{z}} + T\frac{\partial}{\partial T} \frac{f_{\frac{5}{2}}\paren{z}}{f_{\frac{3}{2}}\paren{z}} }_{V}\\
  C_V &= \frac{3}{2}Nk\paren{ \frac{f_{\frac{5}{2}}\paren{z}}{f_{\frac{3}{2}}\paren{z}} + T\frac{f_{\frac{3}{2}}(z) \frac{\partial f_{\frac{5}{2}}(z)}{\partial T} - f_{\frac{5}{2}}(z) \frac{\partial f_{\frac{3}{2}}(z)}{\partial T}}{\left[f_{\frac{3}{2}}(z)\right]^2} }_{V}\\
  C_V &= \frac{3}{2}Nk\paren{ \frac{f_{\frac{5}{2}}\paren{z}}{f_{\frac{3}{2}}\paren{z}} + T\frac{f_{\frac{3}{2}}(z) \frac{\partial f_{\frac{5}{2}}(z)}{\partial z} \frac{\partial z}{\partial T} - f_{\frac{5}{2}}(z) \frac{\partial f_{\frac{3}{2}}(z)}{\partial z}\frac{\partial z}{\partial T}}{\left[f_{\frac{3}{2}}(z)\right]^2} }_{V}\\
  C_V &= \frac{3}{2}Nk\paren{ \frac{f_{\frac{5}{2}}\paren{z}}{f_{\frac{3}{2}}\paren{z}} + T\frac{f_{\frac{3}{2}}(z) \frac{f_{\frac{3}{2}}(z)}{z} \frac{\partial z}{\partial T} - f_{\frac{5}{2}}(z) \frac{f_{\frac{1}{2}}(z)}{z}\frac{\partial z}{\partial T}}{\left[f_{\frac{3}{2}}(z)\right]^2} }_{V}\\
  C_V &= \frac{3}{2}Nk\paren{ \frac{f_{\frac{5}{2}}\paren{z}}{f_{\frac{3}{2}}\paren{z}} + \frac{T}{z} \frac{\partial z}{\partial T}\frac{f_{\frac{3}{2}}(z)^2   - f_{\frac{5}{2}}(z) f_{\frac{1}{2}}(z)}{\left[f_{\frac{3}{2}}(z)\right]^2} }_{V}\\
  C_V &= \frac{3}{2}Nk\paren{ \frac{f_{\frac{5}{2}}\paren{z}}{f_{\frac{3}{2}}\paren{z}} + \frac{T}{z} \paren{-\frac{3}{2}\frac{z}{T}\frac{f_{\frac{3}{2}}(z)}{f_{\frac{1}{2}}(z)}}\frac{f_{\frac{3}{2}}(z)^2   - f_{\frac{5}{2}}(z) f_{\frac{1}{2}}(z)}{\left[f_{\frac{3}{2}}(z)\right]^2} }\\
  C_V &= \frac{3}{2}Nk\paren{ \frac{f_{\frac{5}{2}}\paren{z}}{f_{\frac{3}{2}}\paren{z}} - \frac{3}{2}\frac{f_{\frac{3}{2}}(z)^2   - f_{\frac{5}{2}}(z) f_{\frac{1}{2}}(z)}{f_{\frac{3}{2}}(z)f_{\frac{1}{2}}(z)} }\\
  C_V &= \frac{3}{2}Nk\paren{ \frac{f_{\frac{5}{2}}\paren{z}}{f_{\frac{3}{2}}\paren{z}} - \frac{3}{2}\frac{f_{\frac{3}{2}}(z)^2}{f_{\frac{3}{2}}(z)f_{\frac{1}{2}}(z)}   + \frac{3}{2}\frac{f_{\frac{5}{2}}(z) f_{\frac{1}{2}}(z)}{f_{\frac{3}{2}}(z)f_{\frac{1}{2}}(z)} }\\
  C_V &= \frac{3}{2}Nk\paren{ \frac{f_{\frac{5}{2}}\paren{z}}{f_{\frac{3}{2}}\paren{z}} - \frac{3f_{\frac{3}{2}}(z)}{2f_{\frac{1}{2}}(z)}   + \frac{3f_{\frac{5}{2}}(z)}{2f_{\frac{3}{2}}(z)} }\\
  C_V &= \frac{3}{2}Nk\paren{ \frac{5f_{\frac{5}{2}}\paren{z}}{2f_{\frac{3}{2}}\paren{z}} - \frac{3f_{\frac{3}{2}}(z)}{2f_{\frac{1}{2}}(z)}}\\
  C_V &= Nk\frac{3}{2}\paren{ \frac{5f_{\frac{5}{2}}\paren{z}}{2f_{\frac{3}{2}}\paren{z}} - \frac{3f_{\frac{3}{2}}(z)}{2f_{\frac{1}{2}}(z)}}\\
  C_V &= Nk\paren{\frac{15}{4}\frac{f_{\frac{5}{2}}\paren{z}}{f_{\frac{3}{2}}\paren{z}} - \frac{9}{4}\frac{f_{\frac{3}{2}}(z)}{f_{\frac{1}{2}}(z)}}\\
\end{align*}

\section{}

En el Apendice $E$ del libro de Pathria explican que para $z$ pequeños se cumple que:
\[
  f_v(z) = z - \frac{z^2}{2^v} + \frac{z^3}{3^v} - \ldots
\]

Nos piden encontrar esta serie en terminos de $n\lambda^3$ por lo tanto partamos de la expresión para $n = \frac{N}{V}$ con lo cual:

\begin{align*}
  n &= \frac{g}{\lambda^3}f_{\frac{3}{2}}(z)\\
  n &= \frac{g}{\lambda^3}\paren{z - \frac{z^2}{2^{\frac{3}{2}}} + \frac{z^3}{3^{\frac{3}{2}}} - \ldots}\\
  \frac{n\lambda^3}{g} &= \paren{z - \frac{z^2}{2^{\frac{3}{2}}} + \frac{z^3}{3^{\frac{3}{2}}} - \ldots}\\
  z &\approx \frac{n\lambda^3}{g} + \frac{\paren{n\lambda^3}^2}{2\sqrt{2}g^2}
\end{align*}

Ademas de eso veamos las equivalencias de las funciones:
\begin{align*}
f_{\frac{5}{2}}(z) &\approx z - \frac{z^2}{2^{\frac{5}{2}}} + \cdots = z - \frac{z^2}{4\sqrt{2}} + \cdots, \\
f_{\frac{3}{2}}(z) &\approx z - \frac{z^2}{2^{\frac{3}{2}}} + \cdots = z - \frac{z^2}{2\sqrt{2}} + \cdots, \\
f_{\frac{1}{2}}(z) &\approx z - \frac{z^2}{2^{\frac{1}{2}}} + \cdots = z - \frac{z^2}{\sqrt{2}} + \cdots.
\end{align*}

Ahora tomando en cuenta que
\[
C_V = Nk \left( \frac{15}{4} \frac{f_{\frac{5}{2}}(z)}{f_{\frac{3}{2}}(z)} - \frac{9}{4} \frac{f_{\frac{3}{2}}(z)}{f_{\frac{1}{2}}(z)} \right).
\]

Podemos desarrollar cada una de las fracciones por aparte como

\begin{align*}
\frac{f_{\frac{5}{2}}(z)}{f_{\frac{3}{2}}(z)} &\approx \frac{z - \frac{z^2}{4\sqrt{2}}}{z - \frac{z^2}{2\sqrt{2}}} \approx 1 + \frac{z}{4\sqrt{2}}, \\
\frac{f_{\frac{3}{2}}(z)}{f_{\frac{1}{2}}(z)} &\approx \frac{z - \frac{z^2}{2\sqrt{2}}}{z - \frac{z^2}{\sqrt{2}}} \approx 1 + \frac{z}{2\sqrt{2}}.
\end{align*}

Lo que nos dejaria con un desarrollo como

\begin{align*}
  C_V &\approx Nk \left( \frac{15}{4} \left(1 + \frac{z}{4\sqrt{2}} \right) - \frac{9}{4} \left(1 + \frac{z}{2\sqrt{2}} \right) \right)\\
  C_V &= Nk \left( \frac{15}{4} - \frac{9}{4} + \frac{15}{16\sqrt{2}} z - \frac{9}{8\sqrt{2}} z \right)\\ &= Nk \left( \frac{3}{2} - \frac{3}{16\sqrt{2}} z \right)\\
  C_V &= \frac{3}{2}Nk - \frac{3}{16\sqrt{2}} \frac{n\lambda^3}{g} Nk + \cdots\\
\end{align*}

note que siempre que si $n\lambda^3 > 0$ entonces \[\frac{3}{16\sqrt{2}} \frac{n\lambda^3}{g} Nk > 0\] por lo tanto dado que esto es positivo el termino total seria menor. Es decir:
\begin{align*}
  C_V = \frac{3}{2}Nk - \frac{3}{16\sqrt{2}} \frac{n\lambda^3}{g} Nk &< \frac{3}{2}Nk\\
  C_V &< \frac{3}{2}Nk\\
\end{align*}

\section{}

En este caso usaremos

\[
f_{3/2}(z) \approx \frac{2}{3\sqrt{\pi}} \mu^{3/2} \left[ 1 + \frac{\pi^2}{8} \left( \frac{kT}{\mu} \right)^2 + \cdots \right].
\]

Con lo cual podemos revisar para $\frac{N}{V}$

\begin{align*}
  n &= \frac{g}{\lambda^3} f_{3/2}(z)\\
  n &\approx \frac{g}{\lambda^3} \frac{2}{3\sqrt{\pi}} \mu^{3/2} \left[ 1 + \frac{\pi^2}{8} \left( \frac{kT}{\mu} \right)^2 \right]\\
  n &= \frac{g}{6\pi^2} \left( \frac{2mE_F}{\hbar^2} \right)^{3/2}\\
\end{align*}

Ahora igualando las expresiones para $n$

\begin{align*}
  1 &\approx \left( \frac{\mu}{E_F} \right)^{3/2} + \frac{\pi^2}{8} \left( \frac{T}{T_F} \right)^2 \left( \frac{E_F}{\mu} \right)^{1/2}\\
  \mu &= E_F(1 + \delta)\\
  1 &\approx 1 + \frac{3}{2}\delta + \frac{\pi^2}{8} \left( \frac{T}{T_F} \right)^2\\
  \delta &\approx -\frac{\pi^2}{12} \left( \frac{T}{T_F} \right)^2\\
  \mu(T) &= E_F \left[ 1 - \frac{\pi^2}{12} \left( \frac{T}{T_F} \right)^2 \right]\\
\end{align*}

\section{}

Partimos desde la definición:
\[
  U = \frac{3}{2}NkT\frac{f_{\frac{5}{2}}(z)}{f_{\frac{3}{2}}(z)}
\]

Utilizando la expansión:
\[
  f_v(z) \approx \frac{\paren{\ln z}^v}{\Gamma\paren{v + 1}} \left[1 + \frac{\pi^2}{6}\frac{v(v - 1)}{\paren{\ln z}^2} + \ldots \right]
\]

Con esto entonces podemos encontra
\begin{align*}
f_{\frac{5}{2}}(z) &\approx \frac{(\ln z)^{\frac{5}{2}}}{\Gamma\left(\frac{7}{2}\right)} \left[ 1 + \frac{\pi^2}{6} \frac{\frac{5}{2} \cdot \frac{3}{2}}{(\ln z)^2} \right]\\
f_{\frac{3}{2}}(z) &\approx \frac{(\ln z)^{\frac{3}{2}}}{\Gamma\left(\frac{5}{2}\right)} \left[ 1 + \frac{\pi^2}{6} \frac{\frac{3}{2} \cdot \frac{1}{2}}{(\ln z)^2} \right]
\end{align*}

Con esto entonces podemos encontrar cada una de las fracciones de $U$. Queda:
\begin{align*}
  \frac{f_{\frac{5}{2}}(z)}{f_{\frac{3}{2}}(z)} &\approx \frac{\Gamma\left(\frac{5}{2}\right)}{\Gamma\left(\frac{7}{2}\right)} (\ln z) \left[ 1 + \frac{\pi^2}{8} \frac{1}{(\ln z)^2} \right]\\
  \Gamma\left(\frac{7}{2}\right) &= \frac{15}{8}\sqrt{\pi}\\
  \Gamma\left(\frac{5}{2}\right) &= \frac{3}{4}\sqrt{\pi}\\
  \frac{f_{\frac{5}{2}}(z)}{f_{\frac{3}{2}}(z)} &\approx \frac{2}{5} (\ln z) \left[ 1 + \frac{\pi^2}{8} \frac{1}{(\ln z)^2} \right].
\end{align*}

Con el resultado de la sección anterior tenemos

\begin{align*}
  \ln z = \frac{\mu(T)}{kT} &= \frac{E_F}{kT} \left[ 1 - \frac{\pi^2}{12} \left( \frac{T}{T_F} \right)^2 \right]\\
  U &\approx \frac{3}{2} NkT \cdot \frac{2}{5} \frac{E_F}{kT} \left[ 1 + \frac{\pi^2}{8} \left( \frac{kT}{E_F} \right)^2 \right]\\
  U &\approx \frac{3}{5} N E_F + \frac{3\pi^2}{20} Nk^2 \frac{T^2}{E_F}
\end{align*}

Ahora dado que $C_V = \frac{\partial U}{\partial T}$ lo que nos quedaria como:
\begin{align*}
  C_V &= \frac{\partial U}{\partial T} = \frac{3\pi^2}{10} Nk^2 \frac{T}{E_F}\\
  E_F &= kT_F\\
  C_V &= \frac{\partial U}{\partial T} = \frac{3\pi^2}{10} Nk^2 \frac{T}{kTF}\\
  C_V &= Nk \left\{\frac{\pi^2}{2} \frac{T}{T_F} + o\left( \frac{T}{T_F} \right)\right\}
\end{align*}


\section{}

%%%% PUNTO 2 %%%%%%%%%%%%%%%%%%%%%%%%%%%%%%%%%%%%%%%%%%%%%%%%%%%
\chapter{}

\section{}

\section{}

\section{}

\section{}

%%% PUNTO 3 %%%%%%%%%%%%%%%%%%%%%%%%%%%%%%%%%%%%%%%%%%%%%%%%%%%%
\chapter{}

\section{}

Partimos desde \[\chi = \frac{2n\mu^{*2}}{\paren{\frac{\partial \mu_0 \paren{xN}}{\partial x}}_{x=1/2}}\]

Tenemos que considerar que segun Pathria en la ecuación 8.1.34

\section{}

\section{}

%%%% PUNTO 4 %%%%%%%%%%%%%%%%%%%%%%%%%%%%%%%%%%%%%%%%%%%%%%%%%%%
\chapter{}

\section{}

\section{}

\section{}

\section{}

\section{}

\section{}

\section{}

\end{document}

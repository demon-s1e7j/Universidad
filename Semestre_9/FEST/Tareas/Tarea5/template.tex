\documentclass{report}

\usepackage[spanish]{babel}

\NewDocumentCommand{\paren}{m}{\left(#1\right)}
\NewDocumentCommand{\inangle}{m}{\left<#1\right>}

\documentclass[12pt]{article}
\usepackage{array}
\usepackage{color}
\usepackage{amsthm}
\usepackage{eufrak}
\usepackage{lipsum}
\usepackage{pifont}
\usepackage{yfonts}
\usepackage{amsmath}
\usepackage{amssymb}
\usepackage{ccfonts}
\usepackage{comment} \usepackage{amsfonts}
\usepackage{fancyhdr}
\usepackage{graphicx}
\usepackage{listings}
\usepackage{mathrsfs}
\usepackage{setspace}
\usepackage{textcomp}
\usepackage{blindtext}
\usepackage{enumerate}
\usepackage{microtype}
\usepackage{xfakebold}
\usepackage{kantlipsum}
%\usepackage{draftwatermark}
\usepackage[spanish]{babel}
\usepackage[margin=1.5cm, top=2cm, bottom=2cm]{geometry}
\usepackage[framemethod=tikz]{mdframed}
\usepackage[colorlinks=true,citecolor=blue,linkcolor=red,urlcolor=magenta]{hyperref}

%//////////////////////////////////////////////////////
% Watermark configuration
%//////////////////////////////////////////////////////
%\SetWatermarkScale{4}
%\SetWatermarkColor{black}
%\SetWatermarkLightness{0.95}
%\SetWatermarkText{\texttt{Watermark}}

%//////////////////////////////////////////////////////
% Frame configuration
%//////////////////////////////////////////////////////
\newmdenv[tikzsetting={draw=gray,fill=white,fill opacity=0},backgroundcolor=none]{Frame}

%//////////////////////////////////////////////////////
% Font style configuration
%//////////////////////////////////////////////////////
\renewcommand{\familydefault}{\ttdefault}
\renewcommand{\rmdefault}{tt}

%//////////////////////////////////////////////////////
% Bold configuration
%//////////////////////////////////////////////////////
\newcommand{\fbseries}{\unskip\setBold\aftergroup\unsetBold\aftergroup\ignorespaces}
\makeatletter
\newcommand{\setBoldness}[1]{\def\fake@bold{#1}}
\makeatother

%//////////////////////////////////////////////////////
% Default font configuration
%//////////////////////////////////////////////////////
\DeclareFontFamily{\encodingdefault}{\ttdefault}{%
  \hyphenchar\font=\defaulthyphenchar
  \fontdimen2\font=0.33333em
  \fontdimen3\font=0.16667em
  \fontdimen4\font=0.11111em
  \fontdimen7\font=0.11111em}


\input{macros}
\input{letterfonts}

\title{\Huge{Física Estadistica}\\Tarea 5}
\author{\huge{Sergio Montoya Ramirez}\\ 202112171}
\date{\today}

\begin{document}

\maketitle
\newpage% or \cleardoublepage
% \pdfbookmark[<level>]{<title>}{<dest>}
\pdfbookmark[section]{\contentsname}{toc}
\tableofcontents
\pagebreak

%%% PUNTO 1 %%%%%%%%%%%%%%%%%%%%%%%%%%%%%%%%%%%%%%%%%%%%%%%%%%%%
\chapter{}

\section{}

En las secciones 6.1 y 6.2 del librio Pathria se llego a

\begin{align}
  \frac{PV}{kT} &= \sum_{\varepsilon}\ln\paren{1 + ze^{-\beta\varepsilon}}\label{eq:8-1}\\
  N &= \sum_{\varepsilon} \frac{1}{z^{-1}e^{\beta\varepsilon} + 1}\label{eq:8-2}
\end{align}

Sin embargo
\[
  \sum_{\varepsilon} \to \int_0^\infty g(\varepsilon) d\varepsilon
\]

donde
\[
  g(\varepsilon) d\varepsilon = \frac{V g \sqrt{\varepsilon}}{2\pi^2 \hbar^3} (2m)^{3/2} d\varepsilon,
\] 

Ademas usaremos:
\[
  f_{n}\paren{z} = \frac{1}{\Gamma\paren{n}}\int_0^\infty \frac{x^{n - 1}}{z^{-1}e^{x} + 1}
\]

por lo tanto aplicando en \ref{eq:8-1} y \ref{eq:8-2} tenemos
\begin{enumerate}
  \item Para \ref{eq:8-1}
    \begin{align*}
      \frac{PV}{kT} &= \sum_{\varepsilon}\ln\paren{1 + ze^{-\beta\varepsilon}}\\
      \frac{PV}{kT} &= \int_0^\infty \ln\paren{1 + ze^{-\beta\varepsilon}}\frac{V g \sqrt{\varepsilon}}{2\pi^2 \hbar^3} (2m)^{3/2} d\varepsilon\\
      \frac{PV}{kT} &= \frac{V g }{2\pi^2 \hbar^3} (2m)^{3/2}\int_0^\infty \ln\paren{1 + ze^{-\beta\varepsilon}} \sqrt{\varepsilon}d\varepsilon\\
      x &= \beta x\\
      \varepsilon &= kTx\\
      d\varepsilon &= kTdx\\
      \frac{PV}{kT} &= \frac{V g }{2\pi^2 \hbar^3} (2m)^{3/2}\int_0^\infty \ln\paren{1 + ze^{-x}} \sqrt{kTx}kTdx\\
      \frac{PV}{kT}&=\frac{V g }{2\pi^2 \hbar^3} (2m)^{3/2} \paren{kT}^{\frac{3}{2}}\int_0^\infty \ln\paren{1 + ze^{-x}}\sqrt{x}dx
    \end{align*}

    Ahora para solucionar la integral podemos hacerla por partes de la siguiente manera

    \begin{align*}
      u &= \ln\paren{1 + ze^{-x}}\\
      du &= \frac{-ze^{-x}}{1 + ze^{-x}}dx\\
      dv &= \sqrt{x}dx\\
      v &= \frac{2}{3}x^{\frac{3}{2}}\\
      \int udv &= uv - \int vdu\\
      \int_0^\infty \ln\paren{1 + ze^{-x}}\sqrt{x}dx &= \left[\ln\paren{1 + ze^{-x}}\frac{2}{3}x^{\frac{3}{2}}\right]_{0}^\infty - \int_0^\infty \frac{2}{3}x^{\frac{3}{2}}\frac{-ze^{-x}}{1 + ze^{-x}}dx\\
      &=  \frac{2}{3}\int_0^\infty \frac{x^{\frac{3}{2}}ze^{-x}}{1 + ze^{-x}}dx\\
      &=  \frac{2}{3}\Gamma\paren{\frac{5}{2}}f_{\frac{5}{2}}\paren{z}\\
      \Gamma\paren{\frac{5}{2}} &= \frac{3}{4}\sqrt{\pi}\\
      &=  \frac{\sqrt{\pi}}{2}f_{\frac{5}{2}}\paren{z}
    \end{align*}

    Con esto entonces
    \begin{align*}
      \frac{PV}{kT}&=\frac{V g }{2\pi^2 \hbar^3} (2m)^{3/2} \paren{kT}^{\frac{3}{2}}\frac{\sqrt{\pi}}{2}f_{\frac{5}{2}}\paren{z}\\
      \frac{PV}{kT}&=\frac{V g }{2\pi^2 \hbar^3} (2mkT)^{3/2} \frac{\sqrt{\pi}}{2}f_{\frac{5}{2}}\paren{z}\\
      \frac{PV}{kT}&=\frac{V g }{2\pi^2 \frac{h^3}{8\pi^3}} (2mkT)^{3/2} \frac{\sqrt{\pi}}{2}f_{\frac{5}{2}}\paren{z}\\
      \frac{PV}{kT}&=\frac{V g }{\frac{h^3}{2\pi}} (2mkT)^{3/2} \frac{\sqrt{\pi}}{2}f_{\frac{5}{2}}\paren{z}\\
      \frac{PV}{kT}&=2\pi\frac{V g }{h^3} (2mkT)^{3/2} \frac{\sqrt{\pi}}{2}f_{\frac{5}{2}}\paren{z}\\
      \frac{PV}{kT}&=\frac{V g }{h^3} (2\pi mkT)^{3/2} f_{\frac{5}{2}}\paren{z}\\
      \lambda &= \frac{h}{\sqrt{2\pi mkT}}\\
      \lambda^3 &= \frac{h^3}{\paren{2\pi mkT}^{\frac{3}{2}}}\\
      \frac{1}{\lambda^3} &= \frac{\paren{2\pi mkT}^{\frac{3}{2}}}{h^3}\\
      \frac{PV}{kT}&=\frac{V g }{\lambda^3} f_{\frac{5}{2}}\paren{z}\\
      \frac{P}{kT}&=\frac{g }{\lambda^3} f_{\frac{5}{2}}\paren{z}\\
    \end{align*}
    
  \item Para \ref{eq:8-2}
    \begin{align*}
      N &= \sum_{\varepsilon} \frac{1}{z^{-1}e^{\beta\varepsilon} + 1}\\
      &= \int_0^{\infty} \frac{1}{z^{-1}e^{\beta\varepsilon} + 1} g(\varepsilon) d\varepsilon\\
      &= \int_0^{\infty} \frac{1}{z^{-1}e^{\beta\varepsilon} + 1} \frac{V g \sqrt{\varepsilon}}{2\pi^2 \hbar^3} (2m)^{3/2} d\varepsilon\\
      &= \frac{V g }{2\pi^2 \hbar^3} (2m)^{3/2} \int_0^{\infty} \frac{1}{z^{-1}e^{\beta\varepsilon} + 1} \sqrt{\varepsilon} d\varepsilon\\
      &= \frac{V g }{2\pi^2 \hbar^3} (2m)^{3/2} \int_0^{\infty} \frac{\varepsilon^{1/2}}{z^{-1}e^{\beta\varepsilon} + 1}  d\varepsilon\\
      &= \frac{V g }{2\pi^2 \hbar^3} (2m)^{3/2} \int_0^{\infty} \frac{\varepsilon^{1/2}}{z^{-1}e^{\beta\varepsilon} + 1}  d\varepsilon\\
      x &= \beta\varepsilon\\
      kTx &= \varepsilon\\
      d\varepsilon &= kTdx\\
      &= \frac{V g }{2\pi^2 \hbar^3} (2m)^{3/2} \int_0^{\infty} \frac{\paren{kTx}^{1/2}}{z^{-1}e^{x} + 1}  kTdx\\
      &= \frac{V g }{2\pi^2 \hbar^3} (2mkT)^{3/2} \int_0^{\infty} \frac{\paren{x}^{1/2}}{z^{-1}e^{x} + 1}  dx\\
      &= \frac{V g }{2\pi^2 \hbar^3} (2mkT)^{3/2} \Gamma\paren{\frac{3}{2}} f_{\frac{3}{2}}\paren{z}\\
      &= \frac{V g }{2\pi^2 \hbar^3} (2mkT)^{3/2} \frac{\sqrt{\pi}}{2} f_{\frac{3}{2}}\paren{z}\\
      &= \frac{V g }{\frac{h^3}{2\pi}} (2mkT)^{3/2} \frac{\sqrt{\pi}}{2} f_{\frac{3}{2}}\paren{z}\\
      &= 2\pi\frac{V g }{h^3} (2mkT)^{3/2} \frac{\sqrt{\pi}}{2} f_{\frac{3}{2}}\paren{z}\\
      &= \frac{V g }{h^3} (2\pi mkT)^{3/2} f_{\frac{3}{2}}\paren{z}\\
      &= \frac{V g }{\lambda^3} f_{\frac{3}{2}}\paren{z}\\
      N &= \frac{V g }{\lambda^3} f_{\frac{3}{2}}\paren{z}\\
      \frac{N}{V} &= \frac{g}{\lambda^3} f_{\frac{3}{2}}\paren{z}\\
    \end{align*}
\end{enumerate}

\section{}

Tenemos
\begin{align*}
  U &= kT^2 \paren{\frac{\partial}{\partial T}\frac{PV}{kT}}\\
  U &= kT^2 \paren{\frac{\partial}{\partial T}\frac{Vg}{\lambda^3}f_{\frac{5}{2}}\paren{z}}\\
  U &= kT^2Vg \paren{\frac{\partial}{\partial T}\frac{1}{\lambda^3}f_{\frac{5}{2}}\paren{z}}\\
  U &= kT^2Vg \paren{\frac{\partial}{\partial T}\frac{1}{\lambda^3}f_{\frac{5}{2}}\paren{z} + \frac{1}{\lambda^3}\frac{\partial}{\partial T}f_{\frac{5}{2}}\paren{z}}\\
  U &= kT^2Vg \paren{\frac{3}{2\lambda^3T}f_{\frac{5}{2}}\paren{z} + \frac{1}{\lambda^3}0}\\
  U &= kT^2Vg \frac{3}{2\lambda^3T}f_{\frac{5}{2}}\paren{z}\\
  U &= \frac{3kT^2Vg}{2\lambda^3T}f_{\frac{5}{2}}\paren{z}\\
  \frac{N}{V} &= \frac{g}{\lambda^3} f_{\frac{3}{2}}\paren{z}\\
  \frac{N}{f_{\frac{3}{2}}\paren{z}} &= \frac{gV}{\lambda^3}\\
  U &= \frac{3kT N}{2f_{\frac{3}{2}}\paren{z}}f_{\frac{5}{2}}\paren{z}\\
  U &= \frac{3}{2}kT N\frac{f_{\frac{5}{2}}\paren{z}}{f_{\frac{3}{2}}\paren{z}}\\
\end{align*}

\section{}

Para esto usaremos
\[
  C_V = \paren{\frac{\partial U}{\partial T}}_{V}
\]

Con lo cual:
\begin{align*}
  U &= \frac{3}{2}kT N\frac{f_{\frac{5}{2}}\paren{z}}{f_{\frac{3}{2}}\paren{z}}\\
  C_V &= \paren{\frac{\partial}{\partial T} \frac{3}{2}kT N\frac{f_{\frac{5}{2}}\paren{z}}{f_{\frac{3}{2}}\paren{z}} }_{V}\\
  C_V &= \frac{3}{2}Nk\paren{\frac{\partial}{\partial T} T \frac{f_{\frac{5}{2}}\paren{z}}{f_{\frac{3}{2}}\paren{z}} }_{V}\\
  C_V &= \frac{3}{2}Nk\paren{ \frac{f_{\frac{5}{2}}\paren{z}}{f_{\frac{3}{2}}\paren{z}} + T\frac{\partial}{\partial T} \frac{f_{\frac{5}{2}}\paren{z}}{f_{\frac{3}{2}}\paren{z}} }_{V}\\
  C_V &= \frac{3}{2}Nk\paren{ \frac{f_{\frac{5}{2}}\paren{z}}{f_{\frac{3}{2}}\paren{z}} + T\frac{f_{\frac{3}{2}}(z) \frac{\partial f_{\frac{5}{2}}(z)}{\partial T} - f_{\frac{5}{2}}(z) \frac{\partial f_{\frac{3}{2}}(z)}{\partial T}}{\left[f_{\frac{3}{2}}(z)\right]^2} }_{V}\\
  C_V &= \frac{3}{2}Nk\paren{ \frac{f_{\frac{5}{2}}\paren{z}}{f_{\frac{3}{2}}\paren{z}} + T\frac{f_{\frac{3}{2}}(z) \frac{\partial f_{\frac{5}{2}}(z)}{\partial z} \frac{\partial z}{\partial T} - f_{\frac{5}{2}}(z) \frac{\partial f_{\frac{3}{2}}(z)}{\partial z}\frac{\partial z}{\partial T}}{\left[f_{\frac{3}{2}}(z)\right]^2} }_{V}\\
  C_V &= \frac{3}{2}Nk\paren{ \frac{f_{\frac{5}{2}}\paren{z}}{f_{\frac{3}{2}}\paren{z}} + T\frac{f_{\frac{3}{2}}(z) \frac{f_{\frac{3}{2}}(z)}{z} \frac{\partial z}{\partial T} - f_{\frac{5}{2}}(z) \frac{f_{\frac{1}{2}}(z)}{z}\frac{\partial z}{\partial T}}{\left[f_{\frac{3}{2}}(z)\right]^2} }_{V}\\
  C_V &= \frac{3}{2}Nk\paren{ \frac{f_{\frac{5}{2}}\paren{z}}{f_{\frac{3}{2}}\paren{z}} + \frac{T}{z} \frac{\partial z}{\partial T}\frac{f_{\frac{3}{2}}(z)^2   - f_{\frac{5}{2}}(z) f_{\frac{1}{2}}(z)}{\left[f_{\frac{3}{2}}(z)\right]^2} }_{V}\\
  C_V &= \frac{3}{2}Nk\paren{ \frac{f_{\frac{5}{2}}\paren{z}}{f_{\frac{3}{2}}\paren{z}} + \frac{T}{z} \paren{-\frac{3}{2}\frac{z}{T}\frac{f_{\frac{3}{2}}(z)}{f_{\frac{1}{2}}(z)}}\frac{f_{\frac{3}{2}}(z)^2   - f_{\frac{5}{2}}(z) f_{\frac{1}{2}}(z)}{\left[f_{\frac{3}{2}}(z)\right]^2} }\\
  C_V &= \frac{3}{2}Nk\paren{ \frac{f_{\frac{5}{2}}\paren{z}}{f_{\frac{3}{2}}\paren{z}} - \frac{3}{2}\frac{f_{\frac{3}{2}}(z)^2   - f_{\frac{5}{2}}(z) f_{\frac{1}{2}}(z)}{f_{\frac{3}{2}}(z)f_{\frac{1}{2}}(z)} }\\
  C_V &= \frac{3}{2}Nk\paren{ \frac{f_{\frac{5}{2}}\paren{z}}{f_{\frac{3}{2}}\paren{z}} - \frac{3}{2}\frac{f_{\frac{3}{2}}(z)^2}{f_{\frac{3}{2}}(z)f_{\frac{1}{2}}(z)}   + \frac{3}{2}\frac{f_{\frac{5}{2}}(z) f_{\frac{1}{2}}(z)}{f_{\frac{3}{2}}(z)f_{\frac{1}{2}}(z)} }\\
  C_V &= \frac{3}{2}Nk\paren{ \frac{f_{\frac{5}{2}}\paren{z}}{f_{\frac{3}{2}}\paren{z}} - \frac{3f_{\frac{3}{2}}(z)}{2f_{\frac{1}{2}}(z)}   + \frac{3f_{\frac{5}{2}}(z)}{2f_{\frac{3}{2}}(z)} }\\
  C_V &= \frac{3}{2}Nk\paren{ \frac{5f_{\frac{5}{2}}\paren{z}}{2f_{\frac{3}{2}}\paren{z}} - \frac{3f_{\frac{3}{2}}(z)}{2f_{\frac{1}{2}}(z)}}\\
  C_V &= Nk\frac{3}{2}\paren{ \frac{5f_{\frac{5}{2}}\paren{z}}{2f_{\frac{3}{2}}\paren{z}} - \frac{3f_{\frac{3}{2}}(z)}{2f_{\frac{1}{2}}(z)}}\\
  C_V &= Nk\paren{\frac{15}{4}\frac{f_{\frac{5}{2}}\paren{z}}{f_{\frac{3}{2}}\paren{z}} - \frac{9}{4}\frac{f_{\frac{3}{2}}(z)}{f_{\frac{1}{2}}(z)}}\\
\end{align*}

\section{}

En el Apendice $E$ del libro de Pathria explican que para $z$ pequeños se cumple que:
\[
  f_v(z) = z - \frac{z^2}{2^v} + \frac{z^3}{3^v} - \ldots
\]

Nos piden encontrar esta serie en terminos de $n\lambda^3$ por lo tanto partamos de la expresión para $n = \frac{N}{V}$ con lo cual:

\begin{align*}
  n &= \frac{g}{\lambda^3}f_{\frac{3}{2}}(z)\\
  n &= \frac{g}{\lambda^3}\paren{z - \frac{z^2}{2^{\frac{3}{2}}} + \frac{z^3}{3^{\frac{3}{2}}} - \ldots}\\
  \frac{n\lambda^3}{g} &= \paren{z - \frac{z^2}{2^{\frac{3}{2}}} + \frac{z^3}{3^{\frac{3}{2}}} - \ldots}
\end{align*}

\section{}

\section{}

\section{}

%%%% PUNTO 2 %%%%%%%%%%%%%%%%%%%%%%%%%%%%%%%%%%%%%%%%%%%%%%%%%%%
\chapter{}

\section{}

\section{}

\section{}

\section{}

%%% PUNTO 3 %%%%%%%%%%%%%%%%%%%%%%%%%%%%%%%%%%%%%%%%%%%%%%%%%%%%
\chapter{}

\section{}

Partimos desde \[\chi = \frac{2n\mu^{*2}}{\paren{\frac{\partial \mu_0 \paren{xN}}{\partial x}}_{x=1/2}}\]

Tenemos que considerar que segun Pathria en la ecuación 8.1.34

\section{}

\section{}

%%%% PUNTO 4 %%%%%%%%%%%%%%%%%%%%%%%%%%%%%%%%%%%%%%%%%%%%%%%%%%%
\chapter{}

\section{}

\section{}

\section{}

\section{}

\section{}

\section{}

\section{}

\end{document}

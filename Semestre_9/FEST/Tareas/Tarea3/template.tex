\documentclass{report}

\documentclass[12pt]{article}
\usepackage{array}
\usepackage{color}
\usepackage{amsthm}
\usepackage{eufrak}
\usepackage{lipsum}
\usepackage{pifont}
\usepackage{yfonts}
\usepackage{amsmath}
\usepackage{amssymb}
\usepackage{ccfonts}
\usepackage{comment} \usepackage{amsfonts}
\usepackage{fancyhdr}
\usepackage{graphicx}
\usepackage{listings}
\usepackage{mathrsfs}
\usepackage{setspace}
\usepackage{textcomp}
\usepackage{blindtext}
\usepackage{enumerate}
\usepackage{microtype}
\usepackage{xfakebold}
\usepackage{kantlipsum}
%\usepackage{draftwatermark}
\usepackage[spanish]{babel}
\usepackage[margin=1.5cm, top=2cm, bottom=2cm]{geometry}
\usepackage[framemethod=tikz]{mdframed}
\usepackage[colorlinks=true,citecolor=blue,linkcolor=red,urlcolor=magenta]{hyperref}

%//////////////////////////////////////////////////////
% Watermark configuration
%//////////////////////////////////////////////////////
%\SetWatermarkScale{4}
%\SetWatermarkColor{black}
%\SetWatermarkLightness{0.95}
%\SetWatermarkText{\texttt{Watermark}}

%//////////////////////////////////////////////////////
% Frame configuration
%//////////////////////////////////////////////////////
\newmdenv[tikzsetting={draw=gray,fill=white,fill opacity=0},backgroundcolor=none]{Frame}

%//////////////////////////////////////////////////////
% Font style configuration
%//////////////////////////////////////////////////////
\renewcommand{\familydefault}{\ttdefault}
\renewcommand{\rmdefault}{tt}

%//////////////////////////////////////////////////////
% Bold configuration
%//////////////////////////////////////////////////////
\newcommand{\fbseries}{\unskip\setBold\aftergroup\unsetBold\aftergroup\ignorespaces}
\makeatletter
\newcommand{\setBoldness}[1]{\def\fake@bold{#1}}
\makeatother

%//////////////////////////////////////////////////////
% Default font configuration
%//////////////////////////////////////////////////////
\DeclareFontFamily{\encodingdefault}{\ttdefault}{%
  \hyphenchar\font=\defaulthyphenchar
  \fontdimen2\font=0.33333em
  \fontdimen3\font=0.16667em
  \fontdimen4\font=0.11111em
  \fontdimen7\font=0.11111em}


%From M275 "Topology" at SJSU
\newcommand{\id}{\mathrm{id}}
\newcommand{\taking}[1]{\xrightarrow{#1}}
\newcommand{\inv}{^{-1}}

%From M170 "Introduction to Graph Theory" at SJSU
\DeclareMathOperator{\diam}{diam}
\DeclareMathOperator{\ord}{ord}
\newcommand{\defeq}{\overset{\mathrm{def}}{=}}

%From the USAMO .tex files
\newcommand{\ts}{\textsuperscript}
\newcommand{\dg}{^\circ}
\newcommand{\ii}{\item}

% % From Math 55 and Math 145 at Harvard
% \newenvironment{subproof}[1][Proof]{%
% \begin{proof}[#1] \renewcommand{\qedsymbol}{$\blacksquare$}}%
% {\end{proof}}

\newcommand{\liff}{\leftrightarrow}
\newcommand{\lthen}{\rightarrow}
\newcommand{\opname}{\operatorname}
\newcommand{\surjto}{\twoheadrightarrow}
\newcommand{\injto}{\hookrightarrow}
\newcommand{\On}{\mathrm{On}} % ordinals
\DeclareMathOperator{\img}{im} % Image
\DeclareMathOperator{\Img}{Im} % Image
\DeclareMathOperator{\coker}{coker} % Cokernel
\DeclareMathOperator{\Coker}{Coker} % Cokernel
\DeclareMathOperator{\Ker}{Ker} % Kernel
\DeclareMathOperator{\rank}{rank}
\DeclareMathOperator{\Spec}{Spec} % spectrum
\DeclareMathOperator{\Tr}{Tr} % trace
\DeclareMathOperator{\pr}{pr} % projection
\DeclareMathOperator{\ext}{ext} % extension
\DeclareMathOperator{\pred}{pred} % predecessor
\DeclareMathOperator{\dom}{dom} % domain
\DeclareMathOperator{\ran}{ran} % range
\DeclareMathOperator{\Hom}{Hom} % homomorphism
\DeclareMathOperator{\Mor}{Mor} % morphisms
\DeclareMathOperator{\End}{End} % endomorphism

\newcommand{\eps}{\epsilon}
\newcommand{\veps}{\varepsilon}
\newcommand{\ol}{\overline}
\newcommand{\ul}{\underline}
\newcommand{\wt}{\widetilde}
\newcommand{\wh}{\widehat}
\newcommand{\vocab}[1]{\textbf{\color{blue} #1}}
\providecommand{\half}{\frac{1}{2}}
\newcommand{\dang}{\measuredangle} %% Directed angle
\newcommand{\ray}[1]{\overrightarrow{#1}}
\newcommand{\seg}[1]{\overline{#1}}
\newcommand{\arc}[1]{\wideparen{#1}}
\DeclareMathOperator{\cis}{cis}
\DeclareMathOperator*{\lcm}{lcm}
\DeclareMathOperator*{\argmin}{arg min}
\DeclareMathOperator*{\argmax}{arg max}
\newcommand{\cycsum}{\sum_{\mathrm{cyc}}}
\newcommand{\symsum}{\sum_{\mathrm{sym}}}
\newcommand{\cycprod}{\prod_{\mathrm{cyc}}}
\newcommand{\symprod}{\prod_{\mathrm{sym}}}
\newcommand{\Qed}{\begin{flushright}\qed\end{flushright}}
\newcommand{\parinn}{\setlength{\parindent}{1cm}}
\newcommand{\parinf}{\setlength{\parindent}{0cm}}
% \newcommand{\norm}{\|\cdot\|}
\newcommand{\inorm}{\norm_{\infty}}
\newcommand{\opensets}{\{V_{\alpha}\}_{\alpha\in I}}
\newcommand{\oset}{V_{\alpha}}
\newcommand{\opset}[1]{V_{\alpha_{#1}}}
\newcommand{\lub}{\text{lub}}
\newcommand{\del}[2]{\frac{\partial #1}{\partial #2}}
\newcommand{\Del}[3]{\frac{\partial^{#1} #2}{\partial^{#1} #3}}
\newcommand{\deld}[2]{\dfrac{\partial #1}{\partial #2}}
\newcommand{\Deld}[3]{\dfrac{\partial^{#1} #2}{\partial^{#1} #3}}
\newcommand{\lm}{\lambda}
\newcommand{\uin}{\mathbin{\rotatebox[origin=c]{90}{$\in$}}}
\newcommand{\usubset}{\mathbin{\rotatebox[origin=c]{90}{$\subset$}}}
\newcommand{\lt}{\left}
\newcommand{\rt}{\right}
\newcommand{\paren}[1]{\left(#1\right)}
\newcommand{\bs}[1]{\boldsymbol{#1}}
\newcommand{\exs}{\exists}
\newcommand{\st}{\strut}
\newcommand{\dps}[1]{\displaystyle{#1}}

\newcommand{\sol}{\setlength{\parindent}{0cm}\textbf{\textit{Solution:}}\setlength{\parindent}{1cm} }
\newcommand{\solve}[1]{\setlength{\parindent}{0cm}\textbf{\textit{Solution: }}\setlength{\parindent}{1cm}#1 \Qed}

% Things Lie
\newcommand{\kb}{\mathfrak b}
\newcommand{\kg}{\mathfrak g}
\newcommand{\kh}{\mathfrak h}
\newcommand{\kn}{\mathfrak n}
\newcommand{\ku}{\mathfrak u}
\newcommand{\kz}{\mathfrak z}
\DeclareMathOperator{\Ext}{Ext} % Ext functor
\DeclareMathOperator{\Tor}{Tor} % Tor functor
\newcommand{\gl}{\opname{\mathfrak{gl}}} % frak gl group
\renewcommand{\sl}{\opname{\mathfrak{sl}}} % frak sl group chktex 6

% More script letters etc.
\newcommand{\SA}{\mathcal A}
\newcommand{\SB}{\mathcal B}
\newcommand{\SC}{\mathcal C}
\newcommand{\SF}{\mathcal F}
\newcommand{\SG}{\mathcal G}
\newcommand{\SH}{\mathcal H}
\newcommand{\OO}{\mathcal O}

\newcommand{\SCA}{\mathscr A}
\newcommand{\SCB}{\mathscr B}
\newcommand{\SCC}{\mathscr C}
\newcommand{\SCD}{\mathscr D}
\newcommand{\SCE}{\mathscr E}
\newcommand{\SCF}{\mathscr F}
\newcommand{\SCG}{\mathscr G}
\newcommand{\SCH}{\mathscr H}

% Mathfrak primes
\newcommand{\km}{\mathfrak m}
\newcommand{\kp}{\mathfrak p}
\newcommand{\kq}{\mathfrak q}

% number sets
\newcommand{\RR}[1][]{\ensuremath{\ifstrempty{#1}{\mathbb{R}}{\mathbb{R}^{#1}}}}
\newcommand{\NN}[1][]{\ensuremath{\ifstrempty{#1}{\mathbb{N}}{\mathbb{N}^{#1}}}}
\newcommand{\ZZ}[1][]{\ensuremath{\ifstrempty{#1}{\mathbb{Z}}{\mathbb{Z}^{#1}}}}
\newcommand{\QQ}[1][]{\ensuremath{\ifstrempty{#1}{\mathbb{Q}}{\mathbb{Q}^{#1}}}}
\newcommand{\CC}[1][]{\ensuremath{\ifstrempty{#1}{\mathbb{C}}{\mathbb{C}^{#1}}}}
\newcommand{\PP}[1][]{\ensuremath{\ifstrempty{#1}{\mathbb{P}}{\mathbb{P}^{#1}}}}
\newcommand{\HH}[1][]{\ensuremath{\ifstrempty{#1}{\mathbb{H}}{\mathbb{H}^{#1}}}}
\newcommand{\FF}[1][]{\ensuremath{\ifstrempty{#1}{\mathbb{F}}{\mathbb{F}^{#1}}}}
% expected value
\newcommand{\EE}{\ensuremath{\mathbb{E}}}
\newcommand{\charin}{\text{ char }}
\DeclareMathOperator{\sign}{sign}
\DeclareMathOperator{\Aut}{Aut}
\DeclareMathOperator{\Inn}{Inn}
\DeclareMathOperator{\Syl}{Syl}
\DeclareMathOperator{\Gal}{Gal}
\DeclareMathOperator{\GL}{GL} % General linear group
\DeclareMathOperator{\SL}{SL} % Special linear group

%---------------------------------------
% BlackBoard Math Fonts :-
%---------------------------------------

%Captital Letters
\newcommand{\bbA}{\mathbb{A}}	\newcommand{\bbB}{\mathbb{B}}
\newcommand{\bbC}{\mathbb{C}}	\newcommand{\bbD}{\mathbb{D}}
\newcommand{\bbE}{\mathbb{E}}	\newcommand{\bbF}{\mathbb{F}}
\newcommand{\bbG}{\mathbb{G}}	\newcommand{\bbH}{\mathbb{H}}
\newcommand{\bbI}{\mathbb{I}}	\newcommand{\bbJ}{\mathbb{J}}
\newcommand{\bbK}{\mathbb{K}}	\newcommand{\bbL}{\mathbb{L}}
\newcommand{\bbM}{\mathbb{M}}	\newcommand{\bbN}{\mathbb{N}}
\newcommand{\bbO}{\mathbb{O}}	\newcommand{\bbP}{\mathbb{P}}
\newcommand{\bbQ}{\mathbb{Q}}	\newcommand{\bbR}{\mathbb{R}}
\newcommand{\bbS}{\mathbb{S}}	\newcommand{\bbT}{\mathbb{T}}
\newcommand{\bbU}{\mathbb{U}}	\newcommand{\bbV}{\mathbb{V}}
\newcommand{\bbW}{\mathbb{W}}	\newcommand{\bbX}{\mathbb{X}}
\newcommand{\bbY}{\mathbb{Y}}	\newcommand{\bbZ}{\mathbb{Z}}

%---------------------------------------
% MathCal Fonts :-
%---------------------------------------

%Captital Letters
\newcommand{\mcA}{\mathcal{A}}	\newcommand{\mcB}{\mathcal{B}}
\newcommand{\mcC}{\mathcal{C}}	\newcommand{\mcD}{\mathcal{D}}
\newcommand{\mcE}{\mathcal{E}}	\newcommand{\mcF}{\mathcal{F}}
\newcommand{\mcG}{\mathcal{G}}	\newcommand{\mcH}{\mathcal{H}}
\newcommand{\mcI}{\mathcal{I}}	\newcommand{\mcJ}{\mathcal{J}}
\newcommand{\mcK}{\mathcal{K}}	\newcommand{\mcL}{\mathcal{L}}
\newcommand{\mcM}{\mathcal{M}}	\newcommand{\mcN}{\mathcal{N}}
\newcommand{\mcO}{\mathcal{O}}	\newcommand{\mcP}{\mathcal{P}}
\newcommand{\mcQ}{\mathcal{Q}}	\newcommand{\mcR}{\mathcal{R}}
\newcommand{\mcS}{\mathcal{S}}	\newcommand{\mcT}{\mathcal{T}}
\newcommand{\mcU}{\mathcal{U}}	\newcommand{\mcV}{\mathcal{V}}
\newcommand{\mcW}{\mathcal{W}}	\newcommand{\mcX}{\mathcal{X}}
\newcommand{\mcY}{\mathcal{Y}}	\newcommand{\mcZ}{\mathcal{Z}}


%---------------------------------------
% Bold Math Fonts :-
%---------------------------------------

%Captital Letters
\newcommand{\bmA}{\boldsymbol{A}}	\newcommand{\bmB}{\boldsymbol{B}}
\newcommand{\bmC}{\boldsymbol{C}}	\newcommand{\bmD}{\boldsymbol{D}}
\newcommand{\bmE}{\boldsymbol{E}}	\newcommand{\bmF}{\boldsymbol{F}}
\newcommand{\bmG}{\boldsymbol{G}}	\newcommand{\bmH}{\boldsymbol{H}}
\newcommand{\bmI}{\boldsymbol{I}}	\newcommand{\bmJ}{\boldsymbol{J}}
\newcommand{\bmK}{\boldsymbol{K}}	\newcommand{\bmL}{\boldsymbol{L}}
\newcommand{\bmM}{\boldsymbol{M}}	\newcommand{\bmN}{\boldsymbol{N}}
\newcommand{\bmO}{\boldsymbol{O}}	\newcommand{\bmP}{\boldsymbol{P}}
\newcommand{\bmQ}{\boldsymbol{Q}}	\newcommand{\bmR}{\boldsymbol{R}}
\newcommand{\bmS}{\boldsymbol{S}}	\newcommand{\bmT}{\boldsymbol{T}}
\newcommand{\bmU}{\boldsymbol{U}}	\newcommand{\bmV}{\boldsymbol{V}}
\newcommand{\bmW}{\boldsymbol{W}}	\newcommand{\bmX}{\boldsymbol{X}}
\newcommand{\bmY}{\boldsymbol{Y}}	\newcommand{\bmZ}{\boldsymbol{Z}}
%Small Letters
\newcommand{\bma}{\boldsymbol{a}}	\newcommand{\bmb}{\boldsymbol{b}}
\newcommand{\bmc}{\boldsymbol{c}}	\newcommand{\bmd}{\boldsymbol{d}}
\newcommand{\bme}{\boldsymbol{e}}	\newcommand{\bmf}{\boldsymbol{f}}
\newcommand{\bmg}{\boldsymbol{g}}	\newcommand{\bmh}{\boldsymbol{h}}
\newcommand{\bmi}{\boldsymbol{i}}	\newcommand{\bmj}{\boldsymbol{j}}
\newcommand{\bmk}{\boldsymbol{k}}	\newcommand{\bml}{\boldsymbol{l}}
\newcommand{\bmm}{\boldsymbol{m}}	\newcommand{\bmn}{\boldsymbol{n}}
\newcommand{\bmo}{\boldsymbol{o}}	\newcommand{\bmp}{\boldsymbol{p}}
\newcommand{\bmq}{\boldsymbol{q}}	\newcommand{\bmr}{\boldsymbol{r}}
\newcommand{\bms}{\boldsymbol{s}}	\newcommand{\bmt}{\boldsymbol{t}}
\newcommand{\bmu}{\boldsymbol{u}}	\newcommand{\bmv}{\boldsymbol{v}}
\newcommand{\bmw}{\boldsymbol{w}}	\newcommand{\bmx}{\boldsymbol{x}}
\newcommand{\bmy}{\boldsymbol{y}}	\newcommand{\bmz}{\boldsymbol{z}}

%---------------------------------------
% Scr Math Fonts :-
%---------------------------------------

\newcommand{\sA}{{\mathscr{A}}}   \newcommand{\sB}{{\mathscr{B}}}
\newcommand{\sC}{{\mathscr{C}}}   \newcommand{\sD}{{\mathscr{D}}}
\newcommand{\sE}{{\mathscr{E}}}   \newcommand{\sF}{{\mathscr{F}}}
\newcommand{\sG}{{\mathscr{G}}}   \newcommand{\sH}{{\mathscr{H}}}
\newcommand{\sI}{{\mathscr{I}}}   \newcommand{\sJ}{{\mathscr{J}}}
\newcommand{\sK}{{\mathscr{K}}}   \newcommand{\sL}{{\mathscr{L}}}
\newcommand{\sM}{{\mathscr{M}}}   \newcommand{\sN}{{\mathscr{N}}}
\newcommand{\sO}{{\mathscr{O}}}   \newcommand{\sP}{{\mathscr{P}}}
\newcommand{\sQ}{{\mathscr{Q}}}   \newcommand{\sR}{{\mathscr{R}}}
\newcommand{\sS}{{\mathscr{S}}}   \newcommand{\sT}{{\mathscr{T}}}
\newcommand{\sU}{{\mathscr{U}}}   \newcommand{\sV}{{\mathscr{V}}}
\newcommand{\sW}{{\mathscr{W}}}   \newcommand{\sX}{{\mathscr{X}}}
\newcommand{\sY}{{\mathscr{Y}}}   \newcommand{\sZ}{{\mathscr{Z}}}


%---------------------------------------
% Math Fraktur Font
%---------------------------------------

%Captital Letters
\newcommand{\mfA}{\mathfrak{A}}	\newcommand{\mfB}{\mathfrak{B}}
\newcommand{\mfC}{\mathfrak{C}}	\newcommand{\mfD}{\mathfrak{D}}
\newcommand{\mfE}{\mathfrak{E}}	\newcommand{\mfF}{\mathfrak{F}}
\newcommand{\mfG}{\mathfrak{G}}	\newcommand{\mfH}{\mathfrak{H}}
\newcommand{\mfI}{\mathfrak{I}}	\newcommand{\mfJ}{\mathfrak{J}}
\newcommand{\mfK}{\mathfrak{K}}	\newcommand{\mfL}{\mathfrak{L}}
\newcommand{\mfM}{\mathfrak{M}}	\newcommand{\mfN}{\mathfrak{N}}
\newcommand{\mfO}{\mathfrak{O}}	\newcommand{\mfP}{\mathfrak{P}}
\newcommand{\mfQ}{\mathfrak{Q}}	\newcommand{\mfR}{\mathfrak{R}}
\newcommand{\mfS}{\mathfrak{S}}	\newcommand{\mfT}{\mathfrak{T}}
\newcommand{\mfU}{\mathfrak{U}}	\newcommand{\mfV}{\mathfrak{V}}
\newcommand{\mfW}{\mathfrak{W}}	\newcommand{\mfX}{\mathfrak{X}}
\newcommand{\mfY}{\mathfrak{Y}}	\newcommand{\mfZ}{\mathfrak{Z}}
%Small Letters
\newcommand{\mfa}{\mathfrak{a}}	\newcommand{\mfb}{\mathfrak{b}}
\newcommand{\mfc}{\mathfrak{c}}	\newcommand{\mfd}{\mathfrak{d}}
\newcommand{\mfe}{\mathfrak{e}}	\newcommand{\mff}{\mathfrak{f}}
\newcommand{\mfg}{\mathfrak{g}}	\newcommand{\mfh}{\mathfrak{h}}
\newcommand{\mfi}{\mathfrak{i}}	\newcommand{\mfj}{\mathfrak{j}}
\newcommand{\mfk}{\mathfrak{k}}	\newcommand{\mfl}{\mathfrak{l}}
\newcommand{\mfm}{\mathfrak{m}}	\newcommand{\mfn}{\mathfrak{n}}
\newcommand{\mfo}{\mathfrak{o}}	\newcommand{\mfp}{\mathfrak{p}}
\newcommand{\mfq}{\mathfrak{q}}	\newcommand{\mfr}{\mathfrak{r}}
\newcommand{\mfs}{\mathfrak{s}}	\newcommand{\mft}{\mathfrak{t}}
\newcommand{\mfu}{\mathfrak{u}}	\newcommand{\mfv}{\mathfrak{v}}
\newcommand{\mfw}{\mathfrak{w}}	\newcommand{\mfx}{\mathfrak{x}}
\newcommand{\mfy}{\mathfrak{y}}	\newcommand{\mfz}{\mathfrak{z}}


\title{\Huge{Física Estadistica}\\Tarea 3}
\author{\huge{Sergio Montoya} \\ 202112171}
\date{}

\begin{document}

\maketitle
\newpage% or \cleardoublepage
% \pdfbookmark[<level>]{<title>}{<dest>}
\pdfbookmark[section]{\contentsname}{toc}
\tableofcontents
\pagebreak

% PUNTO 1 %%%%%%%%%%%%%%%%%%%%%%%%%%%%%%%%%%%%%%%%%%%%%%%%%%%%%%%%%%%%%%%%%%%%
\chapter{}

\section{}
Simplemente desarrollemos como:
\begin{align*}
	- \frac{\partial}{\partial \beta} ln \left( Z \right)&= - \frac{\partial}{\partial \beta} \ln \left( \sum_{i} e^{-\beta E_i} \right)\\
	&= - \frac{\sum_{i} \frac{\partial}{\partial \beta} e^{-\beta E_i}}{\sum_{i} e^{-\beta E_i}}\\
	&= - \frac{\sum_{i} - E_i e^{-\beta E_i}}{\sum_{i} e^{-\beta E_i}}\\
	&= \frac{\sum_{i} E_i e^{-\beta E_i}}{\sum_{i} e^{-\beta E_i}}\\
	&= \left< E_i \right>\\
	&= U
\end{align*}

\section{}

\begin{align*}
	F &= - \frac{1}{\beta} \ln Z\\
	S &= - \left( \frac{\partial F}{\partial T} \right)_{N, V}\\
	&= - \frac{\partial}{\partial T} \left( - \frac{1}{\beta} \ln Z \right)\\
	&= \frac{\partial}{\partial T} \left( \frac{1}{\beta} \ln Z \right)\\
	&= \frac{\partial}{\partial T} \left( kT \ln Z \right)\\
	&= k \frac{\partial}{\partial T} \left( T \ln Z \right)\\
	&= k \left( \ln Z + T \frac{\partial}{\partial T} \ln Z \right) \\
	\frac{\partial \ln Z}{\partial \beta} &= \frac{\partial \ln Z}{\partial \beta} \frac{\partial \beta}{\partial T}\\
	\frac{\partial \beta}{\partial T} &=  - \frac{1}{k T^2}\\
	S &= k \left( \ln Z - \frac{T}{k T^2} \frac{\partial \ln Z}{\partial \beta} \right) \\
	&= k \left( \ln Z - \frac{1}{k T} \frac{\partial \ln Z}{\partial \beta} \right) \\
	&= k \left( \ln Z - \frac{1}{k T} \frac{\partial \ln Z}{\partial \beta} \right) \\
	&= k \left( \ln Z - \beta \frac{\partial \ln Z}{\partial \beta} \right) \\
\end{align*}

\section{}

\begin{align*}
	C_V &= \left( \frac{\partial E}{\partial T} \right)_{N,V}\\
	&= \frac{\partial}{\partial T} \left( - \frac{\partial}{\partial \beta} \ln Z \right)\\
	\frac{\partial}{\partial \beta} &= \frac{\partial}{\partial \beta}\frac{\partial \beta}{\partial T}\\
	\frac{\partial}{\partial \beta} &=  - \frac{\partial}{\partial \beta}\frac{1}{kT^2}\\
	C_V &= - \frac{\partial}{\partial \beta} \frac{1}{kT^2} \left(- \frac{\partial}{\partial \beta} \ln Z \right)\\
	&=\frac{\partial}{\partial \beta} \frac{1}{kT^2} \frac{k}{k} \left(\frac{\partial}{\partial\beta} \ln Z \right)\\
	&=\frac{\partial}{\partial \beta} \frac{1}{k^2T^2} k \left(\frac{\partial}{\partial \beta} \ln Z \right)\\
	&= k\beta^2 \left(\frac{\partial^2}{\partial \beta^2} \ln Z \right)
\end{align*}


% PUNTO 2 %%%%%%%%%%%%%%%%%%%%%%%%%%%%%%%%%%%%%%%%%%%%%%%%%%%%%%%%%%%%%%%%%%%%
\chapter{}

Sabemos que para este modelo, la probabilidad \[
	P_i = g_i e^{- \frac{E_i}{k_B T}}
\]

pero el propio enunciado nos da el degeneramiento. Ahora bien, es importante notar que la degeneración total para el estado $n = 2$ es $8$ dada por el enunciado. Sin embargo, esto esta compuesto de dos estados en donde solamente 1 nos interesa (pues tenemos 2s y 2p) por lo tanto partamos estos estados notando que 2s tiene una degeneración de $2$. Por lo tanto juntando todo esto la probabilidad simplemente se nos reduce a:
\begin{align*}
	P(2p) &= \frac{g_{2p} e^{- \frac{E_2}{k T}}}{2 e^{- \frac{E_i}{kT}} + 8 e^{-\frac{E_2}{k T}}}
\end{align*}


% PUNTO 3 %%%%%%%%%%%%%%%%%%%%%%%%%%%%%%%%%%%%%%%%%%%%%%%%%%%%%%%%%%%%%%%%%%%%
\chapter{}

\section{}

En este caso partimo de:
\begin{align*}
	Z &= \sum_{i = 0}^{\infty} e^{-\beta E_i}\\
	Z_1 &= \sum_{i = 0}^{\infty} e^{-\beta \hbar \omega \left( n + \frac{1}{2} \right)}\\
	&= \sum_{i = 0}^{\infty} e^{-\beta \hbar \omega n}  e^{- \beta \hbar \omega\frac{1}{2}}\\
	&= e^{- \beta \hbar \omega\frac{1}{2}} \sum_{i = 0}^{\infty} e^{-\beta \hbar \omega n}  \\
	&= e^{- \beta \hbar \omega\frac{1}{2}} \sum_{i = 0}^{\infty} \left(e^{-\beta \hbar \omega }\right)^n  \\
	\sum_{n=0}^{\infty} y^n &= \frac{1}{1 - y} \text{ Serie Geometrica}\\
	Z_1 &= e^{-\beta \hbar \omega \frac{1}{2}} \frac{1}{1 - e^{-\beta\hbar\omega}}\\
	&=  \frac{1}{e^{-\beta \hbar \omega \frac{1}{2}}\left(1 - e^{-\beta\hbar\omega}\right)}\\
	&=  \frac{1}{e^{-\beta \hbar \omega \frac{1}{2}} - e^{-\frac{1}{2}\beta\hbar\omega}}\\
	&=  \frac{1}{e^{-\beta \hbar \omega \frac{1}{2}} - e^{-\frac{1}{2}\beta\hbar\omega}} \frac{2}{2} \\
	&=  \frac{2}{e^{-\beta \hbar \omega \frac{1}{2}} - e^{-\frac{1}{2}\beta\hbar\omega}} \frac{1}{2} \\
	&=  \frac{1}{\sinh\left( \frac{\beta \hbar \omega}{2} \right) 2}  \\
	&=  \left[{\sinh\left( \frac{\beta \hbar \omega}{2} \right) 2}\right]^{-1} \square \\
\end{align*}

\section{}

En este caso :
\begin{align*}
		Z_N &= \prod_v Z_1\\
		Z_N &= \prod_{v} \frac{1}{2\sinh \left( \frac{\beta \hbar \omega(v)}{2} \right)}\\
\end{align*}

Ahora tenemos la energia libre de Helmholtz como:
\begin{align*}
	F &= - \frac{1}{\beta} \ln Z_N\\
	&= - \frac{1}{\beta} \ln \prod_v Z_1\\
	\ln \left( a * b \right) &= \ln(a) + \ln(b) \implies \ln \prod_n a_n = \sum_n \ln a_n\\
	F &= - \frac{1}{\beta} \sum_{v} \ln Z_1\\
	Z_1 &= e^{-\beta \hbar \omega \frac{1}{2}} \frac{1}{1 - e^{-\beta\hbar\omega}}\\
	F &= - \frac{1}{\beta} \sum_{v} \ln e^{-\beta \hbar \omega \frac{1}{2}} \frac{1}{1 - e^{-\beta\hbar\omega}} \\
	&= - \frac{1}{\beta} \sum_{v} \left(\ln e^{-\beta \hbar \omega \frac{1}{2}} - \ln 1 - e^{-\beta\hbar\omega}\right) \\
	&= - \frac{1}{\beta} \sum_{v} \left( -\beta \hbar \omega \frac{1}{2} - \ln 1 - e^{-\beta\hbar\omega}\right) \\
	&= \sum_{v} \left(\frac{1}{\beta}\beta \hbar \omega \frac{1}{2} + \frac{1}{\beta}\ln \left(1 - e^{-\beta\hbar\omega}\right)\right) \\
	&= \sum_{v} \left( \hbar \omega \frac{1}{2} + \frac{1}{\beta}\ln \left(1 - e^{-\beta\hbar\omega}\right)\right) \\
	&= \sum_{v} F_v\square
\end{align*}

\section{}

En este caso partimos de
\begin{align*}
	U &= - \frac{\partial}{\partial \beta} \ln Z\\
	&= - \frac{\partial}{\partial \beta} \ln \prod_v Z_1\\
	&= - \frac{\partial}{\partial \beta} \sum_v \ln Z_1\\
	&=\sum_v - \frac{\partial}{\partial \beta} \ln Z_1\\
\end{align*}

Ahora definimos: \[
	u_v = - \frac{\partial}{\partial \beta} \ln Z_1\\
\]

Con lo que podemos desarrollar como:
\begin{align*}
	\ln Z_1 &= \ln \left( e^{-\beta \hbar \omega \frac{1}{2}} \frac{1}{1 - e^{-\beta\hbar\omega}} \right)\\
	\ln Z_1 &= \ln \left( e^{-\beta \hbar \omega \frac{1}{2}}\right) + \ln \left( \frac{1}{1 - e^{-\beta\hbar\omega}} \right)\\
	\ln Z_1 &= -\beta \hbar \omega \frac{1}{2} - \ln \left( 1 - e^{-\beta\hbar\omega} \right)
\end{align*}

Que ahora derivamos parcialmente respecto a $\beta$ con lo que tenemos:
\begin{align*}
	u_v &= - \frac{\partial}{\partial \beta} \ln Z_1\\
	&= - \frac{\partial}{\partial \beta} \left( -\beta \hbar \omega \frac{1}{2} - \ln \left( 1 - e^{-\beta\hbar\omega} \right) \right) \\
	&= \frac{\partial}{\partial \beta} \beta \hbar \omega \frac{1}{2} + \frac{\partial}{\partial \beta}\ln \left( 1 - e^{-\beta\hbar\omega} \right) \\
	&= \hbar \omega \frac{1}{2} + \left( \frac{1}{1 - e^{-\beta\hbar\omega}} \right) \frac{\partial - e^{-\beta\hbar\omega}}{\partial \beta}\\
	&= \hbar \omega \frac{1}{2} + \left( \frac{1}{1 - e^{-\beta\hbar\omega}} \right) \hbar\omega e^{-\beta\hbar\omega}\\
	&= \hbar \omega \frac{1}{2} + \left( \frac{e^{-\beta\hbar\omega}}{1 - e^{-\beta\hbar\omega}} \right) \hbar\omega \\
	&= \hbar \omega \frac{1}{2} + \left( \frac{1}{e^{\beta\hbar\omega}\left(1 - e^{-\beta\hbar\omega}\right)} \right) \hbar\omega \\
	&= \hbar \omega \frac{1}{2} + \left( \frac{1}{e^{\beta\hbar\omega} - 1} \right) \hbar\omega \\
	&= \hbar\omega\left[\frac{1}{2} + \left( \frac{1}{e^{\beta\hbar\omega} - 1} \right)\right] \\
	&= \hbar\omega\left[\frac{1}{2} + n_v(\beta)\right]\square \\
\end{align*}

\section{}

En este caso partimos desde el punto anterior
\begin{align*}
	U &=\frac{Na}{2\pi}\int_{-\frac{\pi}{a}}^{+\frac{\pi}{a}}\left[ n_v(\beta) + \frac{1}{2} \right]\hbar \omega(v) dv\\
	U &=\frac{Na}{2\pi}\int_{-\frac{\pi}{a}}^{+\frac{\pi}{a}}\left[ n_v(\beta) + \frac{1}{2} \right]\hbar C_L \left| v \right| dv\\
	&=\frac{Na}{\pi}\int_{0}^{\frac{\pi}{a}}\left[ n_v(\beta) + \frac{1}{2} \right]\hbar C_L v dv\\
	&=\frac{Na}{\pi}\int_{0}^{\frac{\pi}{a}}\left[ \frac{1}{e^{\beta\hbar c_L v} - 1} + \frac{1}{2} \right]\hbar C_L v dv\\
	&=\frac{Na}{\pi}\int_{0}^{\frac{\pi}{a}}\left[ \frac{\hbar C_L v}{e^{\beta\hbar c_L v} - 1} + \frac{C_L v}{2} \right]\hbar dv\\
	x &= \beta \hbar C_L v\\
	dv &= \frac{dx}{\beta \hbar c_L}\\
	U &=\frac{Na}{\pi}\int_{0}^{\frac{\beta\hbar c_L\pi}{a}}\left[ \frac{\frac{x}{\beta}}{e^{x} - 1} + \frac{\frac{x}{\beta}}{2} \right] \frac{dx}{\beta\hbar c_L} \\
	U &=\frac{Na}{\pi}\int_{0}^{\frac{\beta\hbar c_L\pi}{a}}\left[ \frac{x}{e^{x} - 1} + \frac{x}{2} \right] \frac{dx}{\beta^2\hbar c_L} \\
	U &=\frac{Na}{\beta^2\hbar c_L\pi}\int_{0}^{\frac{\beta\hbar c_L\pi}{a}}\left[ \frac{x}{e^{x} - 1} + \frac{x}{2} \right] dx \\
	\Theta &= \frac{\hbar c_L\pi}{k a}\\
	\frac{\Theta}{T} &= \frac{\beta\hbar c_L\pi}{a}\\
	U &=\frac{Na}{\beta^2\hbar c_L\pi}\int_{0}^{\frac{\Theta}{T}}\left[ \frac{x}{e^{x} - 1} + \frac{x}{2} \right] dx \\
\end{align*}

Con lo que podemos  complementar como 
\begin{align*}
	U &=\frac{Na}{\beta^2\hbar c_L\pi}\int_{0}^{\frac{\Theta}{T}}\frac{x}{e^{x} - 1} dx +\int_{0}^{\frac{\Theta}{T}} \frac{x}{2} dx \\
	U &=\frac{Na}{\beta^2\hbar c_L\pi}\int_{0}^{\frac{\Theta}{T}}\frac{x}{e^{x} - 1} dx + \frac{1}{4}\left( \frac{\Theta}{T} \right)^2 \\
	\beta &= \frac{1}{kT}\\
	U &=\frac{k^2T^2 N a}{\hbar c_L\pi}\int_{0}^{\frac{\Theta}{T}}\frac{x}{e^{x} - 1} dx + \frac{1}{4}\left( \frac{\Theta}{T} \right)^2 \\
	U &=N k T^2 \frac{k a}{\hbar c_L\pi}\int_{0}^{\frac{\Theta}{T}}\frac{x}{e^{x} - 1} dx + \frac{1}{4}\left( \frac{\Theta}{T} \right)^2 \\
	U &=N k T^2 \frac{1}{\Theta} \int_{0}^{\frac{\Theta}{T}}\frac{x}{e^{x} - 1} dx + \frac{1}{4}\left( \frac{\Theta}{T} \right)^2 \\
	U &=N k T^2 \frac{1}{\Theta} \left[\int_{0}^{\frac{\Theta}{T}}\frac{x}{e^{x} - 1} dx + \frac{1}{4}\left( \frac{\Theta}{T} \right)^2\right] \\
	U &=N k T^2 \frac{1}{\Theta} \left[\int_{0}^{\frac{\Theta}{T}}\frac{x}{e^{x} - 1} dx + \frac{1}{4}\left( \frac{\Theta}{T} \right)^2\right] \\
	U &=N k T^2 \frac{1}{\Theta} \left[\int_{0}^{\frac{\Theta}{T}}\frac{x}{e^{x} - 1} dx + \frac{1}{4}\left( \frac{\Theta}{T} \right)^2\right] \\
	U &=N k \left[\frac{T^2}{\Theta} \int_{0}^{\frac{\Theta}{T}}\frac{x}{e^{x} - 1} dx + \frac{1}{4}\frac{T^2}{\Theta}\left( \frac{\Theta}{T} \right)^2\right] \\
	U &=N k \left[\frac{T^2}{\Theta} \frac{\Theta}{\Theta} \int_{0}^{\frac{\Theta}{T}}\frac{x}{e^{x} - 1} dx + \frac{1}{4}\frac{T^2}{\Theta}\left( \frac{\Theta}{T} \right)^2\right] \\
	U &=N k \left[\Theta\frac{T^2}{\Theta^2} \int_{0}^{\frac{\Theta}{T}}\frac{x}{e^{x} - 1} dx + \frac{1}{4}\Theta\right] \\
	U &=N k \Theta\left[\frac{T^2}{\Theta^2} \int_{0}^{\frac{\Theta}{T}}\frac{x}{e^{x}- 1} dx + \frac{1}{4}\right]\\
	U &=N k \Theta\left[\frac{1}{4} +\left(\frac{T}{\Theta}\right)^2 \int_{0}^{\frac{\Theta}{T}}\frac{x}{e^{x}- 1} dx\right]
\end{align*}


\section{}

Dado que $\Theta >> T$ entonces hagamos $\frac{\Theta}{T} \to \infty$ Con lo cual desarrollemos primero la integral:
\begin{align*}
	\int_{0}^{\frac{\Theta}{T}}\frac{x}{e^{x}- 1} dx &\to \int_{0}^{\infty}\frac{x}{e^{x}- 1} dx\\
	&= \int_{0}^{\infty}\frac{x}{e^{x}- 1} dx\\
	&= \int_{0}^{\infty} x \sum_{n= 0}^{\infty} e^{-nx} dx\\
	&= \sum_{n = 0}^{\infty} \int_{0}^{\infty}  x e^{-nx} dx\\
	u &= x;\ du = dx\\
	dv &= e^{-nx}dx;\ v = - \frac{1}{n} e^{-nx}\\
	\int udv &= uv -\int vdu\\
	\int_{0}^{\infty}  x e^{-nx} dx &= \left.- \frac{x}{n e^{nx}}\right|_0^\infty + \frac{1}{n} \int_0^\infty e^{-nx} dx\\
	\left.- \frac{x}{n e^{nx}}\right|_0^\infty &= \lim_{x \to \infty}- \frac{x}{n e^{nx}} - 0\\
	\left.- \frac{x}{n e^{nx}}\right|_0^\infty &= 0 - 0 = 0\\
	\int_{0}^{\infty}  x e^{-nx} dx &= \frac{1}{n} \int_0^\infty e^{-nx} dx\\
	\int_{0}^{\infty}  x e^{-nx} dx &= \frac{1}{n} \frac{1}{n}\\
	\int_{0}^{\infty}  x e^{-nx} dx &= \frac{1}{n^2}\\
	\sum_{n = 0}^{\infty} \int_{0}^{\infty}  x e^{-nx} dx &= \sum_{n = 0}^{\infty} \frac{1}{n^2}\\
\end{align*}

Esta es una serie conocida como \textit{Basel Problem} y tiene como resultado $\frac{\pi^2}{6}$. Ahora dado que tenemos
la integral podemos volver a la expresión completa de U
\begin{align*}
	U &=N k \Theta\left[\frac{1}{4} +\left(\frac{T}{\Theta}\right)^2 \int_{0}^{\frac{\Theta}{T}}\frac{x}{e^{x}- 1} x\right]\\
	&=N k \Theta\left[\frac{1}{4} +\left(\frac{T}{\Theta}\right)^2 \frac{\pi^2}{6} \right]
\end{align*}

Ahora con esto podemos simplemente derivar para obtener $C_V$
\begin{align*}
	C_V &= \frac{\partial U}{\partial T}\\
	&= \frac{\partial}{\partial T} N k \Theta\left[\frac{1}{4} +\left(\frac{T}{\Theta}\right)^2 \frac{\pi^2}{6} \right]\\
	&=  N k \Theta\left[\frac{\partial}{\partial T}\left(\frac{T}{\Theta}\right)^2 \frac{\pi^2}{6} \right]\\
	&=  N k \Theta\left[ 2 T\left(\frac{1}{\Theta}\right)^2 \frac{\pi^2}{6} \right]\\
	&=  N k \left[ 2 \left(\frac{T}{\Theta}\right) \frac{\pi^2}{6} \right]\\
	&=  N k \left[ \left(\frac{T}{\Theta}\right) \frac{\pi^2}{3} \right]\\
	&=  N k \left(\frac{T}{\Theta}\right) \frac{\pi^2}{3}\square\\
\end{align*}


% PUNTO 4 %%%%%%%%%%%%%%%%%%%%%%%%%%%%%%%%%%%%%%%%%%%%%%%%%%%%%%%%%%%%%%%%%%%%
\chapter{}

\section{}

Para esto partimos con:
\begin{align*}
	Z &= \frac{1}{N! h^{3N}} \int e^{-\beta \sum_{i = 0}^{N} p_i c} d^{3N}q d^{3N}p\\
	&= \frac{1}{N! h^{3N}} \int e^{-\beta \sum_{i = 0}^{N} p_i c} d^{3N}p \int d^{3N}q\\
	&= \frac{1}{N! h^{3N}} \int e^{-\beta \sum_{i = 0}^{N} p_i c} d^{3N}p V^N\\
	&= \frac{V^N}{N! h^{3N}} \int e^{-\beta \sum_{i = 0}^{N} p_i c} d^{3N}p\\
	&= \frac{V^N}{N! h^{3N}} \left[\int e^{-\beta \sum_{i = 0}^{N} p_i c} p_i^2 \sin\theta dpi d\theta d\phi\right]^N\\
	&= \frac{V^N}{N! h^{3N}} \left[\int_0^\infty e^{-\beta \sum_{i = 0}^{N} p_i c} p_i^2 dpi \int_0^\pi \sin\theta d\theta \int_0^{2\pi} d\phi\right]^N\\
	&= \frac{V^N}{N! h^{3N}} \left[\int_0^\infty e^{-\beta \sum_{i = 0}^{N} p_i c} p_i^2 dpi 4\pi\right]^N\\
	&= \frac{1}{N!} \left[\int_0^\infty e^{-\beta \sum_{i = 0}^{N} p_i c} \frac{V 4pi}{h^3}p_i^2 dpi\right]^N\\
	\int_0^\infty e^{-\beta \sum_{i = 0}^{N} p_i c} \frac{V 4pi}{h^3}p_i^2 dpi &= \frac{8\pi V}{h^3} \frac{1}{\beta^3 C^3}\\
	Z &= \frac{1}{N!}\left( \frac{8\pi V}{h^3} \frac{1}{\beta^3 C^3} \right)^N\\
	&= \frac{1}{N!} \left( 8\pi V \left( \frac{kT}{hC} \right)^3\right)^N
\end{align*}

\section{}

Para este caso
\begin{align*}
	F &= - \frac{1}{\beta} \ln \left( Z \right)\\
	&= - \frac{1}{\beta} \ln \left( \frac{1}{N!} \left( 8\pi V \left( \frac{kT}{hC} \right)^3\right)^N\right)\\
	&= - \frac{1}{\beta} \left( N \ln \left( 8\pi V \left( \frac{kT}{hC} \right)^3\right) - \ln(N!) \right)\\
	&= - \frac{1}{\beta} \left( N \ln \left( 8\pi V \left( \frac{kT}{hC} \right)^3\right) - N\ln(N) + N \right)\\
	&= - \frac{N}{\beta} \left( \ln \left( 8\pi V \left( \frac{kT}{hC} \right)^3\right) - \ln(N) + 1 \right)\\
	&= - \frac{N}{\beta} \left( \ln \left( \frac{8\pi V}{N} \left( \frac{kT}{hC} \right)^3\right) + 1 \right)\\
	&= - kT N \left( \ln \left( \frac{8\pi V}{N} \left( \frac{kT}{hC} \right)^3\right) + 1 \right)\\
\end{align*}

\section{}

Tenemos:
\begin{align*}
	S &= - \left( \frac{\partial F}{\partial T} \right)_{N, V}\\
	&= - \left( \frac{\partial}{\partial T} - kT N \left( \ln \left( \frac{8\pi V}{N} \left( \frac{kT}{hC} \right)^3\right) + 1 \right)\right)_{N, V}\\
	&= -kN \left( \ln \left( \frac{8\pi V}{N} \left( \frac{kT}{hC} \right)^3 \right) + 1\right) - kN\cancel{T} \left( \cancel{\frac{N}{8\pi V}} \cancel{\left( \frac{hc}{kT} \right)^3} \right) \cancel{\frac{8\pi V}{N}} 3 \left( \cancel{\frac{k}{hC}} \right)^3 \cancel{T^2}\\
	&= -kN \left( \ln \left( \frac{8\pi V}{N} \left( \frac{kT}{hC} \right)^3 \right) + 1 + 3 \right)
	&= -kN \left( \ln \left( \frac{8\pi V}{N} \left( \frac{kT}{hC} \right)^3 \right) + 4 \right)
\end{align*}

\section{}

En este caso
\begin{align*}
	U &= - \frac{\partial}{\partial \beta} \ln Z\\
	&= - \frac{\partial}{\partial \beta} \left( \ln \left( \frac{1}{N!} \left( \frac{8\pi V}{h^3} \frac{1}{\beta^3 C^3} \right) \right)^N \right)\\
	&= - \frac{\partial}{\partial \beta} \left(\ln \left( \frac{1}{N!} \right) - N \ln \left( \frac{\beta^3 C^3}{8\pi V} \right)\right)\\
	&= \frac{\partial}{\partial \beta} \left( N \ln \left( \frac{\beta^3 C^3}{8\pi V} \right)\right)\\
	&= N \frac{8\pi V}{\beta^3 C^3} \frac{C^3}{8\pi V} 3 \beta^2\\
	&= \frac{3N}{\beta}\\
	&= 3Nk T
\end{align*}

Ahora nos queda
\begin{align*}
	C_v &= \left( \frac{\partial U}{\partial T} \right)_{N, V}\\
	&= 3Nk
\end{align*}

Por otro lado:
\begin{align*}
	C_p &= \frac{\partial \left( U + PV \right)}{\partial T}\\
	&= \frac{\partial}{\partial T} \left( 3NkT + NkT \right)\\
	&= 4Nk
\end{align*}

Por lo tanto
\begin{align*}
	\gamma &= \frac{C_p}{C_v}\\
	&= \frac{4Nk}{3Nk}\\
	&= \frac{4}{3}
\end{align*}



% PUNTO 5 %%%%%%%%%%%%%%%%%%%%%%%%%%%%%%%%%%%%%%%%%%%%%%%%%%%%%%%%%%%%%%%%%%%%
\chapter{}

\section{}

Partimos desde:
\begin{align*}
	Z &= \frac{1}{h} \int e^{-\beta \frac{p^2}{2m}} dp \int e^{-\beta \left( cq^2 - gq^3 - fq^4 \right)} dq\\
	&= \frac{1}{h} \int e^{-\beta \frac{p^2}{2m}} dp \int e^{-\beta \left( cq^2\right)}e^{-\beta\left(- gq^3 - fq^4 \right)} dq\\
	e^{-\beta\left(- gq^3 - fq^4 \right)} &= \sum_{n = 0}^{\infty} \left(-\beta CU q^3 - Fq^4\right)^n\\
	& = 1 + \beta Cgq^3 + F q^4 + \frac{\beta}{2} g^2 q^6 + \dots\\
	Z&= \gamma \left( \sqrt{\frac{\pi}{\beta C}} + \int_{-\infty}^{\infty} e^{-\beta rq^2} \beta gq^3 dq + \int_{-\infty}^{\infty} e^{-\beta cq^2} \beta F q^4 dq + \int_{-\infty}^{\infty} e^{-\beta cq^2} \frac{1}{2} \beta^2 g^2 q^6 dq\right)\\
	e^{-\beta C q^2}q^4 &= \frac{\partial^2}{\partial q^2} C e^{-\beta Cq^2} \frac{1}{\beta F}\\
	e^{-\beta C q^2}q^6 &= \frac{\partial^3}{\partial q^3} C e^{-\beta Cq^2} \frac{1}{\beta^2 g^2}\\
	\int_{-\infty}^{\infty} e^{-\beta cq^2} q^4 \beta F dq &= \beta F \left( \frac{3}{4} \sqrt{\frac{\pi}{\left( \beta C \right)^2}} \right)\\
	\int_{-\infty}^{\infty} \frac{1}{2} \beta g^2 e^{-\beta Cq^2} q^6 dq &= - \frac{\beta^2 g^2}{2} \left( - \frac{15}{8} \sqrt{\frac{\pi}{\left( \beta C \right)^7}}\right)\\
	Z_1 &= \frac{1}{h} \sqrt{\frac{2m\pi}{\beta C}} \left( \sqrt{\frac{\pi}{\beta C}} + \frac{3}{4} \beta F \sqrt{\frac{\pi}{\left(\beta C\right)^5}} + \frac{15}{16} B^2 g^2 \sqrt{\frac{\pi}{\left( C\beta \right)^2}} \right)\\
	&= \frac{\pi}{\beta h} \sqrt{\frac{2m}{C}} \left( 1 + \frac{3}{4} \frac{F}{\beta C^2} + \frac{15}{16} \frac{g^2}{bC^3} \right)
\end{align*}

\begin{align*}
	U &= - \frac{\partial}{\partial \beta} \ln Z\\
	&= \frac{\partial}{\partial \beta} \left( \ln \left( \frac{\pi}{\beta h} \sqrt{\frac{2m}{C}} \left( 1 + \frac{3}{4} \frac{F}{\beta C^2} + \frac{15}{16} \frac{g^2}{bC^3} \right) \right) \right)\\
	&= \frac{\partial}{\partial \beta} \left( \ln \left( \frac{\pi}{\beta h} \sqrt{\frac{2m}{C}}\right) + \ln \left( 1 + \frac{3}{4} \frac{F}{\beta C^2} + \frac{15}{16} \frac{g^2}{bC^3} \right) \right)\\
	&= \frac{1}{\beta} + \frac{-\frac{3 F}{4\beta^2C^2} + \frac{15 g^2}{16 8^2 c^63}}{1 + \frac{3f}{4\beta C^2} + \frac{\beta g^2}{16 \beta C^3}}\\
	&= \frac{1}{\beta} + \frac{\frac{3F}{4C^2} + \frac{15 g^2}{16 c^3}}{B^2 + \frac{3F\beta}{4C^2} + \frac{15 g^2}{16 C^3}}\\
	U \approx \frac{1}{\beta} + \frac{1}{\beta^2} \left( \frac{3 F}{4C^2} + \frac{15 g^2}{16 C^3} \right)
\end{align*}

\section{}

Tenemos que:
\begin{align*}
	C_V &= \frac{\partial U}{\partial T}\\
	&= \frac{\partial}{\partial T} \left(kT + k^2 T^2 \left( \frac{3 F}{4C^2} + \frac{15 g^2}{16 C^3} \right)\right)\\
	&= k + 2k^2 T \left( \frac{3 F}{4C^2} + \frac{15 g^2}{16 C^3} \right)T\\
\end{align*}

\section{}

Para este caso partimos desde:
\begin{align*}
	\left< q \right> &= \frac{1}{h} \int_{-\infty}^{\infty} e^{-\beta \frac{\beta^2}{2m} + \left( q^i - gq^3 - Fq^4 \right)} q dp dq\\
	&= \frac{\sqrt{\frac{2m \beta}{h z}}} \int_{-\infty}^{\infty} e^{-\beta cq^2} \left( q + \beta \left( g q^4 + Fq^3 \right) \right) + \frac{1}{2} \beta^2 g^2 q^7 dq\\
	&= \frac{\sqrt{\frac{2m\pi}{hz}}} \int_{-\infty}^{\infty} e^{-\beta cq^2} \beta g q^7\\
	&= \frac{\sqrt{\frac{2m\pi}{hz}}} \int_{-\infty}^{\infty} e^{-\beta cq^2} \beta g \left( \frac{3}{4} \sqrt{\frac{\pi}{\beta c^5}} \right)\\
	&= \frac{3}{4} \frac{\pi g}{hz} \sqrt{\frac{2m}{\beta^4 c^5}}\\
	&= \frac{\frac{3\pi g}{4h} \sqrt{\frac{2m}{\beta^4 c^3}}}{\frac{\pi}{\beta h} \sqrt{\frac{2m}{c}} \left( 1 + \frac{3F}{4\beta c^2} + \frac{15 g^2}{16\beta c^2} \right)}\\
\end{align*}
\begin{align*}
	&= \frac{3}{4} \frac{g}{\beta c^2} \left( 1 + \frac{3F}{4\beta c^2} + \frac{15 g^2}{16 \beta c^5} \right)^{-1}\\
	\left< q \right> &= \frac{3g}{4\beta c^2}\\
	\frac{1}{q_0} \frac{\partial \left< q \right>}{\partial T} &= \frac{1}{q_0} \frac{\partial}{\partial T} \left( \frac{3g}{4c^2} kT \right)\\
	&= \frac{1}{q_0} \frac{3}{4} \frac{g}{c^2} k = \alpha
\end{align*}

\section{}

\begin{align*}
	Z_1 &= \frac{1}{h}\sum_{n = 0}^{\infty} e^{- \beta \left[\left( ch + \frac{1}{2}\right) \hbar \omega - X \left( n + \frac{1}{2} \right)^2 \hbar \omega \right]}\\
	&= \sum_{n = 1}^{\infty} e^{-\beta \hbar w \left( n + \frac{1}{2} \right)} - e^{\beta x h w \left( n + \frac{1}{2} \right)}\\
	e^{\beta x w \hbar \left( n + \frac{1}{2} \right)} &= 1 + \beta x w \hbar \left( n + \frac{1}{2} \right)\\
	Z_1 &= \sum_{n = 0}^{\infty} e^{-\beta \hbar w \left( n + \frac{1}{2} \right)} \left( 1 + \beta x w \hbar \left( n + \frac{1}{2} \right) \right)\\
	&= e^{- \frac{\beta k w}{2}}\left( \sum_{k = 0}^{\infty} \left( e^{-\beta k w} \right) \right) + \beta w x \hbar \left( \frac{1}{2}\sum_{n = 0}^{\infty} \left( e^{-\beta \hbar w} \right)^n + \sum_{n = 2}^{\infty} e^{-\beta \hbar \omega n} \right)\\
	&= \frac{\text{csch} \left( \frac{1}{2} \beta \hbar w \right)}{2} \left( 1 + \frac{\beta w x \hbar}{2} \right) - \beta w x \hbar \frac{\partial}{\partial \beta \hbar w} \left( \frac{\text{csch} \left( \frac{1}{2}\beta \hbar w \right)}{2} \right)\\
	&= \frac{\text{csch} \frac{1}{2} \beta \hbar w}{2} \left( 1 + \frac{\beta w x \hbar}{2}\left( \frac{1}{2} \coth \left( \frac{1}{2}\beta \hbar w \right) + 1 \right) \right)
\end{align*}

\section{}

\begin{align*}
	U &= \beta \hbar w\\
	&= - \frac{\partial}{\partial \beta}\ln \left( z_1 \right)\\
	&= - h w \frac{\partial}{\partial u} \ln Z_1\\
	\ln z_1 &= - \ln \left( 2\sinh \left( \frac{u}{2} \right) \right) + \frac{xu}{2} \left( \frac{1}{2} \coth \left( \frac{1}{2} u \right) \right) + 1\\
	U &= - \hbar w \frac{\partial}{\partial u} \left( \frac{2x}{w} + \frac{x h}{12} + \frac{xu^3}{120} \right)
	&= - k w x \left( \frac{2k^2 T^2}{\hbar^2 w^2} - \frac{1}{12} - \frac{\hbar^2 w^2}{40k^2T^2} \right)
\end{align*}

\section{}

\begin{align*}
	C_V &= \frac{\partial U}{\partial T}\\
	&= \frac{\partial U}{\partial u}\frac{\partial u}{\partial} \frac{\partial \beta}{\partial T}\\
	&= \frac{h^2 w^2}{kT^2} x \left( \frac{4}{w^3} - \frac{w}{20} \right)\\
	&= \frac{4 x\hbar^2 w^2}{kT^2 \beta^2 \hbar^2 w^2} \left( \frac{1}{w} + \frac{u^3}{30} \right)\\
	&= 4x k \left( \frac{1}{u} + \frac{u^3}{30} \right)
\end{align*}

\section{}

\begin{align*}
	C_V &= 4x k \left( \frac{kT}{\hbar w} + \frac{\hbar^3 w^3}{k^3 T^3 80}\right)\\
	&= \frac{4x k - T}{\hbar w}\alpha k^2 T \text{ Clasico}\\
	C_V &= \frac{3}{2} k^2 \left( \frac{F}{c^2} + \frac{5g^2}{4c^3} \right)T \alpha k^2 T\text{ Cuantico}
\end{align*}


% PUNTO 6 %%%%%%%%%%%%%%%%%%%%%%%%%%%%%%%%%%%%%%%%%%%%%%%%%%%%%%%%%%%%%%%%%%%%
\chapter{}

\section{}

Podemos tratar esta función como separable para cada grado de libertad. Por lo
tanto se veria algo como:
\begin{align*}
	Z_1 = Z_r Z_\Omega Z_{dip}
\end{align*}

Con lo cual:

\begin{align*}
	Z_r &= \frac{1}{h^3} \int e^{-\beta \frac{p^2}{2m}} d^3p \int d^3 r\\
	&= \left( 2\pi m k T \right)^{\frac{3}{2}} \frac{V}{h^3}
\end{align*}

\begin{align*}
	L^2 &= p_\theta^2 + \frac{p^2_\phi}{\sin \theta}\\
	E_\Omega &= \frac{L^2}{2I}\\
	Z_\Omega &= \frac{1}{h^3}\int e^{-\beta E_\Omega} d^3L d\Omega\\
	Z_\Omega &= \frac{1}{h^3}\int_0^\infty \int_0^\pi \int_0^{2\pi} e^{-\beta \frac{L^2}{2I}} L^2 dLd\cos\theta_L d\phi_L\\
	\int_0^{\pi} d\cos\theta_L &= 2\\
	\int_0^{2\pi} d\phi_L &= 2\pi\\
	u &= \beta \frac{L^2}{2I}\\
	du &= \beta \frac{L}{I} dL\\
	\int_0^\infty L^2 e^{-\beta \frac{L^2}{2I}}dL &= \int_0^\infty L e^{-u}\frac{I}{\beta} du\\
	&= \frac{I}{\beta}\int_0^\infty L e^{-u} du\\
	&= \frac{I}{\beta} \cdot \frac{4I^2}{\beta}\\
	Z_\Omega &= \frac{4\pi^2 IkT}{h^2}
\end{align*}

\begin{align*}
	Z_{dip} &= \frac{1}{4\pi}\int_0^{2\pi}\int_0^\pi e^{\beta \mu E\cos\theta} \sin d\theta d\phi\\
	&= \frac{\sinh \left( \beta\mu E \right)}{\beta \mu E}
\end{align*}

Ahora para construir toda la función de partición simplemente multiplicamos por todos los resultados lo que nos da:
\begin{align*}
	Z_1 &= \left( 2\pi m k T \right)^{\frac{3}{2}} \frac{V}{h^3} \frac{4\pi^2 IkT}{h^2} \frac{\sinh \left( \beta\mu E \right)}{\beta \mu E}
\end{align*}

Para encontrar la función de partición total simplemente es la función de partición de cada molecula y se multiplica por la corrección de Gibbs con lo que nos da:
\begin{align*}
	Z_N &= \frac{1}{N!} \left[ Z_1 \right]^N\\
	Z_N &= \frac{1}{N!} \left[ \left( 2\pi m k T \right)^{\frac{3}{2}} \frac{V}{h^3} \frac{4\pi^2 IkT}{h^2} \frac{\sinh \left( \beta\mu E \right)}{\beta \mu E} \right]^N
\end{align*}

\section{}

En este caso

\begin{align*}
	\left< \mu \cos\theta \right> &= \frac{\int_0^{\pi} \mu \cos\theta e^{\beta \mu E \cos\theta} \sin\theta d\theta}{\int_0^{\pi} e^{\beta \mu E\cos\theta}\sin\theta d\theta}\\
	\int_0^{\pi} e^{\beta \mu E\cos\theta}\sin\theta d\theta &= \frac{\sinh \left( \beta \mu E \right)}{\beta \mu E}\\
	\int_0^{\pi} \mu \cos\theta e^{\beta \mu E \cos\theta} \sin\theta d\theta &= mu \frac{d}{d \left( \beta \mu E \right)} \int_0^\pi e^{\beta \mu E \cos\theta} \sin\theta d\theta\\
	&= \frac{d}{d \left( \beta \mu E \right)} \frac{\sinh \left( \beta \mu E \right)}{\beta \mu E}\\
	&= \frac{\beta\mu E\cosh \left( \beta \mu E \right) - \sinh \left( \beta \mu E \right)}{\left( \beta \mu E^2 \right)}\\
	\left<\mu \cos\theta \right> &= \mu \frac{\cosh \left( \beta \mu E \right)}{\sinh \left( \beta \mu E \right)} - \frac{1}{\beta \mu E}\\
	&= \mu L \left( \beta \mu E \right)
\end{align*}

Donde $L(x) = \coth x - \frac{1}{x}$ es la función de Langevin.

\section{}

Para este caso vamos a usar:
\begin{align*}
	P &= \frac{N}{V} \left<\mu \cos\theta \right>\\
	&= \frac{N}{V} \mu L \left( \beta \mu E \right)\\
	x \ll 1 &\implies L(x) = \frac{x}{3}\\
	P &\approx \frac{N}{V} \mu \left( \frac{\beta \mu E}{3} \right)\\
	&\approx \frac{N}{V} \mu \left( \frac{\beta \mu}{3} \right) E\\
	\alpha &= \frac{N}{V} \mu \left( \frac{\beta \mu}{3} \right)\\
	&\approx \alpha E
\end{align*}

Ahora, para mostar que
\begin{align*}
	\alpha &= \lim_{E \to 0} \frac{\partial \frac{N}{V3}\left< \mu \cos\theta \right>}{\partial E}\\
	\alpha &= \lim_{E \to 0} \frac{N}{V3}\mu\frac{\partial L \left( \beta \mu E \right)}{\partial E}\\
	\alpha &= \lim_{E \to 0} \frac{N}{V3}\mu \left( \beta \mu \left( 1 - L \left( \beta\mu E \right)^2 \right) \right)\\
	\alpha &= \frac{N}{V3}\mu \left( \beta \mu \left( 1 - 0^2 \right) \right)\\
	\alpha &= \frac{N}{V3}\mu \beta \mu\\
	\alpha &= \frac{N}{V}\mu \left(\frac{\beta \mu}{3}\right)
\end{align*}

\section{}

Salimos de
\begin{align*}
	P &= \frac{N}{V} \mu \left( \frac{\beta \mu}{3} \right) E\\
	P &= \epsilon_0 \chi_e E\\
	\chi_e &= \frac{P}{\epsilon_0 E}\\
	&= \frac{N}{V} \frac{\mu^2}{3\epsilon_0 kT}\\
	k &= 1 + \chi_e\\
	&= 1 + \frac{N}{V} \frac{\mu^2}{3\epsilon_0 kT}\\
	&= 1 + f \left( \frac{1}{T} \right)\square
\end{align*}


\end{document}

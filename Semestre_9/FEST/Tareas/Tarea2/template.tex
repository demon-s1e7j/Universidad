\documentclass{report}

\documentclass[12pt]{article}
\usepackage{array}
\usepackage{color}
\usepackage{amsthm}
\usepackage{eufrak}
\usepackage{lipsum}
\usepackage{pifont}
\usepackage{yfonts}
\usepackage{amsmath}
\usepackage{amssymb}
\usepackage{ccfonts}
\usepackage{comment} \usepackage{amsfonts}
\usepackage{fancyhdr}
\usepackage{graphicx}
\usepackage{listings}
\usepackage{mathrsfs}
\usepackage{setspace}
\usepackage{textcomp}
\usepackage{blindtext}
\usepackage{enumerate}
\usepackage{microtype}
\usepackage{xfakebold}
\usepackage{kantlipsum}
%\usepackage{draftwatermark}
\usepackage[spanish]{babel}
\usepackage[margin=1.5cm, top=2cm, bottom=2cm]{geometry}
\usepackage[framemethod=tikz]{mdframed}
\usepackage[colorlinks=true,citecolor=blue,linkcolor=red,urlcolor=magenta]{hyperref}

%//////////////////////////////////////////////////////
% Watermark configuration
%//////////////////////////////////////////////////////
%\SetWatermarkScale{4}
%\SetWatermarkColor{black}
%\SetWatermarkLightness{0.95}
%\SetWatermarkText{\texttt{Watermark}}

%//////////////////////////////////////////////////////
% Frame configuration
%//////////////////////////////////////////////////////
\newmdenv[tikzsetting={draw=gray,fill=white,fill opacity=0},backgroundcolor=none]{Frame}

%//////////////////////////////////////////////////////
% Font style configuration
%//////////////////////////////////////////////////////
\renewcommand{\familydefault}{\ttdefault}
\renewcommand{\rmdefault}{tt}

%//////////////////////////////////////////////////////
% Bold configuration
%//////////////////////////////////////////////////////
\newcommand{\fbseries}{\unskip\setBold\aftergroup\unsetBold\aftergroup\ignorespaces}
\makeatletter
\newcommand{\setBoldness}[1]{\def\fake@bold{#1}}
\makeatother

%//////////////////////////////////////////////////////
% Default font configuration
%//////////////////////////////////////////////////////
\DeclareFontFamily{\encodingdefault}{\ttdefault}{%
  \hyphenchar\font=\defaulthyphenchar
  \fontdimen2\font=0.33333em
  \fontdimen3\font=0.16667em
  \fontdimen4\font=0.11111em
  \fontdimen7\font=0.11111em}


\input{macros}
\input{letterfonts}

\title{\Huge{Fisica Estadistica}\\Tarea 2}
\author{\huge{Sergio Montoya Ramirez}}
\date{202112171}

\begin{document}

\maketitle
\newpage% or \cleardoublepage
% \pdfbookmark[<level>]{<title>}{<dest>}
\pdfbookmark[section]{\contentsname}{toc}
\tableofcontents
\pagebreak

% PUNTO 1
\chapter{}

\section{}

En este caso simplemente tenemos que despejar:
\begin{align*}
  S\left(N, V, E\right) &= Nk\ln\left[\frac{V}{h^3}\left(\frac{4 \pi m E}{3N}\right)^{\frac{3}{2}}\right] + \frac{3}{2}Nk \\
  S &= Nk\ln\left[\frac{V}{h^3}\left(\frac{4 \pi m E}{3N}\right)^{\frac{3}{2}}\right] + \frac{3}{2}Nk \\
  S &= \ln\left[\left(\frac{V}{h^3}\left(\frac{4 \pi m E}{3N}\right)^{\frac{3}{2}}\right)^{Nk}\right] + \frac{3}{2}Nk \\
  e^{S} &= e^{\ln\left[\left(\frac{V}{h^3}\left(\frac{4 \pi m E}{3N}\right)^{\frac{3}{2}}\right)^{Nk}\right] + \frac{3}{2}Nk} \\
  e^{S} &= \left(\frac{V}{h^3}\right)^{Nk}\left(\frac{4 \pi m E}{3N}\right)^{\frac{3}{2}Nk} e^{\frac{3}{2}Nk} \\
  e^{S} \left(\frac{V}{h^3}\right)^{-Nk} e^{-\frac{3}{2}Nk} &= \left(\frac{4 \pi m E}{3N}\right)^{\frac{3}{2}Nk} \\
  \left(e^{S} \left(\frac{V}{h^3}\right)^{-Nk} e^{-\frac{3}{2}Nk}\right)^{\frac{2}{3Nk}} &= \left(\frac{4 \pi m E}{3N}\right) \\
  e^{\frac{2S}{3Nk}} \left(\frac{V}{h^3}\right)^{-\frac{2}{3}} e^{-1} &= \left(\frac{4 \pi m E}{3N}\right) \\
  E &= e^{\frac{2S}{3Nk}} \frac{h^2}{V^{\frac{2}{3}}} e^{-1} \left(\frac{3N}{4 \pi m}\right) \\
  E &= e^{\frac{2S}{3Nk} -1} \left(\frac{3Nh^2}{4 \pi m V^{\frac{2}{3}}}\right)
.\end{align*}

\section{}

Para este caso vamos a usar:
\begin{align*}
  E &= e^{\frac{2S}{3Nk} -1} \left(\frac{3Nh^2}{4 \pi m V^{\frac{2}{3}}}\right)\\
  T &= \frac{\partial E}{\partial S}\\
  T &= \left(\frac{3Nh^2}{4 \pi m V^{\frac{2}{3}}}\right) e^{-1} \frac{\partial e^{\frac{2S}{3Nk}}}{\partial S}\\
  T &= \left(\frac{3Nh^2}{4 \pi m V^{\frac{2}{3}}}\right) e^{-1} \frac{2}{3Nk}e^{\frac{2S}{3Nk}} \\
  T &= \left(\frac{h^2}{k 2 \pi m V^{\frac{2}{3}}}\right) e^{-1} e^{\frac{2S}{3Nk}} \\
  T \left(\frac{k 2 \pi m V^{\frac{2}{3}}}{h^2}\right) &= e^{\frac{2S}{3Nk} - 1} \\
  E &= e^{\frac{2S}{3Nk} -1} \left(\frac{3Nh^2}{4 \pi m V^{\frac{2}{3}}}\right)\\
  E &= T \left(\frac{k 2 \pi m V^{\frac{2}{3}}}{h^2}\right) \left(\frac{3Nh^2}{4 \pi m V^{\frac{2}{3}}}\right)\\
  E &= T \left(\frac{3Nk}{2}\right)\\
  E &= \frac{3}{2} NkT
.\end{align*}

\section{}

Ahora desarrollemos:
\begin{align*}
  E &= e^{\frac{2S}{3Nk} -1} \left(\frac{3Nh^2}{4 \pi m V^{\frac{2}{3}}}\right)\\
  P &= -\frac{\partial E}{\partial V} \\
  P &= -e^{\frac{2S}{3Nk} -1} \left(\frac{3Nh^2}{4 \pi m }\right)\frac{\partial V^{-\frac{2}{3}}}{\partial V} \\
  P &= \frac{2}{3}e^{\frac{2S}{3Nk} -1} \left(\frac{3Nh^2}{4 \pi m }\right)V^{- \frac{5}{3}} \\
  T &= \left(\frac{h^2}{k 2 \pi m V^{\frac{2}{3}}}\right) e^{\frac{2S}{3Nk} - 1} \\
  \frac{P}{T} &= \frac{\frac{2}{3}e^{\frac{2S}{3Nk} -1} \left(\frac{3Nh^2}{4 \pi m }\right)V^{- \frac{5}{3}}}{\left(\frac{h^2}{k 2 \pi m V^{\frac{2}{3}}}\right) e^{\frac{2S}{3Nk} - 1}}\\
  \frac{P}{T} &= \frac{2}{3}\frac{\left(\frac{3Nh^2}{4 \pi m }\right)}{\left(\frac{h^2}{k 2 \pi m V^{\frac{2}{3}}}\right)}V^{- \frac{5}{3}}\\
  \frac{P}{T} &= \frac{2}{3}\left(\frac{3Nh^2k 2 \pi m V^{\frac{2}{3}}}{4 \pi m h^2}\right)V^{- \frac{5}{3}}\\
  \frac{P}{T} &= \frac{4}{3}\left(\frac{3Nh^2k\pi m V^{\frac{2}{3}}}{4 \pi m h^2 V^{\frac{2}{3}}}\right)V^{-1}\\
  \frac{P}{T} &= \left(Nk\right)V^{-1} \\
  PV &= NkT
.\end{align*}

\section{}

\begin{align*}
  C_v &= \frac{\partial E}{\partial T}\\
  &= \frac{\partial \left( \frac{3}{2} NkT\right)}{\partial T}\\
  &= \frac{3}{2} Nk\\
  C_p &= \frac{\partial \left(E + PV\right)}{\partial T}\\
  &= \frac{\partial \left(\frac{3}{2} NkT + NkT\right)}{\partial T}\\
  &= \frac{\partial NkT \left(\frac{3}{2} + 1\right)}{\partial T}\\
  &= \frac{\partial \frac{5}{2}NkT}{\partial T}\\
  &= \frac{5}{2} Nk\\
  \frac{C_p}{C_v} &= \frac{\frac{5}{2}Nk}{\frac{3}{2}Nk}\\
  &= \frac{5\cdot 2}{3 \cdot 2}\\
  &= \frac{5}{3}
.\end{align*}

\section{}

Para este caso necesitamos
\begin{align*}
  E &= e^{\frac{2S}{3Nk} -1} \left(\frac{3Nh^2}{4 \pi m V^{\frac{2}{3}}}\right)\\
  \mu &= \frac{\partial E}{\partial N}\\
  &= \frac{\partial e^{\frac{2S}{3Nk} -1} \left(\frac{3Nh^2}{4 \pi m V^{\frac{2}{3}}}\right)}{\partial N}\\
  &= \frac{\partial e^{\frac{2S}{3Nk} -1}}{\partial N} \left(\frac{3Nh^2}{4 \pi m V^{\frac{2}{3}}}\right) + e^{\frac{2S}{3Nk} -1}\frac{\partial \left(\frac{3Nh^2}{4 \pi m V^{\frac{2}{3}}}\right)}{\partial N}\\
  &= -\frac{2S}{3N^2 k}e^{\frac{2S}{3Nk} -1} \left(\frac{3Nh^2}{4 \pi m V^{\frac{2}{3}}}\right) + e^{\frac{2S}{3Nk} -1}\left(\frac{3h^2}{4 \pi m V^{\frac{2}{3}}}\right)\\
  &= -\frac{2S}{3N k}e^{\frac{2S}{3Nk} -1} \left(\frac{3h^2}{4 \pi m V^{\frac{2}{3}}}\right) + e^{\frac{2S}{3Nk} -1}\left(\frac{3h^2}{4 \pi m V^{\frac{2}{3}}}\right)\\
  &= e^{\frac{2S}{3Nk} -1} \left(\frac{3h^2}{4 \pi m V^{\frac{2}{3}}}\right)\left(1 -\frac{2S}{3N k} \right)
.\end{align*}

Y con esto podemos probar si
\begin{align*}
  \mu\left(\lambda N, \lambda V, \lambda S\right) &= \lambda \mu \left(N, V, S\right)\\
  &= e^{\frac{2\lambda S}{3\lambda Nk} -1} \left(\frac{3h^2}{4 \pi m \lambda^{\frac{2}{3}}V^{\frac{2}{3}}}\right)\left(1 -\frac{2\lambda S}{3\lambda N k} \right)\\
  &= e^{\frac{2S}{3Nk} -1} \left(\frac{3h^2}{4 \pi m \lambda^{\frac{2}{3}}V^{\frac{2}{3}}}\right)\left(1 -\frac{2S}{3N k} \right)
\end{align*}

Que como se ve no se coincide con una cantidad intensiva.

\section{}
\section{}

% PUNTO 2
\chapter{}

% PUNTO 3
\chapter{}

\section{}
\section{}
\section{}

% PUNTO 4
\chapter{}

\section{}
\section{}
\section{}
\section{}
\section{}

% PUNTO 5
\chapter{}

\section{}
\section{}
\section{}

% PUNTO 6
\chapter{}

\section{}
\section{}
\section{}

\end{document}

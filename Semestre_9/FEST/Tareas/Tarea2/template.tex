\documentclass{report}

\documentclass[12pt]{article}
\usepackage{array}
\usepackage{color}
\usepackage{amsthm}
\usepackage{eufrak}
\usepackage{lipsum}
\usepackage{pifont}
\usepackage{yfonts}
\usepackage{amsmath}
\usepackage{amssymb}
\usepackage{ccfonts}
\usepackage{comment} \usepackage{amsfonts}
\usepackage{fancyhdr}
\usepackage{graphicx}
\usepackage{listings}
\usepackage{mathrsfs}
\usepackage{setspace}
\usepackage{textcomp}
\usepackage{blindtext}
\usepackage{enumerate}
\usepackage{microtype}
\usepackage{xfakebold}
\usepackage{kantlipsum}
%\usepackage{draftwatermark}
\usepackage[spanish]{babel}
\usepackage[margin=1.5cm, top=2cm, bottom=2cm]{geometry}
\usepackage[framemethod=tikz]{mdframed}
\usepackage[colorlinks=true,citecolor=blue,linkcolor=red,urlcolor=magenta]{hyperref}

%//////////////////////////////////////////////////////
% Watermark configuration
%//////////////////////////////////////////////////////
%\SetWatermarkScale{4}
%\SetWatermarkColor{black}
%\SetWatermarkLightness{0.95}
%\SetWatermarkText{\texttt{Watermark}}

%//////////////////////////////////////////////////////
% Frame configuration
%//////////////////////////////////////////////////////
\newmdenv[tikzsetting={draw=gray,fill=white,fill opacity=0},backgroundcolor=none]{Frame}

%//////////////////////////////////////////////////////
% Font style configuration
%//////////////////////////////////////////////////////
\renewcommand{\familydefault}{\ttdefault}
\renewcommand{\rmdefault}{tt}

%//////////////////////////////////////////////////////
% Bold configuration
%//////////////////////////////////////////////////////
\newcommand{\fbseries}{\unskip\setBold\aftergroup\unsetBold\aftergroup\ignorespaces}
\makeatletter
\newcommand{\setBoldness}[1]{\def\fake@bold{#1}}
\makeatother

%//////////////////////////////////////////////////////
% Default font configuration
%//////////////////////////////////////////////////////
\DeclareFontFamily{\encodingdefault}{\ttdefault}{%
  \hyphenchar\font=\defaulthyphenchar
  \fontdimen2\font=0.33333em
  \fontdimen3\font=0.16667em
  \fontdimen4\font=0.11111em
  \fontdimen7\font=0.11111em}


%From M275 "Topology" at SJSU
\newcommand{\id}{\mathrm{id}}
\newcommand{\taking}[1]{\xrightarrow{#1}}
\newcommand{\inv}{^{-1}}

%From M170 "Introduction to Graph Theory" at SJSU
\DeclareMathOperator{\diam}{diam}
\DeclareMathOperator{\ord}{ord}
\newcommand{\defeq}{\overset{\mathrm{def}}{=}}

%From the USAMO .tex files
\newcommand{\ts}{\textsuperscript}
\newcommand{\dg}{^\circ}
\newcommand{\ii}{\item}

% % From Math 55 and Math 145 at Harvard
% \newenvironment{subproof}[1][Proof]{%
% \begin{proof}[#1] \renewcommand{\qedsymbol}{$\blacksquare$}}%
% {\end{proof}}

\newcommand{\liff}{\leftrightarrow}
\newcommand{\lthen}{\rightarrow}
\newcommand{\opname}{\operatorname}
\newcommand{\surjto}{\twoheadrightarrow}
\newcommand{\injto}{\hookrightarrow}
\newcommand{\On}{\mathrm{On}} % ordinals
\DeclareMathOperator{\img}{im} % Image
\DeclareMathOperator{\Img}{Im} % Image
\DeclareMathOperator{\coker}{coker} % Cokernel
\DeclareMathOperator{\Coker}{Coker} % Cokernel
\DeclareMathOperator{\Ker}{Ker} % Kernel
\DeclareMathOperator{\rank}{rank}
\DeclareMathOperator{\Spec}{Spec} % spectrum
\DeclareMathOperator{\Tr}{Tr} % trace
\DeclareMathOperator{\pr}{pr} % projection
\DeclareMathOperator{\ext}{ext} % extension
\DeclareMathOperator{\pred}{pred} % predecessor
\DeclareMathOperator{\dom}{dom} % domain
\DeclareMathOperator{\ran}{ran} % range
\DeclareMathOperator{\Hom}{Hom} % homomorphism
\DeclareMathOperator{\Mor}{Mor} % morphisms
\DeclareMathOperator{\End}{End} % endomorphism

\newcommand{\eps}{\epsilon}
\newcommand{\veps}{\varepsilon}
\newcommand{\ol}{\overline}
\newcommand{\ul}{\underline}
\newcommand{\wt}{\widetilde}
\newcommand{\wh}{\widehat}
\newcommand{\vocab}[1]{\textbf{\color{blue} #1}}
\providecommand{\half}{\frac{1}{2}}
\newcommand{\dang}{\measuredangle} %% Directed angle
\newcommand{\ray}[1]{\overrightarrow{#1}}
\newcommand{\seg}[1]{\overline{#1}}
\newcommand{\arc}[1]{\wideparen{#1}}
\DeclareMathOperator{\cis}{cis}
\DeclareMathOperator*{\lcm}{lcm}
\DeclareMathOperator*{\argmin}{arg min}
\DeclareMathOperator*{\argmax}{arg max}
\newcommand{\cycsum}{\sum_{\mathrm{cyc}}}
\newcommand{\symsum}{\sum_{\mathrm{sym}}}
\newcommand{\cycprod}{\prod_{\mathrm{cyc}}}
\newcommand{\symprod}{\prod_{\mathrm{sym}}}
\newcommand{\Qed}{\begin{flushright}\qed\end{flushright}}
\newcommand{\parinn}{\setlength{\parindent}{1cm}}
\newcommand{\parinf}{\setlength{\parindent}{0cm}}
% \newcommand{\norm}{\|\cdot\|}
\newcommand{\inorm}{\norm_{\infty}}
\newcommand{\opensets}{\{V_{\alpha}\}_{\alpha\in I}}
\newcommand{\oset}{V_{\alpha}}
\newcommand{\opset}[1]{V_{\alpha_{#1}}}
\newcommand{\lub}{\text{lub}}
\newcommand{\del}[2]{\frac{\partial #1}{\partial #2}}
\newcommand{\Del}[3]{\frac{\partial^{#1} #2}{\partial^{#1} #3}}
\newcommand{\deld}[2]{\dfrac{\partial #1}{\partial #2}}
\newcommand{\Deld}[3]{\dfrac{\partial^{#1} #2}{\partial^{#1} #3}}
\newcommand{\lm}{\lambda}
\newcommand{\uin}{\mathbin{\rotatebox[origin=c]{90}{$\in$}}}
\newcommand{\usubset}{\mathbin{\rotatebox[origin=c]{90}{$\subset$}}}
\newcommand{\lt}{\left}
\newcommand{\rt}{\right}
\newcommand{\paren}[1]{\left(#1\right)}
\newcommand{\bs}[1]{\boldsymbol{#1}}
\newcommand{\exs}{\exists}
\newcommand{\st}{\strut}
\newcommand{\dps}[1]{\displaystyle{#1}}

\newcommand{\sol}{\setlength{\parindent}{0cm}\textbf{\textit{Solution:}}\setlength{\parindent}{1cm} }
\newcommand{\solve}[1]{\setlength{\parindent}{0cm}\textbf{\textit{Solution: }}\setlength{\parindent}{1cm}#1 \Qed}

% Things Lie
\newcommand{\kb}{\mathfrak b}
\newcommand{\kg}{\mathfrak g}
\newcommand{\kh}{\mathfrak h}
\newcommand{\kn}{\mathfrak n}
\newcommand{\ku}{\mathfrak u}
\newcommand{\kz}{\mathfrak z}
\DeclareMathOperator{\Ext}{Ext} % Ext functor
\DeclareMathOperator{\Tor}{Tor} % Tor functor
\newcommand{\gl}{\opname{\mathfrak{gl}}} % frak gl group
\renewcommand{\sl}{\opname{\mathfrak{sl}}} % frak sl group chktex 6

% More script letters etc.
\newcommand{\SA}{\mathcal A}
\newcommand{\SB}{\mathcal B}
\newcommand{\SC}{\mathcal C}
\newcommand{\SF}{\mathcal F}
\newcommand{\SG}{\mathcal G}
\newcommand{\SH}{\mathcal H}
\newcommand{\OO}{\mathcal O}

\newcommand{\SCA}{\mathscr A}
\newcommand{\SCB}{\mathscr B}
\newcommand{\SCC}{\mathscr C}
\newcommand{\SCD}{\mathscr D}
\newcommand{\SCE}{\mathscr E}
\newcommand{\SCF}{\mathscr F}
\newcommand{\SCG}{\mathscr G}
\newcommand{\SCH}{\mathscr H}

% Mathfrak primes
\newcommand{\km}{\mathfrak m}
\newcommand{\kp}{\mathfrak p}
\newcommand{\kq}{\mathfrak q}

% number sets
\newcommand{\RR}[1][]{\ensuremath{\ifstrempty{#1}{\mathbb{R}}{\mathbb{R}^{#1}}}}
\newcommand{\NN}[1][]{\ensuremath{\ifstrempty{#1}{\mathbb{N}}{\mathbb{N}^{#1}}}}
\newcommand{\ZZ}[1][]{\ensuremath{\ifstrempty{#1}{\mathbb{Z}}{\mathbb{Z}^{#1}}}}
\newcommand{\QQ}[1][]{\ensuremath{\ifstrempty{#1}{\mathbb{Q}}{\mathbb{Q}^{#1}}}}
\newcommand{\CC}[1][]{\ensuremath{\ifstrempty{#1}{\mathbb{C}}{\mathbb{C}^{#1}}}}
\newcommand{\PP}[1][]{\ensuremath{\ifstrempty{#1}{\mathbb{P}}{\mathbb{P}^{#1}}}}
\newcommand{\HH}[1][]{\ensuremath{\ifstrempty{#1}{\mathbb{H}}{\mathbb{H}^{#1}}}}
\newcommand{\FF}[1][]{\ensuremath{\ifstrempty{#1}{\mathbb{F}}{\mathbb{F}^{#1}}}}
% expected value
\newcommand{\EE}{\ensuremath{\mathbb{E}}}
\newcommand{\charin}{\text{ char }}
\DeclareMathOperator{\sign}{sign}
\DeclareMathOperator{\Aut}{Aut}
\DeclareMathOperator{\Inn}{Inn}
\DeclareMathOperator{\Syl}{Syl}
\DeclareMathOperator{\Gal}{Gal}
\DeclareMathOperator{\GL}{GL} % General linear group
\DeclareMathOperator{\SL}{SL} % Special linear group

%---------------------------------------
% BlackBoard Math Fonts :-
%---------------------------------------

%Captital Letters
\newcommand{\bbA}{\mathbb{A}}	\newcommand{\bbB}{\mathbb{B}}
\newcommand{\bbC}{\mathbb{C}}	\newcommand{\bbD}{\mathbb{D}}
\newcommand{\bbE}{\mathbb{E}}	\newcommand{\bbF}{\mathbb{F}}
\newcommand{\bbG}{\mathbb{G}}	\newcommand{\bbH}{\mathbb{H}}
\newcommand{\bbI}{\mathbb{I}}	\newcommand{\bbJ}{\mathbb{J}}
\newcommand{\bbK}{\mathbb{K}}	\newcommand{\bbL}{\mathbb{L}}
\newcommand{\bbM}{\mathbb{M}}	\newcommand{\bbN}{\mathbb{N}}
\newcommand{\bbO}{\mathbb{O}}	\newcommand{\bbP}{\mathbb{P}}
\newcommand{\bbQ}{\mathbb{Q}}	\newcommand{\bbR}{\mathbb{R}}
\newcommand{\bbS}{\mathbb{S}}	\newcommand{\bbT}{\mathbb{T}}
\newcommand{\bbU}{\mathbb{U}}	\newcommand{\bbV}{\mathbb{V}}
\newcommand{\bbW}{\mathbb{W}}	\newcommand{\bbX}{\mathbb{X}}
\newcommand{\bbY}{\mathbb{Y}}	\newcommand{\bbZ}{\mathbb{Z}}

%---------------------------------------
% MathCal Fonts :-
%---------------------------------------

%Captital Letters
\newcommand{\mcA}{\mathcal{A}}	\newcommand{\mcB}{\mathcal{B}}
\newcommand{\mcC}{\mathcal{C}}	\newcommand{\mcD}{\mathcal{D}}
\newcommand{\mcE}{\mathcal{E}}	\newcommand{\mcF}{\mathcal{F}}
\newcommand{\mcG}{\mathcal{G}}	\newcommand{\mcH}{\mathcal{H}}
\newcommand{\mcI}{\mathcal{I}}	\newcommand{\mcJ}{\mathcal{J}}
\newcommand{\mcK}{\mathcal{K}}	\newcommand{\mcL}{\mathcal{L}}
\newcommand{\mcM}{\mathcal{M}}	\newcommand{\mcN}{\mathcal{N}}
\newcommand{\mcO}{\mathcal{O}}	\newcommand{\mcP}{\mathcal{P}}
\newcommand{\mcQ}{\mathcal{Q}}	\newcommand{\mcR}{\mathcal{R}}
\newcommand{\mcS}{\mathcal{S}}	\newcommand{\mcT}{\mathcal{T}}
\newcommand{\mcU}{\mathcal{U}}	\newcommand{\mcV}{\mathcal{V}}
\newcommand{\mcW}{\mathcal{W}}	\newcommand{\mcX}{\mathcal{X}}
\newcommand{\mcY}{\mathcal{Y}}	\newcommand{\mcZ}{\mathcal{Z}}


%---------------------------------------
% Bold Math Fonts :-
%---------------------------------------

%Captital Letters
\newcommand{\bmA}{\boldsymbol{A}}	\newcommand{\bmB}{\boldsymbol{B}}
\newcommand{\bmC}{\boldsymbol{C}}	\newcommand{\bmD}{\boldsymbol{D}}
\newcommand{\bmE}{\boldsymbol{E}}	\newcommand{\bmF}{\boldsymbol{F}}
\newcommand{\bmG}{\boldsymbol{G}}	\newcommand{\bmH}{\boldsymbol{H}}
\newcommand{\bmI}{\boldsymbol{I}}	\newcommand{\bmJ}{\boldsymbol{J}}
\newcommand{\bmK}{\boldsymbol{K}}	\newcommand{\bmL}{\boldsymbol{L}}
\newcommand{\bmM}{\boldsymbol{M}}	\newcommand{\bmN}{\boldsymbol{N}}
\newcommand{\bmO}{\boldsymbol{O}}	\newcommand{\bmP}{\boldsymbol{P}}
\newcommand{\bmQ}{\boldsymbol{Q}}	\newcommand{\bmR}{\boldsymbol{R}}
\newcommand{\bmS}{\boldsymbol{S}}	\newcommand{\bmT}{\boldsymbol{T}}
\newcommand{\bmU}{\boldsymbol{U}}	\newcommand{\bmV}{\boldsymbol{V}}
\newcommand{\bmW}{\boldsymbol{W}}	\newcommand{\bmX}{\boldsymbol{X}}
\newcommand{\bmY}{\boldsymbol{Y}}	\newcommand{\bmZ}{\boldsymbol{Z}}
%Small Letters
\newcommand{\bma}{\boldsymbol{a}}	\newcommand{\bmb}{\boldsymbol{b}}
\newcommand{\bmc}{\boldsymbol{c}}	\newcommand{\bmd}{\boldsymbol{d}}
\newcommand{\bme}{\boldsymbol{e}}	\newcommand{\bmf}{\boldsymbol{f}}
\newcommand{\bmg}{\boldsymbol{g}}	\newcommand{\bmh}{\boldsymbol{h}}
\newcommand{\bmi}{\boldsymbol{i}}	\newcommand{\bmj}{\boldsymbol{j}}
\newcommand{\bmk}{\boldsymbol{k}}	\newcommand{\bml}{\boldsymbol{l}}
\newcommand{\bmm}{\boldsymbol{m}}	\newcommand{\bmn}{\boldsymbol{n}}
\newcommand{\bmo}{\boldsymbol{o}}	\newcommand{\bmp}{\boldsymbol{p}}
\newcommand{\bmq}{\boldsymbol{q}}	\newcommand{\bmr}{\boldsymbol{r}}
\newcommand{\bms}{\boldsymbol{s}}	\newcommand{\bmt}{\boldsymbol{t}}
\newcommand{\bmu}{\boldsymbol{u}}	\newcommand{\bmv}{\boldsymbol{v}}
\newcommand{\bmw}{\boldsymbol{w}}	\newcommand{\bmx}{\boldsymbol{x}}
\newcommand{\bmy}{\boldsymbol{y}}	\newcommand{\bmz}{\boldsymbol{z}}

%---------------------------------------
% Scr Math Fonts :-
%---------------------------------------

\newcommand{\sA}{{\mathscr{A}}}   \newcommand{\sB}{{\mathscr{B}}}
\newcommand{\sC}{{\mathscr{C}}}   \newcommand{\sD}{{\mathscr{D}}}
\newcommand{\sE}{{\mathscr{E}}}   \newcommand{\sF}{{\mathscr{F}}}
\newcommand{\sG}{{\mathscr{G}}}   \newcommand{\sH}{{\mathscr{H}}}
\newcommand{\sI}{{\mathscr{I}}}   \newcommand{\sJ}{{\mathscr{J}}}
\newcommand{\sK}{{\mathscr{K}}}   \newcommand{\sL}{{\mathscr{L}}}
\newcommand{\sM}{{\mathscr{M}}}   \newcommand{\sN}{{\mathscr{N}}}
\newcommand{\sO}{{\mathscr{O}}}   \newcommand{\sP}{{\mathscr{P}}}
\newcommand{\sQ}{{\mathscr{Q}}}   \newcommand{\sR}{{\mathscr{R}}}
\newcommand{\sS}{{\mathscr{S}}}   \newcommand{\sT}{{\mathscr{T}}}
\newcommand{\sU}{{\mathscr{U}}}   \newcommand{\sV}{{\mathscr{V}}}
\newcommand{\sW}{{\mathscr{W}}}   \newcommand{\sX}{{\mathscr{X}}}
\newcommand{\sY}{{\mathscr{Y}}}   \newcommand{\sZ}{{\mathscr{Z}}}


%---------------------------------------
% Math Fraktur Font
%---------------------------------------

%Captital Letters
\newcommand{\mfA}{\mathfrak{A}}	\newcommand{\mfB}{\mathfrak{B}}
\newcommand{\mfC}{\mathfrak{C}}	\newcommand{\mfD}{\mathfrak{D}}
\newcommand{\mfE}{\mathfrak{E}}	\newcommand{\mfF}{\mathfrak{F}}
\newcommand{\mfG}{\mathfrak{G}}	\newcommand{\mfH}{\mathfrak{H}}
\newcommand{\mfI}{\mathfrak{I}}	\newcommand{\mfJ}{\mathfrak{J}}
\newcommand{\mfK}{\mathfrak{K}}	\newcommand{\mfL}{\mathfrak{L}}
\newcommand{\mfM}{\mathfrak{M}}	\newcommand{\mfN}{\mathfrak{N}}
\newcommand{\mfO}{\mathfrak{O}}	\newcommand{\mfP}{\mathfrak{P}}
\newcommand{\mfQ}{\mathfrak{Q}}	\newcommand{\mfR}{\mathfrak{R}}
\newcommand{\mfS}{\mathfrak{S}}	\newcommand{\mfT}{\mathfrak{T}}
\newcommand{\mfU}{\mathfrak{U}}	\newcommand{\mfV}{\mathfrak{V}}
\newcommand{\mfW}{\mathfrak{W}}	\newcommand{\mfX}{\mathfrak{X}}
\newcommand{\mfY}{\mathfrak{Y}}	\newcommand{\mfZ}{\mathfrak{Z}}
%Small Letters
\newcommand{\mfa}{\mathfrak{a}}	\newcommand{\mfb}{\mathfrak{b}}
\newcommand{\mfc}{\mathfrak{c}}	\newcommand{\mfd}{\mathfrak{d}}
\newcommand{\mfe}{\mathfrak{e}}	\newcommand{\mff}{\mathfrak{f}}
\newcommand{\mfg}{\mathfrak{g}}	\newcommand{\mfh}{\mathfrak{h}}
\newcommand{\mfi}{\mathfrak{i}}	\newcommand{\mfj}{\mathfrak{j}}
\newcommand{\mfk}{\mathfrak{k}}	\newcommand{\mfl}{\mathfrak{l}}
\newcommand{\mfm}{\mathfrak{m}}	\newcommand{\mfn}{\mathfrak{n}}
\newcommand{\mfo}{\mathfrak{o}}	\newcommand{\mfp}{\mathfrak{p}}
\newcommand{\mfq}{\mathfrak{q}}	\newcommand{\mfr}{\mathfrak{r}}
\newcommand{\mfs}{\mathfrak{s}}	\newcommand{\mft}{\mathfrak{t}}
\newcommand{\mfu}{\mathfrak{u}}	\newcommand{\mfv}{\mathfrak{v}}
\newcommand{\mfw}{\mathfrak{w}}	\newcommand{\mfx}{\mathfrak{x}}
\newcommand{\mfy}{\mathfrak{y}}	\newcommand{\mfz}{\mathfrak{z}}


\title{\Huge{Fisica Estadistica}\\Tarea 2}
\author{\huge{Sergio Montoya Ramirez}}
\date{202112171}

\begin{document}

\maketitle
\newpage% or \cleardoublepage
% \pdfbookmark[<level>]{<title>}{<dest>}
\pdfbookmark[section]{\contentsname}{toc}
\tableofcontents
\pagebreak

% PUNTO 1
\chapter{}

\section{}

En este caso simplemente tenemos que despejar:
\begin{align*}
  S\left(N, V, E\right) &= Nk\ln\left[\frac{V}{h^3}\left(\frac{4 \pi m E}{3N}\right)^{\frac{3}{2}}\right] + \frac{3}{2}Nk \\
  S &= Nk\ln\left[\frac{V}{h^3}\left(\frac{4 \pi m E}{3N}\right)^{\frac{3}{2}}\right] + \frac{3}{2}Nk \\
  S &= \ln\left[\left(\frac{V}{h^3}\left(\frac{4 \pi m E}{3N}\right)^{\frac{3}{2}}\right)^{Nk}\right] + \frac{3}{2}Nk \\
  e^{S} &= e^{\ln\left[\left(\frac{V}{h^3}\left(\frac{4 \pi m E}{3N}\right)^{\frac{3}{2}}\right)^{Nk}\right] + \frac{3}{2}Nk} \\
  e^{S} &= \left(\frac{V}{h^3}\right)^{Nk}\left(\frac{4 \pi m E}{3N}\right)^{\frac{3}{2}Nk} e^{\frac{3}{2}Nk} \\
  e^{S} \left(\frac{V}{h^3}\right)^{-Nk} e^{-\frac{3}{2}Nk} &= \left(\frac{4 \pi m E}{3N}\right)^{\frac{3}{2}Nk} \\
  \left(e^{S} \left(\frac{V}{h^3}\right)^{-Nk} e^{-\frac{3}{2}Nk}\right)^{\frac{2}{3Nk}} &= \left(\frac{4 \pi m E}{3N}\right) \\
  e^{\frac{2S}{3Nk}} \left(\frac{V}{h^3}\right)^{-\frac{2}{3}} e^{-1} &= \left(\frac{4 \pi m E}{3N}\right) \\
  E &= e^{\frac{2S}{3Nk}} \frac{h^2}{V^{\frac{2}{3}}} e^{-1} \left(\frac{3N}{4 \pi m}\right) \\
  E &= e^{\frac{2S}{3Nk} -1} \left(\frac{3Nh^2}{4 \pi m V^{\frac{2}{3}}}\right)
.\end{align*}

\section{}

Para este caso vamos a usar:
\begin{align*}
  E &= e^{\frac{2S}{3Nk} -1} \left(\frac{3Nh^2}{4 \pi m V^{\frac{2}{3}}}\right)\\
  T &= \frac{\partial E}{\partial S}\\
  T &= \left(\frac{3Nh^2}{4 \pi m V^{\frac{2}{3}}}\right) e^{-1} \frac{\partial e^{\frac{2S}{3Nk}}}{\partial S}\\
  T &= \left(\frac{3Nh^2}{4 \pi m V^{\frac{2}{3}}}\right) e^{-1} \frac{2}{3Nk}e^{\frac{2S}{3Nk}} \\
  T &= \left(\frac{h^2}{k 2 \pi m V^{\frac{2}{3}}}\right) e^{-1} e^{\frac{2S}{3Nk}} \\
  T \left(\frac{k 2 \pi m V^{\frac{2}{3}}}{h^2}\right) &= e^{\frac{2S}{3Nk} - 1} \\
  E &= e^{\frac{2S}{3Nk} -1} \left(\frac{3Nh^2}{4 \pi m V^{\frac{2}{3}}}\right)\\
  E &= T \left(\frac{k 2 \pi m V^{\frac{2}{3}}}{h^2}\right) \left(\frac{3Nh^2}{4 \pi m V^{\frac{2}{3}}}\right)\\
  E &= T \left(\frac{3Nk}{2}\right)\\
  E &= \frac{3}{2} NkT
.\end{align*}

\section{}

Ahora desarrollemos:
\begin{align*}
  E &= e^{\frac{2S}{3Nk} -1} \left(\frac{3Nh^2}{4 \pi m V^{\frac{2}{3}}}\right)\\
  P &= -\frac{\partial E}{\partial V} \\
  P &= -e^{\frac{2S}{3Nk} -1} \left(\frac{3Nh^2}{4 \pi m }\right)\frac{\partial V^{-\frac{2}{3}}}{\partial V} \\
  P &= \frac{2}{3}e^{\frac{2S}{3Nk} -1} \left(\frac{3Nh^2}{4 \pi m }\right)V^{- \frac{5}{3}} \\
  T &= \left(\frac{h^2}{k 2 \pi m V^{\frac{2}{3}}}\right) e^{\frac{2S}{3Nk} - 1} \\
  \frac{P}{T} &= \frac{\frac{2}{3}e^{\frac{2S}{3Nk} -1} \left(\frac{3Nh^2}{4 \pi m }\right)V^{- \frac{5}{3}}}{\left(\frac{h^2}{k 2 \pi m V^{\frac{2}{3}}}\right) e^{\frac{2S}{3Nk} - 1}}\\
  \frac{P}{T} &= \frac{2}{3}\frac{\left(\frac{3Nh^2}{4 \pi m }\right)}{\left(\frac{h^2}{k 2 \pi m V^{\frac{2}{3}}}\right)}V^{- \frac{5}{3}}\\
  \frac{P}{T} &= \frac{2}{3}\left(\frac{3Nh^2k 2 \pi m V^{\frac{2}{3}}}{4 \pi m h^2}\right)V^{- \frac{5}{3}}\\
  \frac{P}{T} &= \frac{4}{3}\left(\frac{3Nh^2k\pi m V^{\frac{2}{3}}}{4 \pi m h^2 V^{\frac{2}{3}}}\right)V^{-1}\\
  \frac{P}{T} &= \left(Nk\right)V^{-1} \\
  PV &= NkT
.\end{align*}

\section{}

\begin{align*}
  C_v &= \frac{\partial E}{\partial T}\\
  &= \frac{\partial \left( \frac{3}{2} NkT\right)}{\partial T}\\
  &= \frac{3}{2} Nk\\
  C_p &= \frac{\partial \left(E + PV\right)}{\partial T}\\
  &= \frac{\partial \left(\frac{3}{2} NkT + NkT\right)}{\partial T}\\
  &= \frac{\partial NkT \left(\frac{3}{2} + 1\right)}{\partial T}\\
  &= \frac{\partial \frac{5}{2}NkT}{\partial T}\\
  &= \frac{5}{2} Nk\\
  \frac{C_p}{C_v} &= \frac{\frac{5}{2}Nk}{\frac{3}{2}Nk}\\
  &= \frac{5\cdot 2}{3 \cdot 2}\\
  &= \frac{5}{3}
.\end{align*}

\section{}

Para este caso necesitamos
\begin{align*}
  E &= e^{\frac{2S}{3Nk} -1} \left(\frac{3Nh^2}{4 \pi m V^{\frac{2}{3}}}\right)\\
  \mu &= \frac{\partial E}{\partial N}\\
  &= \frac{\partial e^{\frac{2S}{3Nk} -1} \left(\frac{3Nh^2}{4 \pi m V^{\frac{2}{3}}}\right)}{\partial N}\\
  &= \frac{\partial e^{\frac{2S}{3Nk} -1}}{\partial N} \left(\frac{3Nh^2}{4 \pi m V^{\frac{2}{3}}}\right) + e^{\frac{2S}{3Nk} -1}\frac{\partial \left(\frac{3Nh^2}{4 \pi m V^{\frac{2}{3}}}\right)}{\partial N}\\
  &= -\frac{2S}{3N^2 k}e^{\frac{2S}{3Nk} -1} \left(\frac{3Nh^2}{4 \pi m V^{\frac{2}{3}}}\right) + e^{\frac{2S}{3Nk} -1}\left(\frac{3h^2}{4 \pi m V^{\frac{2}{3}}}\right)\\
  &= -\frac{2S}{3N k}e^{\frac{2S}{3Nk} -1} \left(\frac{3h^2}{4 \pi m V^{\frac{2}{3}}}\right) + e^{\frac{2S}{3Nk} -1}\left(\frac{3h^2}{4 \pi m V^{\frac{2}{3}}}\right)\\
  &= e^{\frac{2S}{3Nk} -1} \left(\frac{3h^2}{4 \pi m V^{\frac{2}{3}}}\right)\left(1 -\frac{2S}{3N k} \right)
.\end{align*}

Y con esto podemos probar si
\begin{align*}
  \mu\left(\lambda N, \lambda V, \lambda S\right) &= \lambda \mu \left(N, V, S\right)\\
  &= e^{\frac{2\lambda S}{3\lambda Nk} -1} \left(\frac{3h^2}{4 \pi m \lambda^{\frac{2}{3}}V^{\frac{2}{3}}}\right)\left(1 -\frac{2\lambda S}{3\lambda N k} \right)\\
  &= e^{\frac{2S}{3Nk} -1} \left(\frac{3h^2}{4 \pi m \lambda^{\frac{2}{3}}V^{\frac{2}{3}}}\right)\left(1 -\frac{2S}{3N k} \right)
\end{align*}

Que como se ve no se coincide con una cantidad intensiva.

\section{}
\section{}

% PUNTO 2
\chapter{}

Tenemos
\begin{align*}
  \varepsilon &= \frac{hc}{2L}\sqrt{n_x^2 + n_y^2 + n_z^2}  \\
  \sqrt{n_x^2 + n_y^2 + n_z^2} &= \frac{2L}{hc}\varepsilon = \varepsilon^{*}
.\end{align*}

Ahora, extendiendo esto a $N$ partículas y tres dimensiones tenemos:
\begin{align*}
  E^{*} &= \sum_{n=1}^{N} \varepsilon^{*} \\
  V &= L^{3} \\
  E &= \sum_{n=1}^{N} \varepsilon
.\end{align*}

Con lo cual podemos desarrollar:
\begin{align*}
  \Omega = E^{*} &= \frac{2V^{\frac{1}{3}}}{hc}\sum_{i=1}^{N} \varepsilon_i \\
  &= \frac{2V^{\frac{1}{3}}}{hc}E \\
  \Omega \propto V^{\frac{1}{3}}E\\
  S = k\ln\left( \Omega \right) \\
  \implies S \propto V^{\frac{1}{3}}E
.\end{align*}

Para un proceso reversible adiabatico:
\begin{align*}
  V^{\frac{1}{3}}E &= CTE \\
  E &= \frac{CTE}{V^{\frac{1}{3}}}
.\end{align*}

Con lo cual
\begin{align*}
  P &= -\left( \frac{\partial E}{\partial V}  \right)_{N, S} = -\frac{1}{3}\frac{1}{V} -E\\
  &= \frac{CTE}{3V^{\frac{4}{3}}} \\
  &= \frac{CTE}{3V^{\frac{4}{3}}} \\
  P V^{\frac{4}{3}}&= \frac{CTE}{3} \\
  PV^{\gamma} &= cte\\
  \frac{4}{3} &= \gamma
.\end{align*}


% PUNTO 3
\chapter{}

\section{}

Tenemos
\begin{align*}
  \varepsilon &= nhv\\
  \frac{\varepsilon}{hv} &= n = \varepsilon^{*} \\
  E^{*}&= \sum_{n=1}^{N} \varepsilon_n^{*} \\
  E^{*}&= \frac{1}{hv}\sum_{n=1}^{N} \varepsilon_n
.\end{align*}

Ahora, para esto necesitamos entonces
\begin{align*}
  \Omega &= \frac{\left( E^{*} + N - 1 \right)!}{E^{*}! \left( N - 1 \right)!}\\
  S &= k\ln\Omega \\
  &= k\ln\left( \frac{\left( E^{*} + N - 1 \right)!}{E^{*}! \left( N - 1 \right)!} \right)  \\
  &= k \ln\left( \left( E^{*} + N -1 \right)! \right) - k\ln E^{*}! - k\ln\left( \left( N - 1 \right)! \right)  \\
 \frac{S}{k} &= \left( E^{*} + N -1 \right)\ln\left( \left( E^{*} + N -1 \right) \right) -\left( E^{*} + N -1 \right) - E^{*}\ln E^{*} + E^{*} - \left( N - 1 \right) \ln\left( \left( N - 1 \right) \right) + \left( N - 1 \right)   \\
 \frac{S}{k} &=  \left( E^{*} + N -1 \right)\ln\left( \left( E^{*} + N -1 \right) \right) - E^{*}\ln E^{*} - \left( N - 1 \right) \ln\left( \left( N - 1 \right) \right) \\
 \frac{S}{k} &= \left( E^{*} + N -1 \right)\ln\left( \left( E^{*} + N -1 \right) \right) - E^{*}\ln E^{*} - \left( N - 1 \right) \ln\left( \left( N - 1 \right) \right) \\
  \frac{S}{k} &= \left( E^{*} \right)\ln\left( \left( E^{*} + N -1 \right) \right) +  \left( N - 1 \right)\ln\left( \left( E^{*} + N -1 \right)\right) - E^{*}\ln E^{*} - \left( N - 1 \right) \ln\left( \left( N - 1 \right) \right) \\
   \frac{S}{k} &= E^{*}\left( \ln\left(\frac{E^{*} + N - 1}{E^{*}} \right)  \right) + \left( N - 1 \right)\ln \frac{E^{*} + N - 1}{N - 1} \\
   N &\gg 1\\
   \frac{S}{k} &= E^{*}\left( \ln\left(\frac{E^{*} + N}{E^{*}} \right)  \right) + \left( N \right)\ln \frac{E^{*} + N}{N} \\
   \frac{S}{k} &= \frac{E}{hv}\left( \ln\left(\frac{\frac{E}{hv} + N}{\frac{E}{hv}} \right)  \right) + \left( N \right) \ln\frac{\frac{E}{hv} + N}{N} \\
   \frac{S}{k} &= \frac{E}{hv}\left( \ln\left(\frac{E + Nhv}{E} \right)  \right) + \left( N \right) \ln\frac{E + Nhv}{Nhv}\\
   S &= k \left( \frac{E}{hv}\left( \ln\left(\frac{E + Nhv}{E} \right)  \right) + \left( N \right) \ln\frac{E + Nhv}{Nhv} \right) 
.\end{align*}

\section{}

Ahora para este caso:
\begin{align*}
  \frac{1}{T} &= \left( \frac{\partial S}{\partial E}  \right)_{N} \\
  \frac{1}{T} &= k\left( \frac{\partial \frac{E}{hv}\left( \ln\left(\frac{E + Nhv}{E} \right)  \right) + \left( N \right) \ln\frac{E + Nhv}{Nhv}}{\partial E}  \right)_{N} \\
  \frac{1}{kT} &= \frac{1}{hv}\ln\left( \frac{E + Nhv}{E} \right) + \frac{E}{hv}\frac{1}{\frac{E + Nhv}{E}}  + \frac{1}{hv}\frac{1}{\frac{E + Nhv}{Nhv}}\\
  \frac{hv}{kT} &= \ln\left( \frac{E + Nhv}{E} \right) + \frac{1}{\frac{E + Nhv}{E}}\left( 1 - \frac{1}{E} \right)   + \frac{1}{\frac{E + Nhv}{Nhv}} - Nhv\\
  \frac{hv}{kT} &= \ln\left( \frac{E + Nhv}{E} \right) + E\frac{E}{E + Nhv}  + \frac{Nhv}{E + Nhv} - Nhv\\
  \frac{hv}{kT} &= \ln\left( \frac{E + Nhv}{E} \right) + \frac{Nhv + E}{E + Nhv} - Nhv\\
  \frac{hv}{kT} &= \ln\left( \frac{E + Nhv}{E} \right) + 1 - Nhv\\
  \frac{hv}{k\ln\left( \frac{E + Nhv}{E} \right) + 1 - Nhv} &=T
.\end{align*}
\section{}

Para este caso entonces el termino que mas aporta es el primero por lo tanto quedamos con:
\begin{align*}
  T &= \frac{hv}{k \ln\left( 1 + \frac{Nhv}{E} \right) } \\
  &= \frac{hv}{k \frac{Nhv}{E}} \\
  &= \frac{Ehv}{k Nhv} \\
  &= \frac{E}{Nk}
.\end{align*}

% PUNTO 4
\chapter{}

\section{}
  Podemos expresar este sistema simplemente reemplazando:
  \begin{align*}
  	S\left(N, V, E\right) &= Nk\ln\left[\frac{V}{h^3}\left(\frac{4 \pi m E}{3N}\right)^{\frac{3}{2}}\right] + \frac{3}{2}Nk \\
	E &= \frac{3}{2}NkT \\
  	S\left(N, V, T\right) &= Nk\ln\left[\frac{V}{h^3}\left(\frac{4 \pi m \frac{3}{2}NkT}{3N}\right)^{\frac{3}{2}}\right] + \frac{3}{2}Nk \\
  	S\left(N, V, T\right) &=Nk\ln V + Nk\ln\left[\frac{1}{h^3}\left(\frac{2 \pi m kT}{1}\right)^{\frac{3}{2}}\right] + \frac{3}{2}Nk \\
  	S\left(N, V, T\right) &=Nk\ln V + Nk\ln\left[\left(\frac{2 \pi m kT}{h^2}\right)^{\frac{3}{2}}\right] + \frac{3}{2}Nk \\
  	S\left(N, V, T\right) &=Nk\ln V + \frac{3}{2}Nk\ln\left[\left(\frac{2 \pi m kT}{h^2}\right)\right] + \frac{3}{2}Nk \\
  	S\left(N, V, T\right) &=Nk\ln V + \frac{3}{2}Nk\left\{ 1 + \ln\left[\left(\frac{2 \pi m kT}{h^2}\right)\right]\right\} \\
  .\end{align*}

  Con esto entonces lo unico que queda es reemplazar:
  \begin{align*}
  	S_1\left(N_1, V_1, T\right) &=N_1k\ln V_1 + \frac{3}{2}N_1k\left\{ 1 + \ln\left[\left(\frac{2 \pi m kT}{h^2}\right)\right]\right\} \\
  	S_2\left(N_2, V_2, T\right) &=N_2k\ln V_2 + \frac{3}{2}N_2k\left\{ 1 + \ln\left[\left(\frac{2 \pi m kT}{h^2}\right)\right]\right\} \\
	S_T &= \sum_{i=1}^{2} \left[ N_ik\ln V + \frac{3}{2}N_ik\left\{ 1 + \ln\left( \frac{2\pi m kT}{h^2} \right)  \right\}  \right]
  .\end{align*}

\section{}
\begin{align*}
  \Delta S &= S_T - S_1 - S_2
.\end{align*}

Notemos que con estas expresiones el segundo termino se cancelan mutuamente. Por lo tanto, solo nos interesa quedarnos con $N_i k\ln V$ lo que implica
\begin{align*}
  \Delta S &= N_1 k\ln V + N_2 k \ln V - N_1 k \ln V_1 - N_2 k \ln V_2 \\
  &= k\left[ N_1 \ln V + N_2 \ln V - N_1 \ln V_1 - N_2 \ln V_2 \right]  \\
  &= k\left[ N_1 \ln \left( \frac{V}{V_1} \right)  + N_2 \ln\left( \frac{V}{V_2} \right) \right]  \\
  &= k\left[ N_1 \ln \left( \frac{V_1 + V_2}{V_1} \right)  + N_2 \ln\left( \frac{V_1 + V_2}{V_2} \right) \right]
.\end{align*}

\section{}

Ahora en este caso partimos de que $\frac{N_1}{V_1} = \frac{N_2}{V_2} = \frac{N_1 + N_2}{V_1 + V_2} = \delta$. Por lo tanto, podemos despejar como:
\begin{align*}
  \frac{N_1 + N_2}{\delta} &= V_1 + V_2 \\
  \frac{N_1}{\delta} &= V_1 \\
  \frac{N_2}{\delta} &= V_2
.\end{align*}

Ahora si lo ponemos todo dentro nos queda:
\begin{align*}
  \Delta S &= k\left[ N_1 \ln \left( \frac{\frac{N_1 + N_2}{\delta}}{\frac{N_1}{\delta}} \right)  + N_2 \ln\left( \frac{\frac{N_1 + N_2}{\delta}}{\frac{N_2}{\delta}} \right) \right]\\
  &= k\left[ N_1 \ln \left( \frac{N_1 + N_2}{N_1} \right)  + N_2 \ln\left( \frac{N_1 + N_2}{N_2} \right) \right]
.\end{align*}

\section{}

En este caso volvemos a partir de la expresión anterior:
\begin{align*}
  \Delta S &= k\left[ N_1 \ln \left( \frac{N_1 + N_2}{N_1} \right)  + N_2 \ln\left( \frac{N_1 + N_2}{N_2} \right) \right]\\
  &= k\left[ N_1 \ln \left( N_1 + N_2 \right) - N_1\ln\left( N_1 \right)  + N_2 \ln\left( N_1 + N_2\right) - N_2\ln\left( N_2 \right)  \right]\\
  &= k\left[ \left( N_1 + N_2 \right)  \ln \left( N_1 + N_2 \right) - N_1\ln\left( N_1 \right) - N_2\ln\left( N_2 \right)  \right] \\
  &= k\left[ \left( N_1 + N_2 \right)  \ln \left( N_1 + N_2 \right) - N_1\ln\left( N_1 \right) - N_2\ln\left( N_2 \right) + N_1 - N_1 + N_2 - N_2 \right] \\
  &= k\left[ \left( N_1 + N_2 \right)  \ln \left( N_1 + N_2 \right) - \left( N_1 + N_2 \right) - N_1\ln\left( N_1 \right) + N_1 - N_2\ln\left( N_2 \right) + N_2 \right] \\
  &= k\left[ \left( N_1 + N_2 \right)  \ln \left( N_1 + N_2 \right) - \left( N_1 + N_2 \right) - \left(N_1\ln\left( N_1 \right) - N_1\right) + \left(N_2\ln\left( N_2 \right) - N_2\right) \right]
.\end{align*}

Esto dado que $N$ es grande nos permite usar la aproximación de Stirling $\ln\left( N! \right) = N \ln\left( N \right) - N$. Lo que entonces nos permite poner esto como se nos pide:
\begin{align*}
  \Delta S &= k\left[ \left( N_1 + N_2 \right)  \ln \left( N_1 + N_2 \right) - \left( N_1 + N_2 \right) - \left(N_1\ln\left( N_1 \right) - N_1\right) - \left(N_2\ln\left( N_2 \right) - N_2\right) \right] \\
  \Delta S &= k\left[ \ln \left( \left( N_1 + N_2 \right)! \right) - \ln\left( N_1! \right) - \ln\left( N_2! \right)\right]
.\end{align*}

\section{}

Cuando despejamos con la consideración de Gibbs queda:
\begin{align*}
  S\left(N, V, E\right) &= Nk\ln\left[\frac{V}{h^3}\left(\frac{4 \pi m E}{3N}\right)^{\frac{3}{2}}\right] + \frac{3}{2}Nk - k\ln\left( N! \right)  \\
  S\left(N, V, E\right) &= Nk\ln\left[\frac{V}{h^3}\left(\frac{4 \pi m E}{3N}\right)^{\frac{3}{2}}\right] + \frac{3}{2}Nk - kN\ln\left( N \right)+ Nk  \\
  S\left( N, V, E \right) &= Nk\ln\left[ \frac{V}{Nh^{3}}\left( \frac{4\pi mE}{3N} \right)^{\frac{3}{2}} \right] + \frac{5}{2}Nk \\
  &= Nk\ln\left( \frac{V}{N} \right) + Nk\ln\left[ \left( \frac{4\pi mE}{3N h^2} \right)^{\frac{3}{2}} \right] + \frac{5}{2}Nk \\
  &= Nk\ln\left( \frac{V}{N} \right) + \frac{3}{2}Nk\ln\left[ \left( \frac{4\pi mE}{3N h^2} \right) \right] + \frac{5}{2}Nk \\
  &= Nk\ln\left( \frac{V}{N} \right) + \frac{3}{2}Nk\ln\left[ \left( \frac{4\pi m \frac{3}{2} Nk T}{3N h^2} \right) \right] + \frac{5}{2}Nk \\
  &= Nk\ln\left( \frac{V}{N} \right) + \frac{3}{2}Nk\left\{ \frac{5}{3} + \ln \left( \frac{2\pi m kT}{h^2}  \right)  \right\}
.\end{align*}

Ahora volvamos a notar que en el caso de que $\Delta S$ el segundo termino se cancelaría mutuamente. Por lo tanto solo nos interesa el primer caso con lo cual tendríamos:
\begin{align*}
    \Delta S &= k\left[ \left( N_1 + N_2 \right) \ln\left( \frac{V_1 + V_2}{N_1 + N_2} \right) - N_1\ln\left( \frac{V_1}{N_1} \right) - N_2\ln\left( \frac{V_2}{N_2} \right) \right]
.\end{align*}

Ahora, en el caso de $\frac{N_1}{V_1} = \frac{N_2}{V_2} = \frac{N_1 + N_2}{V_1 + V_2} = \delta^{-1}$ nos queda
\begin{align*}
    \Delta S &= k\left[ \left( N_1 + N_2 \right) \ln\left( \delta \right) - N_1\ln\left( \delta \right) - N_2\ln\left( \delta \right) \right]\\
    \Delta S &= k\left[ \left( N_1 + N_2 \right) \ln\left( \delta \right) - \left( N_1 + N_2 \right) \ln\left( \delta \right)\right]\\
    \Delta S &= k\left[0\right]\\
    \Delta S &= 0
.\end{align*}

% PUNTO 5
\chapter{}

\section{}

Si partimos de un espacio de fase entonces la manera en la que estos puntos cambian en el espacio es:
\begin{align*}
  \frac{\partial}{\partial t} \int_{\omega} \rho d\omega\\
  \omega &= d^{3N}q d^{3N}p
.\end{align*}

Ahora bien, el ratio neto en que los puntos se mueven fuera de nuestra superficie queda:
\begin{align*}
  \int_{\sigma} \rho\left( v\cdot \hat{n} \right) d\sigma&= \int_{\omega} div\left( \rho v \right) d\omega \\
  div(pv) &= \sum_{i=1}^{3N} \left\{ \frac{\partial}{\partial q_i} \left( \rho \dot{q_i} \right) + \frac{\partial}{\partial p_i} \left( \rho \dot{p_i}\right)  \right\} 
.\end{align*}

Dado que no hay fuentes ni sumideros entonces sabemos que:
\begin{align*}
  \frac{\partial}{\partial t} \int_{\omega} \rho d\omega &= - \int_{\omega}div\left( \rho v \right) d\omega \\
  \int_{\omega}\left\{ \frac{\partial \rho}{\partial t} + div\left( \rho v \right)  \right\} d\omega &= 0 \\
  \frac{\partial \rho}{\partial t} + div\left( \rho v \right) &= 0 \\
  \frac{\partial \rho}{\partial t} + \sum_{i=1}^{3N} \left( \frac{\partial \rho}{\partial q_i} \dot{q_i} + \frac{\partial \rho}{\partial p_i} \dot{p_i} \right) + \rho \sum_{i=1}^{3N} \left( \frac{\partial \dot{q_i}}{\partial q_i}  + \frac{\partial \dot{p_i}}{\partial p_i}  \right) &= 0 \\
  \frac{\partial \dot{q_i}}{\partial q_i} &= \frac{\partial^2 H}{\partial q_i \partial p_i} = - \frac{\partial \dot{p_i}}{\partial p_i}  \\
  \frac{\partial \rho}{\partial t} + \sum_{i=1}^{3N} \left( \frac{\partial \rho}{\partial q_i} \dot{q_i} + \frac{\partial \rho}{\partial p_i} \dot{p_i} \right) + \rho \sum_{i=1}^{3N} \left( -\frac{\partial \dot{p_i}}{\partial p_i}  + \frac{\partial \dot{p_i}}{\partial p_i}  \right) &= 0 \\
  \frac{\partial \rho}{\partial t} + \sum_{i=1}^{3N} \left( \frac{\partial \rho}{\partial q_i} \dot{q_i} + \frac{\partial \rho}{\partial p_i} \dot{p_i} \right) &= 0 \\
.\end{align*}

\section{}

Partimos del teorema:
\begin{align*}
  \frac{\partial \rho}{\partial t} + \sum_{i=1}^{3N} \left( \frac{\partial \rho}{\partial q_i} \dot{q_i} + \frac{\partial \rho}{\partial p_i} \dot{p_i} \right) &= 0 \\
  \rho &= cte\\
  \frac{\partial cte}{\partial t} + \sum_{i=1}^{3N} \left( \frac{\partial cte}{\partial q_i} \dot{q_i} + \frac{\partial cte}{\partial p_i} \dot{p_i} \right) &= 0 \\
  0 + \sum_{i=1}^{3N} \left( 0 \dot{q_i} + 0 \dot{p_i} \right) &= 0 \\
  0 &= 0
.\end{align*}
\section{}

Partimos del teorema:
\begin{align*}
  \frac{\partial \rho}{\partial t} + \sum_{i=1}^{3N} \left( \frac{\partial \rho}{\partial q_i} \dot{q_i} + \frac{\partial \rho}{\partial p_i} \dot{p_i} \right) &= 0 \\
  \frac{\partial \rho\left[H(q_i,p_i)\right]}{\partial t} + \sum_{i=1}^{3N} \left( \frac{\partial \rho\left[H(q_i,p_i)\right]}{\partial q_i} \dot{q_i} + \frac{\partial \rho\left[H(q_i,p_i)\right]}{\partial p_i} \dot{p_i} \right) &= 0 \\
  0 + \sum_{i=1}^{3N} \left( \rho \frac{\partial \left[H(q_i,p_i)\right]}{\partial q_i} \dot{q_i} + \rho\frac{\partial \left[H(q_i,p_i)\right]}{\partial p_i} \dot{p_i} \right) &= 0 \\
  \frac{\partial \left[H(q_i,p_i)\right]}{\partial p_i} &=  \dot{q_i}\\
  \frac{\partial \left[H(q_i,p_i)\right]}{\partial q_i} &= - \dot{p_i}\\
  0 + \sum_{i=1}^{3N} \left( -\rho\dot{p_i} \dot{q_i} + \rho \dot{q_i} \dot{p_i} \right) &= 0 \\
  0 &= 0 
.\end{align*}

% PUNTO 6
\chapter{}

\section{}

En este caso tenemos solo un grado de libertad por lo tanto tenemos solo una variable a la que llamaremos $\theta$. Por lo tanto, su momento angular seria $p_\theta = mL^2\theta$
\end{document}

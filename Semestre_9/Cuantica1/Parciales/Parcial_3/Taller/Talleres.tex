\documentclass[12pt]{exam}
\usepackage{amsthm}
\usepackage{libertine}
\usepackage[utf8]{inputenc}
\usepackage[margin=1in]{geometry}
\usepackage{amsmath,amssymb}
\usepackage{multicol}
\usepackage[shortlabels]{enumitem}
\usepackage{siunitx}
\usepackage{cancel}
\usepackage{graphicx}
\usepackage{pgfplots}
\usepackage{listings}
\usepackage{tikz}


\pgfplotsset{width=10cm,compat=1.9}
\usepgfplotslibrary{external}
\tikzexternalize

\newcommand{\class}{Cuantica} % This is the name of the course 
\newcommand{\examnum}{Taller 3} % This is the name of the assignment
\newcommand{\examdate}{\today} % This is the due date
\newcommand{\timelimit}{}
\NewDocumentCommand{\bra}{m}{\left<#1\right|}
\NewDocumentCommand{\ket}{m}{\left|#1\right>}
\NewDocumentCommand{\braket}{m m}{\left<#1|#2\right>}
\NewDocumentCommand{\med}{m}{\left<#1\right>}
\NewDocumentCommand{\paren}{m}{\left(#1\right)}





\begin{document}
\pagestyle{plain}
\thispagestyle{empty}

\noindent
\begin{tabular*}{\textwidth}{l @{\extracolsep{\fill}} r @{\extracolsep{6pt}} l}
	\textbf{\class} & \textbf{Name:} & \textit{Sergio Montoya}\\ %Your name here instead, obviously 
	\textbf{\examnum} &&\\
	\textbf{\examdate} &&
\end{tabular*}\\
\rule[2ex]{\textwidth}{2pt}
% ---

\section{Primer Punto}

\subsection{Enunciado}

Considere un oscilador armonico simple en una dimensión. La función de onda $\psi$ en $t = 0$ es
\begin{equation}
	\psi(0) = \frac{1}{\sqrt{5}}\left|1\right> + \frac{2}{\sqrt{5}} \left| 2\right> \label{eq:1}
\end{equation}

Donde $\left| n \right>$ representa el estado propio para una energia $E_n$.
\begin{enumerate}
	\item Escriba la función de onda para cualquier tiempo $t$.
	\item Escriba el valor esperado para la energia.
	\item Encuentre la expresión para el valor esperado de la posición dependiente del tiempo:
		\begin{equation}
			\left<\psi(t)\right|\hat{x}\left|\psi(t)\right>\label{eq:2_3}
		\end{equation}
\end{enumerate}

\subsection{Solución}

\begin{enumerate}
	\item
		Para iniciar debemos considerar que para convertir una función de onda independiente del tiempo a una dependiente del tiempo en este caso debemos multiplicar por $e^{-iE_nt/\hbar}$ en cada uno de los terminos de \ref{eq:1}. Tomando en consideración que para un oscilador armonico se cumple que: \[E_n = \left(n + \frac{1}{2}\right)\hbar\omega\] tendriamos los siguientes valores:
		\begin{align}
			E_{1} &= \left(1 + \frac{1}{2}\right)\hbar\omega\nonumber\\
			&= \frac{3}{2}\hbar\omega\label{eq:e1}\\
			E_2 &= \left(2 + \frac{1}{2}\right)\hbar\omega\nonumber\\
			&= \frac{5}{2}\hbar\omega\label{eq:e2}
		\end{align}
	
		Con lo cual los valores serian:
		\begin{align*}
			e^{-\frac{iE_1t}{\hbar}} &= e^{-\frac{i\frac{3}{2}\hbar\omega t}{\hbar}}\\
			&= e^{-i\frac{3}{2}\omega t}\\
			e^{-\frac{iE_2t}{\hbar}} &= e^{-\frac{i\frac{5}{2}\hbar\omega t}{\hbar}}\\
			&= e^{-i\frac{5}{2}\omega t}
		\end{align*}

		Con lo cual la función de onda para cualquier $t$ seria:
		\begin{equation}
			\psi(t) = \frac{1}{\sqrt{5}}e^{-i\frac{3}{2}\omega t}\left|1\right> + \frac{2}{\sqrt{5}} e^{-i\frac{5}{2}\omega t}\left| 2\right> \label{eq:res_1a}
		\end{equation}
	\item 
		La expresión general para la energia promedio es:
		\begin{equation}
			\bra{\psi(0)}\hat{H}\ket{\psi(0)}\label{eq:E_med}
		\end{equation}

		Teniendo entonces dada la equación \ref{eq:1} nos queda
		\begin{align}
			\med{E} &= \paren{\frac{1}{\sqrt{5}}\bra{1} + \frac{2}{\sqrt{5}}\bra{2}}\hat{H}\paren{\frac{1}{\sqrt{5}}\ket{1} + \frac{2}{\sqrt{5}}\ket{2}}\nonumber\\
			&= \frac{1}{\sqrt{5}}\frac{1}{\sqrt{5}}\bra{1}\hat{H}\ket{1} +
			\frac{1}{\sqrt{5}}\frac{2}{\sqrt{5}}\bra{1}\hat{H}\ket{2} +
			\frac{2}{\sqrt{5}}\frac{1}{\sqrt{5}}\bra{2}\hat{H}\ket{1} +
			\frac{2}{\sqrt{5}}\frac{2}{\sqrt{5}}\bra{2}\hat{H}\ket{2}\nonumber\\
			&= \frac{1}{5}\bra{1}\hat{H}\ket{1} +
			\frac{2}{5}\bra{1}\hat{H}\ket{2} +
			\frac{2}{5}\bra{2}\hat{H}\ket{1} +
			\frac{4}{5}\bra{2}\hat{H}\ket{2}\label{eq:med_1}
		\end{align}

		Ahora bien, tambien necesitaremos que:
		\begin{equation}
			\hat{H}\ket{n} = E_n\ket{n}
		\end{equation}

		que al aplicarlo en \ref{eq:med_1}
		\begin{align*}
			\med{E} &=
			\frac{1}{5}\bra{1}\hat{H}\ket{1} +
			\frac{2}{5}\bra{1}\hat{H}\ket{2} +
			\frac{2}{5}\bra{2}\hat{H}\ket{1} +
			\frac{4}{5}\bra{2}\hat{H}\ket{2}\\
			&=
			\frac{1}{5}\bra{1}E_1\ket{1} +
			\frac{2}{5}\bra{1}E_2\ket{2} +
			\frac{2}{5}\bra{2}E_1\ket{1} +
			\frac{4}{5}\bra{2}E_2\ket{2}\\
			&=
			\frac{1}{5}E_1\braket{1}{1} +
			\frac{2}{5}E_2\braket{1}{2} +
			\frac{2}{5}E_1\braket{2}{1} +
			\frac{4}{5}E_2\braket{2}{2}\\
			&=
			\frac{1}{5}E_1 +
			\frac{2}{5}E_2 \cdot 0 +
			\frac{2}{5}E_1 \cdot 0 +
			\frac{4}{5}E_2\\
			&=
			\frac{1}{5}E_1 +
			\frac{4}{5}E_2\\
		\end{align*}

		Ahora recordando lo encontrado en \ref{eq:e1} y \ref{eq:e2} nos queda:
		\begin{align*}
			\med{E}&=
			\frac{1}{5}E_1 +
			\frac{4}{5}E_2\\
			&=
			\frac{1}{5}\frac{3}{2}\hbar\omega +
			\frac{4}{5}\frac{5}{2}\hbar\omega\\
			&=
			\frac{3}{10}\hbar\omega +
			\frac{20}{10}\hbar\omega\\
			&=
			\frac{23}{10}\hbar\omega\\
		\end{align*}
	\item
		Dada la descripción que tenemos en \ref{eq:2_3} consideremos que
		\begin{equation}
			\hat{x} = \sqrt{\frac{\hbar}{2m\omega}}\paren{a + a^{\dagger}}\label{eq:hatx}
		\end{equation}

		Con esto entonces podemos reemplazar:
		\begin{align}
			\med{\hat{x}}(t) &= \left<\psi(t)\right|\hat{x}\left|\psi(t)\right>\nonumber\\
			&= \paren{\frac{1}{\sqrt{5}}e^{i\frac{3}{2}\omega t}\bra{1} + \frac{2}{\sqrt{5}}e^{i\frac{5}{2}\omega t}\bra{2}}\hat{x}\paren{\frac{1}{\sqrt{5}}e^{-i\frac{3}{2}\omega t}\ket{1} + \frac{2}{\sqrt{5}}e^{-i\frac{5}{2}\omega t}\ket{2}}\nonumber\\
			&=
			\frac{1}{\sqrt{5}}\frac{1}{\sqrt{5}}e^{i\frac{3}{2}\omega t - i \frac{3}{2}\omega t}\bra{1}\hat{x}\ket{1} +
			\frac{1}{\sqrt{5}}\frac{2}{\sqrt{5}}e^{i\frac{3}{2}\omega t - i \frac{5}{2}\omega t}\bra{1}\hat{x}\ket{2}\nonumber\\
			&+ \frac{2}{\sqrt{5}}\frac{1}{\sqrt{5}}e^{i\frac{5}{2}\omega t - i \frac{3}{2}\omega t}\bra{2}\hat{x}\ket{1} +
			\frac{2}{\sqrt{5}}\frac{2}{\sqrt{5}}e^{i\frac{5}{2}\omega t - i \frac{5}{2}\omega t}\bra{2}\hat{x}\ket{2}\nonumber\\
			&=
			\frac{1}{5}\bra{1}\hat{x}\ket{1} +
			\frac{2}{5}e^{-i\omega t}\bra{1}\hat{x}\ket{2} +
			\frac{2}{5}e^{i\omega t}\bra{2}\hat{x}\ket{1} +
			\frac{4}{5}\bra{2}\hat{x}\ket{2}\label{eq:med_2}
		\end{align}

		Podemos saber dada la definición \ref{eq:hatx} con los operadores de creación y aniquilación que para quelos estados no se aniquilen mutuamente deben estar separados por unicamente un paso. Es decir, para nuestro caso concreto con \ref{eq:med_2} tenemos que se aniquilan los puntos con $\bra{n}\hat{x}\ket{n}$. Esto nos deja con:
		\begin{align}
			\med{\hat{x}}(t)
			&=
			\frac{1}{5}\bra{1}\hat{x}\ket{1} +
			\frac{2}{5}e^{-i\omega t}\bra{1}\hat{x}\ket{2} +
			\frac{2}{5}e^{i\omega t}\bra{2}\hat{x}\ket{1} +
			\frac{4}{5}\bra{2}\hat{x}\ket{2}\nonumber\\
			&=
			\frac{2}{5}e^{-i\omega t}\bra{1}\hat{x}\ket{2} +
			\frac{2}{5}e^{i\omega t}\bra{2}\hat{x}\ket{1}\label{eq:med_3}
		\end{align}

		Para cada uno de estos casos veamoslo por aparte:
		\begin{itemize}
			\item $\bra{1}\hat{x}\ket{2}$:
				\begin{align}
					\bra{1}\hat{x}\ket{2} &= \sqrt{\frac{\hbar}{2m\omega}}\bra{1}\paren{a + a^{\dagger}}\ket{2}\nonumber\\
					&= \sqrt{\frac{\hbar}{2m\omega}}\bra{1}\paren{a\ket{2} + a^{\dagger}\ket{2}}\nonumber
				\end{align}

				Considerando que

				\begin{align}
					a\ket{2} &= \sqrt{2}\ket{1}\nonumber\\
					a^{\dagger}\ket{2} &= \sqrt{3}\ket{3}\nonumber
				\end{align}

				Podemos reemplazar directamente y obtener:

				\begin{align}
					\bra{1}\hat{x}\ket{2} &= \sqrt{\frac{\hbar}{2m\omega}}\bra{1}\paren{a\ket{2} + a^{\dagger}\ket{2}}\nonumber\\
					&= \sqrt{\frac{\hbar}{2m\omega}}\bra{1}\paren{\sqrt{2}\ket{1} + \sqrt{3}\ket{3}}\nonumber\\
					&= \sqrt{\frac{\hbar}{2m\omega}}\paren{\bra{1}\sqrt{2}\ket{1} + \bra{1}\sqrt{3}\ket{3}}\nonumber\\
					&= \sqrt{\frac{\hbar}{2m\omega}}\paren{\sqrt{2}\braket{1}{1} + \sqrt{3}\braket{1}{3}}\nonumber\\
					&= \sqrt{\frac{\hbar}{2m\omega}}\sqrt{2}\nonumber\\
					&= \sqrt{\frac{2\hbar}{2m\omega}}\nonumber\\
					&= \sqrt{\frac{\hbar}{m\omega}}\label{eq:1_3_1_2}
				\end{align}
			\item $\bra{2}\hat{x}\ket{1}$:
				\begin{align}
					\bra{2}\hat{x}\ket{1} &= \sqrt{\frac{\hbar}{2m\omega}}\bra{2}\paren{a + a^{\dagger}}\ket{1}\nonumber\\
					&= \sqrt{\frac{\hbar}{2m\omega}}\bra{2}\paren{a\ket{1} + a^{\dagger}\ket{1}}\nonumber
				\end{align}

				Considerando que

				\begin{align}
					a\ket{1} &= \sqrt{1}\ket{0}\nonumber\\
					a^{\dagger}\ket{1} &= \sqrt{2}\ket{2}\nonumber
				\end{align}

				Podemos reemplazar directamente y obtener

				\begin{align}
					\bra{1}\hat{x}\ket{2} &= \sqrt{\frac{\hbar}{2m\omega}}\bra{2}\paren{a\ket{1} + a^{\dagger}\ket{1}}\nonumber\\
					&= \sqrt{\frac{\hbar}{2m\omega}}\bra{2}\paren{\sqrt{1}\ket{0} + \sqrt{2}\ket{2}}\nonumber\\
					&= \sqrt{\frac{\hbar}{2m\omega}}\paren{\bra{2}\sqrt{1}\ket{0} + \bra{2}\sqrt{2}\ket{2}}\nonumber\\
					&= \sqrt{\frac{\hbar}{2m\omega}}\paren{\sqrt{1}\braket{1}{0} + \sqrt{2}\braket{2}{2}}\nonumber\\
					&= \sqrt{\frac{\hbar}{2m\omega}}\sqrt{2}\nonumber\\
					&= \sqrt{\frac{2\hbar}{2m\omega}}\nonumber\\
					&= \sqrt{\frac{\hbar}{m\omega}}\label{eq:1_3_2_1}
				\end{align}
			\item \textbf{Nota:}
				Decidi escribir explicitamente que los resultados eran los mismos para \ref{eq:1_3_2_1} y para \ref{eq:1_3_1_2}. Sin embargo, se podia llegar al mismo resultado notando que
				\[\bra{2}\hat{x}\ket{1} = \paren{\bra{1}\hat{x}\ket{2}}^{*} = \paren{\sqrt{\frac{\hbar}{m\omega}}}^* = \sqrt{\frac{\hbar}{m\omega}}.\]

				Esto es valioso tomarlo en cuenta particularemente si uno se encuentra de afan en un parcial
		\end{itemize}

		Ahora bien, ingresando los resultado \ref{eq:1_3_1_2} y \ref{eq:1_3_2_1} en \ref{eq:med_3} nos quedaria:
		\begin{align*}
			\med{\hat{x}}(t)&=
			\frac{2}{5}e^{-i\omega t}\bra{1}\hat{x}\ket{2} +
			\frac{2}{5}e^{i\omega t}\bra{2}\hat{x}\ket{1}\\
			&=
			\frac{2}{5}e^{-i\omega t}\sqrt{\frac{\hbar}{m\omega}} +
			\frac{2}{5}e^{i\omega t}\sqrt{\frac{\hbar}{m\omega}}\\
			&=
			\frac{2}{5}\sqrt{\frac{\hbar}{m\omega}} \paren{e^{i\omega t} + e^{-i\omega t}}\\
			e^{i\theta} + e^{-i\theta} &= \cos(\theta) + i\sin(\theta) + \cos(\theta) - i\sin(\theta)\\
			&= 2\cos(\theta)\\
			&\implies e^{i\omega t} + e^{-i\omega t} = 2\cos(\omega t)\\
			\med{\hat{x}}(t)
			&=
			\frac{4}{5}\sqrt{\frac{\hbar}{m\omega}} \cos(\omega t)\\
		\end{align*}
\end{enumerate}
\break

\section{Pregunta 2}

\subsection{Enunciado}
La función de onda, en $t = 0$, para un atomo de hidrogeno, esta dada por:
\begin{equation}
	\psi(r, \theta, \phi) = 2\psi_{100} + \psi_{210}
\end{equation}
\begin{enumerate}
	\item Normalice esta función
	\item ¿Cuales son los posibles resultados si se mide individualmente la energia, el momento angular, y la componente $z$ del momento angular?
	\item Calcule la probabilidad de obtener los valores de energia hallados en el punto (b)
	\item Calcule el valor esperado para la energia, el momento angular y la componente $z$ del momento angular.
\end{enumerate}

\subsection{Solución}
\begin{enumerate}
	\item Para normalizar nos interesa que sobre todo el espacio se cumpla que la suma de las probabilidades sea $1$. Esto es equivalente a mostrar
		\begin{equation}
			\int \left|\psi(r,\theta,\phi)\right|^2 dV = 1
		\end{equation}

		Ahora con esto entonces tenemos:
		\begin{align*}
			\int \left|\psi(r,\theta,\phi)\right|^2 dV &= 1\\
			\int \left|N\paren{2\psi_{100} + \psi_{210}}\right|^2 dV &= 1
		\end{align*}
		
		Por ortogonalidad en $\psi_{nlm}$ se cancela cualquier combinación que no sea del mismo estado lo que nos deja con:
		\begin{align*}
			\left|N\right|^2\int 4\left|\psi_{100}\right|^2 + \left|\psi_{210}\right|^2 dV &= 1\\
			\left|N\right|^2\paren{\int 4\left|\psi_{100}\right|^2 dV + \int \left|\psi_{210}\right|^2 dV} &= 1\\
			\left|N\right|^2\paren{4 + 1} &= 1\\
			\left|N\right|^2\paren{5} &= 1\\
			\left|N\right|^2 &= \frac1{5}\\
			\left|N\right| &= \frac1{\sqrt{5}}\\
			N &= \frac1{\sqrt{5}}\\
		\end{align*}
	\item 
\end{enumerate}
\break

\section{Pregunta 3}

\subsection{Enunciado}
Considere un electrón que es preparado en el estado $\ket{\uparrow}$ y que es inyectado en un campo magnético uniforme en la dirección $y$ con magnitud $B_y$, durante un tiempo $t$.
\begin{enumerate}
	\item Escriba el operador de evolución temporal $\hat{U}(t)$ que describe la evolución temporal del estado del electrón como una expansión en la base de autoestados de $\hat{S}^2$ y $\hat{S}_y$.
	\item Escriba al estado del electrón después de un tiempo $t$ como una superposición de los estados en la base $\ket{\uparrow}$ y $\ket{\downarrow}$.
	\item ¿Cual es la probabilidad de que el electrón sea detectado en los estados $\ket{\uparrow_x}$ y $\ket{\downarrow_x}$ después de un tiempo $t$?
	\item ¿Cual es la probabilidad de que el electrón sea detectado en los estados $\ket{\uparrow_y}$ y $\ket{\downarrow_y}$ después de un tiempo $t$?
\end{enumerate}

\subsection{Solución}

\begin{enumerate}
  \item 
    En este caso, el Hamiltoniano de interacción entre el campo magnético y el spin del electrón conmutan con el operador de momento angular $\hat{S}^2$ y $\hat{S}_y$
\end{enumerate}

\end{document}

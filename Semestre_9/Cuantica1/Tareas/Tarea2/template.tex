\documentclass{report}

\documentclass[12pt]{article}
\usepackage{array}
\usepackage{color}
\usepackage{amsthm}
\usepackage{eufrak}
\usepackage{lipsum}
\usepackage{pifont}
\usepackage{yfonts}
\usepackage{amsmath}
\usepackage{amssymb}
\usepackage{ccfonts}
\usepackage{comment} \usepackage{amsfonts}
\usepackage{fancyhdr}
\usepackage{graphicx}
\usepackage{listings}
\usepackage{mathrsfs}
\usepackage{setspace}
\usepackage{textcomp}
\usepackage{blindtext}
\usepackage{enumerate}
\usepackage{microtype}
\usepackage{xfakebold}
\usepackage{kantlipsum}
%\usepackage{draftwatermark}
\usepackage[spanish]{babel}
\usepackage[margin=1.5cm, top=2cm, bottom=2cm]{geometry}
\usepackage[framemethod=tikz]{mdframed}
\usepackage[colorlinks=true,citecolor=blue,linkcolor=red,urlcolor=magenta]{hyperref}

%//////////////////////////////////////////////////////
% Watermark configuration
%//////////////////////////////////////////////////////
%\SetWatermarkScale{4}
%\SetWatermarkColor{black}
%\SetWatermarkLightness{0.95}
%\SetWatermarkText{\texttt{Watermark}}

%//////////////////////////////////////////////////////
% Frame configuration
%//////////////////////////////////////////////////////
\newmdenv[tikzsetting={draw=gray,fill=white,fill opacity=0},backgroundcolor=none]{Frame}

%//////////////////////////////////////////////////////
% Font style configuration
%//////////////////////////////////////////////////////
\renewcommand{\familydefault}{\ttdefault}
\renewcommand{\rmdefault}{tt}

%//////////////////////////////////////////////////////
% Bold configuration
%//////////////////////////////////////////////////////
\newcommand{\fbseries}{\unskip\setBold\aftergroup\unsetBold\aftergroup\ignorespaces}
\makeatletter
\newcommand{\setBoldness}[1]{\def\fake@bold{#1}}
\makeatother

%//////////////////////////////////////////////////////
% Default font configuration
%//////////////////////////////////////////////////////
\DeclareFontFamily{\encodingdefault}{\ttdefault}{%
  \hyphenchar\font=\defaulthyphenchar
  \fontdimen2\font=0.33333em
  \fontdimen3\font=0.16667em
  \fontdimen4\font=0.11111em
  \fontdimen7\font=0.11111em}


%From M275 "Topology" at SJSU
\newcommand{\id}{\mathrm{id}}
\newcommand{\taking}[1]{\xrightarrow{#1}}
\newcommand{\inv}{^{-1}}

%From M170 "Introduction to Graph Theory" at SJSU
\DeclareMathOperator{\diam}{diam}
\DeclareMathOperator{\ord}{ord}
\newcommand{\defeq}{\overset{\mathrm{def}}{=}}

%From the USAMO .tex files
\newcommand{\ts}{\textsuperscript}
\newcommand{\dg}{^\circ}
\newcommand{\ii}{\item}

% % From Math 55 and Math 145 at Harvard
% \newenvironment{subproof}[1][Proof]{%
% \begin{proof}[#1] \renewcommand{\qedsymbol}{$\blacksquare$}}%
% {\end{proof}}

\newcommand{\liff}{\leftrightarrow}
\newcommand{\lthen}{\rightarrow}
\newcommand{\opname}{\operatorname}
\newcommand{\surjto}{\twoheadrightarrow}
\newcommand{\injto}{\hookrightarrow}
\newcommand{\On}{\mathrm{On}} % ordinals
\DeclareMathOperator{\img}{im} % Image
\DeclareMathOperator{\Img}{Im} % Image
\DeclareMathOperator{\coker}{coker} % Cokernel
\DeclareMathOperator{\Coker}{Coker} % Cokernel
\DeclareMathOperator{\Ker}{Ker} % Kernel
\DeclareMathOperator{\rank}{rank}
\DeclareMathOperator{\Spec}{Spec} % spectrum
\DeclareMathOperator{\Tr}{Tr} % trace
\DeclareMathOperator{\pr}{pr} % projection
\DeclareMathOperator{\ext}{ext} % extension
\DeclareMathOperator{\pred}{pred} % predecessor
\DeclareMathOperator{\dom}{dom} % domain
\DeclareMathOperator{\ran}{ran} % range
\DeclareMathOperator{\Hom}{Hom} % homomorphism
\DeclareMathOperator{\Mor}{Mor} % morphisms
\DeclareMathOperator{\End}{End} % endomorphism

\newcommand{\eps}{\epsilon}
\newcommand{\veps}{\varepsilon}
\newcommand{\ol}{\overline}
\newcommand{\ul}{\underline}
\newcommand{\wt}{\widetilde}
\newcommand{\wh}{\widehat}
\newcommand{\vocab}[1]{\textbf{\color{blue} #1}}
\providecommand{\half}{\frac{1}{2}}
\newcommand{\dang}{\measuredangle} %% Directed angle
\newcommand{\ray}[1]{\overrightarrow{#1}}
\newcommand{\seg}[1]{\overline{#1}}
\newcommand{\arc}[1]{\wideparen{#1}}
\DeclareMathOperator{\cis}{cis}
\DeclareMathOperator*{\lcm}{lcm}
\DeclareMathOperator*{\argmin}{arg min}
\DeclareMathOperator*{\argmax}{arg max}
\newcommand{\cycsum}{\sum_{\mathrm{cyc}}}
\newcommand{\symsum}{\sum_{\mathrm{sym}}}
\newcommand{\cycprod}{\prod_{\mathrm{cyc}}}
\newcommand{\symprod}{\prod_{\mathrm{sym}}}
\newcommand{\Qed}{\begin{flushright}\qed\end{flushright}}
\newcommand{\parinn}{\setlength{\parindent}{1cm}}
\newcommand{\parinf}{\setlength{\parindent}{0cm}}
% \newcommand{\norm}{\|\cdot\|}
\newcommand{\inorm}{\norm_{\infty}}
\newcommand{\opensets}{\{V_{\alpha}\}_{\alpha\in I}}
\newcommand{\oset}{V_{\alpha}}
\newcommand{\opset}[1]{V_{\alpha_{#1}}}
\newcommand{\lub}{\text{lub}}
\newcommand{\del}[2]{\frac{\partial #1}{\partial #2}}
\newcommand{\Del}[3]{\frac{\partial^{#1} #2}{\partial^{#1} #3}}
\newcommand{\deld}[2]{\dfrac{\partial #1}{\partial #2}}
\newcommand{\Deld}[3]{\dfrac{\partial^{#1} #2}{\partial^{#1} #3}}
\newcommand{\lm}{\lambda}
\newcommand{\uin}{\mathbin{\rotatebox[origin=c]{90}{$\in$}}}
\newcommand{\usubset}{\mathbin{\rotatebox[origin=c]{90}{$\subset$}}}
\newcommand{\lt}{\left}
\newcommand{\rt}{\right}
\newcommand{\paren}[1]{\left(#1\right)}
\newcommand{\bs}[1]{\boldsymbol{#1}}
\newcommand{\exs}{\exists}
\newcommand{\st}{\strut}
\newcommand{\dps}[1]{\displaystyle{#1}}

\newcommand{\sol}{\setlength{\parindent}{0cm}\textbf{\textit{Solution:}}\setlength{\parindent}{1cm} }
\newcommand{\solve}[1]{\setlength{\parindent}{0cm}\textbf{\textit{Solution: }}\setlength{\parindent}{1cm}#1 \Qed}

% Things Lie
\newcommand{\kb}{\mathfrak b}
\newcommand{\kg}{\mathfrak g}
\newcommand{\kh}{\mathfrak h}
\newcommand{\kn}{\mathfrak n}
\newcommand{\ku}{\mathfrak u}
\newcommand{\kz}{\mathfrak z}
\DeclareMathOperator{\Ext}{Ext} % Ext functor
\DeclareMathOperator{\Tor}{Tor} % Tor functor
\newcommand{\gl}{\opname{\mathfrak{gl}}} % frak gl group
\renewcommand{\sl}{\opname{\mathfrak{sl}}} % frak sl group chktex 6

% More script letters etc.
\newcommand{\SA}{\mathcal A}
\newcommand{\SB}{\mathcal B}
\newcommand{\SC}{\mathcal C}
\newcommand{\SF}{\mathcal F}
\newcommand{\SG}{\mathcal G}
\newcommand{\SH}{\mathcal H}
\newcommand{\OO}{\mathcal O}

\newcommand{\SCA}{\mathscr A}
\newcommand{\SCB}{\mathscr B}
\newcommand{\SCC}{\mathscr C}
\newcommand{\SCD}{\mathscr D}
\newcommand{\SCE}{\mathscr E}
\newcommand{\SCF}{\mathscr F}
\newcommand{\SCG}{\mathscr G}
\newcommand{\SCH}{\mathscr H}

% Mathfrak primes
\newcommand{\km}{\mathfrak m}
\newcommand{\kp}{\mathfrak p}
\newcommand{\kq}{\mathfrak q}

% number sets
\newcommand{\RR}[1][]{\ensuremath{\ifstrempty{#1}{\mathbb{R}}{\mathbb{R}^{#1}}}}
\newcommand{\NN}[1][]{\ensuremath{\ifstrempty{#1}{\mathbb{N}}{\mathbb{N}^{#1}}}}
\newcommand{\ZZ}[1][]{\ensuremath{\ifstrempty{#1}{\mathbb{Z}}{\mathbb{Z}^{#1}}}}
\newcommand{\QQ}[1][]{\ensuremath{\ifstrempty{#1}{\mathbb{Q}}{\mathbb{Q}^{#1}}}}
\newcommand{\CC}[1][]{\ensuremath{\ifstrempty{#1}{\mathbb{C}}{\mathbb{C}^{#1}}}}
\newcommand{\PP}[1][]{\ensuremath{\ifstrempty{#1}{\mathbb{P}}{\mathbb{P}^{#1}}}}
\newcommand{\HH}[1][]{\ensuremath{\ifstrempty{#1}{\mathbb{H}}{\mathbb{H}^{#1}}}}
\newcommand{\FF}[1][]{\ensuremath{\ifstrempty{#1}{\mathbb{F}}{\mathbb{F}^{#1}}}}
% expected value
\newcommand{\EE}{\ensuremath{\mathbb{E}}}
\newcommand{\charin}{\text{ char }}
\DeclareMathOperator{\sign}{sign}
\DeclareMathOperator{\Aut}{Aut}
\DeclareMathOperator{\Inn}{Inn}
\DeclareMathOperator{\Syl}{Syl}
\DeclareMathOperator{\Gal}{Gal}
\DeclareMathOperator{\GL}{GL} % General linear group
\DeclareMathOperator{\SL}{SL} % Special linear group

%---------------------------------------
% BlackBoard Math Fonts :-
%---------------------------------------

%Captital Letters
\newcommand{\bbA}{\mathbb{A}}	\newcommand{\bbB}{\mathbb{B}}
\newcommand{\bbC}{\mathbb{C}}	\newcommand{\bbD}{\mathbb{D}}
\newcommand{\bbE}{\mathbb{E}}	\newcommand{\bbF}{\mathbb{F}}
\newcommand{\bbG}{\mathbb{G}}	\newcommand{\bbH}{\mathbb{H}}
\newcommand{\bbI}{\mathbb{I}}	\newcommand{\bbJ}{\mathbb{J}}
\newcommand{\bbK}{\mathbb{K}}	\newcommand{\bbL}{\mathbb{L}}
\newcommand{\bbM}{\mathbb{M}}	\newcommand{\bbN}{\mathbb{N}}
\newcommand{\bbO}{\mathbb{O}}	\newcommand{\bbP}{\mathbb{P}}
\newcommand{\bbQ}{\mathbb{Q}}	\newcommand{\bbR}{\mathbb{R}}
\newcommand{\bbS}{\mathbb{S}}	\newcommand{\bbT}{\mathbb{T}}
\newcommand{\bbU}{\mathbb{U}}	\newcommand{\bbV}{\mathbb{V}}
\newcommand{\bbW}{\mathbb{W}}	\newcommand{\bbX}{\mathbb{X}}
\newcommand{\bbY}{\mathbb{Y}}	\newcommand{\bbZ}{\mathbb{Z}}

%---------------------------------------
% MathCal Fonts :-
%---------------------------------------

%Captital Letters
\newcommand{\mcA}{\mathcal{A}}	\newcommand{\mcB}{\mathcal{B}}
\newcommand{\mcC}{\mathcal{C}}	\newcommand{\mcD}{\mathcal{D}}
\newcommand{\mcE}{\mathcal{E}}	\newcommand{\mcF}{\mathcal{F}}
\newcommand{\mcG}{\mathcal{G}}	\newcommand{\mcH}{\mathcal{H}}
\newcommand{\mcI}{\mathcal{I}}	\newcommand{\mcJ}{\mathcal{J}}
\newcommand{\mcK}{\mathcal{K}}	\newcommand{\mcL}{\mathcal{L}}
\newcommand{\mcM}{\mathcal{M}}	\newcommand{\mcN}{\mathcal{N}}
\newcommand{\mcO}{\mathcal{O}}	\newcommand{\mcP}{\mathcal{P}}
\newcommand{\mcQ}{\mathcal{Q}}	\newcommand{\mcR}{\mathcal{R}}
\newcommand{\mcS}{\mathcal{S}}	\newcommand{\mcT}{\mathcal{T}}
\newcommand{\mcU}{\mathcal{U}}	\newcommand{\mcV}{\mathcal{V}}
\newcommand{\mcW}{\mathcal{W}}	\newcommand{\mcX}{\mathcal{X}}
\newcommand{\mcY}{\mathcal{Y}}	\newcommand{\mcZ}{\mathcal{Z}}


%---------------------------------------
% Bold Math Fonts :-
%---------------------------------------

%Captital Letters
\newcommand{\bmA}{\boldsymbol{A}}	\newcommand{\bmB}{\boldsymbol{B}}
\newcommand{\bmC}{\boldsymbol{C}}	\newcommand{\bmD}{\boldsymbol{D}}
\newcommand{\bmE}{\boldsymbol{E}}	\newcommand{\bmF}{\boldsymbol{F}}
\newcommand{\bmG}{\boldsymbol{G}}	\newcommand{\bmH}{\boldsymbol{H}}
\newcommand{\bmI}{\boldsymbol{I}}	\newcommand{\bmJ}{\boldsymbol{J}}
\newcommand{\bmK}{\boldsymbol{K}}	\newcommand{\bmL}{\boldsymbol{L}}
\newcommand{\bmM}{\boldsymbol{M}}	\newcommand{\bmN}{\boldsymbol{N}}
\newcommand{\bmO}{\boldsymbol{O}}	\newcommand{\bmP}{\boldsymbol{P}}
\newcommand{\bmQ}{\boldsymbol{Q}}	\newcommand{\bmR}{\boldsymbol{R}}
\newcommand{\bmS}{\boldsymbol{S}}	\newcommand{\bmT}{\boldsymbol{T}}
\newcommand{\bmU}{\boldsymbol{U}}	\newcommand{\bmV}{\boldsymbol{V}}
\newcommand{\bmW}{\boldsymbol{W}}	\newcommand{\bmX}{\boldsymbol{X}}
\newcommand{\bmY}{\boldsymbol{Y}}	\newcommand{\bmZ}{\boldsymbol{Z}}
%Small Letters
\newcommand{\bma}{\boldsymbol{a}}	\newcommand{\bmb}{\boldsymbol{b}}
\newcommand{\bmc}{\boldsymbol{c}}	\newcommand{\bmd}{\boldsymbol{d}}
\newcommand{\bme}{\boldsymbol{e}}	\newcommand{\bmf}{\boldsymbol{f}}
\newcommand{\bmg}{\boldsymbol{g}}	\newcommand{\bmh}{\boldsymbol{h}}
\newcommand{\bmi}{\boldsymbol{i}}	\newcommand{\bmj}{\boldsymbol{j}}
\newcommand{\bmk}{\boldsymbol{k}}	\newcommand{\bml}{\boldsymbol{l}}
\newcommand{\bmm}{\boldsymbol{m}}	\newcommand{\bmn}{\boldsymbol{n}}
\newcommand{\bmo}{\boldsymbol{o}}	\newcommand{\bmp}{\boldsymbol{p}}
\newcommand{\bmq}{\boldsymbol{q}}	\newcommand{\bmr}{\boldsymbol{r}}
\newcommand{\bms}{\boldsymbol{s}}	\newcommand{\bmt}{\boldsymbol{t}}
\newcommand{\bmu}{\boldsymbol{u}}	\newcommand{\bmv}{\boldsymbol{v}}
\newcommand{\bmw}{\boldsymbol{w}}	\newcommand{\bmx}{\boldsymbol{x}}
\newcommand{\bmy}{\boldsymbol{y}}	\newcommand{\bmz}{\boldsymbol{z}}

%---------------------------------------
% Scr Math Fonts :-
%---------------------------------------

\newcommand{\sA}{{\mathscr{A}}}   \newcommand{\sB}{{\mathscr{B}}}
\newcommand{\sC}{{\mathscr{C}}}   \newcommand{\sD}{{\mathscr{D}}}
\newcommand{\sE}{{\mathscr{E}}}   \newcommand{\sF}{{\mathscr{F}}}
\newcommand{\sG}{{\mathscr{G}}}   \newcommand{\sH}{{\mathscr{H}}}
\newcommand{\sI}{{\mathscr{I}}}   \newcommand{\sJ}{{\mathscr{J}}}
\newcommand{\sK}{{\mathscr{K}}}   \newcommand{\sL}{{\mathscr{L}}}
\newcommand{\sM}{{\mathscr{M}}}   \newcommand{\sN}{{\mathscr{N}}}
\newcommand{\sO}{{\mathscr{O}}}   \newcommand{\sP}{{\mathscr{P}}}
\newcommand{\sQ}{{\mathscr{Q}}}   \newcommand{\sR}{{\mathscr{R}}}
\newcommand{\sS}{{\mathscr{S}}}   \newcommand{\sT}{{\mathscr{T}}}
\newcommand{\sU}{{\mathscr{U}}}   \newcommand{\sV}{{\mathscr{V}}}
\newcommand{\sW}{{\mathscr{W}}}   \newcommand{\sX}{{\mathscr{X}}}
\newcommand{\sY}{{\mathscr{Y}}}   \newcommand{\sZ}{{\mathscr{Z}}}


%---------------------------------------
% Math Fraktur Font
%---------------------------------------

%Captital Letters
\newcommand{\mfA}{\mathfrak{A}}	\newcommand{\mfB}{\mathfrak{B}}
\newcommand{\mfC}{\mathfrak{C}}	\newcommand{\mfD}{\mathfrak{D}}
\newcommand{\mfE}{\mathfrak{E}}	\newcommand{\mfF}{\mathfrak{F}}
\newcommand{\mfG}{\mathfrak{G}}	\newcommand{\mfH}{\mathfrak{H}}
\newcommand{\mfI}{\mathfrak{I}}	\newcommand{\mfJ}{\mathfrak{J}}
\newcommand{\mfK}{\mathfrak{K}}	\newcommand{\mfL}{\mathfrak{L}}
\newcommand{\mfM}{\mathfrak{M}}	\newcommand{\mfN}{\mathfrak{N}}
\newcommand{\mfO}{\mathfrak{O}}	\newcommand{\mfP}{\mathfrak{P}}
\newcommand{\mfQ}{\mathfrak{Q}}	\newcommand{\mfR}{\mathfrak{R}}
\newcommand{\mfS}{\mathfrak{S}}	\newcommand{\mfT}{\mathfrak{T}}
\newcommand{\mfU}{\mathfrak{U}}	\newcommand{\mfV}{\mathfrak{V}}
\newcommand{\mfW}{\mathfrak{W}}	\newcommand{\mfX}{\mathfrak{X}}
\newcommand{\mfY}{\mathfrak{Y}}	\newcommand{\mfZ}{\mathfrak{Z}}
%Small Letters
\newcommand{\mfa}{\mathfrak{a}}	\newcommand{\mfb}{\mathfrak{b}}
\newcommand{\mfc}{\mathfrak{c}}	\newcommand{\mfd}{\mathfrak{d}}
\newcommand{\mfe}{\mathfrak{e}}	\newcommand{\mff}{\mathfrak{f}}
\newcommand{\mfg}{\mathfrak{g}}	\newcommand{\mfh}{\mathfrak{h}}
\newcommand{\mfi}{\mathfrak{i}}	\newcommand{\mfj}{\mathfrak{j}}
\newcommand{\mfk}{\mathfrak{k}}	\newcommand{\mfl}{\mathfrak{l}}
\newcommand{\mfm}{\mathfrak{m}}	\newcommand{\mfn}{\mathfrak{n}}
\newcommand{\mfo}{\mathfrak{o}}	\newcommand{\mfp}{\mathfrak{p}}
\newcommand{\mfq}{\mathfrak{q}}	\newcommand{\mfr}{\mathfrak{r}}
\newcommand{\mfs}{\mathfrak{s}}	\newcommand{\mft}{\mathfrak{t}}
\newcommand{\mfu}{\mathfrak{u}}	\newcommand{\mfv}{\mathfrak{v}}
\newcommand{\mfw}{\mathfrak{w}}	\newcommand{\mfx}{\mathfrak{x}}
\newcommand{\mfy}{\mathfrak{y}}	\newcommand{\mfz}{\mathfrak{z}}


\title{\Huge{Some Class}\\Random Examples}
\author{\huge{Your Name}}
\date{}

\begin{document}

\maketitle
\newpage% or \cleardoublepage
% \pdfbookmark[<level>]{<title>}{<dest>}
\pdfbookmark[section]{\contentsname}{toc}
\tableofcontents
\pagebreak

\chapter{}

\chapter{}

\chapter{}

Para comenzar, veamos el hamiltoniano que nos dieron:
\begin{align*}
  H &= \frac{1}{2M}\vec{p}\cdot\vec{p} + \frac{1}{2}M\omega^2 \vec{r}\cdot\vec{r}\\
  H &= \frac{1}{2M}\left( \hat{p}_x^2 + \hat{p}_y^2 + \hat{p}_z^2 \right) + \frac{1}{2}M\omega^2 \left(x^2 + y^2 + z^2\right)\\
  \hat{p}_i &= - i\hbar \frac{\partial}{\partial x_i}\\
  H &= \frac{1}{2M}\left(  - \hbar \frac{\partial^2}{\partial x^2} - \hbar \frac{\partial^2}{\partial y^2} - \hbar \frac{\partial^2}{\partial z^2}\right) + \frac{1}{2}M\omega^2 \left(x^2 + y^2 + z^2\right)\\
  H &= \frac{-\hbar}{2M}\left(  \frac{\partial^2}{\partial x^2} + \frac{\partial^2}{\partial y^2} + \frac{\partial^2}{\partial z^2}\right) + \frac{1}{2}M\omega^2 \left(x^2 + y^2 + z^2\right)\\
\end{align*}

Ahora bien, viendo esto en la ecuación de Schrodinger veriamos:
\begin{align*}
  H\psi(\vec{r}) &= E\psi(\vec{r})\\
  \left[\frac{-\hbar}{2M}\left(  \frac{\partial^2}{\partial x^2} + \frac{\partial^2}{\partial y^2} + \frac{\partial^2}{\partial z^2}\right) + \frac{1}{2}M\omega^2 \left(x^2 + y^2 + z^2\right)\right]\psi &= E\psi
\end{align*}

Ahora asumamos \[\psi = X(x)Y(y)Z(z)\]
\begin{align*}
  \left[\frac{-\hbar}{2M}\left(  \frac{\partial^2}{\partial x^2} + \frac{\partial^2}{\partial y^2} + \frac{\partial^2}{\partial z^2}\right) + \frac{1}{2}M\omega^2 \left(x^2 + y^2 + z^2\right)\right]X(x)Y(y)Z(z) &= EX(x)Y(y)Z(z)\\
  \left[\frac{-\hbar}{2M}\left(  YZ\frac{\partial^2 X}{\partial x^2} + XZ\frac{\partial^2 Y}{\partial y^2} + XY\frac{\partial^2 Z}{\partial z^2}\right) + \frac{1}{2}M\omega^2 \left(x^2 + y^2 + z^2\right)X(x)Y(y)Z(z)\right] &= EX(x)Y(y)Z(z)\\
  \left[\frac{-\hbar}{2M}\left(  YZX'' + XZY'' +  XYZ''\right) + \frac{1}{2}M\omega^2 \left(x^2 + y^2 + z^2\right)XYZ\right] &= EXYZ\\
  \frac{\left[\frac{-\hbar}{2M}\left(  YZX'' + XZY'' +  XYZ''\right) + \frac{1}{2}M\omega^2 \left(x^2 + y^2 + z^2\right)XYZ\right]}{XYZ} &= \frac{EXYZ}{XYZ}\\
  \left[\frac{-\hbar}{2M}\left(  \frac{X''}{X} + \frac{Y''}{Y} +  \frac{Z''}{Z}\right) + \frac{1}{2}M\omega^2 \left(x^2 + y^2 + z^2\right)\right] &= E\\
  E &= E_x + E_y + E_z\\
  \frac{-\hbar}{2M}\frac{X''}{X} + \frac{1}{2}M\omega^2x^2 &= E_x\\
  \frac{-\hbar}{2M}\frac{Y''}{Y} + \frac{1}{2}M\omega^2y^2 &= E_y\\
  \frac{-\hbar}{2M}\frac{Z''}{Z} + \frac{1}{2}M\omega^2x^2 &= E_z\\
\end{align*}

En este caso, puede transformar de esto a las ecuaciones de estado para potenciales armonicos de la siguiente manera:
\begin{align*}
  \frac{-\hbar}{2M}\frac{X''}{X} + \frac{1}{2}M\omega^2x^2 &= E_x\\
  \frac{-\hbar}{2M}\frac{Y''}{Y} + \frac{1}{2}M\omega^2y^2 &= E_y\\
  \frac{-\hbar}{2M}\frac{Z''}{Z} + \frac{1}{2}M\omega^2x^2 &= E_z\\
  \frac{-\hbar}{2M}\frac{X''}{X}X + \frac{1}{2}M\omega^2x^2X &= E_xX\\
  \frac{-\hbar}{2M}\frac{Y''}{Y}Y + \frac{1}{2}M\omega^2y^2Y &= E_yY\\
  \frac{-\hbar}{2M}\frac{Z''}{Z}Z + \frac{1}{2}M\omega^2x^2Z &= E_zZ\\
  \frac{-\hbar}{2M}X'' + \frac{1}{2}M\omega^2x^2X &= E_xX\\
  \frac{-\hbar}{2M}Y'' + \frac{1}{2}M\omega^2y^2Y &= E_yY\\
  \frac{-\hbar}{2M}Z'' + \frac{1}{2}M\omega^2x^2Z &= E_zZ\\
  \left[\frac{-\hbar}{2M}\frac{d^2}{dx^2} + \frac{1}{2}M\omega^2x^2\right]X &= E_xX\\
  \left[\frac{-\hbar}{2M}\frac{d^2}{dy^2} + \frac{1}{2}M\omega^2y^2\right]Y &= E_yY\\
  \left[\frac{-\hbar}{2M}\frac{d^2}{dz^2} + \frac{1}{2}M\omega^2x^2\right]Z &= E_zZ\\
\end{align*}

Ahora bien, con esto ya mostrado podemos solucionar cada uno de estos casos como si se tratace de un armonico simple. Estos armonicos como se mostro en clase (y en las notas) tienen una solución relacionada con los polinomios de Hermit. Dado que nos piden exactamente la energia entonces tambien podemos saber que: \[E_n = \left(n + \frac{1}{2}\right)\hbar\omega,\ n=0,1,2,3,...\]. Ahora dado que tenemos $3$ dimensiones pasamos:
\begin{align*}
  E_x &= \left(n_x + \frac{1}{2}\right)\hbar\omega\\
  E_y &= \left(n_y + \frac{1}{2}\right)\hbar\omega\\
  E_z &= \left(n_z + \frac{1}{2}\right)\hbar\omega\\
  E &= E_x + E_y + E_z\\
  E &= \left(n_x + \frac{1}{2}\right)\hbar\omega + \left(n_y + \frac{1}{2}\right)\hbar\omega + \left(n_z + \frac{1}{2}\right)\hbar\omega\\
  E &= \left(n_x + \frac{1}{2} + n_y + \frac{1}{2} + n_z + \frac{1}{2} \right)\hbar\omega\\
  E &= \left(n_x + n_y + n_z + \frac{3}{2} \right)\hbar\omega
\end{align*}

\chapter{}

\chapter{}

\section*{Ayuda:}

En este caso nos dicen que lo mejor es expandir la función de onda \[A\sin\left(\frac{\pi x}{L}\right)\] en las funciones propias. Para eso vamos a comenzar por la formula de reducción de potencial trigonometrico con lo cual:
\begin{align*}
  \sin^2(x) &= \frac{1 - \cos(2x)}{2}\\
  \sin^4(x) &= \left(\sin^2(x)\right)^2\\
  &= \left(\frac{1 - \cos(2x)}{2}\right)^2\\
  &= \frac{\left( 1 - \cos(2x)\right)^2}{4}\\
  &= \frac{1 - 2\cos(2x) + \cos^2(2x)}{4}\\
  \sin^5(x) &= \sin(x)\sin^4(x) \\
  &= \sin(x)\left(\frac{1 - 2\cos(2x) + \cos^2(2x)}{4}\right)\\
  &= \frac{\sin(x) - 2\sin(x)\cos(2x) + \sin(x)\cos^2(2x)}{4}\\
\end{align*}

Ahora con esto podemos usar las siguientes identidades trigonometricas
\begin{align*}
\sin(x)\cos(2x) &= \frac{1}{2}\left(- \sin(x) + \sin(3x)\right)\\
  \cos^2(2x) &= \frac{1 + \cos(4x)}{2}\\
  \sin(x)\cos^2(2x) &= \frac{\sin(x) + \sin(x)\cos(4x)}{2}\\
  \sin(x)\cos^2(2x) &= \frac{\sin(x) + \frac{\sin(5x) - \sin(3x)}{2}}{2}\\
  &= \frac{1}{2}\sin(x) + \frac{1}{4}\sin(5x) - \frac{1}{4}\sin(3x)\\
\end{align*}

Que reemplazando y calculando nos da:
\begin{align*}
  \sin^5(x) &= \frac{\sin(x) - 2\sin(x)\cos(2x) + \sin(x)\cos^2(2x)}{4}\\
  \sin^5(x) &= \frac{\sin(x) - 2\left(\frac{1}{2}\left(-\sin(x) + \sin(3x)\right) \right) + \frac{1}{2}\sin(x) + \frac{1}{4}\sin(5x) - \frac{1}{4}\sin(3x)}{4}\\
  \sin^5(x) &= \frac{\sin(x) + \sin(x) - \sin(3x) + \frac{1}{2}\sin(x) + \frac{1}{4}\sin(5x) - \frac{1}{4}\sin(3x)}{4}\\
  \sin^5(x) &= \frac{\sin(x)}{4} + \frac{\sin(x)}{4} - \frac{\sin(3x)}{4} + \frac{\frac{1}{2}\sin(x)}{4} + \frac{\frac{1}{4}\sin(5x)}{4} - \frac{\frac{1}{4}\sin(3x)}{4}\\
  \sin^5(x) &= \frac{\sin(x)}{4} + \frac{\sin(x)}{4} - \frac{\sin(3x)}{4} + \frac{\sin(x)}{8} + \frac{\sin(5x)}{16} - \frac{\sin(3x)}{16}\\
  \sin^5(x) &= \frac{4\sin(x)}{16} + \frac{4\sin(x)}{16} - \frac{4\sin(3x)}{16} + \frac{2\sin(x)}{16} + \frac{\sin(5x)}{16} - \frac{\sin(3x)}{16}\\
  \sin^5(x) &= \left( \frac{4}{16} + \frac{4}{16} + \frac{2}{16}\right)\sin(x) - \left( \frac{1}{16} + \frac{4}{16}\right)\sin(3x) + \frac{1}{16} \sin(5x)\\
  \sin^5(x) &= \frac{10}{16}\sin(x) - \frac{5}{16}\sin(3x) + \frac{1}{16} \sin(5x)
\end{align*}

Ahora bien, recordemos que esta funcion en verdad es \[A \sin^5\left(\frac{\pi x}{L}\right)\] por lo tanto reemplazando:
\begin{align*}
  \sin^5(x) &= \frac{10}{16}\sin(x) - \frac{5}{16}\sin(3x) + \frac{1}{16} \sin(5x)\\
  A \sin^5\left(\frac{\pi x}{L}\right) &= A\left[\frac{10}{16}\sin\left(\frac{\pi x}{L}\right) - \frac{5}{16}\sin\left(3\frac{\pi x}{L}\right) + \frac{1}{16} \sin\left(5\frac{\pi x}{L}\right)\right]\\
  \phi_n &= \sqrt{\frac{2}{L}} \sin\left(\frac{n \pi x}{L}\right)\\
  C_1 &= \frac{10}{16}\sqrt{\frac{L}{2}} \\
  C_2 &= -\frac{5}{16}\sqrt{\frac{L}{2}} \\
  C_3 &= \frac{1}{16}\sqrt{\frac{L}{2}} \\
  A \sin^5\left(\frac{\pi x}{L}\right) &= AC_1\phi_1 + AC_2\phi_3 + AC_3\phi_5
\end{align*}

\nt{Note que:
\begin{align*}
  \psi^2 = A^2\left(C_1^2\phi_1^2 + C_2^2\phi_3^2+ C_3^2\phi_5^2\right)
\end{align*}

Dado que $\phi_n\phi_m = 0$ para $n \neq m$ dado que son ortonormales entre si.
}

\section{}

Para este caso tenemos que:
\begin{align*}
  \int_{0}^{L} \left|\psi\left(x\right)\right|^2 dx &= 1\\
  \int_{0}^{L} \left|\psi\left(x\right)\right|^2 dx &= \int_0^L A^2C_1^2\phi_1^2 dx + \int_0^L A^2C_2^2\phi_3^2 dx + \int_0^L A^2C_3^2 \phi_5 dx \\
  \int_{0}^{L} \left|\psi\left(x\right)\right|^2 dx &= A^2C_1^2\int_0^L \phi_1^2 dx + A^2C_2^2\int_0^L \phi_3^2 dx + A^2C_3^2\int_0^L  \phi_5 dx \\
  \int_{0}^{L} \left|\psi\left(x\right)\right|^2 dx &= A^2C_1^2 + A^2C_2^2 + A^2C_3^2 \\
  C_1 &= \frac{10}{16}\sqrt{\frac{L}{2}} \\
  C_2 &= -\frac{5}{16}\sqrt{\frac{L}{2}} \\
  C_3 &= \frac{1}{16}\sqrt{\frac{L}{2}} \\
  C_1^2 &= \frac{100}{256}\frac{L}{2} \\
   &= \frac{100L}{512}\\
  C_2^2 &= \frac{25}{256}{\frac{L}{2}} \\
   &= \frac{25 L}{512}\\
  C_3^2 &= \frac{1}{256}{\frac{L}{2}} \\
   &= \frac{1 L}{512}\\
  \int_{0}^{L} \left|\psi\left(x\right)\right|^2 dx &= A^2\left(C_1^2 + C_2^2 + C_3^2\right) \\
   &= A^2\left( \frac{100L}{512}+ \frac{25 L}{512} + \frac{1 L}{512}\right) \\
   &= A^2\left( \frac{126L}{512}\right) \\
   &= A^2\left( \frac{63L}{256}\right) \\
   A^2\left( \frac{63L}{256}\right) &= 1\\
   A^2 &= \left( \frac{256}{63L}\right)\\
   A &= \left( \frac{16}{\sqrt{63L}}\right)\\
\end{align*}

Que es la constante de normalización.

\section{}
\subsection{}

\nt{
  Note que:
  \begin{align*}
    \phi_n (x) &= \sqrt{\frac{2}{L}}\sin\left(\frac{n \pi x}{L}\right)\\
    \phi_n (L - x) &= \sqrt{\frac{2}{L}}\sin\left(\frac{n \pi (L - x)}{L}\right)\\
    \phi_n (L - x) &= \sqrt{\frac{2}{L}}\sin\left(\frac{n \pi x}{L} - n\pi\right)\\
    \phi_n (L - x) &= \sqrt{\frac{2}{L}}\sin\left(\frac{n \pi x}{L}\right)\\
    \phi_n (L - x) &= \phi_n(x)
  \end{align*}

  Lo que nos permite saber que es simetrico y por lo tanto queda 
  \[\int_0^L x \phi_n dx = \frac{L}{2} \int_0^2 \phi_n dx\]
}

En este caso tenemos:
\begin{align*}
  \left<x\right> &= \int_0^L x\left|\psi(x)\right|^2 dx\\
  &= A^2\left(C_1^2 + C_2^2 + C_3^3\right) \left(\int_0^L x\phi_1^2 dx + \int_0^L x\phi_3^2 dx+ \int_0^L x\phi_1^2 dx\right)\\
  x &= L - x\\
  &= A^2\left(C_1^2 + C_2^2 + C_3^3\right) \left(\int_0^L (L - x)\phi_1^2(L - x) dx + \int_0^L (L - x)\phi_3^2(L - x) dx+ \int_0^L (L - x)\phi_1^2(L - x) dx\right)\\
  &= A^2\left(C_1^2 + C_2^2 + C_3^3\right) \left(\frac{L}{2}\int_0^L \phi_1^2(x) dx + \frac{L}{2}\int_0^L \phi_3^2(x) dx+ \frac{L}{2}\int_0^L \phi_1^2(x) dx\right)\\
  &= A^2\left(C_1^2 + C_2^2 + C_3^3\right) \frac{L}{2}\left(\int_0^L \phi_1^2(x) dx + \int_0^L \phi_3^2(x) dx+ \int_0^L \phi_1^2(x) dx\right)\\
  &= A^2\left(C_1^2 + C_2^2 + C_3^3\right) \frac{L}{2}\\
  1 &= A^2\left(C_1^2 + C_2^2 + C_3^3\right) \\
  &= \frac{L}{2}
\end{align*}

\subsection{}

\begin{align*}
  \left< p \right> &= \int_0^L \psi^* (-i \hbar) \frac{\partial }{\partial x} \psi dx \\
  &= -i \hbar\int_0^L \psi \frac{\partial }{\partial x} \psi dx \\
  &= -i \hbar\int_0^L \psi \frac{d }{dx} \psi dx \\
  u &= \psi\\
  du &= \frac{d}{dx} \psi \\
  dv &= \frac{d}{dx} \psi\\
  v &= \psi\\
  \int udv &= uv - \int vdu\\
  \int_0^L \psi \frac{d }{d x} \psi dx &= \left[\psi^2\right]_0^L - \int_0^L \psi \frac{d}{dx}\psi dx\\
  \int_0^L \psi \frac{d }{d x} \psi dx + \int_0^L \psi \frac{d}{dx}\psi dx &= \left[\psi^2\right]_0^L\\
  2 \int_0^L \psi \frac{d }{d x} \psi dx &= \left[\psi^2(L) - \psi^2(0)\right]\\
  \int_0^L \psi \frac{d }{d x} \psi dx &= 0\\
  \left< p \right> &= -i \hbar\int_0^L \psi \frac{d }{dx} \psi dx \\
  \left< p \right> &= -i \hbar 0\\
  \left< p \right> &= 0
\end{align*}

\subsection{}

En este caso sabemos que \[\sigma_x = \sqrt{\left<x^2\right> - \left<x\right>^2}\] por lo tanto nos falta $\left< x^2 \right>$ que requerimos.

\begin{align*}
  \left< x^2 \right> &= \int_0^L x^2 \psi^2 dx\\
  &= \int_0^L x^2 A^2\left(C_1^2\phi_1^2 + C_2^2\phi_3^2+ C_3^2\phi_5^2\right)dx\\
  &= A^2 \int_0^L x^2 \left(C_1^2\phi_1^2 + C_2^2\phi_3^2+ C_3^2\phi_5^2\right)dx\\
  &= A^2 \left(\int_0^L x^2 C_1^2\phi_1^2 dx+ \int_0^L C_2^2\phi_3^2 dx+ \int_0^L C_3^2\phi_5^2 dx\right)\\
  %SEPARATOR
  \int_0^L x^2 C_1^2\phi_1^2 dx &= \int\limits_{0}^{L} \frac{25 x^{2} \sin^{2}{\left(\frac{\pi x}{L} \right)}}{64}\, dx\\
  &=\frac{25 L^{4} \left(-3 + 2 \pi^{2}\right)}{3072 \pi^{2}}\\
  \int_0^L x^2 C_2^2\phi_3^2 dx &= \int\limits_{0}^{L} \frac{25 x^{2} \sin^{2}{\left(\frac{3 \pi x}{L} \right)}}{256}\, dx\\
  &= \frac{25 L^{4} \left(-1 + 6 \pi^{2}\right)}{36864 \pi^{2}}\\
  \int_0^L x^2 C_3^2\phi_5^2 dx &= \int\limits_{0}^{L} \frac{x^{2} \sin^{2}{\left(\frac{5 \pi x}{L} \right)}}{256}\, dx\\
  &=- \frac{L^{4}}{102400 \pi^{2}} + \frac{L^{4}}{6144}\\
  \left(\int_0^L x^2 C_1^2\phi_1^2 dx+ \int_0^L C_2^2\phi_3^2 dx+ \int_0^L C_3^2\phi_5^2 dx\right) &= \frac{L^{4} \left(-11567 + 9450 \pi^{2}\right)}{460800 \pi^{2}}\\
  \left< x^2 \right> &= \int_0^L x^2 \psi^2 dx\\
  \left< x^2 \right> &= A^2 \left(\int_0^L x^2 C_1^2\phi_1^2 dx+ \int_0^L C_2^2\phi_3^2 dx+ \int_0^L C_3^2\phi_5^2 dx\right)\\
  \left< x^2 \right> &=\frac{L^{3} \left(-11567 + 9450 \pi^{2}\right)}{113400 \pi^{2}}\\
  &= 0.072998393904842 L^{3}
\end{align*}

En este punto se puede ver que muchas de las integrales son complejas. Por lo tanto, consideramos que lo mas prudente era realizarlo de manera computacional. Esto se hizo por medio de sympy con el siguiente codigo:

\lstinputlisting[language=Python]{code/punto_5.py}

Con esto en consideraciónlo unico que falta es regresar a la forma original:
\begin{align*}
  \sigma_x &= \sqrt{\left<x^2\right> - \left<x\right>^2}\\
  &= \sqrt{0.072998393904842 L^{3} - \frac{L^2}{4}}\\
  &= \frac{\sqrt{14} \sqrt{- L^{2} \left(L \left(11567 - 9450 \pi^{2}\right) + 28350 \pi^{2}\right)}}{1260 \pi}
\end{align*}

\subsection{}

En este caso, nos vamos a aprovechar de:
\begin{align*}
  \left< p^2 \right> &= \int\limits_0^L \psi(x)\left( -\hbar^2 \frac{\partial^2}{\partial x^2}\right)\psi(x) dx = 2m\left<E\right>
\end{align*}

Dado que podemos plantear este $\psi$ es expresado en terminos de funciones propias sabemos que: \[E_n = \frac{\hbar^2\pi^2n^2}{2mL^2}\] y dado que todos estos son diferentes entre si entonces podemos expresarlo como:

\begin{align*}
  \left<p^2\right> &= 2m \sum_{n = 1, 3, 5} C_n^2 E_n\\
  &= \sum_{n = 1, 3, 5} A^2 C_n^2 \frac{\hbar^2\pi^2n^2}{L^2}\\
  C_1^2 &= \frac{100L}{512}\\
  C_2^2 &= \frac{25 L}{512}\\
  C_3^2 &= \frac{1 L}{512}\\
  A^2 C_1^2 &= \frac{256}{63 L}\frac{100L}{512}\\
  A^2 C_2^2 &= \frac{256}{63 L}\frac{25 L}{512}\\
  A^2 C_3^2 &= \frac{256}{63 L}\frac{1  L}{512}\\
  A^2 C_1^2 &= \frac{1}{63}\frac{100}{2}\\
  A^2 C_2^2 &= \frac{1}{63}\frac{25 }{2}\\
  A^2 C_3^2 &= \frac{1}{63}\frac{1  }{2}\\
  A^2 C_1^2 &= \frac{100}{126}\\
  A^2 C_2^2 &= \frac{25 }{126}\\
  A^2 C_3^2 &= \frac{1  }{126}\\
  &= \frac{\hbar^2\pi^2}{L^2}\sum_{n = 1, 3, 5} A^2 C_n^2 n^2 \\
  &= \frac{\hbar^2\pi^2}{L^2}\left[\frac{100}{126} + \frac{25 3^2}{126} + \frac{5^2}{126}\right] \\
  &= \frac{\hbar^2\pi^2}{L^2}\left[\frac{25}{9}\right] \\
  \sigma_x &= \sqrt{\left<p^2\right> - \left<p\right>^2}\\
  \sigma_x &= \sqrt{\left<p^2\right>}\\
  \sigma_x &= \sqrt{\frac{\hbar^2\pi^2}{L^2}\left[\frac{25}{9}\right] }\\
  \sigma_x &= \frac{\hbar\pi}{L}\left[\frac{5}{3}\right]\\
\end{align*}

\section{}

En este caso vamos a usar \[
  A \sin^5\left(\frac{\pi x}{L}\right) = AC_1\phi_1 + AC_2\phi_3 + AC_3\phi_5
\] puesto que sabemos que para $\phi_n$ se cumple que,
\begin{align*}
  \phi_n(x, t) &= \phi_n(x)e^{- \frac{i}{\hbar}E_n t}\\
  E_n &= \frac{\hbar^2 \pi^2 n^2}{2m L^2}
\end{align*}

Por lo tanto podemos ponerlo en estos mismos terminos como:
\begin{align*}
  \psi(x) &= A \sin^5\left(\frac{\pi x}{L}\right)\\
  &= AC_1\phi_1 + AC_2\phi_3 + AC_3\phi_5\\
  \psi\left(x, t\right) &= AC_1\phi_1e^{- \frac{i}{\hbar}E_1 t} + AC_2\phi_3e^{- \frac{i}{\hbar}E_3 t} + AC_3\phi_5e^{- \frac{i}{\hbar}E_5 t}
\end{align*}

Con lo cual ya encontramos lo que nos pidieron.

\section{}
\subsection{}

En este caso tenemos:
\begin{align*}
  \left<x\right> &= \int_0^L x\left|\psi(x)\right|^2 dx\\
  &= A^2\left(C_1^2 e^{- \frac{i}{\hbar}(E_1 - E_1) t}+ C_2^2e^{- \frac{i}{\hbar}(E_3 - E_3) t} + C_3^3e^{- \frac{i}{\hbar}(E_5 - E_5) t}\right) \left(\int_0^L x\phi_1^2 dx + \int_0^L x\phi_3^2 dx+ \int_0^L x\phi_1^2 dx\right)\\
  &= A^2\left(C_1^2 e^{0}+ C_2^2e^{0} + C_3^3e^{0}\right) \left(\int_0^L x\phi_1^2 dx + \int_0^L x\phi_3^2 dx+ \int_0^L x\phi_1^2 dx\right)\\
  &= A^2\left(C_1^2 + C_2^2 + C_3^3\right) \left(\int_0^L x\phi_1^2 dx + \int_0^L x\phi_3^2 dx+ \int_0^L x\phi_1^2 dx\right)\\
  \frac{L}{2} &= \left(\int_0^L x\phi_1^2 dx + \int_0^L x\phi_3^2 dx+ \int_0^L x\phi_1^2 dx\right) \text{ Esto sale del punto anterior}\\
  1 &= A^2\left(C_1^2 + C_2^2 + C_3^3\right) \\
  &= \frac{L}{2}
\end{align*}

\subsection{}

\begin{align*}
  \left< p \right> &= \int_0^L \psi^* (-i \hbar) \frac{\partial }{\partial x} \psi dx \\
  &= -i \hbar\int_0^L \psi \frac{\partial }{\partial x} \psi dx \\
  &= -i \hbar\int_0^L \psi \frac{d }{dx} \psi dx \\
  u &= \psi\\
  du &= \frac{\partial}{\partial x} \psi dx\\
  dv &= \frac{\partial}{\partial x} \psi dx\\
  v &= \psi\\
  \int udv &= uv - \int vdu\\
  \int_0^L \psi \frac{\partial }{\partial x} \psi dx &= \left[\psi^2\right]_0^L - \int_0^L \psi \frac{\partial}{\partial x}\psi dx\\
  \int_0^L \psi \frac{\partial }{\partial x} \psi dx + \int_0^L \psi \frac{\partial}{\partial x}\psi dx &= \left[\psi^2\right]_0^L\\
  2 \int_0^L \psi \frac{\partial }{\partial x} \psi dx &= \left[\psi^2(L) - \psi^2(0)\right]\\
  \int_0^L \psi \frac{\partial }{\partial x} \psi dx &= 0\\
  \left< p \right> &= -i \hbar\int_0^L \psi \frac{\partial }{\partial x} \psi dx \\
  \left< p \right> &= -i \hbar 0\\
  \left< p \right> &= 0
\end{align*}

\subsection{}

En este caso sabemos que \[\sigma_x = \sqrt{\left<x^2\right> - \left<x\right>^2}\] por lo tanto nos falta $\left< x^2 \right>$ que requerimos.

\begin{align*}
  \left< x^2 \right> &= \int_0^L x^2 \psi^2 dx\\
  &= \int_0^L x^2 A^2\left(C_1^2\phi_1^2 + C_2^2\phi_3^2+ C_3^2\phi_5^2\right)dx\\
  &= A^2 \int_0^L x^2 \left(C_1^2\phi_1^2 + C_2^2\phi_3^2+ C_3^2\phi_5^2\right)dx\\
  &= A^2 \left(\int_0^L x^2 C_1^2\phi_1^2 dx+ \int_0^L C_2^2\phi_3^2 dx+ \int_0^L C_3^2\phi_5^2 dx\right)\\
  %SEPARATOR
  \int_0^L x^2 C_1^2\phi_1^2 dx &= \int\limits_{0}^{L} \frac{25 x^{2} \sin^{2}{\left(\frac{\pi x}{L} \right)}}{64}\, dx\\
  &=\frac{25 L^{4} \left(-3 + 2 \pi^{2}\right)}{3072 \pi^{2}}\\
  \int_0^L x^2 C_2^2\phi_3^2 dx &= \int\limits_{0}^{L} \frac{25 x^{2} \sin^{2}{\left(\frac{3 \pi x}{L} \right)}}{256}\, dx\\
  &= \frac{25 L^{4} \left(-1 + 6 \pi^{2}\right)}{36864 \pi^{2}}\\
  \int_0^L x^2 C_3^2\phi_5^2 dx &= \int\limits_{0}^{L} \frac{x^{2} \sin^{2}{\left(\frac{5 \pi x}{L} \right)}}{256}\, dx\\
  &=- \frac{L^{4}}{102400 \pi^{2}} + \frac{L^{4}}{6144}\\
  \left(\int_0^L x^2 C_1^2\phi_1^2 dx+ \int_0^L C_2^2\phi_3^2 dx+ \int_0^L C_3^2\phi_5^2 dx\right) &= \frac{L^{4} \left(-11567 + 9450 \pi^{2}\right)}{460800 \pi^{2}}\\
  \left< x^2 \right> &= \int_0^L x^2 \psi^2 dx\\
  \left< x^2 \right> &= A^2 \left(\int_0^L x^2 C_1^2\phi_1^2 dx+ \int_0^L C_2^2\phi_3^2 dx+ \int_0^L C_3^2\phi_5^2 dx\right)\\
  \left< x^2 \right> &=\frac{L^{3} \left(-11567 + 9450 \pi^{2}\right)}{113400 \pi^{2}}\\
  &= 0.072998393904842 L^{3}
\end{align*}

En este punto se puede ver que muchas de las integrales son complejas. Por lo tanto, consideramos que lo mas prudente era realizarlo de manera computacional. Esto se hizo por medio de sympy con el siguiente codigo:

\lstinputlisting[language=Python]{code/punto_5.py}

Con esto en consideraciónlo unico que falta es regresar a la forma original:
\begin{align*}
  \sigma_x &= \sqrt{\left<x^2\right> - \left<x\right>^2}\\
  &= \sqrt{0.072998393904842 L^{3} - \frac{L^2}{4}}\\
  &= \frac{\sqrt{14} \sqrt{- L^{2} \left(L \left(11567 - 9450 \pi^{2}\right) + 28350 \pi^{2}\right)}}{1260 \pi}
\end{align*}

\subsection{}

En este caso, nos vamos a aprovechar de:
\begin{align*}
  \left< p^2 \right> &= \int\limits_0^L \psi(x)\left( -\hbar^2 \frac{\partial^2}{\partial x^2}\right)\psi(x) dx = 2m\left<E\right>
\end{align*}

Dado que podemos plantear este $\psi$ es expresado en terminos de funciones propias sabemos que: \[E_n = \frac{\hbar^2\pi^2n^2}{2mL^2}\] y dado que todos estos son diferentes entre si entonces podemos expresarlo como:

\begin{align*}
  \left<p^2\right> &= 2m \sum_{n = 1, 3, 5} C_n^2 E_n\\
  &= \sum_{n = 1, 3, 5} A^2 C_n^2 \frac{\hbar^2\pi^2n^2}{L^2}\\
  C_1^2 &= \frac{100L}{512}\\
  C_2^2 &= \frac{25 L}{512}\\
  C_3^2 &= \frac{1 L}{512}\\
  A^2 C_1^2 &= \frac{256}{63 L}\frac{100L}{512}\\
  A^2 C_2^2 &= \frac{256}{63 L}\frac{25 L}{512}\\
  A^2 C_3^2 &= \frac{256}{63 L}\frac{1  L}{512}\\
  A^2 C_1^2 &= \frac{1}{63}\frac{100}{2}\\
  A^2 C_2^2 &= \frac{1}{63}\frac{25 }{2}\\
  A^2 C_3^2 &= \frac{1}{63}\frac{1  }{2}\\
  A^2 C_1^2 &= \frac{100}{126}\\
  A^2 C_2^2 &= \frac{25 }{126}\\
  A^2 C_3^2 &= \frac{1  }{126}\\
  &= \frac{\hbar^2\pi^2}{L^2}\sum_{n = 1, 3, 5} A^2 C_n^2 n^2 \\
  &= \frac{\hbar^2\pi^2}{L^2}\left[\frac{100}{126} + \frac{25 3^2}{126} + \frac{5^2}{126}\right] \\
  &= \frac{\hbar^2\pi^2}{L^2}\left[\frac{25}{9}\right] \\
  \sigma_x &= \sqrt{\left<p^2\right> - \left<p\right>^2}\\
  \sigma_x &= \sqrt{\left<p^2\right>}\\
  \sigma_x &= \sqrt{\frac{\hbar^2\pi^2}{L^2}\left[\frac{25}{9}\right] }\\
  \sigma_x &= \frac{\hbar\pi}{L}\left[\frac{5}{3}\right]\\
\end{align*}

\end{document}

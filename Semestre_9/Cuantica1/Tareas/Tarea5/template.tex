\documentclass{report}

\documentclass[12pt]{article}
\usepackage{array}
\usepackage{color}
\usepackage{amsthm}
\usepackage{eufrak}
\usepackage{lipsum}
\usepackage{pifont}
\usepackage{yfonts}
\usepackage{amsmath}
\usepackage{amssymb}
\usepackage{ccfonts}
\usepackage{comment} \usepackage{amsfonts}
\usepackage{fancyhdr}
\usepackage{graphicx}
\usepackage{listings}
\usepackage{mathrsfs}
\usepackage{setspace}
\usepackage{textcomp}
\usepackage{blindtext}
\usepackage{enumerate}
\usepackage{microtype}
\usepackage{xfakebold}
\usepackage{kantlipsum}
%\usepackage{draftwatermark}
\usepackage[spanish]{babel}
\usepackage[margin=1.5cm, top=2cm, bottom=2cm]{geometry}
\usepackage[framemethod=tikz]{mdframed}
\usepackage[colorlinks=true,citecolor=blue,linkcolor=red,urlcolor=magenta]{hyperref}

%//////////////////////////////////////////////////////
% Watermark configuration
%//////////////////////////////////////////////////////
%\SetWatermarkScale{4}
%\SetWatermarkColor{black}
%\SetWatermarkLightness{0.95}
%\SetWatermarkText{\texttt{Watermark}}

%//////////////////////////////////////////////////////
% Frame configuration
%//////////////////////////////////////////////////////
\newmdenv[tikzsetting={draw=gray,fill=white,fill opacity=0},backgroundcolor=none]{Frame}

%//////////////////////////////////////////////////////
% Font style configuration
%//////////////////////////////////////////////////////
\renewcommand{\familydefault}{\ttdefault}
\renewcommand{\rmdefault}{tt}

%//////////////////////////////////////////////////////
% Bold configuration
%//////////////////////////////////////////////////////
\newcommand{\fbseries}{\unskip\setBold\aftergroup\unsetBold\aftergroup\ignorespaces}
\makeatletter
\newcommand{\setBoldness}[1]{\def\fake@bold{#1}}
\makeatother

%//////////////////////////////////////////////////////
% Default font configuration
%//////////////////////////////////////////////////////
\DeclareFontFamily{\encodingdefault}{\ttdefault}{%
  \hyphenchar\font=\defaulthyphenchar
  \fontdimen2\font=0.33333em
  \fontdimen3\font=0.16667em
  \fontdimen4\font=0.11111em
  \fontdimen7\font=0.11111em}


\input{macros}
\input{letterfonts}

\newcommand{\rlangle}[1]{\langle\alpha\left|#1\right|\alpha\rangle}
\newcommand{\inangle}[1]{\langle#1\rangle}

\title{\Huge{Mecanica Cuantica}\\Tarea 5}
\author{\huge{Sergio Montoya}\\ \huge{David Pachon}}
\date{08 de Abril de 2025}

\begin{document}

\maketitle
\newpage% or \cleardoublepage
% \pdfbookmark[<level>]{<title>}{<dest>}
\pdfbookmark[section]{\contentsname}{toc}
\tableofcontents
\pagebreak

\chapter{}

\chapter{}

\section{}

Para mostrar que esta normalizado sumamos cada coeficiente y mostramos que esto equivale a $1$

\begin{align*}
  \left| c_0 \right|^2 +
  \left| c_1 \right|^2 +
  \left| c_2 \right|^2 +
  \left| c_3 \right|^2 &= 1\\
  \left| \frac{\sqrt{2}}{4} \right|^2 +
  \left| \frac{2i}{4} \right|^2 +
  \left| - \frac{i}{4} \right|^2 +
  \left| \frac{3}{4}e^{i \frac{\pi}{3}} \right|^2 &= 1\\
  \frac{2}{16} +
  \frac{4}{16} +
  \frac{1}{16} +
  \left| \frac{3}{4}\right|^2 \left|e^{i \frac{\pi}{3}} \right|^2 &= 1\\
  \frac{2}{16} +
  \frac{4}{16} +
  \frac{1}{16} +
  \frac{9}{16}\left|\cos\left(\frac{\pi}{3}\right) + i\sin\left(\frac{\pi}{3}\right)\right|^2 &= 1\\
  \frac{2}{16} +
  \frac{4}{16} +
  \frac{1}{16} +
  \frac{9}{16}\left(\sqrt{\cos^2\left(\frac{\pi}{3}\right) + \sin^2\left(\frac{\pi}{3}\right)}\right)^2 &= 1\\
  \frac{2}{16} +
  \frac{4}{16} +
  \frac{1}{16} +
  \frac{9}{16}\left( 1 \right)^2 &= 1\\
  \frac{2}{16} +
  \frac{4}{16} +
  \frac{1}{16} +
  \frac{9}{16} &= 1\\
  \frac{2 + 4 + 1 + 9}{16} &= 1\\
  1 &= 1\\
\end{align*}

\section{}

Para encontrar la energia podemos usar la ecuación $4.2.27$ de las notas de clase en donde sabemos que los estados se pueden encontrar como:
$$
E_{n} = \left(n + \frac{1}{2}\right)\hbar \omega
$$

Por lo tanto las energias son:
\begin{align*}
  E_{n} &= \left(n + \frac{1}{2}\right)\hbar \omega\\
  E_{0} &= \left(0 + \frac{1}{2}\right)\hbar \omega\\
  &= \frac{1}{2}\hbar\omega\\
  E_{1} &= \left(1 + \frac{1}{2}\right)\hbar \omega\\
  &= \left(\frac{3}{2}\right)\hbar \omega\\
  E_{2} &= \left(2 + \frac{1}{2}\right)\hbar \omega\\
  &= \left(\frac{5}{2}\right)\hbar \omega\\
  E_{3} &= \left(3 + \frac{1}{2}\right)\hbar \omega\\
  &= \left(\frac{7}{2}\right)\hbar \omega\\
\end{align*}

Ahora bien, las probabilidades son:
\begin{align*}
  P_n &= \left|\left<n | \psi\right>\right|^2\\
  &= \left|c_n\right|^2
\end{align*}

Esto ya lo calculamos en la sección anterior por lo que sabemos que serian:
\begin{align*}
  P_0 &= \frac{2}{16}\\
  P_1 &= \frac{4}{16}\\
  P_2 &= \frac{1}{16}\\
  P_3 &= \frac{9}{16}
\end{align*}

\section{}

Para calcular
\[
  \langle E \rangle = \sum_{n = 0}^{3} P_n E_n
\]

Tomando los resultados de la sección anterior tenemos:
\begin{align*}
  \langle E \rangle &= P_0E_0 + P_1E_1 + P_2E_2 + P_3E_3\\
  &= \frac{2}{16}\left(\frac{1}{2}\hbar\omega\right) +
  \frac{4}{16}\left(\frac{3}{2}\hbar\omega\right) +
  \frac{1}{16}\left(\frac{5}{2}\hbar\omega\right) +
  \frac{9}{16}\left(\frac{7}{2}\hbar\omega\right)\\
  &= \left(\frac{2}{32}\hbar\omega\right) +
  \left(\frac{12}{32}\hbar\omega\right) +
  \left(\frac{5}{32}\hbar\omega\right) +
  \left(\frac{63}{32}\hbar\omega\right)\\
  &= \left(\frac{2 + 12 + 5 + 63}{32}\hbar\omega\right)\\
  &= \left(\frac{82}{32}\hbar\omega\right)\\
  &= \left(\frac{41}{16}\hbar\omega\right)
\end{align*}

\chapter{}

\chapter{}

\section{}

Para solucionar esto partimos desde:

\[
  x = \sqrt{\frac{\hbar}{2m\omega}} \paren{a_{-} + a_{+}};\ p = i \sqrt{\frac{m\omega\hbar}{2}}\paren{a_{+} - a_{-}}
\]

Ahora bien, tomemos que:
\begin{align*}
  \langle \alpha | a_{-} | \alpha \rangle &= \langle \alpha | \alpha | \alpha \rangle \\ &= \alpha \langle \alpha  | \alpha \rangle \\ &= \alpha\\
  \langle \alpha | a_{+} | \alpha \rangle &= \langle \alpha | \alpha^* | \alpha \rangle \\ &= \alpha^* \langle \alpha  | \alpha \rangle \\ &= \alpha^*\\
\end{align*}

Por lo tanto
\begin{align*}
  x &= \sqrt{\frac{\hbar}{2m\omega}} \paren{a_{-} + a_{+}}\\
  \langle x \rangle &= \sqrt{\frac{\hbar}{2m\omega}} \langle \alpha |\paren{a_{-} + a_{+}}| \alpha \rangle\\
  &= \sqrt{\frac{\hbar}{2m\omega}} \paren{\alpha + \alpha^*}\\
  &= \sqrt{\frac{\hbar}{2m\omega}} 2\Re\paren{\alpha}\\
  &= \sqrt{\frac{4\hbar}{2m\omega}} \Re\paren{\alpha}\\
  &= \sqrt{\frac{2\hbar}{m\omega}} \Re\paren{\alpha}
\end{align*}

Para $\langle p \rangle$
\begin{align*}
  p &= i \sqrt{\frac{m\omega\hbar}{2}}\paren{a_{+} - a_{-}}\\
  \langle p \rangle &= i \sqrt{\frac{m\omega\hbar}{2}}\langle \alpha | \paren{a_{+} - a_{-}} | \alpha \rangle\\
  &= i \sqrt{\frac{m\omega\hbar}{2}}\paren{\alpha^* - \alpha} \\
  &= i \sqrt{\frac{m\omega\hbar}{2}}\paren{-2i\Im\paren{\alpha}}\\
  &= \sqrt{\frac{4 m\omega\hbar}{2}}\Im\paren{\alpha}\\
  &= \sqrt{2 m\omega\hbar}\Im\paren{\alpha}
\end{align*}

Ahora con los casos de $\langle x^2 \rangle$ y $\langle p^2 \rangle$

Primero miremos lo siguiente:
\begin{align*}
  x^2 &= \frac{\hbar}{2m\omega} \paren{a_{-} + a_{+}}^2\\
  &= \frac{\hbar}{2m\omega} \paren{a_{-}^2 + a_{+}^2 + a_- a_+ + a_+a_-}\\
  p^2 &= - \frac{m\omega\hbar}{2}\paren{a_{+} - a_{-}}^2\\
  &= - \frac{m\omega\hbar}{2}\paren{a_{+}^2 + a_{-}^2 - a_+a_- - a_-a_+}
\end{align*}

Por lo tanto vamos a necesitar:
\begin{align*}
  \langle \alpha | a_-^2 | \alpha \rangle &= \alpha\langle \alpha | a_- | \alpha \rangle\\
  &= \alpha^2\\
  \langle \alpha | a_+^2 | \alpha \rangle &= \alpha^*\langle \alpha | a_- | \alpha \rangle\\
  &= \paren{\alpha^*}^2\\
  \langle \alpha | a_+a_- | \alpha \rangle &= \alpha \langle \alpha | a_+ | \alpha \rangle\\
  &= \alpha\alpha^* \langle \alpha | \alpha \rangle\\
  &= \left|\alpha\right|^2\\
  \langle \alpha | a_-a_+ | \alpha \rangle &= \langle \alpha | a_+a_- + 1 | \alpha \rangle\\
  &= \alpha \langle \alpha | a_+ | \alpha \rangle + \langle \alpha | 1 | \alpha \rangle\\
  &= \alpha\alpha^* \langle \alpha | \alpha \rangle + 1 \langle \alpha | \alpha \rangle\\
  &= \left|\alpha\right|^2 + 1
\end{align*}

Ya con esto podemos pasar a calcular
\begin{enumerate}
  \item $\langle x^2 \rangle$
    \begin{align*}
      x^2 &= \frac{\hbar}{2m\omega} \paren{a_{-}^2 + a_{+}^2 + a_- a_+ + a_+a_-}\\
      \langle x^2 \rangle &= \frac{\hbar}{2m\omega} \langle \alpha |\paren{a_{-}^2 + a_{+}^2 + a_- a_+ + a_+a_-}| \alpha \rangle\\
      &= \frac{\hbar}{2m\omega} \paren{\rlangle{a_{-}^2} + \rlangle{a_{+}^2} + \rlangle{a_- a_+} + \rlangle{a_+a_-}}\\
      &= \frac{\hbar}{2m\omega} \paren{\alpha^2 + \paren{\alpha^*}^2 + \left|\alpha\right|^2 + \left|\alpha\right|^2 + 1}\\
      &= \frac{\hbar}{2m\omega} \paren{\alpha^2 + \paren{\alpha^*}^2 + 2\left|\alpha\right|^2+ 1}\\
      &= \frac{\hbar}{2m\omega} \paren{2\Re \paren{\alpha}^2 - 2\Im \paren{\alpha}^2 + 2\Re\paren{\alpha}^2 + 2\Im\paren{\alpha}^2 + 1}\\
      &= \frac{\hbar}{2m\omega} \paren{4\Re \paren{\alpha}^2 + 1}\\
      &= \frac{2\hbar}{m\omega}\Re \paren{\alpha}^2 + \frac{\hbar}{2m\omega}
    \end{align*}

  \item $\langle p^2 \rangle$
    \begin{align*}
      p^2 &= - \frac{m\omega\hbar}{2}\paren{a_{+}^2 + a_{-}^2 - a_+a_- - a_-a_+}\\
      \langle p^2 \rangle &= - \frac{m\omega\hbar}{2}\rlangle{a_{+}^2 + a_{-}^2 - a_+a_- - a_-a_+}\\
      &= - \frac{m\omega\hbar}{2}\paren{\rlangle{a_{+}^2} + \rlangle{a_{-}^2} - \rlangle{a_+a_-} - \rlangle{a_-a_+}}\\
      &= - \frac{m\omega\hbar}{2}\paren{\alpha^2 + \paren{\alpha^*}^2 - \left|\alpha\right|^2 - \paren{\left|\alpha\right|^2 + 1}}\\
      &= - \frac{m\omega\hbar}{2}\paren{2\Re\paren{\alpha}^2 - 2\Im\paren{\alpha}^2 - \left|\alpha\right|^2 - \left|\alpha\right|^2 - 1}\\
      &= - \frac{m\omega\hbar}{2}\paren{2\Re\paren{\alpha}^2 - 2\Im\paren{\alpha}^2 - 2\Re\paren{\alpha}^2 - 2\Im\paren{\alpha}^2 - 1}\\
      &= - \frac{m\omega\hbar}{2}\paren{- 4\Im\paren{\alpha}^2 - 1}\\
      &= \frac{m\omega\hbar}{2}\paren{4\Im\paren{\alpha}^2 + 1}\\
      &= \frac{m\omega\hbar}{2}4\Im\paren{\alpha}^2 + \frac{m\omega\hbar}{2}\\
      &= 2m\omega\hbar\Im\paren{\alpha}^2 + \frac{m\omega\hbar}{2}
    \end{align*}
\end{enumerate}

Por lo tanto los resultados son:
\begin{enumerate}
  \item $\langle x \rangle = \sqrt{\frac{2\hbar}{m\omega}} \Re\paren{\alpha}$
  \item $\langle p \rangle = \sqrt{2 m\omega}\Im\paren{\alpha}$
  \item $\langle x^2 \rangle = \frac{2\hbar}{m\omega}\Re \paren{\alpha}^2 + \frac{\hbar}{2m\omega}$
  \item $\langle p^2 \rangle = 2m\omega\hbar\Im\paren{\alpha}^2 + \frac{m\omega\hbar}{2}$
\end{enumerate}

\pagebreak

\section{}

En este caso tenemos:
\[
  \sigma_x = \sqrt{\inangle{x^2} - \inangle{x}^2}
\]

Por lo tanto veamos cuanto es $\inangle{x}^2$

\begin{align*}
  \inangle{x}^2 &= \paren{\sqrt{\frac{2\hbar}{m\omega}} \Re\paren{\alpha}}^2\\
  \inangle{x}^2 &= \frac{2\hbar}{m\omega} \Re\paren{\alpha}^2
\end{align*}

Con esto entonces
\begin{align*}
  \sigma_x &= \sqrt{\inangle{x^2} - \inangle{x}^2}\\
  &= \sqrt{\frac{2\hbar}{m\omega}\Re \paren{\alpha}^2 + \frac{\hbar}{2m\omega} - \frac{2\hbar}{m\omega} \Re\paren{\alpha}^2}\\
  &= \sqrt{\frac{\hbar}{2m\omega}}
\end{align*}

Por el otro lado
\[
  \sigma_p = \sqrt{\inangle{p^2} - \inangle{p}^2}
\]

con
\begin{align*}
  \inangle{p}^2 &= \paren{\sqrt{2 m\omega\hbar}\Im\paren{\alpha}}^2\\
   &= \paren{\sqrt{2 m\omega\hbar}\Im\paren{\alpha}}^2\\
   &= 2 m\omega\hbar\Im\paren{\alpha}^2\\
\end{align*}

De nuevo calculemos
\begin{align*}
  \sigma_p &= \sqrt{\inangle{p^2} - \inangle{p}^2}\\
  &= \sqrt{2m\omega\hbar\Im\paren{\alpha}^2 + \frac{m\omega\hbar}{2} - 2 m\omega\hbar\Im\paren{\alpha}^2}\\
  &= \sqrt{\frac{m\omega\hbar}{2}}
\end{align*}

Ahora al final:
\begin{align*}
  \sigma_x\sigma_p &= \sqrt{\frac{\hbar}{2m\omega}} \cdot \sqrt{\frac{m\omega\hbar}{2}}\\
  \sigma_x\sigma_p &= \sqrt{\frac{\hbar^2 m\omega}{2\cdot2m\omega}}\\
  \sigma_x\sigma_p &= \sqrt{\frac{\hbar^2}{4}}\\
  \sigma_x\sigma_p &= \frac{\hbar}{2}
\end{align*}

\pagebreak

\section{}

Apliquemos $a_-$ de la siguiente manera
\begin{align*}
  a_-\left| \alpha \right> &= \alpha \sum_{n} c_n \left| n \right>\\
  &= \sum_{n} c_n \sqrt{n} \left| n - 1 \right>\\
\end{align*}

Dado que son esencialmente los mismos podemos hacer
\begin{align*}
  \alpha \sum_{n} c_n \left| n \right> &= \sum_{n} c_n \sqrt{n} \left| n - 1 \right>\\
  \sum_{n} \alpha c_n \left| n \right> &= \sum_{n} c_n \sqrt{n} \left| n - 1 \right>\\
  \sum_{n} \alpha c_n \left| n \right> &= \sum_{n} c_{n + 1} \sqrt{n - 1} \left| n \right>\\
  \alpha c_n &= c_{n + 1} \sqrt{n - 1}\\
  \frac{\alpha}{\sqrt{n - 1}} c_n &= c_{n + 1}
\end{align*}

Dada esta definición recursiva podemos reducirla hasta
\[
  \frac{\alpha^n}{\sqrt{n!}} c_0 = c_{n}
\]

\section{}

Tenemos:
\begin{align*}
  \inangle{\alpha | \alpha} &= \sum_{n = 0}^{\infty} \left|c_n\right|^2 = 1\\
  \sum_{n = 0}^{\infty} \left|\frac{\alpha^n}{\sqrt{n!}}\right|^2 &= \left|c_0\right|^2 \sum_{n = 0}^{\infty} \frac{\left|\alpha\right|^{2n}}{n!}\\
  \sum_{n = 0}^{\infty} \left|\frac{\alpha^n}{\sqrt{n!}}\right|^2 &= \left|c_0\right|^2 \sum_{n = 0}^{\infty} \frac{\paren{\left|\alpha\right|^{2}}^n}{n!}
\end{align*}

Esta es una serie exponencial conocida:
\[
  \sum_{k = 0}^\infty \frac{z^k}{k!} = e^z
\]

Por lo tanto

\begin{align*}
  \left|c_0\right|^2 \sum_{n = 0}^{\infty} \frac{\left|\alpha\right|^{2n}}{n!} &= \left|c_0\right|^2 e^{\left|\alpha\right|^2}\\
  \left|c_0\right|^2 e^{\left|\alpha\right|^2} &= 1\\
  \left|c_0\right|^2 &= e^{-\left|\alpha\right|^2}\\
  c_0 &= e^{-\left|\alpha\right|^2/2}
\end{align*}

\pagebreak

\section{}

En los punto anterior definimos que:
\begin{equation*}
  \left| \alpha \right> = \sum_{n} c_n \left| n \right> = \sum_{n} \frac{\alpha^n}{\sqrt{n!}} c_0 \left| n \right> = \sum_{n} \frac{\alpha^n}{\sqrt{n!}} e^{-\left|\alpha\right|^2/2} \left| n \right> = e^{-\left|\alpha\right|^2/2} \sum_{n} \frac{\alpha^n}{\sqrt{n!}}  \left| n \right>
\end{equation*}

Con esto entonces, podemos pasar a $\left| \alpha (t) \right>$ de la manera en la que nos dicen esto queda como
\begin{equation*}
  \left| \alpha(t) \right> = e^{-\left|\alpha\right|^2/2} \sum_{n} \frac{\alpha^n}{\sqrt{n!}}  e^{-i E_n t/ \hbar}\left| n \right>
\end{equation*}

Ahora apliquemos $a_-$

\begin{align*}
  a_{-}\left|\alpha(t)\right> &=e^{-\left|\alpha\right|^2/2} \sum_{n} \frac{\alpha^n}{\sqrt{n!}}  e^{-i E_n t/ \hbar}\sqrt{n}\left| n - 1\right> \\
  a_{-}\left|\alpha(t)\right> &=e^{-\left|\alpha\right|^2/2} \sum_{n + 1} \frac{\alpha^{n + 1}}{\sqrt{n + 1!}} \sqrt{n + 1} e^{-i E_{n + 1} t/ \hbar}\left| n\right> \\
  a_{-}\left|\alpha(t)\right> &=e^{-\left|\alpha\right|^2/2} \sum_{n} \frac{\alpha^{n}}{\sqrt{n!}}\alpha e^{-i E_{n + 1} t/ \hbar}\left| n\right> \\
  E_n &= \hbar\omega\paren{n + \frac{1}{2}}\\
  \implies E_{n + 1} &= \hbar\omega\paren{n + \frac{3}{2}}\\
  \implies e^{-iE_{n + 1}t/\hbar} &= e^{-i\omega t}e^{-i E_n t/\hbar}\\
  a_{-}\left|\alpha(t)\right> &=e^{-\left|\alpha\right|^2/2} \alpha \sum_{n} \frac{\alpha^{n}}{\sqrt{n!}} e^{-i\omega t}e^{-i E_n t/\hbar} \left| n\right> \\
  a_{-}\left|\alpha(t)\right> &=e^{-\left|\alpha\right|^2/2} \alpha e^{-i\omega t}\sum_{n} \frac{\alpha^{n}}{\sqrt{n!}} e^{-i E_n t/\hbar} \left| n\right> \\
  \alpha e^{-i\omega t} &= \alpha(t)\\
  a_{-}\left|\alpha(t)\right> &=\alpha(t) e^{-\left|\alpha\right|^2/2} \sum_{n} \frac{\alpha^{n}}{\sqrt{n!}} e^{-i E_n t/\hbar} \left| n\right> \\
  a_{-}\left|\alpha(t)\right> &=\alpha(t) \left|\alpha(t)\right>
\end{align*}

Esto en esencia quiere decir que este operador oscila de manera coherente con un oscilador clasico de periodo $\omega$. Cosa que para ser honestos tiene sentido pues estamos minimizando la incertidumbre.

\pagebreak

\section{}

Verifiquemos cada caso:

\begin{align*}
  a_{-}\left| 0 \right> &= 0 \cdot \left| 0 \right>\\\
  \alpha &= 0\\
  \left| \alpha \right> &= e^{-\left|\alpha\right|^2/2} \sum_{n} \frac{\alpha^n}{\sqrt{n!}}  \left| n \right>\\
  \left| \alpha = 0 \right> &= e^{-\left| 0 \right|^2/2} \sum_{n} \frac{0^n}{\sqrt{n!}}  \left| n \right>\\
  \left| \alpha = 0 \right> &= \sum_{n} 0  \left| n \right>\\
  \left| \alpha = 0 \right> &= \left| 0 \right>
\end{align*}

Por ultimo
\begin{enumerate}
  \item $\langle x \rangle = \sqrt{\frac{2\hbar}{m\omega}} \Re\paren{0} = 0$
  \item $\langle p \rangle = \sqrt{2 m\omega\hbar}\Im\paren{0} = 0$
  \item $\langle x^2 \rangle = \frac{2\hbar}{m\omega}\Re \paren{0}^2 + \frac{\hbar}{2m\omega} = \frac{\hbar}{2m\omega}$
  \item $\langle p^2 \rangle = 2m\omega\hbar\Im\paren{0}^2 + \frac{m\omega\hbar}{2} = \frac{m\omega\hbar}{2}$
\end{enumerate}

Por lo tanto:
\begin{align*}
  \sigma_x &= \sqrt{\inangle{x^2} - \inangle{x}^2}\\
  &= \sqrt{\frac{\hbar}{2m\omega} - 0}\\
  &= \sqrt{\frac{\hbar}{2m\omega}}\\
  \sigma_p &= \sqrt{\inangle{p^2} - \inangle{p}^2}\\
  &= \sqrt{\frac{m\omega\hbar}{2} - 0}\\
  &= \sqrt{\frac{m\omega\hbar}{2}}\\
  \sigma_x\sigma_p &= \sqrt{\frac{\hbar}{2m\omega}} \cdot \sqrt{\frac{m\omega\hbar}{2}}\\
  \sigma_x\sigma_p &= \sqrt{\frac{\hbar^2}{4}}\\
  \sigma_x\sigma_p &= \frac{\hbar}{2}
\end{align*}

Por lo tanto si es un estado coherente con $\alpha = 0$

\end{document}

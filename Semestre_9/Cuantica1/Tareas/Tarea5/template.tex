\documentclass{report}

\documentclass[12pt]{article}
\usepackage{array}
\usepackage{color}
\usepackage{amsthm}
\usepackage{eufrak}
\usepackage{lipsum}
\usepackage{pifont}
\usepackage{yfonts}
\usepackage{amsmath}
\usepackage{amssymb}
\usepackage{ccfonts}
\usepackage{comment} \usepackage{amsfonts}
\usepackage{fancyhdr}
\usepackage{graphicx}
\usepackage{listings}
\usepackage{mathrsfs}
\usepackage{setspace}
\usepackage{textcomp}
\usepackage{blindtext}
\usepackage{enumerate}
\usepackage{microtype}
\usepackage{xfakebold}
\usepackage{kantlipsum}
%\usepackage{draftwatermark}
\usepackage[spanish]{babel}
\usepackage[margin=1.5cm, top=2cm, bottom=2cm]{geometry}
\usepackage[framemethod=tikz]{mdframed}
\usepackage[colorlinks=true,citecolor=blue,linkcolor=red,urlcolor=magenta]{hyperref}

%//////////////////////////////////////////////////////
% Watermark configuration
%//////////////////////////////////////////////////////
%\SetWatermarkScale{4}
%\SetWatermarkColor{black}
%\SetWatermarkLightness{0.95}
%\SetWatermarkText{\texttt{Watermark}}

%//////////////////////////////////////////////////////
% Frame configuration
%//////////////////////////////////////////////////////
\newmdenv[tikzsetting={draw=gray,fill=white,fill opacity=0},backgroundcolor=none]{Frame}

%//////////////////////////////////////////////////////
% Font style configuration
%//////////////////////////////////////////////////////
\renewcommand{\familydefault}{\ttdefault}
\renewcommand{\rmdefault}{tt}

%//////////////////////////////////////////////////////
% Bold configuration
%//////////////////////////////////////////////////////
\newcommand{\fbseries}{\unskip\setBold\aftergroup\unsetBold\aftergroup\ignorespaces}
\makeatletter
\newcommand{\setBoldness}[1]{\def\fake@bold{#1}}
\makeatother

%//////////////////////////////////////////////////////
% Default font configuration
%//////////////////////////////////////////////////////
\DeclareFontFamily{\encodingdefault}{\ttdefault}{%
  \hyphenchar\font=\defaulthyphenchar
  \fontdimen2\font=0.33333em
  \fontdimen3\font=0.16667em
  \fontdimen4\font=0.11111em
  \fontdimen7\font=0.11111em}


\input{macros}
\input{letterfonts}

\title{\Huge{Mecanica Cuantica}\\Tarea 5}
\author{\huge{Sergio Montoya}\\ \huge{David Pachon}}
\date{}

\begin{document}

\maketitle
\newpage% or \cleardoublepage
% \pdfbookmark[<level>]{<title>}{<dest>}
\pdfbookmark[section]{\contentsname}{toc}
\tableofcontents
\pagebreak

\chapter{}

\chapter{}

\section{}

Para mostrar que esta normalizado sumamos cada coeficiente y mostramos que esto equivale a $1$

\begin{align*}
  \left| c_0 \right|^2 +
  \left| c_1 \right|^2 +
  \left| c_2 \right|^2 +
  \left| c_3 \right|^2 &= 1\\
  \left| \frac{\sqrt{2}}{4} \right|^2 +
  \left| \frac{2i}{4} \right|^2 +
  \left| - \frac{i}{4} \right|^2 +
  \left| \frac{3}{4}e^{i \frac{\pi}{3}} \right|^2 &= 1\\
  \frac{2}{16} +
  \frac{4}{16} +
  \frac{1}{16} +
  \left| \frac{3}{4}\right|^2 \left|e^{i \frac{\pi}{3}} \right|^2 &= 1\\
  \frac{2}{16} +
  \frac{4}{16} +
  \frac{1}{16} +
  \frac{9}{16}\left|\cos\left(\frac{\pi}{3}\right) + i\sin\left(\frac{\pi}{3}\right)\right|^2 &= 1\\
  \frac{2}{16} +
  \frac{4}{16} +
  \frac{1}{16} +
  \frac{9}{16}\left(\sqrt{\cos^2\left(\frac{\pi}{3}\right) + \sin^2\left(\frac{\pi}{3}\right)}\right)^2 &= 1\\
  \frac{2}{16} +
  \frac{4}{16} +
  \frac{1}{16} +
  \frac{9}{16}\left( 1 \right)^2 &= 1\\
  \frac{2}{16} +
  \frac{4}{16} +
  \frac{1}{16} +
  \frac{9}{16} &= 1\\
  \frac{2 + 4 + 1 + 9}{16} &= 1\\
  1 &= 1\\
\end{align*}

\section{}

Para encontrar la energia podemos usar la ecuación $4.2.27$ de las notas de clase en donde sabemos que los estados se pueden encontrar como:
$$
  E_{n} = \left(n + \frac{1}{2}\right)\hbar \omega
$$

Por lo tanto las energias son:
\begin{align*}
  E_{n} &= \left(n + \frac{1}{2}\right)\hbar \omega\\
  E_{0} &= \left(0 + \frac{1}{2}\right)\hbar \omega\\
  &= \frac{1}{2}\hbar\omega\\
  E_{1} &= \left(1 + \frac{1}{2}\right)\hbar \omega\\
  &= \left(\frac{3}{2}\right)\hbar \omega\\
  E_{2} &= \left(2 + \frac{1}{2}\right)\hbar \omega\\
  &= \left(\frac{5}{2}\right)\hbar \omega\\
  E_{3} &= \left(3 + \frac{1}{2}\right)\hbar \omega\\
  &= \left(\frac{7}{2}\right)\hbar \omega\\
\end{align*}

Ahora bien, las probabilidades son:
\begin{align*}
  P_n &= \left|\left<n | \psi\right>\right|^2\\
  &= \left|c_n\right|^2
\end{align*}

Esto ya lo calculamos en la sección anterior por lo que sabemos que serian:
\begin{align*}
  P_0 &= \frac{2}{16}\\
  P_1 &= \frac{4}{16}\\
  P_2 &= \frac{1}{16}\\
  P_3 &= \frac{9}{16}
\end{align*}

\section{}

Para calcular
\[
  \langle E \rangle = \sum_{n = 0}^{3} P_n E_n
\]

Tomando los resultados de la sección anterior tenemos:
\begin{align*}
  \langle E \rangle &= P_0E_0 + P_1E_1 + P_2E_2 + P_3E_3\\
  &= \frac{2}{16}\left(\frac{1}{2}\hbar\omega\right) +
\frac{4}{16}\left(\frac{3}{2}\hbar\omega\right) +
\frac{1}{16}\left(\frac{5}{2}\hbar\omega\right) +
\frac{9}{16}\left(\frac{7}{2}\hbar\omega\right)\\
  &= \left(\frac{2}{32}\hbar\omega\right) +
\left(\frac{12}{32}\hbar\omega\right) +
\left(\frac{5}{32}\hbar\omega\right) +
\left(\frac{63}{32}\hbar\omega\right)\\
  &= \left(\frac{2 + 12 + 5 + 63}{32}\hbar\omega\right)\\
  &= \left(\frac{82}{32}\hbar\omega\right)\\
  &= \left(\frac{41}{16}\hbar\omega\right)
\end{align*}

\chapter{}

\chapter{}

\end{document}

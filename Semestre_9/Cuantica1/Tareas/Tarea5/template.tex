\documentclass{report}

\documentclass[12pt]{article}
\usepackage{array}
\usepackage{color}
\usepackage{amsthm}
\usepackage{eufrak}
\usepackage{lipsum}
\usepackage{pifont}
\usepackage{yfonts}
\usepackage{amsmath}
\usepackage{amssymb}
\usepackage{ccfonts}
\usepackage{comment} \usepackage{amsfonts}
\usepackage{fancyhdr}
\usepackage{graphicx}
\usepackage{listings}
\usepackage{mathrsfs}
\usepackage{setspace}
\usepackage{textcomp}
\usepackage{blindtext}
\usepackage{enumerate}
\usepackage{microtype}
\usepackage{xfakebold}
\usepackage{kantlipsum}
%\usepackage{draftwatermark}
\usepackage[spanish]{babel}
\usepackage[margin=1.5cm, top=2cm, bottom=2cm]{geometry}
\usepackage[framemethod=tikz]{mdframed}
\usepackage[colorlinks=true,citecolor=blue,linkcolor=red,urlcolor=magenta]{hyperref}

%//////////////////////////////////////////////////////
% Watermark configuration
%//////////////////////////////////////////////////////
%\SetWatermarkScale{4}
%\SetWatermarkColor{black}
%\SetWatermarkLightness{0.95}
%\SetWatermarkText{\texttt{Watermark}}

%//////////////////////////////////////////////////////
% Frame configuration
%//////////////////////////////////////////////////////
\newmdenv[tikzsetting={draw=gray,fill=white,fill opacity=0},backgroundcolor=none]{Frame}

%//////////////////////////////////////////////////////
% Font style configuration
%//////////////////////////////////////////////////////
\renewcommand{\familydefault}{\ttdefault}
\renewcommand{\rmdefault}{tt}

%//////////////////////////////////////////////////////
% Bold configuration
%//////////////////////////////////////////////////////
\newcommand{\fbseries}{\unskip\setBold\aftergroup\unsetBold\aftergroup\ignorespaces}
\makeatletter
\newcommand{\setBoldness}[1]{\def\fake@bold{#1}}
\makeatother

%//////////////////////////////////////////////////////
% Default font configuration
%//////////////////////////////////////////////////////
\DeclareFontFamily{\encodingdefault}{\ttdefault}{%
  \hyphenchar\font=\defaulthyphenchar
  \fontdimen2\font=0.33333em
  \fontdimen3\font=0.16667em
  \fontdimen4\font=0.11111em
  \fontdimen7\font=0.11111em}


%From M275 "Topology" at SJSU
\newcommand{\id}{\mathrm{id}}
\newcommand{\taking}[1]{\xrightarrow{#1}}
\newcommand{\inv}{^{-1}}

%From M170 "Introduction to Graph Theory" at SJSU
\DeclareMathOperator{\diam}{diam}
\DeclareMathOperator{\ord}{ord}
\newcommand{\defeq}{\overset{\mathrm{def}}{=}}

%From the USAMO .tex files
\newcommand{\ts}{\textsuperscript}
\newcommand{\dg}{^\circ}
\newcommand{\ii}{\item}

% % From Math 55 and Math 145 at Harvard
% \newenvironment{subproof}[1][Proof]{%
% \begin{proof}[#1] \renewcommand{\qedsymbol}{$\blacksquare$}}%
% {\end{proof}}

\newcommand{\liff}{\leftrightarrow}
\newcommand{\lthen}{\rightarrow}
\newcommand{\opname}{\operatorname}
\newcommand{\surjto}{\twoheadrightarrow}
\newcommand{\injto}{\hookrightarrow}
\newcommand{\On}{\mathrm{On}} % ordinals
\DeclareMathOperator{\img}{im} % Image
\DeclareMathOperator{\Img}{Im} % Image
\DeclareMathOperator{\coker}{coker} % Cokernel
\DeclareMathOperator{\Coker}{Coker} % Cokernel
\DeclareMathOperator{\Ker}{Ker} % Kernel
\DeclareMathOperator{\rank}{rank}
\DeclareMathOperator{\Spec}{Spec} % spectrum
\DeclareMathOperator{\Tr}{Tr} % trace
\DeclareMathOperator{\pr}{pr} % projection
\DeclareMathOperator{\ext}{ext} % extension
\DeclareMathOperator{\pred}{pred} % predecessor
\DeclareMathOperator{\dom}{dom} % domain
\DeclareMathOperator{\ran}{ran} % range
\DeclareMathOperator{\Hom}{Hom} % homomorphism
\DeclareMathOperator{\Mor}{Mor} % morphisms
\DeclareMathOperator{\End}{End} % endomorphism

\newcommand{\eps}{\epsilon}
\newcommand{\veps}{\varepsilon}
\newcommand{\ol}{\overline}
\newcommand{\ul}{\underline}
\newcommand{\wt}{\widetilde}
\newcommand{\wh}{\widehat}
\newcommand{\vocab}[1]{\textbf{\color{blue} #1}}
\providecommand{\half}{\frac{1}{2}}
\newcommand{\dang}{\measuredangle} %% Directed angle
\newcommand{\ray}[1]{\overrightarrow{#1}}
\newcommand{\seg}[1]{\overline{#1}}
\newcommand{\arc}[1]{\wideparen{#1}}
\DeclareMathOperator{\cis}{cis}
\DeclareMathOperator*{\lcm}{lcm}
\DeclareMathOperator*{\argmin}{arg min}
\DeclareMathOperator*{\argmax}{arg max}
\newcommand{\cycsum}{\sum_{\mathrm{cyc}}}
\newcommand{\symsum}{\sum_{\mathrm{sym}}}
\newcommand{\cycprod}{\prod_{\mathrm{cyc}}}
\newcommand{\symprod}{\prod_{\mathrm{sym}}}
\newcommand{\Qed}{\begin{flushright}\qed\end{flushright}}
\newcommand{\parinn}{\setlength{\parindent}{1cm}}
\newcommand{\parinf}{\setlength{\parindent}{0cm}}
% \newcommand{\norm}{\|\cdot\|}
\newcommand{\inorm}{\norm_{\infty}}
\newcommand{\opensets}{\{V_{\alpha}\}_{\alpha\in I}}
\newcommand{\oset}{V_{\alpha}}
\newcommand{\opset}[1]{V_{\alpha_{#1}}}
\newcommand{\lub}{\text{lub}}
\newcommand{\del}[2]{\frac{\partial #1}{\partial #2}}
\newcommand{\Del}[3]{\frac{\partial^{#1} #2}{\partial^{#1} #3}}
\newcommand{\deld}[2]{\dfrac{\partial #1}{\partial #2}}
\newcommand{\Deld}[3]{\dfrac{\partial^{#1} #2}{\partial^{#1} #3}}
\newcommand{\lm}{\lambda}
\newcommand{\uin}{\mathbin{\rotatebox[origin=c]{90}{$\in$}}}
\newcommand{\usubset}{\mathbin{\rotatebox[origin=c]{90}{$\subset$}}}
\newcommand{\lt}{\left}
\newcommand{\rt}{\right}
\newcommand{\paren}[1]{\left(#1\right)}
\newcommand{\bs}[1]{\boldsymbol{#1}}
\newcommand{\exs}{\exists}
\newcommand{\st}{\strut}
\newcommand{\dps}[1]{\displaystyle{#1}}

\newcommand{\sol}{\setlength{\parindent}{0cm}\textbf{\textit{Solution:}}\setlength{\parindent}{1cm} }
\newcommand{\solve}[1]{\setlength{\parindent}{0cm}\textbf{\textit{Solution: }}\setlength{\parindent}{1cm}#1 \Qed}

% Things Lie
\newcommand{\kb}{\mathfrak b}
\newcommand{\kg}{\mathfrak g}
\newcommand{\kh}{\mathfrak h}
\newcommand{\kn}{\mathfrak n}
\newcommand{\ku}{\mathfrak u}
\newcommand{\kz}{\mathfrak z}
\DeclareMathOperator{\Ext}{Ext} % Ext functor
\DeclareMathOperator{\Tor}{Tor} % Tor functor
\newcommand{\gl}{\opname{\mathfrak{gl}}} % frak gl group
\renewcommand{\sl}{\opname{\mathfrak{sl}}} % frak sl group chktex 6

% More script letters etc.
\newcommand{\SA}{\mathcal A}
\newcommand{\SB}{\mathcal B}
\newcommand{\SC}{\mathcal C}
\newcommand{\SF}{\mathcal F}
\newcommand{\SG}{\mathcal G}
\newcommand{\SH}{\mathcal H}
\newcommand{\OO}{\mathcal O}

\newcommand{\SCA}{\mathscr A}
\newcommand{\SCB}{\mathscr B}
\newcommand{\SCC}{\mathscr C}
\newcommand{\SCD}{\mathscr D}
\newcommand{\SCE}{\mathscr E}
\newcommand{\SCF}{\mathscr F}
\newcommand{\SCG}{\mathscr G}
\newcommand{\SCH}{\mathscr H}

% Mathfrak primes
\newcommand{\km}{\mathfrak m}
\newcommand{\kp}{\mathfrak p}
\newcommand{\kq}{\mathfrak q}

% number sets
\newcommand{\RR}[1][]{\ensuremath{\ifstrempty{#1}{\mathbb{R}}{\mathbb{R}^{#1}}}}
\newcommand{\NN}[1][]{\ensuremath{\ifstrempty{#1}{\mathbb{N}}{\mathbb{N}^{#1}}}}
\newcommand{\ZZ}[1][]{\ensuremath{\ifstrempty{#1}{\mathbb{Z}}{\mathbb{Z}^{#1}}}}
\newcommand{\QQ}[1][]{\ensuremath{\ifstrempty{#1}{\mathbb{Q}}{\mathbb{Q}^{#1}}}}
\newcommand{\CC}[1][]{\ensuremath{\ifstrempty{#1}{\mathbb{C}}{\mathbb{C}^{#1}}}}
\newcommand{\PP}[1][]{\ensuremath{\ifstrempty{#1}{\mathbb{P}}{\mathbb{P}^{#1}}}}
\newcommand{\HH}[1][]{\ensuremath{\ifstrempty{#1}{\mathbb{H}}{\mathbb{H}^{#1}}}}
\newcommand{\FF}[1][]{\ensuremath{\ifstrempty{#1}{\mathbb{F}}{\mathbb{F}^{#1}}}}
% expected value
\newcommand{\EE}{\ensuremath{\mathbb{E}}}
\newcommand{\charin}{\text{ char }}
\DeclareMathOperator{\sign}{sign}
\DeclareMathOperator{\Aut}{Aut}
\DeclareMathOperator{\Inn}{Inn}
\DeclareMathOperator{\Syl}{Syl}
\DeclareMathOperator{\Gal}{Gal}
\DeclareMathOperator{\GL}{GL} % General linear group
\DeclareMathOperator{\SL}{SL} % Special linear group

%---------------------------------------
% BlackBoard Math Fonts :-
%---------------------------------------

%Captital Letters
\newcommand{\bbA}{\mathbb{A}}	\newcommand{\bbB}{\mathbb{B}}
\newcommand{\bbC}{\mathbb{C}}	\newcommand{\bbD}{\mathbb{D}}
\newcommand{\bbE}{\mathbb{E}}	\newcommand{\bbF}{\mathbb{F}}
\newcommand{\bbG}{\mathbb{G}}	\newcommand{\bbH}{\mathbb{H}}
\newcommand{\bbI}{\mathbb{I}}	\newcommand{\bbJ}{\mathbb{J}}
\newcommand{\bbK}{\mathbb{K}}	\newcommand{\bbL}{\mathbb{L}}
\newcommand{\bbM}{\mathbb{M}}	\newcommand{\bbN}{\mathbb{N}}
\newcommand{\bbO}{\mathbb{O}}	\newcommand{\bbP}{\mathbb{P}}
\newcommand{\bbQ}{\mathbb{Q}}	\newcommand{\bbR}{\mathbb{R}}
\newcommand{\bbS}{\mathbb{S}}	\newcommand{\bbT}{\mathbb{T}}
\newcommand{\bbU}{\mathbb{U}}	\newcommand{\bbV}{\mathbb{V}}
\newcommand{\bbW}{\mathbb{W}}	\newcommand{\bbX}{\mathbb{X}}
\newcommand{\bbY}{\mathbb{Y}}	\newcommand{\bbZ}{\mathbb{Z}}

%---------------------------------------
% MathCal Fonts :-
%---------------------------------------

%Captital Letters
\newcommand{\mcA}{\mathcal{A}}	\newcommand{\mcB}{\mathcal{B}}
\newcommand{\mcC}{\mathcal{C}}	\newcommand{\mcD}{\mathcal{D}}
\newcommand{\mcE}{\mathcal{E}}	\newcommand{\mcF}{\mathcal{F}}
\newcommand{\mcG}{\mathcal{G}}	\newcommand{\mcH}{\mathcal{H}}
\newcommand{\mcI}{\mathcal{I}}	\newcommand{\mcJ}{\mathcal{J}}
\newcommand{\mcK}{\mathcal{K}}	\newcommand{\mcL}{\mathcal{L}}
\newcommand{\mcM}{\mathcal{M}}	\newcommand{\mcN}{\mathcal{N}}
\newcommand{\mcO}{\mathcal{O}}	\newcommand{\mcP}{\mathcal{P}}
\newcommand{\mcQ}{\mathcal{Q}}	\newcommand{\mcR}{\mathcal{R}}
\newcommand{\mcS}{\mathcal{S}}	\newcommand{\mcT}{\mathcal{T}}
\newcommand{\mcU}{\mathcal{U}}	\newcommand{\mcV}{\mathcal{V}}
\newcommand{\mcW}{\mathcal{W}}	\newcommand{\mcX}{\mathcal{X}}
\newcommand{\mcY}{\mathcal{Y}}	\newcommand{\mcZ}{\mathcal{Z}}


%---------------------------------------
% Bold Math Fonts :-
%---------------------------------------

%Captital Letters
\newcommand{\bmA}{\boldsymbol{A}}	\newcommand{\bmB}{\boldsymbol{B}}
\newcommand{\bmC}{\boldsymbol{C}}	\newcommand{\bmD}{\boldsymbol{D}}
\newcommand{\bmE}{\boldsymbol{E}}	\newcommand{\bmF}{\boldsymbol{F}}
\newcommand{\bmG}{\boldsymbol{G}}	\newcommand{\bmH}{\boldsymbol{H}}
\newcommand{\bmI}{\boldsymbol{I}}	\newcommand{\bmJ}{\boldsymbol{J}}
\newcommand{\bmK}{\boldsymbol{K}}	\newcommand{\bmL}{\boldsymbol{L}}
\newcommand{\bmM}{\boldsymbol{M}}	\newcommand{\bmN}{\boldsymbol{N}}
\newcommand{\bmO}{\boldsymbol{O}}	\newcommand{\bmP}{\boldsymbol{P}}
\newcommand{\bmQ}{\boldsymbol{Q}}	\newcommand{\bmR}{\boldsymbol{R}}
\newcommand{\bmS}{\boldsymbol{S}}	\newcommand{\bmT}{\boldsymbol{T}}
\newcommand{\bmU}{\boldsymbol{U}}	\newcommand{\bmV}{\boldsymbol{V}}
\newcommand{\bmW}{\boldsymbol{W}}	\newcommand{\bmX}{\boldsymbol{X}}
\newcommand{\bmY}{\boldsymbol{Y}}	\newcommand{\bmZ}{\boldsymbol{Z}}
%Small Letters
\newcommand{\bma}{\boldsymbol{a}}	\newcommand{\bmb}{\boldsymbol{b}}
\newcommand{\bmc}{\boldsymbol{c}}	\newcommand{\bmd}{\boldsymbol{d}}
\newcommand{\bme}{\boldsymbol{e}}	\newcommand{\bmf}{\boldsymbol{f}}
\newcommand{\bmg}{\boldsymbol{g}}	\newcommand{\bmh}{\boldsymbol{h}}
\newcommand{\bmi}{\boldsymbol{i}}	\newcommand{\bmj}{\boldsymbol{j}}
\newcommand{\bmk}{\boldsymbol{k}}	\newcommand{\bml}{\boldsymbol{l}}
\newcommand{\bmm}{\boldsymbol{m}}	\newcommand{\bmn}{\boldsymbol{n}}
\newcommand{\bmo}{\boldsymbol{o}}	\newcommand{\bmp}{\boldsymbol{p}}
\newcommand{\bmq}{\boldsymbol{q}}	\newcommand{\bmr}{\boldsymbol{r}}
\newcommand{\bms}{\boldsymbol{s}}	\newcommand{\bmt}{\boldsymbol{t}}
\newcommand{\bmu}{\boldsymbol{u}}	\newcommand{\bmv}{\boldsymbol{v}}
\newcommand{\bmw}{\boldsymbol{w}}	\newcommand{\bmx}{\boldsymbol{x}}
\newcommand{\bmy}{\boldsymbol{y}}	\newcommand{\bmz}{\boldsymbol{z}}

%---------------------------------------
% Scr Math Fonts :-
%---------------------------------------

\newcommand{\sA}{{\mathscr{A}}}   \newcommand{\sB}{{\mathscr{B}}}
\newcommand{\sC}{{\mathscr{C}}}   \newcommand{\sD}{{\mathscr{D}}}
\newcommand{\sE}{{\mathscr{E}}}   \newcommand{\sF}{{\mathscr{F}}}
\newcommand{\sG}{{\mathscr{G}}}   \newcommand{\sH}{{\mathscr{H}}}
\newcommand{\sI}{{\mathscr{I}}}   \newcommand{\sJ}{{\mathscr{J}}}
\newcommand{\sK}{{\mathscr{K}}}   \newcommand{\sL}{{\mathscr{L}}}
\newcommand{\sM}{{\mathscr{M}}}   \newcommand{\sN}{{\mathscr{N}}}
\newcommand{\sO}{{\mathscr{O}}}   \newcommand{\sP}{{\mathscr{P}}}
\newcommand{\sQ}{{\mathscr{Q}}}   \newcommand{\sR}{{\mathscr{R}}}
\newcommand{\sS}{{\mathscr{S}}}   \newcommand{\sT}{{\mathscr{T}}}
\newcommand{\sU}{{\mathscr{U}}}   \newcommand{\sV}{{\mathscr{V}}}
\newcommand{\sW}{{\mathscr{W}}}   \newcommand{\sX}{{\mathscr{X}}}
\newcommand{\sY}{{\mathscr{Y}}}   \newcommand{\sZ}{{\mathscr{Z}}}


%---------------------------------------
% Math Fraktur Font
%---------------------------------------

%Captital Letters
\newcommand{\mfA}{\mathfrak{A}}	\newcommand{\mfB}{\mathfrak{B}}
\newcommand{\mfC}{\mathfrak{C}}	\newcommand{\mfD}{\mathfrak{D}}
\newcommand{\mfE}{\mathfrak{E}}	\newcommand{\mfF}{\mathfrak{F}}
\newcommand{\mfG}{\mathfrak{G}}	\newcommand{\mfH}{\mathfrak{H}}
\newcommand{\mfI}{\mathfrak{I}}	\newcommand{\mfJ}{\mathfrak{J}}
\newcommand{\mfK}{\mathfrak{K}}	\newcommand{\mfL}{\mathfrak{L}}
\newcommand{\mfM}{\mathfrak{M}}	\newcommand{\mfN}{\mathfrak{N}}
\newcommand{\mfO}{\mathfrak{O}}	\newcommand{\mfP}{\mathfrak{P}}
\newcommand{\mfQ}{\mathfrak{Q}}	\newcommand{\mfR}{\mathfrak{R}}
\newcommand{\mfS}{\mathfrak{S}}	\newcommand{\mfT}{\mathfrak{T}}
\newcommand{\mfU}{\mathfrak{U}}	\newcommand{\mfV}{\mathfrak{V}}
\newcommand{\mfW}{\mathfrak{W}}	\newcommand{\mfX}{\mathfrak{X}}
\newcommand{\mfY}{\mathfrak{Y}}	\newcommand{\mfZ}{\mathfrak{Z}}
%Small Letters
\newcommand{\mfa}{\mathfrak{a}}	\newcommand{\mfb}{\mathfrak{b}}
\newcommand{\mfc}{\mathfrak{c}}	\newcommand{\mfd}{\mathfrak{d}}
\newcommand{\mfe}{\mathfrak{e}}	\newcommand{\mff}{\mathfrak{f}}
\newcommand{\mfg}{\mathfrak{g}}	\newcommand{\mfh}{\mathfrak{h}}
\newcommand{\mfi}{\mathfrak{i}}	\newcommand{\mfj}{\mathfrak{j}}
\newcommand{\mfk}{\mathfrak{k}}	\newcommand{\mfl}{\mathfrak{l}}
\newcommand{\mfm}{\mathfrak{m}}	\newcommand{\mfn}{\mathfrak{n}}
\newcommand{\mfo}{\mathfrak{o}}	\newcommand{\mfp}{\mathfrak{p}}
\newcommand{\mfq}{\mathfrak{q}}	\newcommand{\mfr}{\mathfrak{r}}
\newcommand{\mfs}{\mathfrak{s}}	\newcommand{\mft}{\mathfrak{t}}
\newcommand{\mfu}{\mathfrak{u}}	\newcommand{\mfv}{\mathfrak{v}}
\newcommand{\mfw}{\mathfrak{w}}	\newcommand{\mfx}{\mathfrak{x}}
\newcommand{\mfy}{\mathfrak{y}}	\newcommand{\mfz}{\mathfrak{z}}


\newcommand{\rlangle}[1]{\langle\alpha\left|#1\right|\alpha\rangle}
\newcommand{\inangle}[1]{\langle#1\rangle}

\title{\Huge{Mecanica Cuantica}\\Tarea 5}
\author{\huge{Sergio Montoya}\\ \huge{David Pachon}}
\date{08 de Abril de 2025}

\begin{document}

\maketitle
\newpage% or \cleardoublepage
% \pdfbookmark[<level>]{<title>}{<dest>}
\pdfbookmark[section]{\contentsname}{toc}
\tableofcontents
\pagebreak

\chapter{}

\chapter{}

\section{}

Para mostrar que esta normalizado sumamos cada coeficiente y mostramos que esto equivale a $1$

\begin{align*}
  \left| c_0 \right|^2 +
  \left| c_1 \right|^2 +
  \left| c_2 \right|^2 +
  \left| c_3 \right|^2 &= 1\\
  \left| \frac{\sqrt{2}}{4} \right|^2 +
  \left| \frac{2i}{4} \right|^2 +
  \left| - \frac{i}{4} \right|^2 +
  \left| \frac{3}{4}e^{i \frac{\pi}{3}} \right|^2 &= 1\\
  \frac{2}{16} +
  \frac{4}{16} +
  \frac{1}{16} +
  \left| \frac{3}{4}\right|^2 \left|e^{i \frac{\pi}{3}} \right|^2 &= 1\\
  \frac{2}{16} +
  \frac{4}{16} +
  \frac{1}{16} +
  \frac{9}{16}\left|\cos\left(\frac{\pi}{3}\right) + i\sin\left(\frac{\pi}{3}\right)\right|^2 &= 1\\
  \frac{2}{16} +
  \frac{4}{16} +
  \frac{1}{16} +
  \frac{9}{16}\left(\sqrt{\cos^2\left(\frac{\pi}{3}\right) + \sin^2\left(\frac{\pi}{3}\right)}\right)^2 &= 1\\
  \frac{2}{16} +
  \frac{4}{16} +
  \frac{1}{16} +
  \frac{9}{16}\left( 1 \right)^2 &= 1\\
  \frac{2}{16} +
  \frac{4}{16} +
  \frac{1}{16} +
  \frac{9}{16} &= 1\\
  \frac{2 + 4 + 1 + 9}{16} &= 1\\
  1 &= 1\\
\end{align*}

\section{}

Para encontrar la energia podemos usar la ecuación $4.2.27$ de las notas de clase en donde sabemos que los estados se pueden encontrar como:
$$
E_{n} = \left(n + \frac{1}{2}\right)\hbar \omega
$$

Por lo tanto las energias son:
\begin{align*}
  E_{n} &= \left(n + \frac{1}{2}\right)\hbar \omega\\
  E_{0} &= \left(0 + \frac{1}{2}\right)\hbar \omega\\
  &= \frac{1}{2}\hbar\omega\\
  E_{1} &= \left(1 + \frac{1}{2}\right)\hbar \omega\\
  &= \left(\frac{3}{2}\right)\hbar \omega\\
  E_{2} &= \left(2 + \frac{1}{2}\right)\hbar \omega\\
  &= \left(\frac{5}{2}\right)\hbar \omega\\
  E_{3} &= \left(3 + \frac{1}{2}\right)\hbar \omega\\
  &= \left(\frac{7}{2}\right)\hbar \omega\\
\end{align*}

Ahora bien, las probabilidades son:
\begin{align*}
  P_n &= \left|\left<n | \psi\right>\right|^2\\
  &= \left|c_n\right|^2
\end{align*}

Esto ya lo calculamos en la sección anterior por lo que sabemos que serian:
\begin{align*}
  P_0 &= \frac{2}{16}\\
  P_1 &= \frac{4}{16}\\
  P_2 &= \frac{1}{16}\\
  P_3 &= \frac{9}{16}
\end{align*}

\section{}

Para calcular
\[
  \langle E \rangle = \sum_{n = 0}^{3} P_n E_n
\]

Tomando los resultados de la sección anterior tenemos:
\begin{align*}
  \langle E \rangle &= P_0E_0 + P_1E_1 + P_2E_2 + P_3E_3\\
  &= \frac{2}{16}\left(\frac{1}{2}\hbar\omega\right) +
  \frac{4}{16}\left(\frac{3}{2}\hbar\omega\right) +
  \frac{1}{16}\left(\frac{5}{2}\hbar\omega\right) +
  \frac{9}{16}\left(\frac{7}{2}\hbar\omega\right)\\
  &= \left(\frac{2}{32}\hbar\omega\right) +
  \left(\frac{12}{32}\hbar\omega\right) +
  \left(\frac{5}{32}\hbar\omega\right) +
  \left(\frac{63}{32}\hbar\omega\right)\\
  &= \left(\frac{2 + 12 + 5 + 63}{32}\hbar\omega\right)\\
  &= \left(\frac{82}{32}\hbar\omega\right)\\
  &= \left(\frac{41}{16}\hbar\omega\right)
\end{align*}

\chapter{}

\chapter{}

\section{}

Para solucionar esto partimos desde:

\[
  x = \sqrt{\frac{\hbar}{2m\omega}} \paren{a_{-} + a_{+}};\ p = i \sqrt{\frac{m\omega\hbar}{2}}\paren{a_{+} - a_{-}}
\]

Ahora bien, tomemos que:
\begin{align*}
  \langle \alpha | a_{-} | \alpha \rangle &= \langle \alpha | \alpha | \alpha \rangle \\ &= \alpha \langle \alpha  | \alpha \rangle \\ &= \alpha\\
  \langle \alpha | a_{+} | \alpha \rangle &= \langle \alpha | \alpha^* | \alpha \rangle \\ &= \alpha^* \langle \alpha  | \alpha \rangle \\ &= \alpha^*\\
\end{align*}

Por lo tanto
\begin{align*}
  x &= \sqrt{\frac{\hbar}{2m\omega}} \paren{a_{-} + a_{+}}\\
  \langle x \rangle &= \sqrt{\frac{\hbar}{2m\omega}} \langle \alpha |\paren{a_{-} + a_{+}}| \alpha \rangle\\
  &= \sqrt{\frac{\hbar}{2m\omega}} \paren{\alpha + \alpha^*}\\
  &= \sqrt{\frac{\hbar}{2m\omega}} 2\Re\paren{\alpha}\\
  &= \sqrt{\frac{4\hbar}{2m\omega}} \Re\paren{\alpha}\\
  &= \sqrt{\frac{2\hbar}{m\omega}} \Re\paren{\alpha}
\end{align*}

Para $\langle p \rangle$
\begin{align*}
  p &= i \sqrt{\frac{m\omega\hbar}{2}}\paren{a_{+} - a_{-}}\\
  \langle p \rangle &= i \sqrt{\frac{m\omega\hbar}{2}}\langle \alpha | \paren{a_{+} - a_{-}} | \alpha \rangle\\
  &= i \sqrt{\frac{m\omega\hbar}{2}}\paren{\alpha^* - \alpha} \\
  &= i \sqrt{\frac{m\omega\hbar}{2}}\paren{-2i\Im\paren{\alpha}}\\
  &= \sqrt{\frac{4 m\omega\hbar}{2}}\Im\paren{\alpha}\\
  &= \sqrt{2 m\omega\hbar}\Im\paren{\alpha}
\end{align*}

Ahora con los casos de $\langle x^2 \rangle$ y $\langle p^2 \rangle$

Primero miremos lo siguiente:
\begin{align*}
  x^2 &= \frac{\hbar}{2m\omega} \paren{a_{-} + a_{+}}^2\\
  &= \frac{\hbar}{2m\omega} \paren{a_{-}^2 + a_{+}^2 + a_- a_+ + a_+a_-}\\
  p^2 &= - \frac{m\omega\hbar}{2}\paren{a_{+} - a_{-}}^2\\
  &= - \frac{m\omega\hbar}{2}\paren{a_{+}^2 + a_{-}^2 - a_+a_- - a_-a_+}
\end{align*}

Por lo tanto vamos a necesitar:
\begin{align*}
  \langle \alpha | a_-^2 | \alpha \rangle &= \alpha\langle \alpha | a_- | \alpha \rangle\\
  &= \alpha^2\\
  \langle \alpha | a_+^2 | \alpha \rangle &= \alpha^*\langle \alpha | a_- | \alpha \rangle\\
  &= \paren{\alpha^*}^2\\
  \langle \alpha | a_+a_- | \alpha \rangle &= \alpha \langle \alpha | a_+ | \alpha \rangle\\
  &= \alpha\alpha^* \langle \alpha | \alpha \rangle\\
  &= \left|\alpha\right|^2\\
  \langle \alpha | a_-a_+ | \alpha \rangle &= \langle \alpha | a_+a_- + 1 | \alpha \rangle\\
  &= \alpha \langle \alpha | a_+ | \alpha \rangle + \langle \alpha | 1 | \alpha \rangle\\
  &= \alpha\alpha^* \langle \alpha | \alpha \rangle + 1 \langle \alpha | \alpha \rangle\\
  &= \left|\alpha\right|^2 + 1
\end{align*}

Ya con esto podemos pasar a calcular
\begin{enumerate}
  \item $\langle x^2 \rangle$
    \begin{align*}
      x^2 &= \frac{\hbar}{2m\omega} \paren{a_{-}^2 + a_{+}^2 + a_- a_+ + a_+a_-}\\
      \langle x^2 \rangle &= \frac{\hbar}{2m\omega} \langle \alpha |\paren{a_{-}^2 + a_{+}^2 + a_- a_+ + a_+a_-}| \alpha \rangle\\
      &= \frac{\hbar}{2m\omega} \paren{\rlangle{a_{-}^2} + \rlangle{a_{+}^2} + \rlangle{a_- a_+} + \rlangle{a_+a_-}}\\
      &= \frac{\hbar}{2m\omega} \paren{\alpha^2 + \paren{\alpha^*}^2 + \left|\alpha\right|^2 + \left|\alpha\right|^2 + 1}\\
      &= \frac{\hbar}{2m\omega} \paren{\alpha^2 + \paren{\alpha^*}^2 + 2\left|\alpha\right|^2+ 1}\\
      &= \frac{\hbar}{2m\omega} \paren{2\Re \paren{\alpha}^2 - 2\Im \paren{\alpha}^2 + 2\Re\paren{\alpha}^2 + 2\Im\paren{\alpha}^2 + 1}\\
      &= \frac{\hbar}{2m\omega} \paren{4\Re \paren{\alpha}^2 + 1}\\
      &= \frac{2\hbar}{m\omega}\Re \paren{\alpha}^2 + \frac{\hbar}{2m\omega}
    \end{align*}

  \item $\langle p^2 \rangle$
    \begin{align*}
      p^2 &= - \frac{m\omega\hbar}{2}\paren{a_{+}^2 + a_{-}^2 - a_+a_- - a_-a_+}\\
      \langle p^2 \rangle &= - \frac{m\omega\hbar}{2}\rlangle{a_{+}^2 + a_{-}^2 - a_+a_- - a_-a_+}\\
      &= - \frac{m\omega\hbar}{2}\paren{\rlangle{a_{+}^2} + \rlangle{a_{-}^2} - \rlangle{a_+a_-} - \rlangle{a_-a_+}}\\
      &= - \frac{m\omega\hbar}{2}\paren{\alpha^2 + \paren{\alpha^*}^2 - \left|\alpha\right|^2 - \paren{\left|\alpha\right|^2 + 1}}\\
      &= - \frac{m\omega\hbar}{2}\paren{2\Re\paren{\alpha}^2 - 2\Im\paren{\alpha}^2 - \left|\alpha\right|^2 - \left|\alpha\right|^2 - 1}\\
      &= - \frac{m\omega\hbar}{2}\paren{2\Re\paren{\alpha}^2 - 2\Im\paren{\alpha}^2 - 2\Re\paren{\alpha}^2 - 2\Im\paren{\alpha}^2 - 1}\\
      &= - \frac{m\omega\hbar}{2}\paren{- 4\Im\paren{\alpha}^2 - 1}\\
      &= \frac{m\omega\hbar}{2}\paren{4\Im\paren{\alpha}^2 + 1}\\
      &= \frac{m\omega\hbar}{2}4\Im\paren{\alpha}^2 + \frac{m\omega\hbar}{2}\\
      &= 2m\omega\hbar\Im\paren{\alpha}^2 + \frac{m\omega\hbar}{2}
    \end{align*}
\end{enumerate}

Por lo tanto los resultados son:
\begin{enumerate}
  \item $\langle x \rangle = \sqrt{\frac{2\hbar}{m\omega}} \Re\paren{\alpha}$
  \item $\langle p \rangle = \sqrt{2 m\omega}\Im\paren{\alpha}$
  \item $\langle x^2 \rangle = \frac{2\hbar}{m\omega}\Re \paren{\alpha}^2 + \frac{\hbar}{2m\omega}$
  \item $\langle p^2 \rangle = 2m\omega\hbar\Im\paren{\alpha}^2 + \frac{m\omega\hbar}{2}$
\end{enumerate}

\pagebreak

\section{}

En este caso tenemos:
\[
  \sigma_x = \sqrt{\inangle{x^2} - \inangle{x}^2}
\]

Por lo tanto veamos cuanto es $\inangle{x}^2$

\begin{align*}
  \inangle{x}^2 &= \paren{\sqrt{\frac{2\hbar}{m\omega}} \Re\paren{\alpha}}^2\\
  \inangle{x}^2 &= \frac{2\hbar}{m\omega} \Re\paren{\alpha}^2
\end{align*}

Con esto entonces
\begin{align*}
  \sigma_x &= \sqrt{\inangle{x^2} - \inangle{x}^2}\\
  &= \sqrt{\frac{2\hbar}{m\omega}\Re \paren{\alpha}^2 + \frac{\hbar}{2m\omega} - \frac{2\hbar}{m\omega} \Re\paren{\alpha}^2}\\
  &= \sqrt{\frac{\hbar}{2m\omega}}
\end{align*}

Por el otro lado
\[
  \sigma_p = \sqrt{\inangle{p^2} - \inangle{p}^2}
\]

con
\begin{align*}
  \inangle{p}^2 &= \paren{\sqrt{2 m\omega\hbar}\Im\paren{\alpha}}^2\\
   &= \paren{\sqrt{2 m\omega\hbar}\Im\paren{\alpha}}^2\\
   &= 2 m\omega\hbar\Im\paren{\alpha}^2\\
\end{align*}

De nuevo calculemos
\begin{align*}
  \sigma_p &= \sqrt{\inangle{p^2} - \inangle{p}^2}\\
  &= \sqrt{2m\omega\hbar\Im\paren{\alpha}^2 + \frac{m\omega\hbar}{2} - 2 m\omega\hbar\Im\paren{\alpha}^2}\\
  &= \sqrt{\frac{m\omega\hbar}{2}}
\end{align*}

Ahora al final:
\begin{align*}
  \sigma_x\sigma_p &= \sqrt{\frac{\hbar}{2m\omega}} \cdot \sqrt{\frac{m\omega\hbar}{2}}\\
  \sigma_x\sigma_p &= \sqrt{\frac{\hbar^2 m\omega}{2\cdot2m\omega}}\\
  \sigma_x\sigma_p &= \sqrt{\frac{\hbar^2}{4}}\\
  \sigma_x\sigma_p &= \frac{\hbar}{2}
\end{align*}

\pagebreak

\section{}

Apliquemos $a_-$ de la siguiente manera
\begin{align*}
  a_-\left| \alpha \right> &= \alpha \sum_{n} c_n \left| n \right>\\
  &= \sum_{n} c_n \sqrt{n} \left| n - 1 \right>\\
\end{align*}

Dado que son esencialmente los mismos podemos hacer
\begin{align*}
  \alpha \sum_{n} c_n \left| n \right> &= \sum_{n} c_n \sqrt{n} \left| n - 1 \right>\\
  \sum_{n} \alpha c_n \left| n \right> &= \sum_{n} c_n \sqrt{n} \left| n - 1 \right>\\
  \sum_{n} \alpha c_n \left| n \right> &= \sum_{n} c_{n + 1} \sqrt{n - 1} \left| n \right>\\
  \alpha c_n &= c_{n + 1} \sqrt{n - 1}\\
  \frac{\alpha}{\sqrt{n - 1}} c_n &= c_{n + 1}
\end{align*}

Dada esta definición recursiva podemos reducirla hasta
\[
  \frac{\alpha^n}{\sqrt{n!}} c_0 = c_{n}
\]

\section{}

Tenemos:
\begin{align*}
  \inangle{\alpha | \alpha} &= \sum_{n = 0}^{\infty} \left|c_n\right|^2 = 1\\
  \sum_{n = 0}^{\infty} \left|\frac{\alpha^n}{\sqrt{n!}}\right|^2 &= \left|c_0\right|^2 \sum_{n = 0}^{\infty} \frac{\left|\alpha\right|^{2n}}{n!}\\
  \sum_{n = 0}^{\infty} \left|\frac{\alpha^n}{\sqrt{n!}}\right|^2 &= \left|c_0\right|^2 \sum_{n = 0}^{\infty} \frac{\paren{\left|\alpha\right|^{2}}^n}{n!}
\end{align*}

Esta es una serie exponencial conocida:
\[
  \sum_{k = 0}^\infty \frac{z^k}{k!} = e^z
\]

Por lo tanto

\begin{align*}
  \left|c_0\right|^2 \sum_{n = 0}^{\infty} \frac{\left|\alpha\right|^{2n}}{n!} &= \left|c_0\right|^2 e^{\left|\alpha\right|^2}\\
  \left|c_0\right|^2 e^{\left|\alpha\right|^2} &= 1\\
  \left|c_0\right|^2 &= e^{-\left|\alpha\right|^2}\\
  c_0 &= e^{-\left|\alpha\right|^2/2}
\end{align*}

\pagebreak

\section{}

En los punto anterior definimos que:
\begin{equation*}
  \left| \alpha \right> = \sum_{n} c_n \left| n \right> = \sum_{n} \frac{\alpha^n}{\sqrt{n!}} c_0 \left| n \right> = \sum_{n} \frac{\alpha^n}{\sqrt{n!}} e^{-\left|\alpha\right|^2/2} \left| n \right> = e^{-\left|\alpha\right|^2/2} \sum_{n} \frac{\alpha^n}{\sqrt{n!}}  \left| n \right>
\end{equation*}

Con esto entonces, podemos pasar a $\left| \alpha (t) \right>$ de la manera en la que nos dicen esto queda como
\begin{equation*}
  \left| \alpha(t) \right> = e^{-\left|\alpha\right|^2/2} \sum_{n} \frac{\alpha^n}{\sqrt{n!}}  e^{-i E_n t/ \hbar}\left| n \right>
\end{equation*}

Ahora apliquemos $a_-$

\begin{align*}
  a_{-}\left|\alpha(t)\right> &=e^{-\left|\alpha\right|^2/2} \sum_{n} \frac{\alpha^n}{\sqrt{n!}}  e^{-i E_n t/ \hbar}\sqrt{n}\left| n - 1\right> \\
  a_{-}\left|\alpha(t)\right> &=e^{-\left|\alpha\right|^2/2} \sum_{n + 1} \frac{\alpha^{n + 1}}{\sqrt{n + 1!}} \sqrt{n + 1} e^{-i E_{n + 1} t/ \hbar}\left| n\right> \\
  a_{-}\left|\alpha(t)\right> &=e^{-\left|\alpha\right|^2/2} \sum_{n} \frac{\alpha^{n}}{\sqrt{n!}}\alpha e^{-i E_{n + 1} t/ \hbar}\left| n\right> \\
  E_n &= \hbar\omega\paren{n + \frac{1}{2}}\\
  \implies E_{n + 1} &= \hbar\omega\paren{n + \frac{3}{2}}\\
  \implies e^{-iE_{n + 1}t/\hbar} &= e^{-i\omega t}e^{-i E_n t/\hbar}\\
  a_{-}\left|\alpha(t)\right> &=e^{-\left|\alpha\right|^2/2} \alpha \sum_{n} \frac{\alpha^{n}}{\sqrt{n!}} e^{-i\omega t}e^{-i E_n t/\hbar} \left| n\right> \\
  a_{-}\left|\alpha(t)\right> &=e^{-\left|\alpha\right|^2/2} \alpha e^{-i\omega t}\sum_{n} \frac{\alpha^{n}}{\sqrt{n!}} e^{-i E_n t/\hbar} \left| n\right> \\
  \alpha e^{-i\omega t} &= \alpha(t)\\
  a_{-}\left|\alpha(t)\right> &=\alpha(t) e^{-\left|\alpha\right|^2/2} \sum_{n} \frac{\alpha^{n}}{\sqrt{n!}} e^{-i E_n t/\hbar} \left| n\right> \\
  a_{-}\left|\alpha(t)\right> &=\alpha(t) \left|\alpha(t)\right>
\end{align*}

Esto en esencia quiere decir que este operador oscila de manera coherente con un oscilador clasico de periodo $\omega$. Cosa que para ser honestos tiene sentido pues estamos minimizando la incertidumbre.

\pagebreak

\section{}

Verifiquemos cada caso:

\begin{align*}
  a_{-}\left| 0 \right> &= 0 \cdot \left| 0 \right>\\\
  \alpha &= 0\\
  \left| \alpha \right> &= e^{-\left|\alpha\right|^2/2} \sum_{n} \frac{\alpha^n}{\sqrt{n!}}  \left| n \right>\\
  \left| \alpha = 0 \right> &= e^{-\left| 0 \right|^2/2} \sum_{n} \frac{0^n}{\sqrt{n!}}  \left| n \right>\\
  \left| \alpha = 0 \right> &= \sum_{n} 0  \left| n \right>\\
  \left| \alpha = 0 \right> &= \left| 0 \right>
\end{align*}

Por ultimo
\begin{enumerate}
  \item $\langle x \rangle = \sqrt{\frac{2\hbar}{m\omega}} \Re\paren{0} = 0$
  \item $\langle p \rangle = \sqrt{2 m\omega\hbar}\Im\paren{0} = 0$
  \item $\langle x^2 \rangle = \frac{2\hbar}{m\omega}\Re \paren{0}^2 + \frac{\hbar}{2m\omega} = \frac{\hbar}{2m\omega}$
  \item $\langle p^2 \rangle = 2m\omega\hbar\Im\paren{0}^2 + \frac{m\omega\hbar}{2} = \frac{m\omega\hbar}{2}$
\end{enumerate}

Por lo tanto:
\begin{align*}
  \sigma_x &= \sqrt{\inangle{x^2} - \inangle{x}^2}\\
  &= \sqrt{\frac{\hbar}{2m\omega} - 0}\\
  &= \sqrt{\frac{\hbar}{2m\omega}}\\
  \sigma_p &= \sqrt{\inangle{p^2} - \inangle{p}^2}\\
  &= \sqrt{\frac{m\omega\hbar}{2} - 0}\\
  &= \sqrt{\frac{m\omega\hbar}{2}}\\
  \sigma_x\sigma_p &= \sqrt{\frac{\hbar}{2m\omega}} \cdot \sqrt{\frac{m\omega\hbar}{2}}\\
  \sigma_x\sigma_p &= \sqrt{\frac{\hbar^2}{4}}\\
  \sigma_x\sigma_p &= \frac{\hbar}{2}
\end{align*}

Por lo tanto si es un estado coherente con $\alpha = 0$

\end{document}

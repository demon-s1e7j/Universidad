\documentclass{report}

\documentclass[12pt]{article}
\usepackage{array}
\usepackage{color}
\usepackage{amsthm}
\usepackage{eufrak}
\usepackage{lipsum}
\usepackage{pifont}
\usepackage{yfonts}
\usepackage{amsmath}
\usepackage{amssymb}
\usepackage{ccfonts}
\usepackage{comment} \usepackage{amsfonts}
\usepackage{fancyhdr}
\usepackage{graphicx}
\usepackage{listings}
\usepackage{mathrsfs}
\usepackage{setspace}
\usepackage{textcomp}
\usepackage{blindtext}
\usepackage{enumerate}
\usepackage{microtype}
\usepackage{xfakebold}
\usepackage{kantlipsum}
%\usepackage{draftwatermark}
\usepackage[spanish]{babel}
\usepackage[margin=1.5cm, top=2cm, bottom=2cm]{geometry}
\usepackage[framemethod=tikz]{mdframed}
\usepackage[colorlinks=true,citecolor=blue,linkcolor=red,urlcolor=magenta]{hyperref}

%//////////////////////////////////////////////////////
% Watermark configuration
%//////////////////////////////////////////////////////
%\SetWatermarkScale{4}
%\SetWatermarkColor{black}
%\SetWatermarkLightness{0.95}
%\SetWatermarkText{\texttt{Watermark}}

%//////////////////////////////////////////////////////
% Frame configuration
%//////////////////////////////////////////////////////
\newmdenv[tikzsetting={draw=gray,fill=white,fill opacity=0},backgroundcolor=none]{Frame}

%//////////////////////////////////////////////////////
% Font style configuration
%//////////////////////////////////////////////////////
\renewcommand{\familydefault}{\ttdefault}
\renewcommand{\rmdefault}{tt}

%//////////////////////////////////////////////////////
% Bold configuration
%//////////////////////////////////////////////////////
\newcommand{\fbseries}{\unskip\setBold\aftergroup\unsetBold\aftergroup\ignorespaces}
\makeatletter
\newcommand{\setBoldness}[1]{\def\fake@bold{#1}}
\makeatother

%//////////////////////////////////////////////////////
% Default font configuration
%//////////////////////////////////////////////////////
\DeclareFontFamily{\encodingdefault}{\ttdefault}{%
  \hyphenchar\font=\defaulthyphenchar
  \fontdimen2\font=0.33333em
  \fontdimen3\font=0.16667em
  \fontdimen4\font=0.11111em
  \fontdimen7\font=0.11111em}


\input{macros}
\input{letterfonts}

\title{\Huge{Mecanica Cuantica}\\Tarea 4}
\author{\huge{Sergio Montoya}\\\huge{David Pachon}}
\date{}

\begin{document}

\maketitle
\newpage% or \cleardoublepage
% \pdfbookmark[<level>]{<title>}{<dest>}
\pdfbookmark[section]{\contentsname}{toc}
\tableofcontents
\pagebreak

\chapter{}

\section{}

Para iniciar este punto simplemente podemos solucionar:

\begin{align*}
	\hat{A}^2 - 3\hat{A} + 2 &= 0 \\
	\hat{A} &= \frac{-b \pm \sqrt{b^2 - 4ac}}{2a}\\
	&= \frac{3 \pm \sqrt{3^2 - 4(1)(2)}}{2(1)} \\
	&= \frac{3 \pm \sqrt{9 - 8}}{2}\\
	&= \frac{3 \pm \sqrt{1}}{2}\\
	&= \frac{3 \pm 1}{2}\\
	A_+ &= \frac{3 + 1}{2} = \frac{4}{2} = 2\\
	A_- &= \frac{3 - 1}{2} = \frac{2}{2} = 1
\end{align*}

Dado que $\lambda_1 \neq \lambda_2$ entonces $\hat{A}$ es diagonalizable en su base canónica. Esto quiere decir que podemos escribir $A$ como:

\[
	A = 
	\begin{bmatrix}
		2 & 0 \\
		0 & 1
	\end{bmatrix}
\]

Lo que nos sera de mucha utilidad para los siguientes puntos.

\section{}

En este punto podemos utilizar que ya encontramos los eigenvalues de la matriz $A$. Lo que seria:
\begin{align*}
	\hat{A}\left| \psi_1 \right> &= \lambda_1 \psi_1\\
	\hat{A}\left| \psi_1 \right> &= 1 \psi_1\\
	\begin{bmatrix}
		2 & 0 \\
		0 & 1
	\end{bmatrix}
	\begin{bmatrix}
		\psi_1 \\
		\psi_2
	\end{bmatrix}
	&= \begin{bmatrix}
		\psi_1 \\
		\psi_2
	\end{bmatrix}\\
	&= \begin{bmatrix}
		2 \psi_1 + 0 \psi_2 \\
		0 \psi_1 + 1 \psi_2 
	\end{bmatrix}\\
	\begin{bmatrix}
		2 \psi_1 \\
		\psi_2 
	\end{bmatrix} &= \begin{bmatrix}
		\psi_1 \\
		\psi_2
	\end{bmatrix}\\
	\implies \left| \psi_1 \right> &= \begin{bmatrix} 0 \\ 1 \end{bmatrix}\\
\end{align*}

Para el segundo eigenvector:

\begin{align*}
	\hat{A}\left| \psi_2 \right> &= \lambda_2 \psi_2\\
	\hat{A}\left| \psi_2 \right> &=  2 \psi_2\\
	\begin{bmatrix}
		2 & 0 \\
		0 & 1
	\end{bmatrix}
	\begin{bmatrix}
		\psi_1 \\
		\psi_2
	\end{bmatrix}
	&= 2\begin{bmatrix}
		\psi_1 \\
		\psi_2
	\end{bmatrix}\\
	\begin{bmatrix}
		2 \psi_1 \\
		\psi_2
	\end{bmatrix}
	&= \begin{bmatrix}
		2\psi_1 \\
		2\psi_2
	\end{bmatrix}\\
	\implies \left| \psi_2 \right> &= \begin{bmatrix} 1 \\ 0 \end{bmatrix}\\
\end{align*}

Aunque para ser completamente honesto, este resultado es trivial, puesto que la matriz $A$ con la que estamos trabajando fue construida en la base de sus vectores propios diagonalizando sus eigenvalues.


\section{}

Lo unico que nos falta para mostrar que $A$ es un observable (dado que ya mostramos que es diagonalizable) es que $A$ es hermitica. Lo cual significa que tenemos que mostrar
\begin{align}
	\hat{A}^\dagger &= \bar{\hat{A}}^*\\
	&= \begin{bmatrix}
		2 & 0 \\
		0 & 1
	\end{bmatrix}^*\\
	&= \begin{bmatrix}
		2 & 0 \\
		0 & 1
	\end{bmatrix}\\
	\hat{A} &= \begin{bmatrix}
		2 & 0 \\
		0 & 1
	\end{bmatrix}\\
	\implies \hat{A}^\dagger &= \hat{A}
\end{align}

Con esto ya tenemos que $A$ es hermitica. Ademas, como sabemos que $A$ tiene valores propios reales. Esto demuestra que $A$ es observable.

\chapter{}

En este caso iniciamos por definir $\hat{v}$ como:
\[
	\hat{v_i} = \frac{1}{m}\left( \hat{p_i} - qA_i \right)
\]

Ahora, podemos calcular el conmutador $\left[ v_i, v_j \right]$ como:
\begin{align*}
	\left[ v_i, v_j \right] &= \left[ \frac{1}{m}\left( \hat{p_i} - qA_i \right), \frac{1}{m}\left( \hat{p_j} - qA_j \right) \right]\\
	3.\quad \left[ \alpha \hat{\Omega}, \hat{\Lambda} \right] &= \alpha \left[ \hat{\Omega}, \hat{\Lambda} \right], \\
	\left[ v_i, v_j \right] &= \frac{1}{m^2}\left[ \hat{p_i} - qA_i, \hat{p_j} - qA_j \right]\\
	2.\quad \left[ \hat{\Omega}, \hat{\Lambda} + \hat{\Sigma} \right] &= \left[ \hat{\Omega}, \hat{\Lambda} \right] + \left[ \hat{\Omega}, \hat{\Sigma} \right], \\
	\left[ v_i, v_j \right] &= \frac{1}{m^2}\left(\left[ \hat{p_i}, \hat{p_j}\right] - q\left[\hat{p_i}, A_j\right] - q\left[A_i, \hat{p_j}\right] + q^2 \left[A_i, A_j \right]\right)\\
	\left[ \hat{p_i}, \hat{p_j}\right] &= 0\\
	\left[ A_i, A_j \right] &= 0\\
	\left[ v_i, v_j \right] &= \frac{1}{m^2}\left( - q\left[\hat{p_i}, A_j\right] - q\left[A_i, \hat{p_j}\right] \right)\\
	\left[ v_i, v_j \right] &= \frac{- q}{m^2}\left( \left[\hat{p_i}, A_j\right] + \left[A_i, \hat{p_j}\right] \right)\\
	\left[\hat{p_i}, A_j\right] &= \hat{p_i}A_j - A_j\hat{p_i}\\
	\left[\hat{p_i}, A_j\right] \psi &= -i\hbar\frac{\partial A_j}{\partial x_i}\psi - i\hbar A_j\frac{\partial \psi}{\partial x_i} + A_j i\hbar\frac{\partial \psi}{\partial x_j}\\
	\left[\hat{p_i}, A_j\right] \psi &= -i\hbar\frac{\partial A_j}{\partial x_i}\psi \\
	\left[\hat{p_i}, A_j\right] &= -i\hbar\frac{\partial A_j}{\partial x_i} \\
	\left[A_i, \hat{p_j}\right] &= - \left[ \hat{p_j}, A_i \right]\\
	&= i\hbar\frac{\partial A_i}{\partial x_j} \\
	\left[ v_i, v_j \right] &= \frac{- q}{m^2}\left( -i\hbar\frac{\partial A_j}{\partial x_i} + i\hbar\frac{\partial A_i}{\partial x_j} \right)\\
	\left[ v_i, v_j \right] &= \frac{i\hbar q}{m^2}\left( \frac{\partial A_j}{\partial x_i} - \frac{\partial A_i}{\partial x_j} \right)\\
	\left[ v_i, v_j \right] &= \frac{i\hbar q}{m^2}\epsilon_{ijk}\left( \nabla \times A  \right)_k
\end{align*}

\end{document}

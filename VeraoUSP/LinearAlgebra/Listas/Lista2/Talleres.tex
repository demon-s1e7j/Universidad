\documentclass[5pt]{exam}
\usepackage{amsthm}
\usepackage{libertine}
\usepackage[utf8]{inputenc}
\usepackage[margin=1in]{geometry}
\usepackage{amsmath,amssymb}
\usepackage{multicol}
\usepackage[shortlabels]{enumitem}
\usepackage{siunitx}
\usepackage{cancel}
\usepackage{graphicx}
\usepackage{pgfplots}
\usepackage{listings}
\usepackage{tikz}
\usepackage{mathrsfs}


\pgfplotsset{width=10cm,compat=1.9}
\usepgfplotslibrary{external}
\tikzexternalize

\newcommand{\class}{Linear Algebra} % This is the name of the course 
\newcommand{\examnum}{Lista 1} % This is the name of the assignment
\newcommand{\examdate}{\today} % This is the due date
\newcommand{\timelimit}{}





\begin{document}
\pagestyle{plain}
\thispagestyle{empty}

\noindent
\begin{tabular*}{\textwidth}{l @{\extracolsep{\fill}} r @{\extracolsep{6pt}} l}
  \textbf{\class} & \textbf{Name:} & \textit{Sergio Montoya}\\ %Your name here instead, obviously 
  \textbf{\examnum} &&\\
  \textbf{\examdate} &&
\end{tabular*}\\
\rule[2ex]{\textwidth}{2pt}
% ---

\section{P.4}
\subsection{b}
Let $W$ be a subset of $\mathbb{F}^3$, with $\mathbb{F}$ being $\mathbb{R}$ or $\mathbb{C}$, caracterized by
\begin{equation}
  W = \{(x, y, z) \in \mathbb{F}^3 | x \in \mathbb{Z}\}
\end{equation}

then $W$ is not a sub-space of $\mathbb{F}^3$ because there exists a vector $v = (1, 0, 0)$ and a number (i.e. $\pi$) such that $v \in W$ but $\pi v = (\pi, 0, 0) \notin W$

\subsection{c}

considering that the polinom 0 has $grau(0) < 2$ then the subset
\begin{equation}
  W = \{p(t) \in P_n(\mathbb{F}) | grau(p) \ge 2\}
\end{equation}

cannot be a subspace of $P(\mathbb{F})$

\section{P.5}
\subsection{a}

Let $W$ be a subset of $M(n, \mathbb{F})$, with $\mathbb{F}$ being $\mathbb{R}$ or $\mathbb{C}$, caracterized by
\begin{equation}
  W = \left\{A \in M(n, \mathbb{F})| tr(A) = \sum_{i = 1}^n a_{ii} = 0\right\}
\end{equation}.

Then we can assert
\begin{equation}
  \forall A_1, A_2 \in W, \forall \alpha \in \mathbb{F} : tr(\alpha A_1) = 0 \land tr(A_1 + A_2) = 0
\end{equation}.

Let $A_1, A_2 \in W$ and $\alpha \in \mathbb{F}$. Then
\begin{align*}
  tr(\alpha A_1) = \sum_{i = 1}^n \alpha a_{ii} = \alpha \sum_{i = 1}^n a_{ii} = \alpha tr(A) = \alpha 0 = 0
\end{align*}

And
\begin{align*}
  tr(A_1 + A_2) = \sum_{i = 1}^n a^1_{ii} + a^2_{ii} = \left(\sum_{i = 1}^n a^1_{ii}\right) +  \left(\sum_{i = 1} a^2_{ii}\right) = tr(A_1) + tr(A_2) = 0 + 0 = 0
\end{align*}

therefore, $W$ is a subspace of $M(n, \mathbb{F})$

\subsection{b}

Let $W$ be a subset of $M(n, \mathbb{F})$, with $\mathbb{F}$ being $\mathbb{R}$ or $\mathbb{C}$, caracterized by
\begin{equation}
  W = \left\{A \in M(n, \mathbb{F})| \det(A) = 0\right\}
\end{equation}.

We can proof that $W$ is not a subspace of $M(n, \mathbb{F})$ by providing two matrixes $M_1, M_2$ that have $\det(M_i) = 0$ but $M_1 + M_2 \neq 0$. Let's take an $n > 1$ (because $n \le 1$ would have this subspace just because the only posible matrix with $\det(M) = 0$ is $(0)$) and take the matrix with only ones in the diagonal except for $a_{nn}$ and the matrix with all $0$ except in the element $a_nn$. You can see that
\begin{align*}
  \begin{pmatrix}
    1 & \ldots & 0\\
    \vdots & \ddots & \vdots\\
    0 & \ldots & 0
  \end{pmatrix} +
  \begin{pmatrix}
    0 & \ldots & 0\\
    \vdots & \ddots & \vdots\\
    0 & \ldots & 1
  \end{pmatrix} = 
  \begin{pmatrix}
    1 & \ldots & 0\\
    \vdots & \ddots & \vdots\\
    0 & \ldots & 1
  \end{pmatrix}
\end{align*}

Now you get the identity matrix, and therefore $\det(I) \neq 0$ so it can not be a subspace.

\section{P.9}

\subsection{a}

\begin{align*}
  &\implies \begin{pmatrix}
  1 & 1 & 1 & 1 & -1 & \vline & 0\\
  1 & -1 & -1 & 2 & -1 & \vline & 0
  \end{pmatrix}\\
  (-1)(I) + (II) &\implies \begin{pmatrix}
  1 & 1 & 1 & 1 & -1 & \vline & 0\\
  0 & -2 & -2 & 1 & 0 & \vline & 0
  \end{pmatrix}\\
  (-\frac{1}{2})(II) &\implies \begin{pmatrix}
  1 & 1 & 1 & 1 & -1 & \vline & 0\\
  0 & 1 & 1 & -\frac{1}{2} & 0 & \vline & 0
  \end{pmatrix}\\
  (-1)(II) + (I) &\implies \begin{pmatrix}
  1 & 0 & 0 & \frac{3}{2} & -1 & \vline & 0\\
  0 & 1 & 1 & -\frac{1}{2} & 0 & \vline & 0
  \end{pmatrix}\\
  &\begin{cases}
    x + \frac{3}{2}w - z = 0\\
    y + z - \frac{w}{2} = 0
  \end{cases}\\
  &\begin{cases}
    x = - \frac{3}{2}w + z\\
    y = - z + \frac{w}{2}
  \end{cases}\\
  &\left\{\left(- \frac{3}{2}w + z, -z + \frac{w}{2}, z, w, 0\right)\vline w, z \in \mathbb{F}\right\}\\
  &\left[\left(-\frac{3}{2}, \frac{1}{2}, 0, 1, 0\right), \left(1, -1, 1, 0, 0\right)\right]\\
  &\dim\left(\left[\left(-\frac{3}{2}, \frac{1}{2}, 0, 1, 0\right), \left(1, -1, 1, 0, 0\right)\right]\right) = 2
\end{align*}

\subsection{b}

\begin{align*}
  &\implies \begin{pmatrix}
    4 & 3 & - 1 & 1 & \vline & 0\\
    1 & -1 & 2 & -1 & \vline & 0
  \end{pmatrix}\\
  &\implies \begin{pmatrix}
    1 & -1 & 2 & -1 & \vline & 0\\
    4 & 3 & - 1 & 1 & \vline & 0
  \end{pmatrix}\\
  &\implies \begin{pmatrix}
    1 & -1 & 2 & -1 & \vline & 0\\
    0 & 7 & -9 & 5 & \vline & 0
  \end{pmatrix}\\
  &\implies \begin{pmatrix}
    1 & -1 & 2 & -1 & \vline & 0\\
    0 & 1 & -\frac{9}{7} & \frac{5}{7} & \vline & 0
  \end{pmatrix}\\
  & \implies
  \begin{cases}
    x - y + 2 z - t = 0\\
    y - \frac{9}{7}z + \frac{5}{7} t = 0
  \end{cases}\\
  & \implies
  \begin{cases}
    x - y + 2 z - t = 0\\
    y = \frac{9}{7}z - \frac{5}{7} t
  \end{cases}\\
  & \implies
  \begin{cases}
    x - \frac{9}{7} z + \frac{5}{7} t + 2 z - t = 0\\
    y = \frac{9}{7}z - \frac{5}{7} t
  \end{cases}\\
  & \implies
  \begin{cases}
    x = - \frac{5}{7} z + \frac{2}{7} t\\
    y = \frac{9}{7}z - \frac{5}{7} t
  \end{cases}\\
  & \implies
  \left\{(-\frac{5}{7} z + \frac{2}{7} t, \frac{9}{7} z - \frac{5}{7} t, z, t) \vline z, t \in \mathbb{R}\right\}\\
  & \implies
  \left[\left(-\frac{5}{7}, \frac{9}{7}, 1, 0\right),  \left(\frac{2}{7}, -\frac{5}{7}, 0, 1\right)\right]\\
  & \implies
  \dim\left(\left[\left(-\frac{5}{7}, \frac{9}{7}, 1, 0\right),  \left(\frac{2}{7}, -\frac{5}{7}, 0, 1\right)\right]\right) = 2
\end{align*}

\section{P.12}

\begin{align*}
  S &\implies \begin{pmatrix}
    3 & 0 & 2 & 0 & \vline & 0\\
    -6 & 3 & -1 & -1 & \vline & 0\\
    0 & 1 & 1 & 1 & \vline & 0\\
    9 & -3 & 3 & 1 & \vline & 0\\
    -9 & 5 & -1 & 1 & \vline & 0
  \end{pmatrix}\\
  (\frac{1}{3})(I) &\implies \begin{pmatrix}
    1 & 0 & \frac{2}{3} & 0 & \vline & 0\\
    -6 & 3 & -1 & -1 & \vline & 0\\
    0 & 1 & 1 & 1 & \vline & 0\\
    9 & -3 & 3 & 1 & \vline & 0\\
    -9 & 5 & -1 & 1 & \vline & 0
  \end{pmatrix}\\
  ((6)(I) + (II)) &\implies \begin{pmatrix}
    1 & 0 & \frac{2}{3} & 0 & \vline & 0\\
    0 & 3 & 3 & -1 & \vline & 0\\
    0 & 1 & 1 & 1 & \vline & 0\\
    9 & -3 & 3 & 1 & \vline & 0\\
    -9 & 5 & -1 & 1 & \vline & 0
  \end{pmatrix}\\
\end{align*}
\begin{align*}
  ((-9)(I) + (IV)) &\implies \begin{pmatrix}
    1 & 0 & \frac{2}{3} & 0 & \vline & 0\\
    0 & 3 & 3 & -1 & \vline & 0\\
    0 & 1 & 1 & 1 & \vline & 0\\
    0 & -3 & -3 & 1 & \vline & 0\\
    -9 & 5 & -1 & 1 & \vline & 0
  \end{pmatrix}\\
  ((9)(I) + (V)) &\implies \begin{pmatrix}
    1 & 0 & \frac{2}{3} & 0 & \vline & 0\\
    0 & 3 & 3 & -1 & \vline & 0\\
    0 & 1 & 1 & 1 & \vline & 0\\
    0 & -3 & -3 & 1 & \vline & 0\\
    0 & 5 & 5 & 1 & \vline & 0
  \end{pmatrix}\\
  ((III) \leftrightarrow (II)) &\implies \begin{pmatrix}
    1 & 0 & \frac{2}{3} & 0 & \vline & 0\\
    0 & 1 & 1 & 1 & \vline & 0\\
    0 & 3 & 3 & -1 & \vline & 0\\
    0 & -3 & -3 & 1 & \vline & 0\\
    0 & 5 & 5 & 1 & \vline & 0
  \end{pmatrix}\\
  ((III) + (IV)) &\implies \begin{pmatrix}
    1 & 0 & \frac{2}{3} & 0 & \vline & 0\\
    0 & 1 & 1 & 1 & \vline & 0\\
    0 & 3 & 3 & -1 & \vline & 0\\
    0 & 0 & 0 & 0 & \vline & 0\\
    0 & 5 & 5 & 1 & \vline & 0
  \end{pmatrix}\\
  ((-3)(II) + (III)) &\implies \begin{pmatrix}
    1 & 0 & \frac{2}{3} & 0 & \vline & 0\\
    0 & 1 & 1 & 1 & \vline & 0\\
    0 & 0 & 0 & -4 & \vline & 0\\
    0 & 0 & 0 & 0 & \vline & 0\\
    0 & 5 & 5 & 1 & \vline & 0
  \end{pmatrix}\\
  ((-5)(II) + (V)) &\implies \begin{pmatrix}
    1 & 0 & \frac{2}{3} & 0 & \vline & 0\\
    0 & 1 & 1 & 1 & \vline & 0\\
    0 & 0 & 0 & -4 & \vline & 0\\
    0 & 0 & 0 & 0 & \vline & 0\\
    0 & 0 & 0 & -4 & \vline & 0
  \end{pmatrix}\\
  ((-1)(III) + (V)) &\implies \begin{pmatrix}
    1 & 0 & \frac{2}{3} & 0 & \vline & 0\\
    0 & 1 & 1 & 1 & \vline & 0\\
    0 & 0 & 0 & -4 & \vline & 0\\
    0 & 0 & 0 & 0 & \vline & 0\\
    0 & 0 & 0 & 0 & \vline & 0
  \end{pmatrix}\\
  ((-\frac{1}{4})(III) + (V)) &\implies \begin{pmatrix}
    1 & 0 & \frac{2}{3} & 0 & \vline & 0\\
    0 & 1 & 1 & 1 & \vline & 0\\
    0 & 0 & 0 & -4 & \vline & 0\\
    0 & 0 & 0 & 0 & \vline & 0\\
    0 & 0 & 0 & 0 & \vline & 0
  \end{pmatrix}\\
  ((-1)(III) + (II)) &\implies \begin{pmatrix}
    1 & 0 & \frac{2}{3} & 0 & \vline & 0\\
    0 & 1 & 1 & 0 & \vline & 0\\
    0 & 0 & 0 & 1 & \vline & 0\\
    0 & 0 & 0 & 0 & \vline & 0\\
    0 & 0 & 0 & 0 & \vline & 0
  \end{pmatrix}\\
\end{align*}

With this we demonstrate that this subspace has dimention 3.

\begin{align*}
  \begin{cases}
    x = 12
    y = 7
    \frac{2}{3} x + y = 15
    z = 7
  \end{cases}
\end{align*}

As you can see this equations area already solved and it actually works!

\section{P.14}

\subsection{a}

\begin{align*}
  W_1 \cap W_2 &= \left\{(x, y, z, t) \in \mathbb{R}^4 \vline x + y = 0; z - t = 0; x - y -z + t = 0\right\}\\
  W_1 \cap W_2 &\implies \begin{pmatrix}
    1 & 1 & 0 & 0 & \vline & 0\\
    0 & 0 & 1 & -1 & \vline & 0\\
    1 & -1 & -1 & 1 & \vline & 0
  \end{pmatrix}\\
  (-1)(I) + (III) &\implies \begin{pmatrix}
    1 & 1 & 0 & 0 & \vline & 0\\
    0 & 0 & 1 & -1 & \vline & 0\\
    0 & -2 & -1 & 1 & \vline & 0
  \end{pmatrix}\\
  (II) \leftrightarrow (III) &\implies \begin{pmatrix}
    1 & 1 & 0 & 0 & \vline & 0\\
    0 & -2 & -1 & 1 & \vline & 0\\
    0 & 0 & 1 & -1 & \vline & 0
  \end{pmatrix}\\
  (-\frac{1}{2})(II) &\implies \begin{pmatrix}
    1 & 1 & 0 & 0 & \vline & 0\\
    0 & 1 & \frac{1}{2} & -\frac{1}{2} & \vline & 0\\
    0 & 0 & 1 & -1 & \vline & 0
  \end{pmatrix}\\
  (-1)(II) + (I) &\implies \begin{pmatrix}
    1 & 0 & -\frac{1}{2} & \frac{1}{2} & \vline & 0\\
    0 & 1 & \frac{1}{2} & -\frac{1}{2} & \vline & 0\\
    0 & 0 & 1 & -1 & \vline & 0
  \end{pmatrix}\\
  (-\frac{1}{2})(III) + (II) &\implies \begin{pmatrix}
    1 & 0 & -\frac{1}{2} & \frac{1}{2} & \vline & 0\\
    0 & 1 & 0 & 0 & \vline & 0\\
    0 & 0 & 1 & -1 & \vline & 0
  \end{pmatrix}\\
  (\frac{1}{2})(III) + (I) &\implies \begin{pmatrix}
    1 & 0 & 0 & 0 & \vline & 0\\
    0 & 1 & 0 & 0 & \vline & 0\\
    0 & 0 & 1 & -1 & \vline & 0
  \end{pmatrix}\\
  &\implies \begin{cases}
    x = 0\\
    y = 0\\
    z = t
  \end{cases}\\
  &\implies \left[(0, 0, 1, 1)\right]
\end{align*}

Meaning that the dimention is $1$.

\subsection{b}

Let's start by getting $W_1$ and $W_2$
\begin{align*}
  W_1 &\implies \begin{cases}
    x = -y\\
    z = t
  \end{cases}\\
  W_1 &= \left[(1, -1, 0 ,0), (0, 0, 1, 1)\right]\\
  W_2 &\implies x = y + z - t\\
  W_2 &= \left[(1, 1, 0, 0), (1, 0, 1, 0), (-1, 0, 0, 1)\right]
\end{align*}

We can develop all the steps necessary whoever we can take a simpler approach knowing that we have the dimention of $W_1 \cap W_2$ and that is:

\begin{align*}
  \dim(W_1 + W_2) &= \dim(W_1) + \dim(W_2) - \dim(W_1 \cap W_2)\\
  &= 2 + 3 - 1\\
  &= 4
\end{align*}

and knwoing that both subspaces are in $\mathbb{R}^4$ we know that $W_1 + W_2 = \mathbb{R}^4$

\subsection{c}

It is not a direct sum, we know that by getting the dimention of $W_1 \cap W_2$. Given that $\dim(W_1 \cap W_2) \neq 0$ we know that it is not a direct sum.

\subsection{d}

As we discuss above it is, and that helps a lot by removing the need for calculating $W_1 + W_2$ as we know it have dimention $4$ and that made it mandatory to be $\mathbb{R}^4$ express in, probably, a different basis.

\section{P.21}

\subsection{a}

\begin{align*}
  (x + 1)^2 &= x^2 + x + 1\\
  &= 1 \cdot x^2 + 1 \cdot x + 1 \cdot 1
\end{align*}

this already shows the coordinates in $\mathscr{B}$ (Long live the canon basis!)

\subsection{b}

\subsubsection{$P_{\mathscr{C} \to \mathscr{B}}$}
\begin{align*}
  (x + 1)^2 = x^2 + x + 1 &\implies (1, 1, 1)\\
  x + 4 &\implies (4, 1, 0)\\
  3x &\implies (0, 3, 0)\\
  P_{\mathscr{C} \to \mathscr{B}} &= \begin{pmatrix}
    1 & 1 & 1\\
    4 & 1 & 0\\
    0 & 3 & 0
  \end{pmatrix}
\end{align*}

Check:
\begin{align*}
  a (x + 1)^2 + b (x + 4) + c (3x) &= a x^2 + a x + a + b x + 4b + 3cx\\
  &= a x^2 + (a + b + 3 c) x + (a + 4b)\\
  P_{\mathscr{C} \to \mathscr{B}} v &=
  \begin{pmatrix}
    1 & 1 & 1\\
    4 & 1 & 0\\
    0 & 3 & 0
  \end{pmatrix}
  \begin{pmatrix}
    a \\
    b \\
    c
  \end{pmatrix}_\mathscr{C}\\
  &= \begin{pmatrix}
    a + 4b\\
    a + b + 3c\\
    a
  \end{pmatrix}_\mathscr{B}\\
  &= a x^2 + (a + b + 3c) x + (a + 4b)\square
\end{align*}

\subsubsection{$P_{\mathscr{B} \to \mathscr{C}}$}
\begin{align*}
  P_{\mathscr{B} \to \mathscr{C}} &= P^{-1}_{\mathscr{C} \to \mathscr{B}}\\
  P_{\mathscr{C} \to \mathscr{B}} &\implies \begin{pmatrix}
    1 & 1 & 1 & \vline & 1 & 0 & 0\\
    4 & 1 & 0 & \vline & 0 & 1 & 0\\
    0 & 3 & 0 & \vline & 0 & 0 & 1
  \end{pmatrix}\\
  (-4)(I) + (II) &\implies \begin{pmatrix}
    1 & 1 & 1 & \vline & 1 & 0 & 0\\
    0 & -3 & -4 & \vline & -4 & 1 & 0\\
    0 & 3 & 0 & \vline & 0 & 0 & 1
  \end{pmatrix}\\
  (II) \leftrightarrow (III) &\implies \begin{pmatrix}
    1 & 1 & 1 & \vline & 1 & 0 & 0\\
    0 & 3 & 0 & \vline & 0 & 0 & 1\\
    0 & -3 & -4 & \vline & -4 & 1 & 0
  \end{pmatrix}\\
  (\frac{1}{3})(II) &\implies \begin{pmatrix}
    1 & 1 & 1 & \vline & 1 & 0 & 0\\
    0 & 1 & 0 & \vline & 0 & 0 & \frac{1}{3}\\
    0 & -3 & -4 & \vline & -4 & 1 & 0
  \end{pmatrix}\\
  (-1)(II) + (I) &\implies \begin{pmatrix}
    1 & 0 & 1 & \vline & 1 & 0 & -\frac{1}{3}\\
    0 & 1 & 0 & \vline & 0 & 0 & \frac{1}{3}\\
    0 & -3 & -4 & \vline & -4 & 1 & 0
  \end{pmatrix}\\
  (3)(II) + (III) &\implies \begin{pmatrix}
    1 & 0 & 1 & \vline & 1 & 0 & -\frac{1}{3}\\
    0 & 1 & 0 & \vline & 0 & 0 & \frac{1}{3}\\
    0 & 0 & -4 & \vline & -4 & 1 & 1
  \end{pmatrix}\\
  (-\frac{1}{4})(III) &\implies \begin{pmatrix}
    1 & 0 & 1 & \vline & 1 & 0 & -\frac{1}{3}\\
    0 & 1 & 0 & \vline & 0 & 0 & \frac{1}{3}\\
    0 & 0 & 1 & \vline & 1 & -\frac{1}{4} & -\frac{1}{4}
  \end{pmatrix}\\
  (-1)(III) + (I) &\implies \begin{pmatrix}
    1 & 0 & 0 & \vline & 0 & \frac{1}{4} & -\frac{1}{12}\\
    0 & 1 & 0 & \vline & 0 & 0 & \frac{1}{3}\\
    0 & 0 & 1 & \vline & 1 & -\frac{1}{4} & -\frac{1}{4}
  \end{pmatrix}\\
  P_{\mathscr{B} \to \mathscr{C}} &=
  \begin{pmatrix}
  0 & \frac{1}{4} & -\frac{1}{12}\\
  0 & 0 & \frac{1}{3}\\
  1 & -\frac{1}{4} & -\frac{1}{4}
  \end{pmatrix}\\
\end{align*}

Check:
\begin{align*}
  a (x + 1)^2 + b (x + 4) + c (3x) &= a x^2 + a x + a + b x + 4b + 3cx\\
  &= a x^2 + (a + b + 3 c) x + (a + 4b)\\
  P_{\mathscr{B} \to \mathscr{C}} v &=
  \begin{pmatrix}
  0 & \frac{1}{4} & -\frac{1}{12}\\
  0 & 0 & \frac{1}{3}\\
  1 & -\frac{1}{4} & -\frac{1}{4}
  \end{pmatrix}
  \begin{pmatrix}
    a + 4b\\
    a + b + 3c\\
    a
  \end{pmatrix}_\mathscr{B}\\
  &= \begin{pmatrix}
    a\\
    \frac{a}{4} + b - \frac{a}{4}\\
    -\frac{a}{12} - \frac{b}{3} + \frac{a}{3} + \frac{b}{3} + c - \frac{a}{4}
  \end{pmatrix}_\mathscr{C}\\
  &= \begin{pmatrix}
    a\\
    b\\
    c
  \end{pmatrix}_\mathscr{C}\\
  &= a (x + 1)^2 + b (x + 4) + c (3x)
\end{align*}

\subsubsection{$P_{\mathscr{D} \to \mathscr{B}}$}
\begin{align*}
  1 - x^2 &\implies (1, 0, -1)\\
  x(x + 1) = x^2 + x &\implies (0, 1, 1)\\
  x^2 + 1 &\implies (1, 0, 1)\\
  P_{\mathscr{D} \to \mathscr{B}} &= \begin{pmatrix}
    1 & 0 & -1\\
    0 & 1 & 1\\
    1 & 0 & 1
  \end{pmatrix}
\end{align*}

Check:
\begin{align*}
  a (1 - x^2) + b (x(x + 1)) + c (x^2 + 1) &= a - ax^2 + bx^2 + bx + cx^2 + c\\
  &= - ax^2 + bx^2 + cx^2 + bx + c + a\\
  &= (-a + b + c)x^2 + b x + (c + a)\\
  P_{\mathscr{C} \to \mathscr{B}} v &=
  \begin{pmatrix}
    1 & 0 & -1\\
    0 & 1 & 1\\
    1 & 0 & 1
  \end{pmatrix}
  \begin{pmatrix}
    a \\
    b \\
    c
  \end{pmatrix}_\mathscr{D}\\
  &= \begin{pmatrix}
    a + c\\
    b\\
    -a + b + c
  \end{pmatrix}_\mathscr{B}\\
  &= (-a + b + c)x^2 + b x + (a + c)\square
\end{align*}

\subsubsection{$P_{\mathscr{B} \to \mathscr{D}}$}
\begin{align*}
  P_{\mathscr{B} \to \mathscr{D}} &= P^{-1}_{\mathscr{D} \to \mathscr{B}}\\
  P_{\mathscr{D} \to \mathscr{B}} &\implies \begin{pmatrix}
    1 & 0 & -1 & \vline & 1 & 0 & 0\\
    0 & 1 & 1 & \vline & 0 & 1 & 0\\
    1 & 0 & 1 & \vline & 0 & 0 & 1
  \end{pmatrix}\\
  (-1)(I) + (III) &\implies \begin{pmatrix}
    1 & 0 & -1 & \vline & 1 & 0 & 0\\
    0 & 1 & 1 & \vline & 0 & 1 & 0\\
    0 & 0 & 2 & \vline & -1 & 0 & 1
  \end{pmatrix}\\
  (\frac{1}{2})(III) &\implies \begin{pmatrix}
    1 & 0 & -1 & \vline & 1 & 0 & 0\\
    0 & 1 & 1 & \vline & 0 & 1 & 0\\
    0 & 0 & 1 & \vline & -\frac{1}{2} & 0 & \frac{1}{2}
  \end{pmatrix}\\
  (1)(III) + (I) &\implies \begin{pmatrix}
    1 & 0 & 0 & \vline & \frac{1}{2} & 0 & \frac{1}{2}\\
    0 & 1 & 1 & \vline & 0 & 1 & 0\\
    0 & 0 & 1 & \vline & -\frac{1}{2} & 0 & \frac{1}{2}
  \end{pmatrix}\\
  (-1)(III) + (II) &\implies \begin{pmatrix}
    1 & 0 & 0 & \vline & \frac{1}{2} & 0 & \frac{1}{2}\\
    0 & 1 & 0 & \vline & \frac{1}{2} & 1 & -\frac{1}{2}\\
    0 & 0 & 1 & \vline & -\frac{1}{2} & 0 & \frac{1}{2}
  \end{pmatrix}\\
  P_{\mathscr{B} \to \mathscr{D}} &=
  \begin{pmatrix}
  \frac{1}{2} & 0 & \frac{1}{2}\\
  \frac{1}{2} & 1 & -\frac{1}{2}\\
  -\frac{1}{2} & 0 & \frac{1}{2}
  \end{pmatrix}
\end{align*}

Check:
\begin{align*}
  a (1 - x^2) + b (x(x + 1)) + c (x^2 + 1) &= a - ax^2 + bx^2 + bx + cx^2 + c\\
  &= - ax^2 + bx^2 + cx^2 + bx + c + a\\
  &= (-a + b + c)x^2 + b x + (c + a)\\
  P_{\mathscr{B} \to \mathscr{D}} v &=
  \begin{pmatrix}
  \frac{1}{2} & 0 & \frac{1}{2}\\
  \frac{1}{2} & 1 & -\frac{1}{2}\\
  -\frac{1}{2} & 0 & \frac{1}{2}
  \end{pmatrix}
  \begin{pmatrix}
    a + c\\
    b\\
    -a + b + c
  \end{pmatrix}_\mathscr{B}\\
  &= \begin{pmatrix}
    \frac{a}{2} + \frac{c}{2} + \frac{b}{2} + \frac{a}{2} - \frac{b}{2} - \frac{c}{2}\\
    b\\
    \frac{a}{2} + \frac{c}{2} - \frac{b}{2} - \frac{a}{2} + \frac{b}{2} + \frac{c}{2}
  \end{pmatrix}_\mathscr{C}\\
  &= \begin{pmatrix}
    a\\
    b\\
    c
  \end{pmatrix}_\mathscr{C}\\
  &= a (1 - x^2) + b (x(x + 1)) + c (x^2 + 1)
\end{align*}

\section{c}

The coordinate for $a(x)$ is trivial for $\mathscr{C}$ because it's an element of the basis so it is only $(1, 0, 0)$. Now for the next part we have
\begin{align*}
  P_{\mathscr{B} \to \mathscr{D}} v &=
  \begin{pmatrix}
  \frac{1}{2} & 0 & \frac{1}{2}\\
  \frac{1}{2} & 1 & -\frac{1}{2}\\
  -\frac{1}{2} & 0 & \frac{1}{2}
  \end{pmatrix}
  \begin{pmatrix}
    1\\
    1\\
    1
  \end{pmatrix}_\mathscr{B}\\
  &= 
  \begin{pmatrix}
    \frac{1}{2} + \frac{1}{2} - \frac{1}{2}\\
    1\\
    \frac{1}{2} - \frac{1}{2} + \frac{1}{2}
  \end{pmatrix}_\mathscr{D}\\
  &= 
  \begin{pmatrix}
    \frac{1}{2}\\
    1\\
    \frac{1}{2}
  \end{pmatrix}_\mathscr{D}\\
\end{align*}

Check:
\begin{align*}
  \frac{1}{2}(1 - x^2) + 1 (x(x + 1)) + \frac{1}{2}(x^2 + 1) &= - \frac{x^2}{2} + \frac{1}{2} + x^2 + x + \frac{x^2}{2} + \frac{1}{2}\\
   &= x^2 + x + 1\square
\end{align*}

\section{P.23}

\subsection{c}
It is (just beacause it is described as a linear combination of coordinates in $\mathbb{R}^2$). However, let's check it.
\begin{align*}
  T ((x_1, y_1) + \alpha (x_2, y_2)) &= T (x_1 + \alpha x_2, y1 + \alpha y_2)\\
  &= (x_1 + \alpha x_2 + y_1 + \alpha y_2, x_1 + \alpha x_2 - y_1 - \alpha y_2, y_1 + \alpha y_2)\\
  T(x_1, y_1) + \alpha T(x_2, y_2) &= (x_1 + y_1, x_1 - y_1, y_1) + (\alpha x_2 + \alpha y_2, \alpha x_2 - \alpha y_2, \alpha y_2)\\
  &= (x_1 + \alpha x_2 + y_1 + \alpha y_2, x_1 + \alpha x_2 - y_1 - \alpha y_2, y_1 + \alpha y_2)\\
  T ((x_1, y_1) + \alpha (x_2, y_2)) &= T(x_1, y_1) + \alpha T(x_2, y_2)
\end{align*}
Meaning that it is a linear transformation.


\subsection{d}

Let's first check how it looks for a polinomial
\begin{align*}
  T(a t^2 + b t + c) &= t^2(a t^2 + b t + c)''\\
  &= t^2(2a t + b)'\\
  &= 2a t^2
\end{align*}

Now let's check if this is linear (it is beacuse it is basically applying a linear transformation two times, but lets check it).

\begin{align*}
  T((a t^2 + b t + c) + \alpha (d t^2 + e t + f)) &= T((a + \alpha d) t^2 + (b + \alpha e) t + (c + \alpha f))\\
  &= 2(a + \alpha d)t^2\\
  T(a t^2 + b t + c) + \alpha T(d t^2 + e t + f) &= 2at^2 + 2 \alpha d t^2\\
  &= 2(a + \alpha d)t^2\\
  T((a t^2 + b t + c) + \alpha (d t^2 + e t + f)) &= T(a t^2 + b t + c) + \alpha T(d t^2 + e t + f)\\
\end{align*}

It is a Linear transformation

\section{P.27}

It is actually quite simple to make this problem, what we need is a linear combination of $x$, $y$ and $z$ that equals to $0$ when the coordinates solve this equations:
\begin{align*}
  \begin{cases}
    x + y + z = 0\\
    x - 2y - 3z = 0
  \end{cases}
\end{align*}

But that equations already solve to 0 (and are 2). So we can take the transformation as

\begin{align*}
  T(x, y, z) &= (x + y + z, x - 2y - 3z)
\end{align*}

This already gave us the transformation, now let's check what is the image by first representhing this as the sum of vectors.
\begin{align*}
  T(x, y, z) &= x(1, 1) + y(1, -2) + z(1, -3)
\end{align*}

And now we should check what is the space that this generates

\begin{align*}
  T(x, y, z) &\implies \begin{pmatrix}
    1 & 1 & \vline & 0\\
    1 & - 2 & \vline & 0\\
    1 & - 3 & \vline & 0
  \end{pmatrix}\\
  (-1)(I) + (II) &\implies \begin{pmatrix}
    1 & 1 & \vline & 0\\
    0 & - 3 & \vline & 0\\
    1 & - 3 & \vline & 0
  \end{pmatrix}\\
  (-1)(I) + (III) &\implies \begin{pmatrix}
    1 & 1 & \vline & 0\\
    0 & - 3 & \vline & 0\\
    0 & - 4 & \vline & 0
  \end{pmatrix}\\
  (-\frac{1}{3})(II) &\implies \begin{pmatrix}
    1 & 1 & \vline & 0\\
    0 & 1 & \vline & 0\\
    0 & - 4 & \vline & 0
  \end{pmatrix}\\
  (-1)(II) + (I) &\implies \begin{pmatrix}
    1 & 0 & \vline & 0\\
    0 & 1 & \vline & 0\\
    0 & - 4 & \vline & 0
  \end{pmatrix}\\
  (4)(II) + (III) &\implies \begin{pmatrix}
    1 & 0 & \vline & 0\\
    0 & 1 & \vline & 0\\
    0 & 0 & \vline & 0
  \end{pmatrix}\\
\end{align*}

This actually mean that it generates $\mathbb{R}^2$

\section{P.34}

\begin{align*}
  F(1, 2, 1) &= (1 + 1, 2 - 2) = (2, 0)\\
  F(0, 1, 1) &= (0 + 1, 1 - 2) = (1, -1)\\
  F(0, 3, -1) &= (0 - 1, 3 + 2) = (-1, 5)
\end{align*}

Now we need to represent that in the base $\mathscr{C}$

\begin{align*}
  (2,  0) &= \alpha_1 (1, 5) + \beta_1 (2, -1)\\
  (1, -1) &= \alpha_2 (1, 5) + \beta_2 (2, -1)\\
  (-1, 5) &= \alpha_3 (1, 5) + \beta_3 (2, -1)\\
  &\implies \begin{pmatrix}
    1 & 2 & \vline & 2 & 1 & -1\\
    5 & -1 & \vline & 0 & -1 & 5
  \end{pmatrix}\\
  (-5)(I) + (II) &\implies \begin{pmatrix}
    1 & 2 & \vline & 2 & 1 & -1\\
    0 & -11 & \vline & -10 & -6 & 10
  \end{pmatrix}\\
  (-\frac{1}{11})(II) &\implies \begin{pmatrix}
    1 & 2 & \vline & 2 & 1 & -1\\
    0 & 1 & \vline & \frac{10}{11} & \frac{6}{11} & -\frac{10}{11}
  \end{pmatrix}\\
  (-2)(II) + (I) &\implies \begin{pmatrix}
    1 & 0 & \vline & \frac{2}{11} & -\frac{1}{11} & \frac{9}{11}\\
    0 & 1 & \vline & \frac{10}{11} & \frac{6}{11} & -\frac{10}{11}
  \end{pmatrix}\\
  \left[F\right]_\mathscr{B}^\mathscr{C} &= \frac{1}{11} \begin{pmatrix}
   2 & -1 & 9\\
   10& 6 & -10
  \end{pmatrix}
\end{align*}


\end{document}

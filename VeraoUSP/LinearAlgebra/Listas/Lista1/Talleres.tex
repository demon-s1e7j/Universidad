\documentclass[5pt]{exam}
\usepackage{amsthm}
\usepackage{libertine}
\usepackage[utf8]{inputenc}
\usepackage[margin=1in]{geometry}
\usepackage{amsmath,amssymb}
\usepackage{multicol}
\usepackage[shortlabels]{enumitem}
\usepackage{siunitx}
\usepackage{cancel}
\usepackage{graphicx}
\usepackage{pgfplots}
\usepackage{listings}
\usepackage{tikz}


\pgfplotsset{width=10cm,compat=1.9}
\usepgfplotslibrary{external}
\tikzexternalize

\newcommand{\class}{Linear Algebra} % This is the name of the course 
\newcommand{\examnum}{Lista 1} % This is the name of the assignment
\newcommand{\examdate}{\today} % This is the due date
\newcommand{\timelimit}{}





\begin{document}
\pagestyle{plain}
\thispagestyle{empty}

\noindent
\begin{tabular*}{\textwidth}{l @{\extracolsep{\fill}} r @{\extracolsep{6pt}} l}
  \textbf{\class} & \textbf{Name:} & \textit{Sergio Montoya}\\ %Your name here instead, obviously 
  \textbf{\examnum} &&\\
  \textbf{\examdate} &&
\end{tabular*}\\
\rule[2ex]{\textwidth}{2pt}
% ---

\section{P.1}
\subsection{e}
\begin{align*}
  S = \begin{cases}
    ix + z = 2i\\
    2x - iz = 4\\
    -ix + z = -i
  \end{cases} &\implies \begin{pmatrix}
    i & 1 & \vline & 2i\\
    2 & -i & \vline & 4\\
    -i & 1 & \vline & -i
  \end{pmatrix}\\
  -i(I) &\implies 
  \begin{pmatrix}
    1 & -i & \vline & 2\\
    2 & -i & \vline & 4\\
    -i & 1 & \vline & -i
  \end{pmatrix}\\
  (-2)(I) + (II) \to (II) &\implies 
  \begin{pmatrix}
    1 & -i & \vline & 2\\
    0 & i & \vline & 0\\
    -i & 1 & \vline & -i
  \end{pmatrix}\\
  (i)(I) + (III) \to (III) &\implies 
  \begin{pmatrix}
    1 & -i & \vline & 2\\
    0 & i & \vline & 0\\
    0 & 2 & \vline & i
  \end{pmatrix}\\
  (-i)(II) &\implies 
  \begin{pmatrix}
    1 & -i & \vline & 2\\
    0 & 1 & \vline & 0\\
    0 & 2 & \vline & i
  \end{pmatrix}\\
  (i)(II) + (I) \to (I) &\implies 
  \begin{pmatrix}
    1 & 0 & \vline & 2\\
    0 & 1 & \vline & 0\\
    0 & 2 & \vline & i
  \end{pmatrix}\\
  (-2)(II) + (III) \to (III) &\implies 
  \begin{pmatrix}
    1 & 0 & \vline & 2\\
    0 & 1 & \vline & 0\\
    0 & 0 & \vline & i
  \end{pmatrix}
\end{align*}

It can't exists a solution!

\subsection{g}
\begin{align*}
  S = \begin{cases}
    x - 2y + 3z = 0\\
    2x - y + 2z = 0\\
    3x + y + 2z = 0
  \end{cases} &\implies \begin{pmatrix}
    1 & -2 & 3 & \vline & 0\\
    2 & -1 & 2 & \vline & 0\\
    3 &  1 & 2 & \vline & 0
  \end{pmatrix}\\
  (-2)(I) + (II) \to (II)
  &\implies \begin{pmatrix}
    1 & -2 & 3 & \vline & 0\\
    0 & 3 & -4 & \vline & 0\\
    3 &  1 & 2 & \vline & 0
  \end{pmatrix}\\
  (-3)(I) + (II) \to (II)
  &\implies \begin{pmatrix}
    1 & -2 & 3 & \vline & 0\\
    0 & 3 & -4 & \vline & 0\\
    0 &  7 & -7 & \vline & 0
  \end{pmatrix}\\
  \left(\frac{1}{3}\right)(II)
  &\implies \begin{pmatrix}
    1 & -2 & 3 & \vline & 0\\
    0 & 1 & -\frac{4}{3} & \vline & 0\\
    0 &  7 & -7 & \vline & 0
  \end{pmatrix}\\
  \left(2\right)(II) + (I) \to (I)
  &\implies \begin{pmatrix}
    1 & 0 & \frac{1}{3} & \vline & 0\\
    0 & 1 & -\frac{4}{3} & \vline & 0\\
    0 &  7 & -7 & \vline & 0
  \end{pmatrix}\\
  \left(-7\right)(II) + (III) \to (III)
  &\implies \begin{pmatrix}
    1 & 0 &  \frac{1}{3} & \vline & 0\\
    0 & 1 & -\frac{4}{3} & \vline & 0\\
    0 & 0 & -\frac{7}{8} & \vline & 0
  \end{pmatrix}\\
  \left(-\frac{8}{7}\right)(III)
  &\implies \begin{pmatrix}
    1 & 0 &  \frac{1}{3} & \vline & 0\\
    0 & 1 & -\frac{4}{3} & \vline & 0\\
    0 & 0 & 1 & \vline & 0
  \end{pmatrix}\\
  \left(\frac{3}{4}\right)(III) + (II) \to (II)
  &\implies \begin{pmatrix}
    1 & 0 &  \frac{1}{3} & \vline & 0\\
    0 & 1 & 1 & \vline & 0\\
    0 & 0 & 1 & \vline & 0
  \end{pmatrix}\\
  \left(\frac{3}{1}\right)(III) + (II) \to (II)
  &\implies \begin{pmatrix}
    1 & 0 & 0 & \vline & 0\\
    0 & 1 & 0 & \vline & 0\\
    0 & 0 & 1 & \vline & 0
  \end{pmatrix}\\
\end{align*}

Trivial solution!

\subsection{i}

\begin{align*}
  S = \begin{cases}
    x_1 + 3x_2 + 2x_3 + 3x_4 - 7x_5 = 14\\
    2x_1 + 6x_2 + x_3 - 2x_4 + 5x_5 = -2\\
    x_1 + 3x_2 - x_3 + 2x_5 = -1
  \end{cases} &\implies \begin{pmatrix}
    1 & 3 &  2 &  3 & -7 &\vline& 14\\
    2 & 6 &  1 & -2 &  5 &\vline& -2\\
    1 & 3 & -1 &  0 &  2 &\vline& -1
  \end{pmatrix}\\
  (-2)(I) + (II) \to (II)
  &\implies \begin{pmatrix}
    1 & 3 &  2 &  3 & -7 &\vline& 14\\
    0 & 0 & -3 & -8 & 19 &\vline& -30\\
    1 & 3 & -1 &  0 &  2 &\vline& -1
  \end{pmatrix}\\
  (-1)(I) + (III) \to (III)
  &\implies \begin{pmatrix}
    1 & 3 &  2 &  3 & -7 &\vline& 14\\
    0 & 0 & -3 & -8 & 19 &\vline& -30\\
    0 & 0 & -3 & -3 &  9 &\vline& -15
  \end{pmatrix}\\
  \left(-\frac{1}{3}\right)(II)
  &\implies \begin{pmatrix}
    1 & 3 &  2 &  3 & -7 &\vline& 14\\
    0 & 0 & 1 & \frac{8}{3} & -\frac{19}{3} &\vline& 10\\
    0 & 0 & -3 & -3 &  9 &\vline& -15
  \end{pmatrix}\\
  \left(3\right)(II) + (III) \to (III)
  &\implies \begin{pmatrix}
    1 & 3 &  2 &  3 & -7 &\vline& 14\\
    0 & 0 & 1 & \frac{8}{3} & -\frac{19}{3} &\vline& 10\\
    0 & 0 & 0 & 5 & -10 &\vline& 15
  \end{pmatrix}\\
  \left(\frac{1}{5}\right)(III)
  &\implies \begin{pmatrix}
    1 & 3 &  2 &  3 & -7 &\vline& 14\\
    0 & 0 & 1 & \frac{8}{3} & -\frac{19}{3} &\vline& 10\\
    0 & 0 & 0 & 1 & -2 &\vline& 3
  \end{pmatrix}\\
  \left(-\frac{8}{3}\right)(III) + (II) \to (II)
  &\implies \begin{pmatrix}
    1 & 3 &  2 &  3 & -7 &\vline& 14\\
    0 & 0 & 1 & 0 & -1 &\vline& 2\\
    0 & 0 & 0 & 1 & -2 &\vline& 3
  \end{pmatrix}\\
  \left(-3\right)(III) + (I) \to (I)
  &\implies \begin{pmatrix}
    1 & 3 &  2 &  0 & -1 &\vline& 5\\
    0 & 0 & 1 & 0 & -1 &\vline& 2\\
    0 & 0 & 0 & 1 & -2 &\vline& 3
  \end{pmatrix}\\
  \left(-2\right)(II) + (I) \to (I)
  &\implies \begin{pmatrix}
    1 & 3 & 0 & 0 &  1 &\vline& 1\\
    0 & 0 & 1 & 0 & -1 &\vline& 2\\
    0 & 0 & 0 & 1 & -2 &\vline& 3
  \end{pmatrix}
\end{align*}

This left us with the sistem:
\begin{align*}
  x_1 + 3x_2 + x_5 = 1\\
  x_3 -  x_5 = 2\\
  x_4 - 2x_5 = 2
\end{align*}

Wich implies
\begin{align*}
  x_1 &= 1 - 3x_2 - x_5\\
  x_2 &= x_2\\
  x_3 &= 2 + x_5\\
  x_4 &= 2 + 2x_5\\
  x_5 &= x_5
\end{align*}

\subsection{n}

\begin{align*}
  S = \begin{cases}
    x + y + z = 4\\
    2x + 5y - 2z = 3\\
    x + 7y - 7z = 5
  \end{cases} &\implies \begin{pmatrix}
    1 & 1 &  1 &\vline & 4\\
    2 & 5 & -2 &\vline & 3\\
    1 & 7 & -7 &\vline & 5
  \end{pmatrix}\\
  \left(-2\right)(I) + (II) \to (II)
  &\implies \begin{pmatrix}
    1 & 1 &  1 &\vline & 4\\
    0 & 3 & -4 &\vline & -5\\
    1 & 7 & -7 &\vline & 5
  \end{pmatrix}\\
  \left(-1\right)(I) + (III) \to (III)
  &\implies \begin{pmatrix}
    1 & 1 &  1 &\vline & 4\\
    0 & 3 & -4 &\vline & -5\\
    0 & 6 & -8 &\vline & 1
  \end{pmatrix}\\
  \left(\frac{1}{3}\right)(II)
  &\implies \begin{pmatrix}
    1 & 1 &  1 &\vline & 4\\
    0 & 1 & -\frac{4}{3} &\vline & -\frac{5}{3}\\
    0 & 6 & -8 &\vline & 1
  \end{pmatrix}\\
  \left(-1\right)(II) + (I) \to (I)
  &\implies \begin{pmatrix}
    1 & 0 &  \frac{7}{3} &\vline & \frac{17}{3}\\
    0 & 1 & -\frac{4}{3} &\vline & -\frac{5}{3}\\
    0 & 6 & -8 &\vline & 1
  \end{pmatrix}\\
  \left(-6\right)(II) + (I) \to (I)
  &\implies \begin{pmatrix}
    1 & 0 &  \frac{7}{3} &\vline & \frac{17}{3}\\
    0 & 1 & -\frac{4}{3} &\vline & -\frac{5}{3}\\
    0 & 0 & 0 &\vline & 11
  \end{pmatrix}\\
\end{align*}

Doesn't exists a solution!

\section{P.6}

Before any subsection lets get the reduce row echelon form.

\begin{align*}
  S = \begin{cases}
    x + y + kz = 2\\
    3x + 4y + 2z = k\\
    2x + 3y - z = 1
  \end{cases} &\implies \begin{pmatrix}
    1 & 1 &  k &\vline & 2\\
    3 & 4 &  2 &\vline & k\\
    2 & 3 & -1 &\vline & 1\\
  \end{pmatrix}\\
  \left(-3\right)(I) + (II) \to (II)
  &\implies \begin{pmatrix}
    1 & 1 &  k &\vline & 2\\
    0 & 1 &  2 - 3k &\vline & k - 6\\
    2 & 3 & -1 &\vline & 1\\
  \end{pmatrix}\\
  \left(-2\right)(I) + (III) \to (III)
  &\implies \begin{pmatrix}
    1 & 1 &  k &\vline & 2\\
    0 & 1 &  2 - 3k &\vline & k - 6\\
    0 & 2 & -1 - 2k &\vline & -3\\
  \end{pmatrix}\\
  \left(-1\right)(II) + (I) \to (I)
  &\implies \begin{pmatrix}
    1 & 0 &  - 2 + 4k &\vline & 8 - k\\
    0 & 1 &    2 - 3k &\vline & k - 6\\
    0 & 2 & -1 - 2k &\vline & -3\\
  \end{pmatrix}\\
  \left(-2\right)(II) + (III) \to (III)
  &\implies \begin{pmatrix}
    1 & 0 &  - 2 + 4k &\vline & 8 - k\\
    0 & 1 &    2 - 3k &\vline & k - 6\\
    0 & 0 &  - 3 + 4k &\vline & 9 - 2k\\
  \end{pmatrix}\\
\end{align*}

\begin{align*}
  S = \begin{cases}
    x + y + kz = 2\\
    3x + 4y + 2z = k\\
    2x + 3y - z = 1
  \end{cases} &\implies \begin{pmatrix}
    1 & 1 &  k &\vline & 2\\
    3 & 4 &  2 &\vline & k\\
    2 & 3 & -1 &\vline & 1\\
  \end{pmatrix}\\
  (-3)(I) + (II) \to (II) &\implies
  \begin{pmatrix}
    1 & 1 &        k &\vline & 2\\
    0 & 1 &  2 - 3k &\vline & k - 6\\
    2 & 3 &      -1 &\vline & 1
  \end{pmatrix} \\
  (-2)(I) + (II) \to (III) &\implies
  \begin{pmatrix}
    1 & 1 &        k &\vline & 2\\
    0 & 1 &  2 - 3k &\vline & k - 6\\
    0 & 1 & -1 - 2k &\vline & -3
  \end{pmatrix} \\
  (-1)(II) + (I) \to (I) &\implies
  \begin{pmatrix}
    1 & 0 &  4k - 2 &\vline & 8 - k\\
    0 & 1 &  2 - 3k &\vline & k - 6\\
    0 & 1 & -1 - 2k &\vline & -3
  \end{pmatrix} \\
  (-1)(II) + (III) \to (III) &\implies
  \begin{pmatrix}
    1 & 0 &  4k - 2 &\vline & 8 - k\\
    0 & 1 &  2 - 3k &\vline & k - 6\\
    0 & 0 &   k - 3 &\vline & 3 - k
  \end{pmatrix}
\end{align*}

\subsection{a}
For an only solution we can try $k = 4$ as it get us in:
\begin{align*}
  \begin{pmatrix}
    1 & 0 &  16 - 2 &\vline & 8 - 4\\
    0 & 1 &  2 - 12 &\vline & 4 - 6\\
    0 & 0 &   4 - 3 &\vline & 3 - 4
  \end{pmatrix}\\
  \begin{pmatrix}
    1 & 0 &   14 &\vline & 4\\
    0 & 1 &  -10 &\vline & -2\\
    0 & 0 &   1 &\vline & -1
  \end{pmatrix}\\
  (10)(III) + (II) \to (II) &\implies
  \begin{pmatrix}
    1 & 0 &   14 &\vline & 4\\
    0 & 1 &  0 &\vline & -12\\
    0 & 0 &   1 &\vline & -1
  \end{pmatrix}\\
  (-14)(III) + (I) \to (I) &\implies
  \begin{pmatrix}
    1 & 0 &   0 &\vline & 18\\
    0 & 1 &  0 &\vline & -12\\
    0 & 0 &   1 &\vline & -1
  \end{pmatrix}
\end{align*}

And the solution is trivial (from there)

\subsection{b}

For infinite solutions take $k = 3$
\begin{align*}
  \begin{pmatrix}
    1 & 0 &  12 - 2 &\vline & 8 - 3\\
    0 & 1 &  2 - 9 &\vline & 3 - 6\\
    0 & 0 &   3 - 3 &\vline & 3 - 3
  \end{pmatrix}\\
  \begin{pmatrix}
    1 & 0 &   10 &\vline & 5\\
    0 & 1 &  -7 &\vline & -3\\
    0 & 0 &   0 &\vline & 0
  \end{pmatrix}\\
\end{align*}

Therefore,
\begin{align*}
  x + 10 y &= 5\\
  y - 7  z &= -3\\
  \implies x &= 5 - 10 y\\
  \implies y &= y\\
  \implies z &= \frac{3 + y}{7}\\
\end{align*}

\subsection{c}

There is no possible k values such that this sistem does not have any solution. Particularly because the last row is consistent (meaning that there is no way to make the row of variables 0 and the result different to 0)
\begin{align*}
  \begin{pmatrix}
    1 & 0 &  4k - 2 &\vline & 8 - k\\
    0 & 1 &  2 - 3k &\vline & k - 6\\
    0 & 0 &   k - 3 &\vline & 3 - k
  \end{pmatrix}
\end{align*}

\section{P.11}
\subsection{a}

\begin{align*}
  A &= \begin{pmatrix}
    -2 & -1 &  1\\
    5 &  3 & -1\\
    3 &  1 & -3
  \end{pmatrix} \\
  (A|I) &\implies 
  \begin{pmatrix}
    -2 & -1 &  1 &\vline & 1 & 0 & 0\\
    5 &  3 & -1 &\vline & 0 & 1 & 0\\
    3 &  1 & -3 &\vline & 0 & 0 & 1
  \end{pmatrix} \\
  (I) \rightarrow -\frac{1}{2}(I) &\implies 
  \begin{pmatrix}
    1 & \frac{1}{2} & -\frac{1}{2} &\vline & -\frac{1}{2} & 0 & 0\\
    5 &  3 & -1 &\vline & 0 & 1 & 0\\
    3 &  1 & -3 &\vline & 0 & 0 & 1
  \end{pmatrix} \\
  (II) \rightarrow (II) - 5(I) &\implies 
  \begin{pmatrix}
    1 & \frac{1}{2} & -\frac{1}{2} &\vline & -\frac{1}{2} & 0 & 0\\
    0 & \frac{1}{2} & \frac{3}{2} &\vline & \frac{5}{2} & 1 & 0\\
    3 &  1 & -3 &\vline & 0 & 0 & 1
  \end{pmatrix} \\
  (III) \rightarrow (III) - 3(I) &\implies 
  \begin{pmatrix}
    1 & \frac{1}{2} & -\frac{1}{2} &\vline & -\frac{1}{2} & 0 & 0\\
    0 & \frac{1}{2} & \frac{3}{2} &\vline & \frac{5}{2} & 1 & 0\\
    0 & -\frac{1}{2} & -\frac{3}{2} &\vline & \frac{3}{2} & 0 & 1
  \end{pmatrix} \\
  (I) \rightarrow (I) - (II) &\implies 
  \begin{pmatrix}
    1 & 0 & -2 &\vline & -3 & -1 & 0\\
    0 & \frac{1}{2} & \frac{3}{2} &\vline & \frac{5}{2} & 1 & 0\\
    0 & -\frac{1}{2} & -\frac{3}{2} &\vline & \frac{3}{2} & 0 & 1
  \end{pmatrix} \\
  (III) \rightarrow (III) + (II) &\implies 
  \begin{pmatrix}
    1 & 0 & -2 &\vline & -3 & -1 & 0\\
    0 & \frac{1}{2} & \frac{3}{2} &\vline & \frac{5}{2} & 1 & 0\\
    0 & 0 & 0 &\vline & 4 & 1 & 1
  \end{pmatrix} \\
  (II) \rightarrow 2(II) &\implies 
  \begin{pmatrix}
    1 & 0 & -2 &\vline & -3 & -1 & 0\\
    0 & 1 & 3 &\vline & 5 & 2 & 0\\
    0 & 0 & 0 &\vline & 4 & 1 & 1
  \end{pmatrix}
\end{align*}

\subsection{b}

\begin{align*}
  B &= \begin{pmatrix}
    i & -1\\
    1 + 2i & -3
  \end{pmatrix} \\
  (B|I) &= 
  \begin{pmatrix}
    i & -1 &\vline & 1 & 0\\
    1 + 2i & -3 &\vline & 0 & 1
  \end{pmatrix} \\
  (I) \rightarrow (-i)(I) &\implies
  \begin{pmatrix}
    1 & i &\vline & -i & 0\\
    1 + 2i & -3 &\vline & 0 & 1
  \end{pmatrix} \\
  (i) \rightarrow (-i)(i) &\implies
  \begin{pmatrix}
    1 & i &\vline & -i & 0\\
    1 + 2i & -3 &\vline & 0 & 1
  \end{pmatrix} \\
  (II) \rightarrow (-1 - 2i)(I) + (II) &\implies
  \begin{pmatrix}
    1 & i &\vline & -i & 0\\
    0 & (-i - 1) &\vline & (i - 2) & 1
  \end{pmatrix} \\
  (II) \rightarrow \left(\frac{-1 + i}{2}\right)(II) &\implies
  \begin{pmatrix}
    1 & i &\vline & -i & 0\\
    0 & 1 &\vline & (\frac{1}{2} - \frac{3}{2}i) & \frac{-1 + i}{2}
  \end{pmatrix} \\
  (I) \rightarrow \left(-i\right)(II) + (I) &\implies
  \begin{pmatrix}
    1 & 0 &\vline & (-\frac{3}{2} - \frac{3i}{2}) & \frac{i + 1}{2}\\
    0 & 1 &\vline & (\frac{1}{2} - \frac{3}{2}i) & \frac{-1 + i}{2}
  \end{pmatrix} \\
\end{align*}

Check:
\begin{align*}
  \begin{pmatrix}
    i & -1\\
    1 + 2i & -3
  \end{pmatrix}
  \begin{pmatrix}
    (-\frac{3}{2} - \frac{3i}{2}) & \frac{i + 1}{2}\\
    (\frac{1}{2} - \frac{3}{2}i) & \frac{-1 + i}{2}
  \end{pmatrix} =
  \begin{pmatrix}
    1 & 0\\
    0 & 1
  \end{pmatrix} 
\end{align*}

\subsection{c}

\begin{align*}
  B &= \begin{pmatrix}
     1 & 0 & 1\\
    -1 & 3 & 1\\
     0 & 1 & 1
  \end{pmatrix} \\
  (B|I) &\implies
  \begin{pmatrix}
    1 & 0 & 1 & \vline & 1 & 0 & 0\\
   -1 & 3 & 1 & \vline & 0 & 1 & 0\\
    0 & 1 & 1 & \vline & 0 & 0 & 1
  \end{pmatrix} \\
  (I) + (II) &\implies
  \begin{pmatrix}
    1 & 0 & 1 & \vline & 1 & 0 & 0\\
    0 & 3 & 2 & \vline & 1 & 1 & 0\\
    0 & 1 & 1 & \vline & 0 & 0 & 1
  \end{pmatrix} \\
  (II) \leftrightarrow (III) &\implies
  \begin{pmatrix}
    1 & 0 & 1 & \vline & 1 & 0 & 0\\
    0 & 1 & 1 & \vline & 0 & 0 & 1\\
    0 & 3 & 2 & \vline & 1 & 1 & 0
  \end{pmatrix} \\
  (-3)(II) + (III) &\implies
  \begin{pmatrix}
    1 & 0 & 1 & \vline & 1 & 0 & 0\\
    0 & 1 & 1 & \vline & 0 & 0 & 1\\
    0 & 0 & -1 & \vline & 1 & 1 & -3
  \end{pmatrix} \\
  (-1)(III) &\implies
  \begin{pmatrix}
    1 & 0 & 1 & \vline & 1 & 0 & 0\\
    0 & 1 & 1 & \vline & 0 & 0 & 1\\
    0 & 0 & 1 & \vline & -1 & -1 & 3
  \end{pmatrix} \\
  (-1)(III) + (II) &\implies
  \begin{pmatrix}
    1 & 0 & 1 & \vline & 1 & 0 & 0\\
    0 & 1 & 0 & \vline & 1 & 1 & -2\\
    0 & 0 & 1 & \vline & -1 & -1 & 3
  \end{pmatrix} \\
  (-1)(III) + (I) &\implies
  \begin{pmatrix}
    1 & 0 & 0 & \vline & 2 & 1 & -3\\
    0 & 1 & 0 & \vline & 1 & 1 & -2\\
    0 & 0 & 1 & \vline & -1 & -1 & 3
  \end{pmatrix} \\
\end{align*}

Check:
\begin{align*}
  \begin{pmatrix}
     1 & 0 & 1\\
    -1 & 3 & 1\\
     0 & 1 & 1
  \end{pmatrix}
  \begin{pmatrix}
    2 & 1 & -3\\
    1 & 1 & -2\\
    -1 & -1 & 3
  \end{pmatrix} &= \begin{pmatrix}
    2 - 1 & 1 - 1 & -3 + 3\\
    -2 + 3 - 1 & -1 + 3 - 1 & 3 -6 + 3\\
    1 - 1 & 1 - 1 & -2 + 3
  \end{pmatrix}\\
  &= \begin{pmatrix}
    1 & 0 & 0\\
    0 & 1 & 0\\
    0 & 0 & 1
  \end{pmatrix}
\end{align*}

\section{P.12}

\begin{align*}
  \begin{pmatrix}
    2 & 3\\
    3 & 4
  \end{pmatrix} & \implies
  \begin{pmatrix}
    2 & 3 &\vline & 1 & 0\\
    3 & 4 &\vline & 0 & 1
  \end{pmatrix}\\
  \frac{1}{2}(I) &\implies
  \begin{pmatrix}
    1 & \frac{3}{2} &\vline & \frac{1}{2} & 0\\
    3 & 4 &\vline & 0 & 1
  \end{pmatrix}\\
  (-3)(I) + (II) &\implies
  \begin{pmatrix}
    1 & \frac{3}{2} &\vline & \frac{1}{2} & 0\\
    0 & -\frac{1}{2} &\vline & -\frac{3}{2} & 1
  \end{pmatrix}\\
  (-2)(II) &\implies
  \begin{pmatrix}
    1 & \frac{3}{2} &\vline & \frac{1}{2} & 0\\
    0 & 1 &\vline & 3 & -2
  \end{pmatrix}\\
  (-\frac{3}{2})(II) + (I) &\implies
  \begin{pmatrix}
    1 & 0 &\vline & -4 & 3\\
    0 & 1 &\vline & 3 & -2
  \end{pmatrix}\\
\end{align*}

Check:
\begin{align*}
  \begin{pmatrix}
    -4 & 3\\
    3 & -2
  \end{pmatrix}
  \begin{pmatrix}
    2 & 3\\
    3 & 4
  \end{pmatrix}
  &= \begin{pmatrix}
    -8 + 9 & -12 + 12\\
    6 - 6 & 9 - 8
  \end{pmatrix}\\
  &= \begin{pmatrix}
    1 & 0\\
    0 & 1
  \end{pmatrix}
\end{align*}

\section{P.17}

\begin{align*}
  A = \begin{pmatrix}
    -5 & 3 & 1 \\
    0 & -1 & 2 \\
    4 & -1 & -3 \end{pmatrix} &\implies \begin{pmatrix} -5 & 3 & 1 & \vline & 4 & 11\\
    0 & -1 & 2 & \vline & 4 & -2\\
    4 & -1 & -3 & \vline & -7 & -6
  \end{pmatrix}\\
  (-\frac{1}{5})(I) &\implies
  \begin{pmatrix}
    1 & -\frac{3}{5} & -\frac{1}{5} & \vline & -\frac{4}{5} & -\frac{11}{5}\\
    0 & -1 & 2 & \vline & 4 & -2\\
    4 & -1 & -3 & \vline & -7 & -6
  \end{pmatrix}\\
  (-4)(I) + (III) &\implies
  \begin{pmatrix}
    1 & -\frac{3}{5} & -\frac{1}{5} & \vline & -\frac{4}{5} & -\frac{11}{5}\\
    0 & -1 & 2 & \vline & 4 & -2\\
    0 & \frac{7}{5} & -\frac{11}{5} & \vline & -\frac{19}{5} & \frac{14}{5}
  \end{pmatrix}\\
  (-1)(II) &\implies
  \begin{pmatrix}
    1 & -\frac{3}{5} & -\frac{1}{5} & \vline & -\frac{4}{5} & -\frac{11}{5}\\
    0 & 1 & -2 & \vline & -4 & 2\\
    0 & \frac{7}{5} & -\frac{11}{5} & \vline & -\frac{19}{5} & \frac{14}{5}
  \end{pmatrix}\\
  (\frac{3}{5})(II) + (I) &\implies
  \begin{pmatrix}
    1 & 0 & -\frac{7}{5} & \vline & -\frac{16}{5} & -1\\
    0 & 1 & -2 & \vline & -4 & 2\\
    0 & \frac{7}{5} & -\frac{11}{5} & \vline & -\frac{19}{5} & \frac{14}{5}
    (- \frac{7}{5})(II) + (III) &\implies
  \end{pmatrix}\\
  \begin{pmatrix}
    1 & 0 & -\frac{7}{5} & \vline & -\frac{16}{5} & -1\\
    0 & 1 & -2 & \vline & -4 & 2\\
    0 & 0 & \frac{3}{5} & \vline & \frac{9}{5} & 0
  \end{pmatrix}\\
  (\frac{5}{3})(III) &\implies
  \begin{pmatrix}
    1 & 0 & -\frac{7}{5} & \vline & -\frac{16}{5} & -1\\
    0 & 1 & -2 & \vline & -4 & 2\\
    0 & 0 & 1 & \vline & 3 & 0
  \end{pmatrix}\\
  (2)(III) + (II) &\implies
  \begin{pmatrix}
    1 & 0 & -\frac{7}{5} & \vline & -\frac{16}{5} & -1\\
    0 & 1 & 0 & \vline & 2 & 2\\
    0 & 0 & 1 & \vline & 3 & 0
  \end{pmatrix}\\
  (\frac{7}{5})(III) + (I) &\implies
  \begin{pmatrix}
    1 & 0 & 0 & \vline & 1 & -1\\
    0 & 1 & 0 & \vline & 2 & 2\\
    0 & 0 & 1 & \vline & 3 & 0
  \end{pmatrix}\\
\end{align*}

Verification:

\begin{align*}
  A \cdot B &= \begin{pmatrix}
    -5 & 3 & 1 \\
    0 & -1 & 2 \\
    4 & -1 & -3
  \end{pmatrix} \begin{pmatrix}
    1 & -1\\
    2 & 2\\
    3 & 0
  \end{pmatrix}\\
  &= \begin{pmatrix}
    -5 + 6 + 3 & 5 + 6\\
    -2 + 6 & -2\\
    4 + -2 -9 & -4 -2\\
  \end{pmatrix}\\
  &= \begin{pmatrix}
    4 & 11\\
    4 & -2\\
    -7 & -6\\
  \end{pmatrix}
  \square
\end{align*}



\end{document}

  \documentclass[12pt]{exam}
\usepackage{amsthm}
\usepackage{libertine}
\usepackage[utf8]{inputenc}
\usepackage[margin=1in]{geometry}
\usepackage{amsmath,amssymb}
\usepackage{multicol}
\usepackage[shortlabels]{enumitem}
\usepackage{siunitx}
\usepackage{cancel}
\usepackage{graphicx}
\usepackage{pgfplots}
\usepackage{listings}
\usepackage[spanish]{babel}
\usepackage{tikz}


\newcommand{\class}{Teoria de Grupos para Mecanica Cuantica} % This is the name of the course 
\newcommand{\examnum}{Ejercicio Clase 1} % This is the name of the assignment
\newcommand{\examdate}{\today} % This is the due date
\newcommand{\timelimit}{}





\begin{document}
\pagestyle{plain}
\thispagestyle{empty}

\noindent
\begin{tabular*}{\textwidth}{l @{\extracolsep{\fill}} r @{\extracolsep{6pt}} l}
	\textbf{\class} & \textbf{Name:} & \textit{Sergio Montoya}\\ %Your name here instead, obviously 
	\textbf{\examnum} &&\\
	\textbf{\examdate} &&
\end{tabular*}\\
\rule[2ex]{\textwidth}{2pt}
% ---

\section{Exercise 1.7}
\subsection{symmetries}

\subsubsection{$E$}

\begin{tikzpicture}


% Primer cuadrado (izquierda)
\draw[line width=1pt] (0,0) rectangle (2,2);
\draw (0,0) node[left] {A};
\draw (0,2) node[left] {B};
\draw (2,2) node[right] {C};
\draw (2,0) node[right] {D};

%\draw[line width = 1pt] (0, 0) -- (2, 2);

% Segundo cuadrado (derecha)
\draw[line width=1pt] (5,0) rectangle (7,2);
\draw (5,0) node[left] {A};
\draw (5,2) node[left] {B};
\draw (7,2) node[right] {C};
\draw (7,0) node[right] {D};

% Flecha sin estilo -> (dibujamos la punta manualmente)
\draw[line width=1pt] (3,1) -- (4.5,1);
% Añadimos punta de flecha manualmente
\fill (4.6,1) -- (4.3,0.9) -- (4.3,1.1) -- cycle;

\end{tikzpicture}

\subsubsection{$T_1$}
\begin{tikzpicture}
% Primer cuadrado (izquierda)
\draw[line width=1pt] (0,0) rectangle (2,2);
\draw (0,0) node[left] {A};
\draw (0,2) node[left] {B};
\draw (2,2) node[right] {C};
\draw (2,0) node[right] {D};

\draw[line width = 1pt] (-0.5, 1) -- (2.5, 1);

% Segundo cuadrado (derecha)
\draw[line width=1pt] (5,0) rectangle (7,2);
\draw (5,0) node[left] {B};
\draw (5,2) node[left] {A};
\draw (7,2) node[right] {D};
\draw (7,0) node[right] {C};

% Flecha sin estilo -> (dibujamos la punta manualmente)
\draw[line width=1pt] (3,1) -- (4.5,1);
% Añadimos punta de flecha manualmente
\fill (4.6,1) -- (4.3,0.9) -- (4.3,1.1) -- cycle;

\end{tikzpicture}
\subsubsection{$T_2$}

\begin{tikzpicture}

% Primer cuadrado (izquierda)
\draw[line width=1pt] (0,0) rectangle (2,2);
\draw (0,0) node[left] {A};
\draw (0,2) node[left] {B};
\draw (2,2) node[right] {C};
\draw (2,0) node[right] {D};

\draw[line width = 1pt] (1, 2.5) -- (1, -0.5);

% Segundo cuadrado (derecha)
\draw[line width=1pt] (5,0) rectangle (7,2);
\draw (5,0) node[left] {D};
\draw (5,2) node[left] {C};
\draw (7,2) node[right] {B};
\draw (7,0) node[right] {A};

% Flecha sin estilo -> (dibujamos la punta manualmente)
\draw[line width=1pt] (3,1) -- (4.5,1);
% Añadimos punta de flecha manualmente
\fill (4.6,1) -- (4.3,0.9) -- (4.3,1.1) -- cycle;

\end{tikzpicture}

\subsubsection{$D_1$}

\begin{tikzpicture}

% Primer cuadrado (izquierda)
\draw[line width=1pt] (0,0) rectangle (2,2);
\draw (0,0) node[left] {A};
\draw (0,2) node[left] {B};
\draw (2,2) node[right] {C};
\draw (2,0) node[right] {D};

\draw[line width = 1pt] (0, 0) -- (2, 2);

% Segundo cuadrado (derecha)
\draw[line width=1pt] (5,0) rectangle (7,2);
\draw (5,0) node[left] {A};
\draw (5,2) node[left] {D};
\draw (7,2) node[right] {C};
\draw (7,0) node[right] {B};

% Flecha sin estilo -> (dibujamos la punta manualmente)
\draw[line width=1pt] (3,1) -- (4.5,1);
% Añadimos punta de flecha manualmente
\fill (4.6,1) -- (4.3,0.9) -- (4.3,1.1) -- cycle;

\end{tikzpicture}

\subsubsection{$D_2$}

\begin{tikzpicture}

% Primer cuadrado (izquierda)
\draw[line width=1pt] (0,0) rectangle (2,2);
\draw (0,0) node[left] {A};
\draw (0,2) node[left] {B};
\draw (2,2) node[right] {C};
\draw (2,0) node[right] {D};

\draw[line width = 1pt] (0, 2) -- (2, 0);

% Segundo cuadrado (derecha)
\draw[line width=1pt] (5,0) rectangle (7,2);
\draw (5,0) node[left] {C};
\draw (5,2) node[left] {B};
\draw (7,2) node[right] {A};
\draw (7,0) node[right] {D};

% Flecha sin estilo -> (dibujamos la punta manualmente)
\draw[line width=1pt] (3,1) -- (4.5,1);
% Añadimos punta de flecha manualmente
\fill (4.6,1) -- (4.3,0.9) -- (4.3,1.1) -- cycle;

\end{tikzpicture}

\subsubsection{$R_1$}

\begin{tikzpicture}

% Primer cuadrado (izquierda)
\draw[line width=1pt] (0,0) rectangle (2,2);
\draw (0,0) node[left] {A};
\draw (0,2) node[left] {B};
\draw (2,2) node[right] {C};
\draw (2,0) node[right] {D};

% Flecha curva para rotación
\draw[thick] (1.5,1) arc (0:280:0.5);
% Cabeza de flecha
\fill (1.1, 0.4) -- (1.2, 0.6 ) -- (1, 0.6) -- cycle;
\draw (1.5, 1) node[left] {$90^{\circ}$};

% Segundo cuadrado (derecha)
\draw[line width=1pt] (5,0) rectangle (7,2);
\draw (5,0) node[left] {B};
\draw (5,2) node[left] {C};
\draw (7,2) node[right] {D};
\draw (7,0) node[right] {A};

% Flecha sin estilo -> (dibujamos la punta manualmente)
\draw[line width=1pt] (3,1) -- (4.5,1);
% Añadimos punta de flecha manualmente
\fill (4.6,1) -- (4.3,0.9) -- (4.3,1.1) -- cycle;

\end{tikzpicture}

\subsubsection{$R_2$}

\begin{tikzpicture}

% Primer cuadrado (izquierda)
\draw[line width=1pt] (0,0) rectangle (2,2);
\draw (0,0) node[left] {A};
\draw (0,2) node[left] {B};
\draw (2,2) node[right] {C};
\draw (2,0) node[right] {D};

% Flecha curva para rotación
\draw[thick] (1.5,1) arc (0:280:0.5);
% Cabeza de flecha
\fill (1.1, 0.4) -- (1.2, 0.6 ) -- (1, 0.6) -- cycle;
\draw (1.5, 1) node[left] {$180^{\circ}$};

% Segundo cuadrado (derecha)
\draw[line width=1pt] (5,0) rectangle (7,2);
\draw (5,0) node[left] {C};
\draw (5,2) node[left] {D};
\draw (7,2) node[right] {A};
\draw (7,0) node[right] {B};

% Flecha sin estilo -> (dibujamos la punta manualmente)
\draw[line width=1pt] (3,1) -- (4.5,1);
% Añadimos punta de flecha manualmente
\fill (4.6,1) -- (4.3,0.9) -- (4.3,1.1) -- cycle;

\end{tikzpicture}

\subsubsection{$R_3$}

\begin{tikzpicture}

% Primer cuadrado (izquierda)
\draw[line width=1pt] (0,0) rectangle (2,2);
\draw (0,0) node[left] {A};
\draw (0,2) node[left] {B};
\draw (2,2) node[right] {C};
\draw (2,0) node[right] {D};

% Flecha curva para rotación
\draw[thick] (1.5,1) arc (0:280:0.5);
% Cabeza de flecha
\fill (1.1, 0.4) -- (1.2, 0.6 ) -- (1, 0.6) -- cycle;
\draw (1.5, 1) node[left] {$270^{\circ}$};

% Segundo cuadrado (derecha)
\draw[line width=1pt] (5,0) rectangle (7,2);
\draw (5,0) node[left] {D};
\draw (5,2) node[left] {A};
\draw (7,2) node[right] {B};
\draw (7,0) node[right] {C};

% Flecha sin estilo -> (dibujamos la punta manualmente)
\draw[line width=1pt] (3,1) -- (4.5,1);
% Añadimos punta de flecha manualmente
\fill (4.6,1) -- (4.3,0.9) -- (4.3,1.1) -- cycle;

\end{tikzpicture}

\subsection{Order}

The order (as you can count) is 8

\subsection{Multiplication Table}

\begin{tabular}{c|cccccccc}  % 9 columnas centradas
   & $E$ & $T_1$ & $T_2$ & $D_1$ & $D_2$ & $R_1$ & $R_2$ & $R_3$\\
   \hline
 $E$ & $E$ & $T_1$ & $T_2$ & $D_1$ & $D_2$ & $R_1$ & $R_2$ & $R_3$\\
 $T_1$ & $T_1$ & $E$ & $R_2$ & $R_3$ & $R_1$ & $D_2$ & $T_2$ & $D_1$\\
 $T_2$ & $T_2$ & $R_2$ & $E$ & $R_1$ & $R_3$ & $D_1$ & $T_1$ & $D_2$\\
 $D_1$ & $D_1$ & $R_1$ & $R_3$ & $E$ & $R_2$ & $T_1$ & $D_2$ & $T_2$\\
 $D_2$ & $D_2$ & $R_3$ & $R_1$ & $R_2$ & $E$ & $T_2$ & $D_1$ & $T_1$\\
 $R_1$ & $R_1$ & $D_1$ & $D_2$ & $T_2$ & $T_1$ & $R_2$ & $R_3$ & $E$\\
 $R_2$ & $R_2$ & $T_2$ & $T_1$ & $D_2$ & $D_1$ & $R_3$ & $E$ & $R_1$\\
 $R_3$ & $R_3$ & $D_2$ & $D_1$ & $T_1$ & $T_2$ & $E$ & $R_1$ & $R_2$
\end{tabular}

\subsection{Subgroups}

\begin{tikzpicture}

% Primer cuadrado (izquierda)
\draw (1.5,0) node[left] {$\{E, T_1, T_2, D_1, D_2, R_1, R_2, R_3\}$};

\draw (-5,-3) node[left] {$\{E, D_1, D_2, R_2\}$};
\draw (0,-3) node[left] {$\{E, R_1, R_2, R_3\}$};
\draw (5,-3) node[left] {$\{E, T_1, T_2, R_2\}$};

\draw (-1.5, -0.5) -- (-6, -2.5) ;
\draw (-1.5, -0.5) -- (-1.5, -2.5) ;
\draw (-1.5, -0.5) -- (3.5, -2.5) ;

\draw (-5.5,-6) node[left] {$\{E, D_1\}$};
\draw (-3,-6) node[left] {$\{E, D_2\}$};
\draw (-0.5,-6) node[left] {$\{E, R_2\}$};
\draw (2,-6) node[left] {$\{E, T_1\}$};
\draw (5,-6) node[left] {$\{E, T_2\}$};

\draw (-1.5, -5.5) -- (-1.5, -3.5) ;
\draw (-1.5, -5.5) -- (-7, -3.5) ;
\draw (-4, -5.5) -- (-7, -3.5);
\draw (-6.5, -5.5) -- (-7, -3.5);
\draw (-1.5, -5.5) -- (4.5, -3.5) ;
\draw (1, -5.5) -- (4.5, -3.5);
\draw (4, -5.5) -- (4.5, -3.5);

\draw (-0.85,-8) node[left] {$\{E\}$};

\draw (-1.5, -6.5) -- (-1.45, -7.5) ;
\draw (-1.5, -6.5) -- (-1.45, -7.5) ;
\draw (-4, -6.5) -- (-1.45, -7.5);
\draw (-6.5, -6.5) -- (-1.45, -7.5);
\draw (-1.5, -6.5) -- (-1.45, -7.5) ;
\draw (1, -6.5) -- (-1.45, -7.5);
\draw (4, -6.5) -- (-1.45, -7.5);

\end{tikzpicture}



\end{document}

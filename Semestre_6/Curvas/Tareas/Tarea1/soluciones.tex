\documentclass[12pt]{article}
\usepackage{array}
\usepackage{color}
\usepackage{amsthm}
\usepackage{eufrak}
\usepackage{lipsum}
\usepackage{pifont}
\usepackage{yfonts}
\usepackage{amsmath}
\usepackage{amssymb}
\usepackage{ccfonts}
\usepackage{comment} \usepackage{amsfonts}
\usepackage{fancyhdr}
\usepackage{graphicx}
\usepackage{listings}
\usepackage{mathrsfs}
\usepackage{setspace}
\usepackage{textcomp}
\usepackage{blindtext}
\usepackage{enumerate}
\usepackage{microtype}
\usepackage{xfakebold}
\usepackage{kantlipsum}
%\usepackage{draftwatermark}
\usepackage[spanish]{babel}
\usepackage[margin=1.5cm, top=2cm, bottom=2cm]{geometry}
\usepackage[framemethod=tikz]{mdframed}
\usepackage[colorlinks=true,citecolor=blue,linkcolor=red,urlcolor=magenta]{hyperref}

%//////////////////////////////////////////////////////
% Watermark configuration
%//////////////////////////////////////////////////////
%\SetWatermarkScale{4}
%\SetWatermarkColor{black}
%\SetWatermarkLightness{0.95}
%\SetWatermarkText{\texttt{Watermark}}

%//////////////////////////////////////////////////////
% Frame configuration
%//////////////////////////////////////////////////////
\newmdenv[tikzsetting={draw=gray,fill=white,fill opacity=0},backgroundcolor=none]{Frame}

%//////////////////////////////////////////////////////
% Font style configuration
%//////////////////////////////////////////////////////
\renewcommand{\familydefault}{\ttdefault}
\renewcommand{\rmdefault}{tt}

%//////////////////////////////////////////////////////
% Bold configuration
%//////////////////////////////////////////////////////
\newcommand{\fbseries}{\unskip\setBold\aftergroup\unsetBold\aftergroup\ignorespaces}
\makeatletter
\newcommand{\setBoldness}[1]{\def\fake@bold{#1}}
\makeatother

%//////////////////////////////////////////////////////
% Default font configuration
%//////////////////////////////////////////////////////
\DeclareFontFamily{\encodingdefault}{\ttdefault}{%
  \hyphenchar\font=\defaulthyphenchar
  \fontdimen2\font=0.33333em
  \fontdimen3\font=0.16667em
  \fontdimen4\font=0.11111em
  \fontdimen7\font=0.11111em}



\begin{document}
    %//////////////////////////////////////////////////////
% Heading Configuration
%//////////////////////////////////////////////////////
\pagestyle{fancy}
\thispagestyle{plain}
\fancyhead[RO,L]{\textbf{Geometría de Curvas y Superficies (MATE-2411)}}
\fancyhead[LO,L]{\textbf{Tarea 1}}
\setlength{\headheight}{16.0pt}

%//////////////////////////////////////////////////////
% Subsections Configuration
%//////////////////////////////////////////////////////
\renewcommand*\thesubsection{\arabic{subsection}}
\newcounter{counter}
\newlength{\palabra}
\settowidth{\palabra}{counter 999.}
\newcommand{\makeboxlabel}[1]{\fbox{#1.}\hfill}

%//////////////////////////////////////////////////////
% Personalized commands configuration
%//////////////////////////////////////////////////////
\newcommand{\N}{\mathbb{N}}
\newcommand{\Z}{\mathbb{Z}}
\newcommand{\Q}{\mathbb{Q}}
\newcommand{\R}{\mathbb{R}}
\newcommand{\C}{\mathbb{C}}
\newcommand{\re}{\operatorname{Re}}
\newcommand{\im}{\operatorname{Im}}
\newcommand{\Aut}{\operatorname{Aut}}
\newcommand{\GCD}{\operatorname{GCD}}
\newcommand{\LCD}{\operatorname{LCD}}
\linespread{1} %Line spacing

%//////////////////////////////////////////////////////
% Inline code configuration
%//////////////////////////////////////////////////////
\lstset{
gobble=5,
numbers=left,
frame=single,
framerule=1pt,
showtabs=False,
showspaces=False,
showstringspaces=False,
backgroundcolor=\color{gray}}

%//////////////////////////////////////////////////////
% Problem list configuration
%//////////////////////////////////////////////////////
\newenvironment{problems}
  {\begin{list}
     {{\fbseries Problem \arabic{counter}.}}
    {\usecounter{counter}
     \setlength{\labelsep}{1em}
     \setlength{\itemsep}{2pt}
     \setlength{\leftmargin}{2em}
     \setlength{\rightmargin}{0cm}
     \setlength{\itemindent}{1em} }}
{\end{list}}

%//////////////////////////////////////////////////////
% Appendix configuration
%//////////////////////////////////////////////////////
\newenvironment{Appendix}
  {\begin{list}
     {{\fbseries Lemma \arabic{counter}.}}
    {\usecounter{counter}
     \setlength{\labelsep}{1em}
     \setlength{\itemsep}{2pt}
     \setlength{\leftmargin}{2em}
     \setlength{\rightmargin}{0cm}
     \setlength{\itemindent}{1em} }}
{\end{list}}

%//////////////////////////////////////////////////////
% Notes configuration
%//////////////////////////////////////////////////////
\newenvironment{notes}
  {\begin{list}
     {{\fbseries Note \arabic{counter}.}}
    {\usecounter{counter}
     \setlength{\labelsep}{1em}
     \setlength{\itemsep}{2pt}
     \setlength{\leftmargin}{2em}
     \setlength{\rightmargin}{0cm}
     \setlength{\itemindent}{1em} }}
{\end{list}}

%//////////////////////////////////////////////////////
% Activity Information
%//////////////////////////////////////////////////////
\vspace*{-1cm}
\hrule width \hsize \kern 1mm \hrule width \hsize height 2pt
\begin{center}
   \parbox[c]{.32\textwidth}{
   \hspace{1cm}\\
   Sergio Montoya Ramirez\\
   202112171}
%   Luis Ernesto Tejón Rojas\\
%   202113150}
   \hspace*{\fill}
   \parbox[c]{.35\textwidth}{\centering
   Universidad de Los Andes\\
   Tarea 1\\
   Geometría de Curvas y Superficies\\
   }
   \hspace*{\fill}
   \parbox[c]{.3\textwidth}{
   \begin{flushleft}
      Bogotá D.C., Colombia\\
      \today
   \end{flushleft}}
\end{center}
\hrule width \hsize height 2pt \kern 1mm \hrule width \hsize

\bigskip


\bigskip



    \begin{enumerate}
      \item 
	\begin{enumerate}
	  \item En este caso necesitamos dos $v_1$ y $v_2$ tales que al hacer producto punto entre ellos y $n$ el resultado sea 0 para ello entonces:
	    \begin{align*}
	      n\cdot v_{1} &= \left( a,b,c \right) \cdot \left( x,y,z \right)  \\
	      &= a\cdot x+b\cdot y+c\cdot z \\
	      x &= b \\
	      y &= -a\\
	      z &= 0 \\
	      &= a\cdot b + b \cdot \left( -a \right) + 0 \\
	      &= 0 \\
	    .\end{align*}

	    Y por otro lado

	    \begin{align*}
	      n\cdot v_2 &= \left( a,b,c \right) \cdot  \left( x,y,z \right)  \\
	      &= a\cdot x+b\cdot y+c\cdot z \\
	      x &= 0 \\
	      y &= c \\
	      z &= -b \\
	      &= a\cdot 0 + b\cdot c + c\cdot \left( -b \right)  \\
	      &= 0 + bc - bc \\
	      &= 0 \\
	    .\end{align*}
	  \item En este caso necesitariamos un vector normal al plano que generan por lo tanto hacemos producto cruz entre ambos vectores
	    \begin{align*}
	      v_1\times v_2 &= \left( b,-a,0 \right) \times \left( 0,c,-b \right) \\
	      &= \left( ba, b^2, bc \right)  \\
	    .\end{align*}

	    Y ya con esto podemos escribir la ecuación de un plano a partir de su vector normal
	    \begin{align*}
	      P_{v_1,v_2} &= ba x + b^2y + bc z + d = 0\\
	      -d &= ba\cdot b - b^2 a + bc 0 \\
	      -d &= b^2a - b^2a \\
	      -d &= 0 \\
	    .\end{align*}
	    
	    Por lo tanto el plano es:
	    \begin{align*}
	      0 &= ba x + b^2 y + bc z \\
	    .\end{align*}

	    Ahora necesitamos dos vectores $u_1$ y $u_2$ que sean perpendiculares entre si y que esten en el plano. Para esto y por simplificarnos tomemos $u_1=v_1$ dado que este vector ya esta en el plano y entonces encontremos un vector perpendicular a este con:
	    \begin{align*}
	      v_1\cdot u_2 &= \left( a,-b,0 \right) \cdot \left( x,y,z \right)  \\
	      &= ax - by \\
	      ax &= by \\
	      0 &= ba x + b^2 y + bc z \\
	      x &= b^2 c \\
	      y &= bac \\
	      z &= -2b^2a \\
	      0 &= b^{3}ac + b^{3}ac - 2b^{3}ac \\
	      b^2ac &= b^2ac \\
	    .\end{align*}

	    Tambien lo podiamos conseguir con un producto cruz entre el vector $v_1$ y el normal que encontramos previemente.
	  \item Dado que sabemos que este plano pasa por el origen. Por lo tanto podemos paramtrizar simplemente con:
	    \begin{align*}
	      p\left( t \right) &= R\cos\left( t \right) v_1 + R\sin\left( t \right) u_2\\
	      &= R\cos\left( t \right) \left( a,-b,0 \right) + R\sin\left( t \right) \left( b^2c,bac,-2b^2a \right)
	    .\end{align*}
	\end{enumerate}
      \item 
	\begin{enumerate}
	  \item En este caso, la aproximación de Taylor de segundo orden seria:
	    \begin{align*}
	      f\left( x \right) &= f\left( a \right) + \frac{f'\left( a \right) }{1!}\left( x-a \right) + \frac{f''\left( a \right) }{2!}\left( x-a \right)^2\\
	      a &= 0 \\
	      &= f\left( 0 \right) + f'\left( 0 \right) x + \frac{1}{2}f''\left( 0 \right) x^2 \\
	      &= \frac{1}{2}f''\left( 0 \right) x^2 \\
	      &= \frac{1}{2r}x^2
	    .\end{align*}
	  \item En este caso tenemos que primero despejar $y$ 
	    \begin{align*}
	      x^2 + \left( y-r \right)^2 &= r^2 \\
	      x^2 + y^2 -2yr + r^2 &= r^2 \\
	      x^2 + y^2 - 2yr &= 0 \\
	      y &= \frac{-b \pm \sqrt{b^2-4ac} }{2a} \\
	      &= \frac{2r\pm\sqrt{4r^2-4x^2} }{2} \\
	      &= \frac{2\left( r\pm\sqrt{r^2-x^2}  \right) }{2} \\
	      &= r\pm\sqrt{r^2-x^2}  \\
	    .\end{align*}

	    Ahora con esto, podemos sacar el polinomio de segundo orden que en este caso nos requiere saber:
	    \begin{align*}
	      y\left( 0 \right) &= r \pm \sqrt{r^2-0}= r \pm r = 0 \\
	      y' &= \pm \frac{2x}{2\sqrt{r^2 - x^2} }  \\
	      y'\left( 0 \right) &= \pm \frac{2\left( 0 \right) }{2\sqrt{r^2-0} }=0 \\
	      y'' &= \pm 2 \frac{r^2}{\left( r^2-x^2 \right)^{\frac{3}{2}}}\\
	      y''\left( 0 \right) &= \pm 2 \frac{r^2}{\left( r^2 \right)^{\frac{3}{2}}}=2 \frac{r^2}{r^{3}}=\frac{2}{r} \\
	      y\left( x \right) &= y\left( 0 \right) + y'\left( 0 \right) x + \frac{1}{2}y''\left( 0 \right) x^2\\
	      &= 0 + 0 +\frac{1}{r}x^2 \\
	      &= \frac{1}{r}x^2
	    .\end{align*}
	\end{enumerate}
      \item 
	\begin{enumerate}
	  \item Recta Tangente. Llamemos a los gradientes de $F=\Vec{n_1}$ y $G=\Vec{n_2}$ entonces si sacamos el producto cruz de ambos obtenemos un vector perpendicular a ambos que podremos reemplazar en la ecuación de la recta y encontrar su valor. En este caso tendremos $\Vec{v} = \Vec{n_1}\times \Vec{n_2}$. Y reemplazamos esto y nos da:
	 \begin{align*}
	   r\left( t \right) &= (0,0,0) + \Vec{v}t\\
			     &= \Vec{v}t
	 .\end{align*}
       \item En este caso utilizaremos:
	 \begin{align*}
	   \frac{dr}{ds} &= \frac{\frac{dr}{dt}}{\frac{ds}{dt}}\\
			 &= \frac{\Vec{v}}{ds} \\
	 .\end{align*}
       \item 

	\end{enumerate}
    \end{enumerate}
\end{document}

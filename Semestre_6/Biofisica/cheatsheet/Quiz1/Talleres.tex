  \documentclass[12pt]{exam}
\usepackage{amsthm}
\usepackage{libertine}
\usepackage[utf8]{inputenc}
\usepackage[margin=1in]{geometry}
\usepackage{amsmath,amssymb}
\usepackage{multicol}
\usepackage[shortlabels]{enumitem}
\usepackage{siunitx}
\usepackage{cancel}
\usepackage{graphicx}
\usepackage{pgfplots}
\usepackage{listings}
\usepackage{tikz}


\pgfplotsset{width=10cm,compat=1.9}
\usepgfplotslibrary{external}
\tikzexternalize

\newcommand{\class}{Moderna - Complementaria} % This is the name of the course 
\newcommand{\examnum}{Quiz 3} % This is the name of the assignment
\newcommand{\examdate}{\today} % This is the due date
\newcommand{\timelimit}{}





\begin{document}
\pagestyle{plain}
\thispagestyle{empty}

\noindent
\begin{tabular*}{\textwidth}{l @{\extracolsep{\fill}} r @{\extracolsep{6pt}} l}
	\textbf{\class} & \textbf{Name:} & \textit{Sergio Montoya}\\ %Your name here instead, obviously 
	\textbf{\examnum} &&\\
	\textbf{\examdate} &&
\end{tabular*}\\
\rule[2ex]{\textwidth}{2pt}
% ---

\begin{enumerate}
  \item \textbf{Volumen y Areas}
    \begin{enumerate}
      \item \textit{Esfera}
	\begin{itemize}
	  \item Volumen: $\frac{4}{3}\pi r^{3}$
	  \item Superficie: $4\pi r^2$
	\end{itemize}
      \item Cilindro
	\begin{itemize}
	  \item Volumen: $\pi r^2\cdot h$
	  \item Superficie: $2\pi rh + 2\pi r^2$
	\end{itemize}
    \end{enumerate}
  \item \textbf{Equivalencias: }
    \begin{enumerate}
      \item S.cereviciae: esfera de radio $2.6\mu m$
      \item Nucleo: Esfera de radio $20\mu m$
      \item Histona: Cilindro de $6nm$ de alto y radio $3.5nm$
      \item Bases: 150 bases por histona | $1.2\times 10^{7}$ en 16 cromosomas | $\frac{1nm^{3}}{bp}$
      \item Nucleosomas: $\frac{200 bases}{nucleosoma}$
      \item Densidad Empaque:  $\frac{bases \times medida}{v_{nucleo}}$
      \item Proteinas: $60\times E.coli$
      \item Lipidos: Tienen un Area de  $0.5nm^2$ |  $\frac{2\times 0.5\times A_{levadura}}{A_{lipido}}$
      \item Mitocondria: Hay 40 mitocondrias por celula | $V_{total}=40\times 10 \frac{4\pi}{3}\left( \frac{3}{8} \right)^{3}\mu m^3=9\mu m^{3}$
      \item Membrana Interna: Area superficial de $70\mu m^2$
      \item Reticulo Endoplasmatico: $V_{ER}=\frac{4\pi}{3}R_{afuera}^{3}-\frac{4\pi}{3}R_{adentro}^{3}$ | $A_{ER}= \frac{8\pi}{3}\left( R_{afuera}^{3}-R_{nucleo}^{3} \right)=15\times 10^{4}\mu m^2 $
    \end{enumerate}
\end{enumerate}

\end{document}

\documentclass{report}

\documentclass[12pt]{article}
\usepackage{array}
\usepackage{color}
\usepackage{amsthm}
\usepackage{eufrak}
\usepackage{lipsum}
\usepackage{pifont}
\usepackage{yfonts}
\usepackage{amsmath}
\usepackage{amssymb}
\usepackage{ccfonts}
\usepackage{comment} \usepackage{amsfonts}
\usepackage{fancyhdr}
\usepackage{graphicx}
\usepackage{listings}
\usepackage{mathrsfs}
\usepackage{setspace}
\usepackage{textcomp}
\usepackage{blindtext}
\usepackage{enumerate}
\usepackage{microtype}
\usepackage{xfakebold}
\usepackage{kantlipsum}
%\usepackage{draftwatermark}
\usepackage[spanish]{babel}
\usepackage[margin=1.5cm, top=2cm, bottom=2cm]{geometry}
\usepackage[framemethod=tikz]{mdframed}
\usepackage[colorlinks=true,citecolor=blue,linkcolor=red,urlcolor=magenta]{hyperref}

%//////////////////////////////////////////////////////
% Watermark configuration
%//////////////////////////////////////////////////////
%\SetWatermarkScale{4}
%\SetWatermarkColor{black}
%\SetWatermarkLightness{0.95}
%\SetWatermarkText{\texttt{Watermark}}

%//////////////////////////////////////////////////////
% Frame configuration
%//////////////////////////////////////////////////////
\newmdenv[tikzsetting={draw=gray,fill=white,fill opacity=0},backgroundcolor=none]{Frame}

%//////////////////////////////////////////////////////
% Font style configuration
%//////////////////////////////////////////////////////
\renewcommand{\familydefault}{\ttdefault}
\renewcommand{\rmdefault}{tt}

%//////////////////////////////////////////////////////
% Bold configuration
%//////////////////////////////////////////////////////
\newcommand{\fbseries}{\unskip\setBold\aftergroup\unsetBold\aftergroup\ignorespaces}
\makeatletter
\newcommand{\setBoldness}[1]{\def\fake@bold{#1}}
\makeatother

%//////////////////////////////////////////////////////
% Default font configuration
%//////////////////////////////////////////////////////
\DeclareFontFamily{\encodingdefault}{\ttdefault}{%
  \hyphenchar\font=\defaulthyphenchar
  \fontdimen2\font=0.33333em
  \fontdimen3\font=0.16667em
  \fontdimen4\font=0.11111em
  \fontdimen7\font=0.11111em}


\input{macros}
\input{letterfonts}

\title{\Huge{Topicos en Biofisica}\\Tarea 1}
\author{\huge{Sergio Montoya Ramirez}}
\date{}

\begin{document}

\maketitle
\newpage% or \cleardoublepage
% \pdfbookmark[<level>]{<title>}{<dest>}
\pdfbookmark[section]{\contentsname}{toc}
\tableofcontents
\pagebreak

\chapter{Pregunta 1}
\section{Geometria de E.Coli}

En este caso, nos piden utilizar la figura 2.1 del libro. En particular usaremos la $C$. Basados en esa imagen dividiremos a \textit{E.Coli} en dos figuras geometricas que nos permitan aproximarnos. Diremos que es un cilindro de radio $\frac{1}{2}\mu m$ y altura $1 \mu m$ junto con una esfera (partida en dos y que estara en cada lado) de radio  $\frac{1}{2} \mu m$ en este caso debemos calcular la superficie y volumen de ambas figuras.

\subsection{Volumen}

\begin{enumerate}
  \item \textbf{Cilindro}
    \begin{align*}
      V_{c} &= r^2\pi h \\
      &= \pi \frac{1}{4}\left( 1 \right) \mu m^{3}\\
      &= \frac{\pi}{4} \mu m^{3}
    .\end{align*}
  \item \textbf{Esfera}
    \begin{align*}
      V_{e} &= \frac{4}{3}\pi r^{3} \\
      &= \frac{4}{3}\pi \left( \frac{1}{2} \right)^{3} \\
      &= \frac{4}{3}\frac{\pi}{8} \\
      &= \frac{4\pi}{24}\mu m^{3}
    .\end{align*}
  \item \textbf{Total}
    \begin{align*}
      V_{t} &= V_{c} + V_{e} \\
     &= \frac{\pi}{4} + \frac{4\pi}{24} \\
     &= \frac{6\pi+4\pi}{24} \\
     &= \frac{10\pi}{24} \\
     &\approx 1 \mu m^{3}
    .\end{align*}
  \item \textbf{Conversión}
    \begin{align*}
      1 \mu m^{3} = 1 \mu m^{3} \frac{1 fl}{1 \mu m^{3}} = 1 fl
    .\end{align*}
\end{enumerate}

\subsection{Superficie}

\begin{enumerate}
  \item \textbf{Cilindro}
    \begin{align*}
      S_{c} &= 2\pi rh \\
      &= 2\pi \frac{1}{2} \mu m^2 \\
      &= \pi \mu m^2
    .\end{align*}
  \item \textbf{Esfera}
    \begin{align*}
      S_{e} &=  4 \pi r^2\\
      &= 4 \pi \frac{1}{4} \\
      &= \pi \mu m^2
    .\end{align*}
  \item \textbf{Total}
    \begin{align*}
      S_{t} &= S_{c} + S_{e} \\
      &= \pi + \pi \mu m^2 \\
      &= 2\pi \mu m^2 \\
      &\approx 6 \mu m^2
    .\end{align*}
\end{enumerate}

\subsection{Masa}
si asumimos que \textit{E.Coli} tiene una densidad similar a la del agua tendriamos:
\begin{align*}
  V_{t} &= 1 fl \\
  \rho &= 1 \frac{kg}{l} \frac{10^{-15}l}{1 fl} \\
  m &= v_{t}\rho = 10^{-15}kg \\ \\
  &= 1 pg
.\end{align*}

\section{Bacterias en el Cuerpo}

Desde el punto anterior sabemos cual es la masa aproximada de j=
\end{document}

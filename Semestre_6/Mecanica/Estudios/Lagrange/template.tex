\documentclass{report}

\documentclass[12pt]{article}
\usepackage{array}
\usepackage{color}
\usepackage{amsthm}
\usepackage{eufrak}
\usepackage{lipsum}
\usepackage{pifont}
\usepackage{yfonts}
\usepackage{amsmath}
\usepackage{amssymb}
\usepackage{ccfonts}
\usepackage{comment} \usepackage{amsfonts}
\usepackage{fancyhdr}
\usepackage{graphicx}
\usepackage{listings}
\usepackage{mathrsfs}
\usepackage{setspace}
\usepackage{textcomp}
\usepackage{blindtext}
\usepackage{enumerate}
\usepackage{microtype}
\usepackage{xfakebold}
\usepackage{kantlipsum}
%\usepackage{draftwatermark}
\usepackage[spanish]{babel}
\usepackage[margin=1.5cm, top=2cm, bottom=2cm]{geometry}
\usepackage[framemethod=tikz]{mdframed}
\usepackage[colorlinks=true,citecolor=blue,linkcolor=red,urlcolor=magenta]{hyperref}

%//////////////////////////////////////////////////////
% Watermark configuration
%//////////////////////////////////////////////////////
%\SetWatermarkScale{4}
%\SetWatermarkColor{black}
%\SetWatermarkLightness{0.95}
%\SetWatermarkText{\texttt{Watermark}}

%//////////////////////////////////////////////////////
% Frame configuration
%//////////////////////////////////////////////////////
\newmdenv[tikzsetting={draw=gray,fill=white,fill opacity=0},backgroundcolor=none]{Frame}

%//////////////////////////////////////////////////////
% Font style configuration
%//////////////////////////////////////////////////////
\renewcommand{\familydefault}{\ttdefault}
\renewcommand{\rmdefault}{tt}

%//////////////////////////////////////////////////////
% Bold configuration
%//////////////////////////////////////////////////////
\newcommand{\fbseries}{\unskip\setBold\aftergroup\unsetBold\aftergroup\ignorespaces}
\makeatletter
\newcommand{\setBoldness}[1]{\def\fake@bold{#1}}
\makeatother

%//////////////////////////////////////////////////////
% Default font configuration
%//////////////////////////////////////////////////////
\DeclareFontFamily{\encodingdefault}{\ttdefault}{%
  \hyphenchar\font=\defaulthyphenchar
  \fontdimen2\font=0.33333em
  \fontdimen3\font=0.16667em
  \fontdimen4\font=0.11111em
  \fontdimen7\font=0.11111em}


\input{macros}
\input{letterfonts}

\title{\Huge{Some Class}\\Random Examples}
\author{\huge{Your Name}}
\date{}

\begin{document}

\maketitle
\newpage% or \cleardoublepage
% \pdfbookmark[<level>]{<title>}{<dest>}
\pdfbookmark[section]{\contentsname}{toc}
\tableofcontents
\pagebreak

\chapter{Principio de D'Alembert para una particula}

Suponga que tiene una particula $m$ que se mueve en $R^{3}$ que a su ves tiene una ligadura \textbf{Holonoma}

\nt{Los tipos de Ligaduras son:
  \begin{enumerate}
    \item Holonomas: Ligaduras que dependen de la posición y el tiempo.
    \item No Holonomas: Ligaduras que dependen de la posición, el tiempo y la velocidad.
  \end{enumerate}
}

Esta ligadura tiene la forma: \[
  \ell \left( \Vec{r},t \right) = 0
.\] ahora con esto podemos aplicar la segunda ley de Newton:
\begin{align*}
  m \frac{d^2 \Vec{r}}{d t^2} = \Vec{F}_{ext} + \Vec{R}
.\end{align*}

donde $R$ es la fuerza de ligadura, es decir las fuerzas reacción de la ligadura.

Con esto obtenemos  $3$ ecuaciones diferenciales distintas (una por cada coordenada) con $6$ incognitas. En particular las incognitas son:
\begin{enumerate}
  \item $\Vec{r} \left( t \right) = \left( x,y,z \right) $ 
  \item $\Vec{R} \implies 3$
\end{enumerate}

pero ademas tenemos la ecuación de la ligadura. por lo que tenemos $4$ ecuaciónes. Aun asi, el numero de incognitas supera al de ecuaciones y en consecuencia (tal como esta planteado) es irresoluble. Ahora desarrollemos por otro lado.
\end{document}

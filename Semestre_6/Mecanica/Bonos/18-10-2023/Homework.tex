\documentclass[12pt]{article}
	
\usepackage[margin=1in]{geometry}		% For setting margins
\usepackage{amsmath}				% For Math
\usepackage{fancyhdr}				% For fancy header/footer
\usepackage{graphicx}				% For including figure/image
\usepackage{cancel}					% To use the slash to cancel out stuff in work

%%%%%%%%%%%%%%%%%%%%%%
% Set up fancy header/footer
\pagestyle{fancy}
\fancyhead[LO,L]{Sergio Montoya Ramirez}
\fancyhead[CO,C]{Mecanica - Bono 18-10-2023}
\fancyhead[RO,R]{\today}
\fancyfoot[LO,L]{}
\fancyfoot[CO,C]{\thepage}
\fancyfoot[RO,R]{}
\renewcommand{\headrulewidth}{0.4pt}
\renewcommand{\footrulewidth}{0.4pt}
%%%%%%%%%%%%%%%%%%%%%%

\begin{document}
\section*{Descripción del Problema}

Imaginese usted que tiene una cuenta en una bara infinita y que estamos en un universo en el que los campos no tienen efecto (o su efecto es tan minimo que podemos ignorarlo). Ahora bien, suponga que esta barra esta girando a una velocidad angular constante.

\section*{Analisis Intuitivo}

Una de las primeras cosas que nos podemos dar cuenta es que el sistema mas eficiente para desarrollar este problema es con coordenadas esfericas. Por lo tanto, vamos a tomar estas coordenadas como base y con eso nos damos cuenta de varias cosas bastante interesantes. Primero, notemos que en el caso en el que estemos en un sistema bidimensional (es decir, asumiendo que la barra no tiene una inclinación) podemos simplificar el problema con esto:
\begin{enumerate}
  \item La velocidad angular de la cuenta y de la barra es la misma.
  \item La cuenta experimentaria una fuerza centrifuga.
\end{enumerate}

\section*{Desarrollo $\varphi\left( t \right) $ }

En este caso sabemos que la velocidad angular es la misma que en el caso de la barra. Por lo tanto, la descripción de la velocidad en $\varphi\left( t \right) $. Por lo tanto, simplemente necesitamos notar que el angulo varia a una velocidad constante que es $\omega$. Por lo tanto, y asumiendo que podemos plantear que inicia en $0$ (cosa que podemos pues la falta de marcos de referencia nos lo permite) nos damos cuenta que el angulo se describe como $\omega t$

\section*{Desarrollo $r\left( t \right) $ }

Ahora bien, en este caso sabemos que la fuerza que experimentaria la cuenta es fuerza centrifuga que haria que su velocidad se acelerara (ya con esto podemos ir viendo que terminara siendo una espiral logaritmica). La fuerza centrifuga es \[
F_{cf} = m\omega\times \left( \omega \times r \right) 
.\] que en este caso desarrollamos:
\begin{align*}
  ma = m\omega^2r\\
  a = \omega^2r
.\end{align*}

\end{document}

\documentclass[a4paper, amsfonts, amssymb, amsmath, reprint, showkeys, nofootinbib, twoside]{revtex4-1}
\usepackage[spanish]{babel}
\usepackage[utf8]{inputenc}
\usepackage{float}
\usepackage[colorinlistoftodos, color=green!40, prependcaption]{todonotes}
\usepackage{amsthm}
\usepackage{mathtools}
\usepackage{physics}
\usepackage{xcolor}
\usepackage{graphicx}
\usepackage[left=23mm,right=13mm,top=35mm,columnsep=15pt]{geometry}
\usepackage{adjustbox}
\usepackage{placeins}
\usepackage[T1]{fontenc}
\usepackage{lipsum}
\usepackage{tikz}
\usepackage{circuitikz}
\usepackage{csquotes}
\usepackage[normalem]{ulem}
\useunder{\uline}{\ul}{}
\usepackage[pdftex, pdftitle={Article}, pdfauthor={Author}]{hyperref} % For hyperlinks in the PDF
%\setlength{\marginparwidth}{2.5cm}
\bibliographystyle{apsrev4-1}

\begin{document}

%El título del experimento realizado es importante.
\title{Analisis de circuitos resistivos. Manejo de multímetro}


\author{Sergio Montoya Ramirez}
\email[Correo institucional: ]{s.montoyar2@uniandes.edu.co}

%Si necesitan poner un segundo autor, deben eliminar los porcentajes (%) iniciales.

\affiliation{Universidad de los Andes, Bogotá, Colombia.}

\date{\today} % Si lo dejan vacío no les saldrá fecha. La fecha que se muestra es del día en que se compila.

\begin{abstract}

  Este texto es el preinforme de la sesión 2 del experimento 3 del curso Electronica para Ciencias de la Universidad de los Andes durante el semestre 202310. Tiene como objetivo la preparación para el posterior desarrollo de la sesión previamente enunciada.

\end{abstract}

\maketitle

\section{Thévenin y Norton}

En este caso, partimos de el siguiente centimetro

\begin{figure}[h!]
  \begin{center}
        \begin{circuitikz}
      \draw (0,0)
      to[battery2,v_=$5V$,*-] (0,2) % The voltage source
      to[R=$2k$] (2,2)
      to[R=$20k$,*-*] (2,0) % The resistor
      to[short] node[ground] {} (0,0);
      \draw (2,2)
      to[short] (4,2)
      to[R=$4.7k$,*-*] (4,0)
      to[short] (2,0);
      \draw (4,2)
      to[R=$1k$,-*] (6,2)
      (4,0) to[short,-*] (6,0);
   \end{circuitikz}
    \caption{Circuito inicial}
  \end{center}
\end{figure}

Este circuito es equivalente (por equivalencia de fuentes) a este otro:
\begin{figure}[h!]
  \begin{center}
        \begin{circuitikz}
      \draw (0,0)
      to[V,v=$2.5mA$,*-] (0,2) % The voltage source
      to[short] (2,2)
      to[R=$2k$,*-*] (2,0) % The resistor
      to[short] node[ground] {} (0,0);
      \draw (2,2)
      to[short] (3,2)
      to[R=$20k$,*-*] (3,0)
      to[short] (2,0);
      \draw (3,2)
      to[short] (4,2)
      (4,2) to[R=$4.7k$,*-*] (4,0)
      (3,0) to[short] (4,0);
      \draw (4,2)
      to[R=$1k$,-*] (6,2)
      (4,0) to[short,-*] (6,0);
   \end{circuitikz}
    \caption{Circuito en donde reemplazamos las equivalencias de fuentes}
  \end{center}
\end{figure}

Ahora con esto podemos calcular la resistencia de Thevenin con lo cual podemos sumar los inversos de cada resistencia en paralelo como sigue:

\begin{figure}[h!]
  \begin{center}
        \begin{circuitikz}
      \draw (0,0)
      to[R=$2k$] (0,2) % The voltage source
      to[short] (2,2)
      to[R=$20k$,*-*] (2,0) % The resistor
      to[short] node[ground] {} (0,0);
      \draw (2,2)
      to[short] (4,2)
      to[R=$4.7k$,*-*] (4,0)
      to[short] (2,0);
      \draw (4,2)
      to[R=$1k$,-*] (6,2)
      (4,0) to[short,-*] (6,0);
   \end{circuitikz}
    \caption{Circuito inicial}
  \end{center}
\end{figure}

\begin{align*}
  R_{||}= \left( \frac{1}{2k} + \frac{1}{20k} + \frac{1}{4.7k} \right)^{-1} \\
  = 1.3110
.\end{align*}

que fue obtenido con calculadora. Entonces en este momento el circuito se ve de este estilo:
\begin{figure}[h!]
  \begin{center}
    \begin{circuitikz}
      \draw(0,0)
      to[R=$1.311k$ ](0,2)
      (0,0) to[short] (2,0)
      (0,2) to[R=$1k$] (2,2);
    \end{circuitikz}
  \end{center}
\end{figure}
que entonces nos queda con una resistencia de:
\begin{align*}
  R_{total} &= 1.311k + 1k \\
  &= 2.3k
.\end{align*}

Ahora bien, para calcular el voltaje de circuito abierto nos aprovechamos de este circuito anterior y le ponemos la fuente (recordemos que era de corriente por que habiamos hecho el cambio justo antes) y podemos entonces saber cuanto deberia valer esta fuente de voltaje que seria:
\begin{align*}
  V_{f}&= I_{f}R_{||} = 2.5 \cdot 1.311 \approx 3.3 V \\
.\end{align*}
con lo cual quedamos con el circuito:

\begin{figure}[h!]
  \begin{center}
    \begin{circuitikz}
      \draw(0,0)
      to[battery2,v_=$3.3V$](0,2)
      (0,0) to[short] (4,0)
      (0,2) to[R=$1.311k$] (2,2)
      to[R=$1k$ ] (4,2);
    \end{circuitikz}
  \end{center}
\end{figure}

Por lo tanto, se tiene que el circuito de thevenin es 
\begin{figure}[h!]
  \begin{center}
    \begin{circuitikz}
      \draw(0,0)
      to[battery2,v_=$3.3V$](0,2)
      (0,0) to[short] (2,0)
      (0,2) to[R=$2.311k$] (2,2);
    \end{circuitikz}
  \end{center}
\end{figure}

y la equivalencia de Norton
\begin{figure}[h!]
  \begin{center}
    \begin{circuitikz}
      \draw(0,0)
      to[V,v=$1.4mA$](0,2)
      (0,0) to[short] (4,0)
    (2,2) to[R=$2.311$ ] (2,0)
      (0,2) to[short] (4,2);
    \end{circuitikz}
  \end{center}
\end{figure}
\section{Calcular Voltajes}
El circuito general es:
\begin{figure}[h!]
  \begin{center}
    \begin{circuitikz}
      \draw(0,0)
      to[battery2,v_=$3.3V$] (0,2)
      to[R=$10k$ ](2,2)
      to[R=$1k$] (2,0) node[ground] {} to[short] (0,0);
      \draw(2,2)
      to[R=$2.2k$ ](4,2)
      to[battery2,v_=$5V$ ] (4,0)
      to[short] (2,0);
    \end{circuitikz}
  \end{center}
\end{figure}


Ahora tenemos que encontrar para cada resistencia y cada rama su corriente en diferentes cirscunstancias
\subsection{}

Donde llamaremos nodo $A$ al nodo entre la fuente de y la resistencia de $10k$, nodo B al nodo entre las 3 resistencias y nodo C al unico que queda.

Con la fuente de $3.3V$ y la otra apagada. Con lo cual el circuito queda:
\begin{figure}[h!]
  \begin{center}
    \begin{circuitikz}
      \draw(0,0)
      to[battery2,v_=$3.3V$] (0,2)
      to[R=$10k$ ](2,2)
      to[R=$1k$] (2,0)
      node[ground] {} to[short] (0,0);
      \draw(2,2)
      to[R=$2.2k$ ](4,2)
      to[short ] (4,0)
      to[short] (2,0);
    \end{circuitikz}
  \end{center}
\end{figure}

con lo cual tenemos
\begin{align*}
  \text{Nodo A: }I_1=I_2\\
  \text{Nodo B:} I_2=I_3+I_4\\
  \text{Nodo C:} I_1 =I_3+I_4\\
  \text{10k: } V_A-V_B = I_2\cdot 10k\\
  \text{1k:} V_B-V_C = I_4\cdot 1k\\
  \text{2.2k:} V_A-V_C = I_{3}\cdot 2.2k\\
  \text{Fuente: }V_{A}-V_{C}&= 3.3V \\
  \text{Tierra:} V_C=0
.\end{align*}

Con estas ecuaciones podemos encontrar el voltaje de $V_{B}$ con lo siguiente
\begin{align*}
  V_{C}&= 0 \\
  V_{A}&= 3.3 V \\
  V_{B}&= V_{A}-I_2\cdot 10k \\
  &= V_{A}-\left( I_3+I_4 \right) 10k \\
  &= V_{A}-V_{B}\left( \frac{1}{1k}+\frac{1}{2.2k} \right) 10k \\
  \left( 1+\left( \frac{3.2}{2.2k}10k \right)  \right) V_{B}&= V_{A} \\
  V_{B}&= \frac{V_{A}}{1+\frac{3.2}{2.2k}}=0.21V\\
  I_{4}&= \frac{V_{B}}{2.2k}=0.095 \\
  I_{3}&= \frac{V_{B}}{1k}=0.21 \\
  I_{1}&= I_{2}=0.305
.\end{align*}

Ahora para calcular por ramas tenemos
\begin{align*}
  I_{izquierda}=I_{1}&=I_{2}=0.305mA\\
  I_{derecha}=I_{4}&=0.095mA  \\
  I_{centro}&= 0.21mA
.\end{align*}

\subsection{}

Donde llamaremos nodo $A$ al nodo entre la fuente de y la resistencia de $10k$, nodo B al nodo entre las 3 resistencias y nodo C al unico que queda.

Con la fuente de $3.3$ Apagada y la de $5V$ prendida tendriamos
\begin{figure}[h!]
  \begin{center}
    \begin{circuitikz}
      \draw(0,0)
      to[short] (0,2)
      to[R=$10k$ ](2,2)
      to[R=$1k$] (2,0)
      node[ground] {} to[short] (0,0);
      \draw(2,2)
      to[R=$2.2k$ ](4,2)
      to[battery2,v_=$5V$ ] (4,0)
      to[short] (2,0);
    \end{circuitikz}
  \end{center}
\end{figure}

\begin{align*}
  \text{Nodo A: }I_1=I_2\\
  \text{Nodo B:} I_2=I_3+I_4\\
  \text{Nodo C:} I_1 =I_3+I_4\\
  \text{10k: } V_B-V_C = I_4\cdot 10k\\
  \text{1k:} V_B-V_C = I_3\cdot 1k\\
  \text{2.2k:} V_A-V_B = I_{2}\cdot 2.2k\\
  \text{Fuente: } V_A - V_C = 5V\\
  \text{Tierra:} V_C=0
.\end{align*}

Ahora entonces
\begin{align*}
  V_{B}&= V_{A}-I_{2}\cdot 2.2k=V_{A}-\left( I_{3}+I_{4} \right) 2.2k \\
  &= V_{A}-\left( \frac{1}{1k}+\frac{1}{10k} \right)V_{B}2.2k  \\
  \left( 1+\frac{24.2}{10} \right) V_{B}&= V_{A} \\
  V_{B}&= \frac{V_{A}}{\frac{34.2}{10}}=1.46V \\
  I_{4}&= \frac{V_B}{10k}=0.146mA \\
  I_{3}&= \frac{V_{B}}{1k}=1.46 mA \\
  I_{2}&= I_1=1.606 mA
.\end{align*}

Ahora bien:
\begin{align*}
  V_{2.2k} &= V_A - V_B = 5V-1.46V=3.54V \\
  V_{10k} &= V_B - V_C = 1.46V \\
  V_{1k} &= V_B - V_C = 1.46V
.\end{align*}
y por ramas

\begin{align*}
  I_{izquierda}&= I_{4}=0.146mA \\
  I_{derecha}&= I_1=1.606mA \\
  I_{centro}&= 1.46mA
.\end{align*}

\subsection{}
Con las dos fuentes encendidas debemos usar el principio de superposición
\begin{align*}
  V_{2.2k}&= 3.54V-0.21V=3.33V \\
  V_{10k}&= 1.46V-3.09V=-1.63V \\
  V_{1k} &= 1.46V+0.21V=1.67V \\
  I_{centro} &= 1.67 \\
  I_{izquierda} &= -0.159 mA \\
  I_{derecha} &= 1.511 mA
.\end{align*}
\section{Nota:}
Para la realización de este preinforme se utilizaron las diapositivas vistas en la clase del lunes 28 de agosto y se hablo con compañeros para la solución de inquietudes en los ejercicios.
\end{document}

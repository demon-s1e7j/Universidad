  \documentclass[12pt]{exam}
\usepackage{amsthm}
\usepackage{libertine}
\usepackage[utf8]{inputenc}
\usepackage[margin=1in]{geometry}
\usepackage{amsmath,amssymb}
\usepackage{multicol}
\usepackage[shortlabels]{enumitem}
\usepackage{siunitx}
\usepackage{cancel}
\usepackage{graphicx}
\usepackage{pgfplots}
\usepackage{listings}
\usepackage{tikz}
\usepackage{circuitikz} \usepackage{float}

\pgfplotsset{width=10cm,compat=1.9}
\usepgfplotslibrary{external}
\tikzexternalize

\newcommand{\class}{Electronica Para Ciencias} % This is the name of the course 
\newcommand{\examnum}{Tarea 1} % This is the name of the assignment
\newcommand{\examdate}{\today} % This is the due date
\newcommand{\timelimit}{}





\begin{document}
\pagestyle{plain}
\thispagestyle{empty}

\noindent
\begin{tabular*}{\textwidth}{l @{\extracolsep{\fill}} r @{\extracolsep{6pt}} l}
	\textbf{\class} & \textbf{Name:} & \textit{:0}\\ %Your name here instead, obviously 
	\textbf{\examnum} &&\\
	\textbf{\examdate} &&
\end{tabular*}\\
\rule[2ex]{\textwidth}{2pt}
% ---

\begin{enumerate}
  \item Circuito Inicial:
    \begin{figure}[H]
      \begin{center}
        \begin{circuitikz}
          \draw(0,0)
	  to[V,v=$3mA$ ](0,2)
	  to[short](2,2)
	  to[R=$2k$ ](2,0)
	  to[short](0,0);
	  \draw(2,2)
	  to[battery,v_=$2V$ ](5,2)
	  to[R=$1k$](5,0)
	  to[short](2,0);
	  \draw(5,2)
	  to[R=$3k$ ](8,2)
	  to[battery,v_=$10V$ ](8,0)
	  to[short](5,0);
        \end{circuitikz}
      \end{center}
    \end{figure}

    Lo primeo que hacemos es por equivalencia de fuentes volver $2k$ en serie con una fuente de voltaje de $5V$

    En este caso debemos trabajar con cada una de las fuentes por aparte apagando las otras cuando no estemos trabajando con ellas:

    Iniciamos con:
    \begin{figure}[H]
      \begin{center}
        \begin{circuitikz}
          \draw(0,0)
	  to[battery,v_=$5V$ ](0,2)
	  to[R=$2k$](2,2)
	  to[R= $1k$](2,0)
	  to[short](0,0);
	  \draw(2,2)
	  to[R= $3k$ ](4,2)
	  to[short](4,0)
	  to[short](2,0);
        \end{circuitikz}
      \end{center}
    \end{figure}

    \begin{itemize}
      \item Nodo A: $I_1=I_2$
      \item Nodo B: $I_2=I_3+I_4$ 
      \item Nodo C: $I_3+I_4=I_1$ 
      \item $R_2$ : $V_A-V_B=I_2\cdot 2k$
      \item $R_1$ : $V_B-V_C=I_3\cdot 1k$ 
      \item $R_3$ : $V_B - V_C=I_4\cdot 3k$
      \item Fuente : $V_A - V_C = 5V$ 
      \item Tierra : $V_C = 0$ 
    \end{itemize}

    \begin{align*}
      V_B = V_AI_2\left( 2k \right) &= V_A - \left( I_3 + I_4 \right) 2k \\
      &= V_A - \left( \frac{V_B}{1k} + \frac{V_B}{3k} \right) 2k \\
      V_B + 2V_B + \frac{2}{3}V_B &= V_A \\
      V_B &= \frac{18}{11}V \\
      I_3 &= \frac{18}{11}mA\\
      I_4 &= \frac{18}{33}mA
    .\end{align*}

    \begin{figure}[H]
      \begin{center}
        \begin{circuitikz}
          \draw(0,0)
	  to[R=$2k$ ](0,2)
	  to[battery, v_ = $2V$ ](2,2)
	  to[R=$1k$ ](2,0)
	  to[short] node[ground] {} (0,0);
	  \draw(2,2)
	  to[R=$3k$ ](4,2)
	  to[short](4,0)
	  to[short](2,0);
        \end{circuitikz}
      \end{center}
    \end{figure}

    \begin{itemize}
      \item Nodo A: $I_4=I_1$
      \item Nodo B: $I_1=I_2+I_3$ 
      \item Nodo C: $I_2+I_3=I_4$ 
      \item $R_2$ : $V_C-V_A=I_4\cdot 2k$
      \item $R_1$ : $V_B-V_C=I_3\cdot 1k$ 
      \item $R_3$ : $V_B - V_C=I_2\cdot 3k$
      \item Fuente : $V_B - V_A = 2V$ 
      \item Tierra : $V_C = 0$ 
    \end{itemize}

    \begin{align*}
      V_C - V_A &= -V_A\\
      -V_A &= \left( I_2+I_3 \right) 2k \\
      &= \frac{4}{3}V_B \\
     \frac{4}{3}V_B &= 2V-V_B \\
     \frac{11}{3}V_B &= 2V \\
     V_B &= \frac{6}{11}V \\
     V_A &= V_B - 2V \\
     &= -\frac{16}{11}V \\
     I_2 &= \frac{6}{33}mA =  \frac{2}{11}\\
     I_3 &= \frac{6}{11}\\
     I_1 &= I_4 =  \frac{8}{11}mA
    .\end{align*}

    \begin{figure}[H]
      \begin{center}
        \begin{circuitikz}
          \draw(0,0)
	  to[R=$2k$ ](0,2)
	  to[short](2,2)
	  to[R=$1k$ ] node[ground] {} (2,0)
	  to[short](0,0);
	  \draw(2,0)
	  to[short](4,0)
	  to[battery, v_=$10V$ ](4,2)
	  to[R=$3k$ ](2,2);
        \end{circuitikz}
      \end{center}
    \end{figure}

    \begin{itemize}
      \item Nodo A: $I_1=I_2$
      \item Nodo B: $I_2=I_4+I_3$ 
      \item Nodo C: $I_4+I_3=I_2$ 
      \item $R_2$ : $V_B-V_C=I_4\cdot 2k$
      \item $R_1$ : $V_B-V_C=I_3\cdot 1k$ 
      \item $R_3$ : $V_A - V_B=I_2\cdot 3k$
      \item Fuente : $V_A - V_C = 10V$ 
      \item Tierra : $V_C = 0$ 
    \end{itemize}

    \begin{align*}
      V_B &= V_A - I_2 3k = V_A - \left( I_4 + I_3 \right) 3k \\
      &= V_A - \left( \frac{V_B}{2k} + \frac{V_B}{1k} \right) 3k \\
      V_B + \frac{3}{2}V_B + 3V_B &= V_A\\
      \frac{11}{2}V_B &= 10V \\
      V_B &= \frac{20}{11}V \\
      I_3 &= \frac{V_B}{1k}=\frac{20}{11}mA \\
      I_4 &= \frac{V_B}{2k} = \frac{10}{11}mA \\
      I_2 &= I_1 = \frac{30}{11}mA \\
    .\end{align*}

    Ahora con esto sumamos todos los aportes y queda:
    \begin{itemize}
      \item Fuente $2V$ : $\frac{24}{11}mA + \frac{8}{11}mA - \frac{10}{11}mA=\frac{22}{11}mA=2mA$
      \item Fuente $10V$ : $-\frac{6}{11}mA - \frac{2}{11}mA + \frac{30}{11}mA=\frac{22}{11}mA=2mA$ 
      \item Fuente $3mA$ : $6V - \frac{16}{11}V + \frac{20}{11}V = \frac{70}{11}V$
    \end{itemize}

  \item 

    El circuito con el que vamos a trabajar es:
    \begin{figure}[H]
      \begin{center}
        \begin{circuitikz}
          \draw(0,2)
	  to[short,o-*](1,2)
	  to[R=$30$,*-*](1,0)
	  to[short,*-o](0,0);
	  \draw(1,0)
	  to[R=$40$,*-*](4,0)
	  to[V,v=$2A$,*-*](4,2)
	  to[R=$20$,*-*](1,2);
	  \draw(4,0)
	  to[short](7,0)
	  to[R=$10$ ](7,2)
	  to[battery,v_=$10V$ ](4,2);
        \end{circuitikz}
      \end{center}
    \end{figure}

    Ahora bien, para el voltaje de Thevenin aplicamos equivalencia de fuentes en el circuito original desarrollando como sigue:

    \begin{figure}[H]
      \begin{center}
        \begin{circuitikz}
          \draw(0,2)
	  to[short,o-*](1,2)
	  to[R=$30$,*-*](1,0)
	  to[short,*-o](0,0);
	  \draw(1,0)
	  to[R=$40$,*-*](4,0)
	  to[V,v=$2A$,*-*](4,2)
	  to[R=$20$,*-*](1,2);
	  \draw(4,0)
	  to[short](7,0)
	  to[R=$10$ ](7,2)
	  to[battery,v_=$10V$ ](4,2);
        \end{circuitikz}
      \end{center}
    \end{figure}
   
    \begin{figure}[H]
      \begin{center}
        \begin{circuitikz}
          \draw(0,2)
	  to[short,o-*](1,2)
	  to[R=$30$,*-*](1,0)
	  to[short,*-o](0,0);
	  \draw(1,0)
	  to[R=$40$,*-*](4,0)
	  to[V,v=$2A$,*-*](4,2)
	  to[R=$20$,*-*](1,2);
	  \draw(4,0)
	  to[short](6,0)
	  to[R=$10$ ](6,2)
	  to[short](4,2);
	  \draw(6,0)
	  to[short](8,0)
	  to[V,v=$1A$ ](8,2)
	  to[short](6,2);
        \end{circuitikz}
      \end{center}
      \caption{Para esta se utilizo equivalencia de una fuente de voltaje de $10V$ con una resistencia de $10$ por lo que queda una fuente de $I = \frac{V}{R} = 1A$ }
    \end{figure}

    \begin{figure}[H]
      \begin{center}
        \begin{circuitikz}
          \draw(0,2)
	  to[short,o-*](1,2)
	  to[R=$30$,*-*](1,0)
	  to[short,*-o](0,0);
	  \draw(1,0)
	  to[R=$40$](4,0)
	  (4,2)to[R=$20$](1,2);
	  \draw(4,0)
	  to[short](6,0)
	  (6,2)to[R=$10$](4,2);
	  \draw(6,0)
	  to[short](8,0)
	  to[V,v=$1A$ ](8,2)
	  to[battery,v_=$20V$](6,2);
        \end{circuitikz}
      \end{center}
      \caption{Para esto se utilizo equivalencia de una fuente de corriente $2A$ con la resistencia de $10$ por lo que queda una fuente $V = IR = 20V$ }
    \end{figure}

    \begin{figure}[H]
      \begin{center}
        \begin{circuitikz}
          \draw(0,2)
	  to[short,o-*](1,2)
	  to[R=$30$,*-*](1,0)
	  to[short,*-o](0,0);
	  \draw(1,0)
	  to[R=$40$](4,0)
	  (4,2)to[R=$30$](1,2);
	  \draw(4,0)
	  to[short](6,0)
	  to[V,v=$1A$ ](6,2)
	  to[battery,v_=$20V$ ](4,2);
        \end{circuitikz}
      \end{center}
    \end{figure}

    \begin{figure}[H]
      \begin{center}
        \begin{circuitikz}
          \draw(0,2)
	  to[short,o-*](1,2)
	  to[R=$30$,*-*](1,0)
	  to[short,*-o](0,0);
	  \draw(1,0)
	  to[R=$40$](4,0)
	  to[R=$30$ ](4,2)
	  to[short](1,2);
	  \draw(4,0)
	  to[short](6,0)
	  to[V,v=$1A$ ](6,2)
	  to[short](4,2);
	  \draw(6,0)
	  to[short](8,0)
	  to[V,v=$0.6A$ ](8,2)
	  to[short](6,2);
        \end{circuitikz}
      \end{center}
      \caption{Para esto se utilizo equivalencia de una fuente de Voltaje $20V$ con una resistencia de $30$ por lo que queda $I=\frac{V}{R}=0.6A$ }
    \end{figure}

    \begin{figure}[H]
      \begin{center}
        \begin{circuitikz}
          \draw(0,2)
	  to[short,o-*](1,2)
	  to[R=$30$,*-*](1,0)
	  to[short,*-o](0,0);
	  \draw(1,0)
	  to[R=$40$](4,0)
	  (4,2)to[short](1,2);
	  \draw(4,0)
	  to[R=$30$](6,0)
	  (6,2)to[short](4,2);
	  \draw(6,2)
	  to[short](10,2)
	  (10,0)to[V,v=$0.6A$ ](10,2)
	  (10,0)to[battery,v_=$30V$](6,0);
        \end{circuitikz}
      \end{center}
      \caption{Para esto se utilizo equivalencia de una fuente de corriente $1A$ y una resistencia $30$ por lo que queda $V=IR=30V$ }
    \end{figure}

    \begin{figure}[H]
      \begin{center}
        \begin{circuitikz}
          \draw(0,2)
	  to[short,o-*](1,2)
	  to[R=$30$,*-*](1,0)
	  to[short,*-o](0,0);
	  \draw(1,0)
	  to[R=$70$ ](4,0)
	  (6,0)to[battery,v_=$30V$ ](4,0)
	  (6,0)to[V,v_=$0.6A$ ](6,2)
	  to[short](1,2);
        \end{circuitikz}
      \end{center}
    \end{figure}

    \begin{figure}[H]
      \begin{center}
        \begin{circuitikz}
          \draw(0,2)
	  to[short,o-*](1,2)
	  to[R=$30$,*-*](1,0)
	  to[short,*-o](0,0);
	  \draw(1,0)
	  to[short](4,0)
	  to[V,v=$0.6A$ ](4,2)
	  to[short](1,2);
	  \draw(4,0)
	  to[short](6,0)
	  to[R=$70$ ](6,2)
	  to[short](4,2);
	  \draw(6,0)
	  to[short](8,0)
	  to[V,v=$0.4A$ ](8,2)
	  to[short](6,2);
        \end{circuitikz}
      \end{center}
      \caption{Para esto se utilizo equivalencia de una fuente de Voltaje $30V$ con una resistencia de $70$ por lo que queda  $I = \frac{V}{R}=0.4A$ }
    \end{figure}

    \begin{figure}[H]
      \begin{center}
        \begin{circuitikz}
	  \draw(0,2)
	  to[R=$30$,o-](2,2)
	  to[battery,v_=$18V$ ](4,2)
	  to[R=$70$ ](4,0)
	  (0,0)to[battery,v_=$28V$,o-](4,0);
        \end{circuitikz}
      \end{center}
      \caption{Para esto se utilizo equivalencia entre dos fuentes una de $0.6A$ y una de $0.4A$ que estaban relacionadas en paralelo con resistencias de $30$ y $70$ respectivamente por lo que queda $V=I_1R_1=18$ y $V=I_2R_2=28$}
    \end{figure}

    Ahora con esto resulta mucho mas facil encontrar el voltaje de Thevenin dado que es un circuito abierto sabemos que $I=0$ y por lo tanto, el voltaje de las resistencias es $0$ y en consecuencia podemos sumarlo y nos da: \[
      10V
    .\] 

    Ahora bien, para calcular la corriente de corto circuito cerramos el circuito anterior y analisamos 
      
    \begin{figure}[H]
      \begin{center}
        \begin{circuitikz}
	  \draw(0,0)
	  to[short,*-*](0,2)
	  to[R=$30$](2,2)
	  to[battery,v_=$18V$ ](4,2)
	  to[R=$70$ ](4,0)
	  (0,0)to[battery,v_=$28V$](4,0);
        \end{circuitikz}
      \end{center}
    \end{figure}

    Ahora bien, si analizamos este circuito por mallas sabemos que:
    \begin{align*}
      28V = 18V + I\cdot 70 + I\cdot 30	\\
      10V = I\left( 100 \right) \\
      I = \frac{1}{10}A
    .\end{align*}

    Con estos dos valores ya encontrados podemos calcular la resistencia equivalente con el ultimo circuito que es simplemente apagar las dos fuentes y nos quedamos con:

    \begin{figure}[H]
      \begin{center}
        \begin{circuitikz}
	  \draw(0,2)
	  to[R=$30$,o-](2,2)
	  to[short](4,2)
	  to[R=$70$ ](4,0)
	  to[short,-o](0,0);
        \end{circuitikz}
      \end{center}
    \end{figure}

    Con lo que queda que $R_{eq}=100$ y tambien lo podemos mostrar con: \[
      R_{eq}=\frac{V_{th}}{I_{th}}=\frac{10}{\frac{1}{10}} = 100
    .\] 

    Ahora para el circuito de Thevenin simplemente tenemos que:

    \begin{figure}[H]
      \begin{center}
        \begin{circuitikz}
	  \draw(0,0)
	  to[short](2,0)
	  to[battery,v_=$10V$ ](2,2)
	  to[R=$100$ ](0,2);
        \end{circuitikz}
      \end{center}
    \end{figure}
    Ahora para el circuito de Norton simplemente tenemos que:

    \begin{figure}[H]
      \begin{center}
        \begin{circuitikz}
          \draw(0,0)
	  to[short](2,0)
	  to[R=$100$ ](2,2)
	  to[short](0,2);
	  \draw(2,0)
	  to[short](4,0)
	  to[V,v=$\frac{1}{10}$](4,2)
	  to[short](2,2);
        \end{circuitikz}
      \end{center}
    \end{figure}

  \item Trabajaremos con el circuito:

    \begin{figure}[H]
      \begin{center}
        \begin{circuitikz}
         \draw(0,0)
	 to[R=$3k$ ](0,4)
	 to[short](2,4)
	 to[battery,v_=$3V$ ](2,2)
	 to[C=$1u$](2,0)
	 to[short](0,0);
	 \draw(2,2)
	 to[switch](4,2)
	 to[short](4,4)
	 to[R=$4k$ ](2,4);
	 \draw(2,0)
	 to[R=$2k$ ](4,0)
	 to[short](4,2);
        \end{circuitikz}
      \end{center}
    \end{figure}

    En este caso para $t=0$ podemos entender el capacitor como un circuito abierto por lo que $I_4 = 0$ y con esto:
    \begin{align*}
      I_3 &= \frac{V_B}{2k}\\
      I_5 &= \frac{V_A}{3k} \\
      I_3 + I_5 &= \frac{V_B}{2k} + \frac{V_A}{3k}= 0 = I_4 \\
      V_A - V_B &= V_A - \frac{2V_A}{3} = \frac{V_A}{3} = 3V \\
      V_A &= 9V \\
      I_3 &= I_2 = \frac{V_A - V_B}{4k\Omega}=\frac{3}{4}mA \\
    .\end{align*}

\end{enumerate}

\end{document}

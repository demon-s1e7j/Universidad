\documentclass[12pt]{exam}
\usepackage{amsthm}
\usepackage{libertine}
\usepackage[utf8]{inputenc}
\usepackage[margin=1in]{geometry}
\usepackage{amsmath,amssymb}
\usepackage{multicol}
\usepackage[shortlabels]{enumitem}
\usepackage{siunitx}
\usepackage{cancel}
\usepackage{graphicx}
\usepackage{pgfplots}
\usepackage{listings}
\usepackage{tikz}


\pgfplotsset{width=10cm,compat=1.9}
\usepgfplotslibrary{external}
\tikzexternalize

\newcommand{\class}{Cosmos} % This is the name of the course 
\newcommand{\examnum}{Ensayo Final} % This is the name of the assignment
\newcommand{\examdate}{02/12/2022} % This is the due date
\newcommand{\timelimit}{}
\newenvironment{Figura}
  {\par\medskip\noindent\minipage{\linewidth}}
  {\endminipage\par\medskip}




\begin{document}
\pagestyle{plain}
\thispagestyle{empty}

\noindent
\begin{tabular*}{\textwidth}{l @{\extracolsep{\fill}} r @{\extracolsep{6pt}} l}
\textbf{\class} & \textbf{Name:} & \textit{Sergio Montoya Ramírez}\\ %Your name here instead, obviously 
\textbf{\examnum} &&\\
\textbf{\examdate} &&\\
\end{tabular*}\\
\rule[2ex]{\textwidth}{2pt}
% ---
\begin{center}
    \textbf{Consecuencias Fílosoficas de un Universo Infinito}
\end{center}
\begin{multicols}{2}
Los seres humanos siempre hemos sentido que somos especiales. Desde el principio de las epocas creimos que en nosotros
se encontraba una cualidad magica que nos hacia diferentes de todas las otras especies. Eso es lo que hemos creido desde
epocas muy antiguas y cada vez mas el universo nos muestra nuestro error y prepotencia. En este ensayo defenderemos que
\textit{El hecho de que creamos que el universo es infinito tiene profundas implicaciones fílosoficas en la manera en la
que nos persivimos nosotros mismos y a nuestro entorno.} Para esto, dividiremos el texto en dos secciones principales. En 
la primera hablaremos sobre la percepción historica que ha tenido el ser humano de si mismo y de los otros durante la
historia. En la segunda, Hablaremos sobre las consecuencias particulares que tiene un universo infinito y como estas 
peculiaridades son el origen de muchos problemas fílosoficos interesantes.

\section*{El ser humano se ha mirado al espejo durante la historia}
\subsection*{El orgullo Prometeico}
Ubo un tiempo en donde solo existian los dioses. Estos fueron los encargados de crear la vida y las especies. Entre ellos
se les asigno a Prometeo y Epimeteo el trabajo de distribuir las capacidades a cada una de las especies. De manera muy
astuta Epimeteo logro convenser a Prometeo de dejarle este trabajo \textit{"Despues de hacer yo el reparto tu lo inspeccionas"}
dijo Epimeteo para convenserlo. De esa manera se dio y Epimeteo comenzo a repartir las capacidades. A algunos, los hacia fuertes
pero lentos. A otros, los hacia agiles pero no les daba la suficiente fuerza. A algunos incluso llego a dotarlos con resistencia
o fuerza en sus caparazones. Asi fue Epimeteo repartiendo las habilidades hasta llegar a la ultima de las criuturas. Sin embargo,
esta criatura le causo un coonflicto, Se le habian acabado las capacidades. Ya a estos no los podia dotar de fuerza, agilidad,
velocidad o resistencia, no quedaba nada. En ese momento se hacerca prometeo a, como habian acordado, inspeccionar el
reparto pues ya ese era el dia destinado para acabar. Prometeo entonces afanado y preocupado se le ocurre una idea, robar
el fuego Hefesto y la sabiduria de Atenea. Asi fue, prometeo robo ambas cosas y asi el animal mas debil consiguio aquello
que ninguno de los otros tenia. Autodeterminación. Con el fuego y la sabiduria de como hacerlo, el ser humano cocino su comida
se calento en el invierno y limpio grandes cantidades de tierra para su beneficio. Las consecuencias de esto para prometeo
fueron muchas y celebres durante la historia. Pero los efectos que sus actos tuvieron fueron aun mas severos.

Con este pequeño mito iniciamos nuestro recorrido. Así nos vimos durante muchos años. Es una visión dicotomica pero muy interesante.
Por una parte somos el mas desdichado de los animales, en el reparto de las capacidades no se nos dio la fuerza, la agilidad o
ninguna otra caracteriztica especial. Sin embargo, por otro lado recibimos la mas fuerte de todas las bendiciones, 
la autodeterminación y el fuego. Con esto de alguna manera ya no dependiamos de los dioses. No nos importaba que Zeus o Demeter
estuvieran tristes o enojados. Podiamos cosechar, cazar y criar nuesto alimento. Podiamos mantenernos calientes en invierno
y frescos en verano. Aun tambien importante, nosotros no eramos el centro o causa de eso. No eramos un animal en origen mas
especial que los otros y fue un Dios el que lo hizo y ese mismo dios el que sufrio el castigo.
\subsection*{El Hombre como Centro del \\Universo.}
En el principio no habia nada, luego Dios dijo \textit{fiat lux} y se hizo la luz. Esas fueron las primeras palabras que dijo
Dios y con ellas inicio el tiempo. Se propuso crearlo todo y asi lo hizo. Pasaron muchas cosas y Dios creo muchas cosas 
(Incluyendo los Animales sin el ser humano) pero cuando acabo todo esto lo sintio vacio, faltaba un ser que lo gobernara todo.
Asi pues, Dios decidio crear un animal a su imagen y semejanza. Un ser que estuviera para disfrutar de su creación y gobernarla.
De esta manera creo al ser humano y lo puso en el mundo. Luego de esto le asigno una serie de reglas que el ser humano no cumplio,
sin embargo, para los propositos de este texto estos ya no son importantes aunque sus consecuencias fílosoficas son muy interesantes,
se recomienda al lector averiguar sobre las mismas.

Con esta visión nace una que algunas personas siguen manteniendo hoy en día. Somos el centro del universo, una especie de regla
y mandatario. Un ser creado para gobernar y subyugar a las otras especies por su dominio. Esto nos hacia el centro del universo.
Creiamos que todo giraba alrededor y que toda la creación era para nosotros. Como no ser el centro del universo si el universo
fue creado para nosotros? Así duramos muchos siglos y aun lo creen algunas personas. Somos el centro del universo, una especie
singular y especial ante los ojos de Dios.
\subsection*{El Giro Copernicano}
Pocas veces en la historia se a dado un golpe tan polemico y revolucionario como se hizo en 1543 cuando por fin se publico
\textit{De revolutionibus orbium coelestium} de Nicolás Copérnico (Lo habia terminado unos años antes). En este se hacia
un estudio de las orbitas de los planetas y se concluia una cosa que ahora a muchos nos parece obvia, La tierra \textbf{NO}
esta en el centro del universo. Eso era casi un imposible, ¿Como el ser mas perfecto y aquel al que Diós mismo habia hecho
dueño de todo podia no estar en un lugar tan especial como el centro del universo? 

La verdad nadie estaba dispuesto a responder una pregunta Teologica tan compleja y por tanto se Castigo a aquellos que osaban mantenerla. Se les persiguio y asesino
pero la ciencia ya se habia revolucionado. Por primera vez en la historia la ciencia se solto de sus cadenas y desobedecio a la religión
Así pues, grandes cientificos comenzaron a trabajar y crearon lo que a todas luces era un mundo nuevo. Un mundo en donde no solo
la tierra no era el centro del universo si no que ademas se podia modelar con fuerzas y numeros. Las observaciones se 
hicieron mucho mas logicas y el universo comenzo a tomar un sentido que hasta ahora nunca habia tenido. Esto evidentemente
cayo en la fílosofia como un balde de agua fresca. Donde quedaban los Dioses en un mundo de numeros. Cual era el lugar del hombre.
Todas las cosas deben ser descubiertas por la razon decia descartes y por tanto solo existen tres sustancias, la res Cogita, la res Divina 
y la res Extensa. Quizas solo hay una sustancia respondia Spinoza, La res Divina cuyas cualidades pasan por ser extensas y pensar.
Hume en otra esquina los miraba con desden y les decia: No hay cosa como la justificación racional de las cosas, Lo unico valido
y verdadero es aquello que puedes experimentar. Por ultimo en ese cuarto habia un niño, un niño que con destreza se para y dice
"Existen las verdades A-priori y A-posteriori. pero aun mas, Existen las verdades cientificas a posteriori. Este niño, es la 
representación del espiritu Kantiano. Aquel que estaba encargado de unificarlo todo y comenzar una manera de ver el universo.
En donde el hombre no era mas que otro animal y la tierra nada mas que otro planeta mas. De hecho es tanta la influencia de 
kant que incluso este capitulo se llama en su honor. El Giro Copernicano fue un termino que el mismo Kant acuño para el cambio
que la revolución cientifica trajo al mundo.

\subsection*{El Big Bang y el Universo Uniforme e Infinito}

Pasaron muchos años desde eso, la ciencia avanzo como era de esperarse viento en popa y cuando se comenzo a sentir estancada
todo exploto. \textit{Newton estaba equivocado.} El universo no obedecia sus leyes si no unas mas complejas pensadas por Einstein.
Estas leyes, traian consigo grandes consecuenciás pero una que afectaba especialmente mucho a Einstein era el hecho de que en un universo
como el suyo era posible que este no fuera infinito. Por lo tanto metio con su teoria una constante que le permitia hacer que su 
universo fuera mucho mas que eso y entonces por tanto fuera Infinito en el tiempo. Que hubiese existido desde siempre. Sin embargo,
no paso mucho tiempo antes de que salieran grandes cientificos a contradecirlo. En este caso fueron Edwin Hubble y George Lemaitre
los que se encargaron de mostrar que el universo si habia tenido un inicio. Hubble dio el punta pie observacional, \textit{Los 
cuerpos celestes mientras mas lejos estan mas rapidos se alejan.} Pero fue Lemaitre el que se encargo de darle sentido a esto de mantener
teorica. El huevo primigenio lo llamo sin embargo sus contrincantes un una burla le pusieron el nombre mucho mas comercial
de Big Bang con el cual esta teoria dio su gran salto a la fama. 

Esta era ya una teoria fuerte, Sin embargo, se hizo innegable con el descubrimiento del fondo cosmico de microondas. 
Aun mas, con este descubrimiento se  vio algo aun mas preocupante para como nos vemos, Parece que el universo es homogeneo,
Absolutamente todos los lugares del cosmos parecen ser mas o menos parecidos en terminos de materia, parecen estar hechos
de lo mismo en una distribución extrañamente similar y esto trae consecuencias brutales para no existencia, no solo no somos
el centro del universo, este ni siquiera existe y a gran escala somos solo un grumo en un universo practicamente homogeno y
uniforme en todas partes. Es el golpe definitivo, ya no hay dioses que repartan caracterizticas, o que nos pidan que gobernemos
el mundo. Pero tampoco hay matematicas especiales en esta parte o algo notablemente interesante y aun asi vivimos tan
sumergidos en nuestra vida y entorno que el hecho de que solo nos encontraramos asi es casi desolador. Miramos al vacio
y este nos devolvio la mirada con una abrumadora e infinita realidad.

\section*{El Infinito y sus consecuencias.}
\subsection*{Como que soy solo particulas?}
Los seres humanos estamos hechos de particulas y espero esto no tome por sorpresa a nadie. Pero si, todos simplemente materia
que puede reconocerse en un espejo y por medio de procesos especialmente complejos pensar. Para mi, un ejemplo bastante interesante
son las maquinas de Turing. Hay una cantidad inmensa de maquinas de turing. Desde las cosas mas absurdas como el juego
de la vida de Conway hasta grandes y complejas como los transistores modernos. Sin embargo, todas estan en un sentido relacionadas
todas tienen en su centro un principio esencial que hace que al final todo lo que puedas hacer con una maquina de Turing lo puedes 
hacer en otra. Algo así es el ser humano, Es claro, creemos que somos especiales, tenemos algo que llamamos "vida" y otra cosa que 
denominamos inteligencia. Pero la abrumadora realidad es que si supieramos como acomodar la materia podriamos imitar estos procesos
con cualquier otra materia. Entonces, ya que aceptamos que somos materia y una configuración muy particular. Que pasa cuando hacemos
los calculos de como serian las posibilidades. Por ejemplo, cual es la posibilidad de que en este preciso momento mientras usted
lee esto aparezca a su lado una taza de cafe perfectamente servida que se materializo de la "nada" (realmente se habria materializado
de su entorno) pues seguramente muy baja me diria usted, se tendrian que alinear Muchas particulas en una manera muy particular
para que esto ocurriera y tendria razón. Sin embargo, tome encuenta que estamos en un espacio infinito, en todo el universo, en este momento
en un universo infinito y homogeno hay a su vez infinitas particulas de las cuales en cualquier momento podria ocurrir.

Pogamos esto en un ejemplo que tal vez asume demasiado. Digamos que la posibilidad de que esto pase es de $10^{-3}\%$ no se usted, pero
si a mi me dijeran que esa es mi posibilidad de ganar en una apuesta yo no apostaria (No diga esto enfrente de un consumidor
regular de la loteria, por favor) Sin embargo, cual es la posibilidad de que efectivamente no pase. Pues $(100\% - 10^{-3}\%)$ claro,
sin embargo, aqui viene lo interesante, cual es la posibilidad de que esto no pase dos veces en ese caso seria $(0.999999)^2 = 0.999998$
Oh, la probabilidad de que esto no pase se redujo un poco. Pues probemos una vez mas, para un tercer intento seria $(0.999999)^2 = 0.999996$
Se redujo otro poco, pues bien, con esta información (y dado que soy matematico) hagamos una serie $S_n = (1-10^{-3})^n$.
Pido perdon a todos los matematicos que esten leyendo este texto pero solo por propositos divulgativos voy a graficar esto como
si fuese una función (Esto no es un argumento formal pero sinceramente es mucho mas divertido si se los puedo mostrar graficamente)
\begin{Figura}
    \centering
    \includegraphics[width=0.9\textwidth]{Sucesión.jpeg}
  \end{Figura}
Oh wow, resulta que lo que estabamos viendo se mantiene, la Sucesión sigue bajando y bajando hasta el infinito. Claro que 
los argumentos teoricos son mucho mas potentes sin embargo a lo que quiero llegar es que eventualmente sera 0 y como habiamos
planteado en ese caso si tenemos infinitas oportunidades es basicamente obligatorio que en algun momento esto ocurra.

Esto puede sonar simple y logico, sin embargo, tiene consecuencias muy profundas en como nos vemos a nosotros mismos. Ahora,
no es solo que no estemos en el centro del universo si no que cualquier cosa que pueda físicamente ocurrir a ocurrido. No solamente
somos seres sin importancia si no que por ahi hay un conjunto de particulas que esencialmente son iguales a nosotros. Existe
todo un universo infinito, si el universo se puede modelar en un computador ya se ha modelado en alguna parte del universo y
absolutamente cualquier cosa en la que puedas pensar (E incluso muchas en las que no) ya han ocurrido de manera obligatoria.
\end{multicols}
\begin{thebibliography}{1}
    \bibitem[1] La concepción del humano en la historia del pensamiento occidental. Recuperado de https://bit.ly/3Pc8X7P
\end{thebibliography}
\end{document}

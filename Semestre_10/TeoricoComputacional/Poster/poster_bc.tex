% Unofficial University of Cambridge Poster Template
% https://github.com/andiac/gemini-cam
% a fork of https://github.com/anishathalye/gemini
% also refer to https://github.com/k4rtik/uchicago-poster

\documentclass[final]{beamer}

% ====================
% Packages
% ====================

\usepackage[T1]{fontenc}
\usepackage{lmodern}
\usepackage[orientation=portrait,size=custom,width=100,height=120,scale=1.0]{beamerposter}
\usetheme{gemini}
\usecolortheme{nott}
\usepackage{graphicx}
\usepackage{booktabs}
\usepackage{tikz}
\usepackage{pgfplots}
\pgfplotsset{compat=1.14}
\usepackage{anyfontsize}
\renewcommand{\normalsize}{\fontsize{38}{40}\selectfont} % Aumenta de ~24pt a ~28pt


% ====================
% Lengths
% ====================

% If you have N columns, choose \sepwidth and \colwidth such that
% (N+1)*\sepwidth + N*\colwidth = \paperwidth
\newlength{\sepwidth}
\newlength{\colwidth}
\setlength{\sepwidth}{0.025\paperwidth}
\setlength{\colwidth}{0.45\paperwidth}

\newcommand{\separatorcolumn}{\begin{column}{\sepwidth}\end{column}}

% ====================
% Title
% ====================

\title{Calculadora Agnostica de Anchos Equivalentes}

\author{Sergio Montoya Ramirez \inst{1} \and Benjamin Oostra Vannoppen \inst{1}}

\institute[shortinst]{\inst{1} Universidad de los Andes}

% ====================
% Footer (optional)
% ====================

% \footercontent{
%   \href{https://utfpr.edu.br/ct/ppgca}{utfpr.edu.br/ct/ppgca} \hfill
%   Mostra de Trabalhos do PPGCA --- TechTalks 2024 \hfill
%   \href{mailto:ppgca-ct@utfpr.edu.br}{ppgca-ct@utfpr.edu.br}}
% (can be left out to remove footer)


% ====================
% Logo (optional)
% ====================

% use this to include logos on the left and/or right side of the header:
\logoright{\includegraphics[height=5cm]{./logos/logo-uniandes.png}}
%\logoleft{\hspace{20ex}\includegraphics[height=3.5cm]{logos/ppgca-logo.png}}

% ====================
% Body
% ====================

\begin{document}

% Refer to https://github.com/k4rtik/uchicago-poster
% logo: https://www.cam.ac.uk/brand-resources/about-the-logo/logo-downloads
% \addtobeamertemplate{headline}{}
% {
%     \begin{tikzpicture}[remember picture,overlay]
%       \node [anchor=north west, inner sep=3cm] at ([xshift=-2.5cm,yshift=1.75cm]current page.north west)
%       {\includegraphics[height=7cm]{logos/unott-logo.eps}}; 
%     \end{tikzpicture}
% }

\begin{frame}[t]
\begin{columns}[t]
\separatorcolumn

\begin{column}{\colwidth}

  \begin{block}{Introducción}
  Uno de las mejores maneras de determinar los materiales de los que esta compuesto un cuerpo celeste es por medio de la espectrografia. En este campo, uno de los valores más importantes es el ancho equivalente, una medida que cuantifica la intensidad total de una línea de absorción o emisión. A diferencia de una simple medición de profundidad, el ancho equivalente encapsula toda la energía faltante o sobrante en la línea, representándola como el ancho de un rectángulo de área equivalente. En este proyecto, se desarrolla una calculadora agnostica que permita calcular anchos equivalentes para espectros en Vizier y una interfas CLI para espectros locales.
  \end{block}

  \begin{block}{Objetivos}

    \begin{itemize}

      \item Estudiar el calculo de anchos equivalentes para espectrografia
      \item Desarrollar una calculadora para anchos equivalentes agnosticos
      \item Calcular los anchos equivalentes para un espectro solar (ID: J/A+A/587/A65)

    \end{itemize}

  \end{block}

  \begin{block}{Ancho Equivalente}
    En la luz que generan (o que choca) con cualquier elemento se encuentra una gran cantidad de información. Se puede ver ahi los espectros de donde se puede saber muchas cosas, desde composición quimica hasta velocidad. Ahora bien, no siempre los espectros funcionan de la misma manera, en caso de que la luz sea muy poca o la resolución del espectroscopio no sea suficiente se podria complicar la comparación de espectros. Por lo tanto, se hace uso de anchos equivalentes. Estos son en esencia, una manera en la que podemos pasar de espectros, que tienen formas particulares y puntos dependientes de la muestra, a anchos de rectangulos equivalentes en area (y que por tanto representan la misma absorción de radiación). Para esto la forma en que se hace es:
    \begin{equation}
      EW = \int \frac{(F_c - F_\lambda)}{F_c} d\lambda
      \label{eq:ew_eq1}
    \end{equation}

    \begin{figure}
      \centering
      \includegraphics[width=0.8\colwidth]{./imgs/ew_img1.png}
      \caption{Grafica de Explicación para el Area Equivalente}
      \label{fig:ew_fg1}
    \end{figure}

  \end{block}

  \begin{block}{Sumatoria de Riemann}
    Dada una recta se pueden sumar facilmente los rectangulos que forman sus aristas de modo tal que se tenga un resultado aproximado de la integral.

    \begin{figure}
      \centering
      \includegraphics[width=0.8\colwidth]{./imgs/rm_img1.png}
      \caption{Grafica Demostración sumas de Rieamann}
      \label{fig:ew_fg1}
    \end{figure}
  \end{block}

\end{column}

\separatorcolumn

\begin{column}{\colwidth}

  \begin{block}{Metodo de los Trapecios}

    En un espiritu similar al previamente enunciado para las sumatorias de Riemann el metodo de los trapecios es una mejora a este metodo donde en vez de utilizar rectangulos se usan trapecios que por su forma se ajustan mucho mejor a la recta en cuestion.

    \begin{figure}
      \centering
      \includegraphics[width=0.8\colwidth]{./imgs/tp_img1.png}
      \caption{Grafica Demostración Metodo Trapecios}
      \label{fig:ew_fg1}
    \end{figure}

  \end{block}

  
  \begin{block}{Caso de Uso: Visible and near-infrared solar spectra}

    Habiendo ya hablado de los terminos generales de la aplicación lo mejor es mostrar el funcionamiento de la misma.

    Para un caso de uso con un espectro de luz visible y cercana al infrarojo se quisieron sacar las areas equivalentes.

    Puede ver todos la información (y los datos) en \textit{https://cdsarc.u-strasbg.fr/viz-bin/qcat?J/A+A/587/A65}. Con esto entonces se pidio que calculara el area equivalente para picos que pasaran un umbral (en este caso que su absorción sea mayor a 0.8) con lo cual se obtuvo

    \begin{figure}
      \centering
      \includegraphics[width=0.8\colwidth]{./imgs/ex_img1.png}
      \caption{Grafica de Ejemplo Area Equivalente}
      \label{fig:ew_fg1}
    \end{figure}

  \end{block}

  \begin{block}{Conclusiones}
    \begin{itemize}
      \item 
    \end{itemize}
  \end{block}


  \begin{block}{Referencias}

    \nocite{*}
    \footnotesize{\bibliographystyle{plain}\bibliography{poster}}

  \end{block}

\end{column}
\separatorcolumn



\end{columns}
\end{frame}

\end{document}

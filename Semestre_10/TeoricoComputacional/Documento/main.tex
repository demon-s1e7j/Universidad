\documentclass[a4paper, amsfonts, amssymb, amsmath, reprint, showkeys, nofootinbib, twoside]{revtex4-1}
\usepackage[spanish]{babel}
\usepackage[utf8]{inputenc}
\usepackage{float}
\usepackage[colorinlistoftodos, color=green!40, prependcaption]{todonotes}
\usepackage{amsthm}
\usepackage{mathtools}
\usepackage{physics}
\usepackage{xcolor}
\usepackage{graphicx}
\usepackage[left=23mm,right=13mm,top=35mm,columnsep=15pt]{geometry} 
\usepackage{adjustbox}
\usepackage{placeins}
\usepackage[T1]{fontenc}
\usepackage{lipsum}
\usepackage{csquotes}
\usepackage[normalem]{ulem}
\useunder{\uline}{\ul}{}
\usepackage[pdftex, pdftitle={Article}, pdfauthor={Author}]{hyperref} % For hyperlinks in the PDF
\usepackage[utf8]{inputenc}
\usepackage{enumitem}
\usepackage{xcolor}
\usepackage{booktabs}
\usepackage{longtable}
\usepackage{array}
\usepackage{siunitx}
%\setlength{\marginparwidth}{2.5cm}
\bibliographystyle{aasjournal}

\begin{document}

%El título del experimento realizado es importante.
\title{Calculadora General de Anchos Equivalentes}


\author{Sergio Montoya Ramirez}
\email[Correo institucional: ]{s.montoyar2@uniandes.edu.co}

%Si necesitan poner un segundo autor, deben eliminar los porcentajes (%) iniciales.
  
%\author{Second Author}
%\email{Second.Author@institution.edu}

\affiliation{Universidad de los Andes, Bogotá, Colombia.}

\date{\today} % Si lo dejan vacío no les saldrá fecha. La fecha que se muestra es del día en que se compila.

\begin{abstract}

  Este proyecto desarrolló una calculadora versátil para el cálculo de anchos equivalentes en espectros astronómicos, con el objetivo de simplificar la caracterización de cuerpos celestes mediante espectroscopía. Se implementó un programa gráfico con un sistema modular de clases y herencia, que incluye dos métodos principales: uno basado en límites explícitos (utilizando sumas de Riemann) para análisis rápidos, y otro de simetría (empleando el método de los trapecios) para la detección automática de múltiples picos. La herramienta se aplicó exitosamente a datos del catálogo VizieR (J/A+A/587/A65) en la región del infrarrojo cercano (4348–5000 $cm^{-1}$), filtrando líneas con absorción superior al $80\%$. Los anchos equivalentes obtenidos se transformaron a unidades de longitud de onda (nm) y se reportaron en picómetros, demostrando la capacidad de la calculadora para convertir espectros complejos en valores numéricos comparables. La arquitectura modular y el flujo de usuario intuitivo permiten una extensión sencilla y un análisis iterativo. Se concluye que esta herramienta representa una contribución práctica para la comunidad astronómica, agilizando el procesamiento de espectros y facilitando comparaciones sistemáticas entre observaciones.
  %En este se describen brevemente los objetivos y los resultados del trabajo, por lo tanto se debe dar información completa pero corta del contenido del trabajo. Se debe indicar qué fue lo que se hizo, cómo se hizo y cuáles fueron los resultados obtenidos de forma EXPLÍCITA. Por ejemplo: se obtuvo un valor para la constante de Planck de $h=(5.9\pm 0.5)\times 10^{-34}\,\text{J s}$ \footnote{Utilizar esta forma para reportar los datos, note que las unidades están en un cuadro de texto. TODO debe llevar unidades e incertidumbre, siempre.}. A lo largo de todo el informe por favor utilizar el diccionario de overleaf. Tener ideas claras y concisas de lo que se hizo y de los resultados obtenidos. Que genere interés en leer el resto del artículo. El informe, que está en formato de artículo científico debe ser auto-contenido. Un lector que no haya ido al laboratorio, o que no sepa del experimento debe ser capaz de leerlo y entender todo lo que allí de plantea.

\end{abstract}

\maketitle

\section{Introducción}
Una de las mejores formas de conocer un cuerpo celeste es mediante la espectroscopía. Con este método se puede conocer multitud de variables esenciales para la caracterización del cuerpo, como su composición o su rotación. Sin embargo, los espectros crudos resultan particularmente difíciles de procesar, pues dependen, entre otros factores, de la resolución, la tasa de medida o la intensidad de la luz recibida. Uno de los métodos utilizados en la comunidad astronómica es el cálculo de áreas equivalentes para la caracterización de espectros. El área equivalente de un espectro es, esencialmente, la cantidad de energía absorbida (o emitida, dependiendo del tipo de espectroscopía) que una cierta longitud de onda experimenta. Se le denomina ancho equivalente porque, al calcularse en un espectro normalizado, corresponde con el ancho de un rectángulo de altura uno que ocupa la misma área en el espectro. De este modo, podemos pasar de representar espectros como una línea continua con picos a representarlos como una serie de números mucho más fáciles de comparar entre sí. Este proyecto tiene como objetivo desarrollar una calculadora versátil para calcular anchos equivalentes en múltiples espectros. Para ello, se creó un programa gráfico con un sistema de clases y herencia, así como un par de métodos que permiten determinar anchos equivalentes para espectros en VizieR o en archivos locales en formato XLSX o CSV. Puede acceder a la calculadora en el \href{https://github.com/demon-s1e7j/Universidad/tree/main/Semestre\_10/TeoricoComputacional/FirstIteration}{repositorio}

\section{Marco Teórico}

Según \cite{StellarPhysics}, un ancho equivalente se define como el ancho que tendría un rectángulo de altura uno (por estar normalizado) y que absorbe la misma cantidad de energía que la línea atómica correspondiente. En la figura \ref{fig:ew_graph} puede observarse un ejemplo de ancho equivalente para un espectro idealizado. En ella se aprecia cómo el espectro representa una curva y el ancho equivalente es un rectángulo de altura 1 y menor anchura, pero que en total ocupa la misma área en el espectro; es decir, absorbe la misma energía. Así, podemos entender una de las grandes ventajas de los anchos equivalentes: podemos representar líneas atómicas como un número que indica la misma cantidad de energía absorbida por parte del elemento en esa longitud de onda concreta, simplificando considerablemente su comparación.

\begin{figure}[H]
\centering
\includegraphics[width=0.45\textwidth]{./Fig/ew_img1.png}
\caption{Representación de un ancho equivalente. Se observa un espectro idealizado con una única línea atómica, de modo que su ancho equivalente corresponde a un rectángulo que ocupa la misma área (aunque su anchura sea menor que la de la línea procesada, ya que su altura es 1).}
\label{fig:ew_graph}
\end{figure}

Ahora bien, los cálculos de anchos equivalentes se han simplificado considerablemente con la incorporación de computadoras en nuestra época, dada la gran cantidad de metodologías computacionalmente eficientes que existen para calcular el área bajo una curva. En nuestro caso, utilizaremos uno de los algoritmos más conocidos para el cálculo de áreas bajo una curva: el método de los trapecios. Este es una mejora respecto a un método previo conocido como sumas de Riemann. Vamos a desarrollar, entonces, las sumas de Riemann como contexto para luego explicar la mejora que representa el método de los trapecios.

Imagine que tiene una curva (dado que nosotros estamos trabajando con espectros, la tomaremos como una curva de absorción; sin embargo, este método es aplicable a cualquier curva). Esta curva puede dividirse en múltiples rectángulos que ocupan un área desde el punto de normalización hasta algún punto de la curva. Puede ver una simulación con pocos rectángulos (apenas 6) en la figura \ref{fig:rm_graph1}. Esta figura nos permite entender un poco la aproximación que estamos usando, pero también muestra una de sus mayores debilidades: la cantidad de área que no se correlaciona entre los rectángulos y la curva objetivo es alta. Puede verse en la gráfica \ref{fig:rm_graph2} cómo esto mejora considerablemente al aumentar el número de rectángulos. Sin embargo, incluso en este caso, la correspondencia no es particularmente exacta.

Para estos casos es cuando se puede utilizar el método de los trapecios. Este consiste, en esencia, en los mismos rectángulos de antes, pero esta vez no los aproximaremos con un lado recto (particularmente el lado que limita con la curva), sino que asumiremos que la diferencia entre dos puntos consecutivos es una recta, lo que equivale a formar un pequeño triángulo sobre nuestro rectángulo, ajustándose mejor a la curva. La forma que adquiere entonces el elemento con el que se va a sumar es la de un trapecio. Puede observarse, bajo las mismas condiciones anteriores, la mejora que este método representa frente al anterior en las figuras \ref{fig:tp_graph1} y \ref{fig:tp_graph2}. Comparar la exactitud de un método con el otro escapa por completo del alcance planeado para este documento. Sin embargo, confiamos en que el lector, con estos ejemplos, podrá darse una idea de las diferencias y ventajas de cada método.

\section{Metodología}

El objetivo de este trabajo era desarrollar una calculadora capaz de calcular áreas equivalentes para una gran diversidad de espectros. Con este fin, podemos dividir el sistema implementado en tres características principales. Primero, el flujo del usuario, es decir, la secuencia general que debe seguir un usuario para calcular los espectros. El segundo aspecto es el modelo de clases de la aplicación, que, aunque resulta increíblemente simple, permite buena parte de su modularidad y mejora considerablemente las capacidades de la calculadora. La tercera y última categoría son los algoritmos específicos implementados para la detección de picos y para el cálculo de áreas equivalentes. Para cubrir estos tres puntos, dividiremos esta sección en tres subsecciones.

\subsection{Flujo del Usuario}

Para un usuario de nuestra aplicación, lo primero que deberá hacer es determinar el catálogo con el que trabajará. Inicialmente tendrá que definir el ID de VizieR para buscarlo en la plataforma. Nosotros le presentaremos las opciones posibles, de modo que pueda escoger el artículo que le interesa en una tabla. Otra manera es iniciando el programa con un solo argumento que sea el nombre del archivo en el que estan los datos (que puede ser formato excel o formato csv). Luego de esto, deberá determinar cuál de todas las tablas es en la que le interesa trabajar, pudiendo escoger entre las distintas tablas del catálogo aquella que contenga la información relevante. Posteriormente, escogerá el modelo con el que va a trabajar y tendrá que completar la información necesaria para todos los modelos (que siempre coincide en ser la misma). Puede ver una lista de los inputs necesarios en el Apéndice \ref{an:inputs_necesarios}. Con esto, aplica el método con la calculadora y verifica los resultados. Si está satisfecho, terminará y continuará con las áreas equivalentes para realizar su trabajo. En caso contrario, puede iterar volviendo a escoger el método y cambiando las variables seleccionadas. Podemos ver la representación de este flujo en la figura \ref{fig:flujo_usuario}.

\begin{figure}[H]
\centering
\includegraphics[width=0.45\textwidth]{./Fig/user_diagram.png}
  \caption{Representación del flujo que seguirá un usuario en nuestra aplicación. Los pasos son: selección de datos (bien sea con ID de VizieR o con el nombre del archivo que los contenga como argumento), selección de la tabla necesaria, selección del método a usar, completado de los inputs necesarios, verificación de satisfacción con los resultados y, finalmente, aprobación y continuación con la interpretación. El flujo es esencialmente lineal, excepto en el paso de comprobación, donde, en caso de fallar, se vuelve a escoger un método y se prueba de nuevo.}
\label{fig:flujo_usuario}
\end{figure}

\subsection{Modelo de Clases}

A diferencia de la subsección anterior, aquí no nos interesa explicar todos los pasos ni todo el sistema de clases, sino solo los métodos y su interacción. En particular, nos interesa mostrar que todos los métodos heredan de una clase base (llamada BaseMethod). Esta es la clase que define el comportamiento básico. Pueden observarse los elementos básicos con los que cuenta en el Apéndice \ref{an:inputs_necesarios}. De esta clase heredan todas las demás. Una característica principal de esta clase es un método abstracto llamado calculate, que será sobrescrito por todas sus subclases para cambiar el comportamiento. La explicación detallada de los algoritmos utilizados para cada método se dará en la subsección \ref{subsec:algorithms}. Puede verse un diagrama simplificado en la figura \ref{fig:classes}.

\begin{figure}[H]
\centering
\includegraphics[width=0.45\textwidth]{./Fig/uml_class.png}
  \caption{Diagrama que representa el sistema de clases de los métodos de cálculo de áreas equivalentes. Muestra cómo, heredando desde BaseMethod, se puede crear cualquier protocolo para calcular áreas equivalentes, resultando en un sistema modular y extensible.}
\label{fig:classes}
\end{figure}

\subsection{Algoritmos}\label{subsec:algorithms}

En nuestro proyecto utilizamos dos algoritmos que definen el comportamiento. El primero es un algoritmo con límites explícitos, que confía plenamente en que el usuario le ha proporcionado los límites correctos para una sola línea atómica. El segundo método permite calcular, mediante un criterio de simetría, los límites de diferentes líneas atómicas y, por lo tanto, determinar distintos picos en una región mucho más extensa. A continuación, explicaremos cada uno de estos métodos.

\subsubsection{Límites Explícitos}
Como explicamos previamente, este algoritmo es relativamente sencillo y directo. No realiza preprocesamiento de los datos ni verificación. Funciona principalmente como una aproximación inicial para observar un dato rápidamente, pues es considerablemente más rápido que el otro método. Calcula el área mediante sumas de Riemann. Puede verse en la figura \ref{fig:explicit_limits} una explicación de su flujo.


\begin{figure}[H]
\centering
\includegraphics[width=0.45\textwidth]{./Fig/diagrama_calculate.png}
  \caption{Diagrama de flujo de la función calculate() que implementa el cálculo numérico de integrales mediante sumas de Riemann. El proceso inicia con una consulta SQL parametrizada para extraer datos de una base de datos astronómica, aplicando transformaciones matemáticas condicionales según la configuración del usuario. El núcleo del algoritmo itera sobre los pares de datos independiente-dependiente para calcular rectángulos de aproximación usando el método del punto izquierdo, acumulando progresivamente la suma total. Finalmente, dependiendo de las opciones de configuración, genera visualizaciones gráficas y archivos de resultados.}
\label{fig:explicit_limits}
\end{figure}

\subsubsection{Método de Simetría}\label{sec:met_sim}

A diferencia del caso anterior, este es un método mucho más complejo, con al menos dos fases. La primera fase consiste en descubrir aquellos picos en el rango indicado que cumplen con las condiciones. Para ello, recorre inicialmente el espectro identificando picos más altos que el umbral definido. Luego, expande los picos tomando uno a uno los puntos. Es simétrico porque se encarga de revisar que, si uno de los lados no es aproximadamente simétrico al otro, se queda con el más alto. Esto lo hace para evitar casos como líneas atómicas que se entrecruzan o ruido. Una vez verificados los picos, calcula el área bajo estos mediante el método de los trapecios. Dada la complejidad de este algoritmo, su diagrama de flujo se dividió en múltiples imágenes, que pueden verse en la sección \ref{an:diagramas_symmetry}.

\subsubsection{Método de Simetría}

Aprovechando que ya tenemos el metodo de simetria creamos una opción adicional, en donde se puede calcular areas equivalentes para picos ya conocidos. Para esto se da una lista de los picos necesarios para revisar (separados por cambios de linea) y la calculadora adquiere la información necesaria para encontrar los datos de los picos por el mismo metodo que \ref{sec:met_sim}

\section{Análisis de Datos y Resultados}

Con el objetivo de utilizar este proceso, se calcularán los anchos equivalentes para el espectro de \cite{vizier:J/A+A/587/A65}. Estos son datos de espectros para luz visible y cercana al infrarrojo. Particularmente, nos interesa encontrar los anchos equivalentes para datos de esta última sección del espectro, pues no están reportados como en la primera parte. Para ello, tomaremos los datos con número de onda entre 4348 y 5000 $cm^{-1}$. Además, nos quedaremos con aquellos elementos que superen el 80\% de absorción. Introducimos esos datos en la calculadora y recibimos los resultados que pueden verse en la figura \ref{fig:results_1}. Los resultados que devuelve son preliminares; en particular, no corresponden en muchos casos con los picos que nos interesan y presentan errores. Afortunadamente, el programa asigna un área de 0 a aquellos picos que, aunque detectados, no son calculables (por ejemplo, si es solamente un punto en 0). Además, aquí ocurre el primer problema: dado que estábamos trabajando en $cm^{-1}$, debemos primero convertirlo a nm (longitud de onda) y hacer lo respectivo con el ancho equivalente para expresarlo en picómetros. Con este procesamiento, se obtienen los datos que pueden encontrarse en la tabla de la sección \ref{an:results}. Los resultados pueden observarse en la gráfica \ref{fig:area}.

\begin{figure}[H]
\centering
\includegraphics[width=0.45\textwidth]{./Fig/results.png}
  \caption{Resultados del cálculo de ancho equivalente para el espectro mostrado en la gráfica. Se muestran todos los picos que serán reportados en el archivo de resultados.}
\label{fig:results_1}
\end{figure}

\begin{figure}[H]
\centering
\includegraphics[width=0.45\textwidth]{./Fig/Area.png}
  \caption{Gráfica de anchos equivalentes para diferentes longitudes de onda. Se representan los anchos equivalentes en picómetros respecto a la longitud de onda en nanómetros, siempre que superen una absorción del 80\%.}
\label{fig:area}
\end{figure}

\section{Conclusiones}

Este proyecto ha desarrollado exitosamente una calculadora versátil para el cálculo de anchos equivalentes en espectros astronómicos, cumpliendo con el objetivo principal de crear una herramienta que simplifica considerablemente el procesamiento y análisis de datos espectroscópicos. La implementación de un sistema modular basado en clases con herencia ha demostrado ser particularmente efectiva, permitiendo la creación de múltiples métodos de cálculo que pueden extenderse fácilmente según las necesidades específicas del usuario.

La calculadora ha demostrado su utilidad práctica al aplicarse a datos reales del catálogo VizieR, específicamente para espectros en el infrarrojo cercano. Los resultados obtenidos, aunque requirieron un procesamiento adicional para la conversión de unidades (de $cm^{-1}$ a nm y picómetros), muestran que el método de simetría implementado es capaz de identificar y caracterizar líneas atómicas incluso en regiones espectrales complejas. La capacidad de establecer un umbral de absorción (en este caso del $80\%$) permite filtrar los picos más relevantes, mejorando la calidad de los resultados finales.

La comparación entre los dos métodos implementados—límites explícitos y método de simetría—revela ventajas y desventajas de cada enfoque. Mientras que el método de límites explícitos resulta más rápido y adecuado para análisis preliminares, el método de simetría ofrece mayor robustez al detectar automáticamente múltiples picos y establecer sus límites mediante criterios de simetría. Esta complementariedad enriquece la herramienta, permitiendo al usuario seleccionar el enfoque más apropiado según el tipo de análisis requerido.

La arquitectura del sistema, particularmente el flujo de usuario implementado (selección de catálogo → elección de tabla → selección de método → cálculo → verificación), ha demostrado ser intuitiva y eficiente. La posibilidad de iterar en el proceso de cálculo, ajustando parámetros según los resultados obtenidos, proporciona flexibilidad que resulta valiosa en el trabajo científico real, donde la exploración de diferentes configuraciones es frecuentemente necesaria.

En cuanto a las limitaciones del proyecto, se identificó que el preprocesamiento de datos (especialmente la conversión de unidades) debe ser considerado cuidadosamente por el usuario, ya que afecta directamente la interpretación de los resultados. Futuras versiones de la calculadora podrían incorporar automatizaciones para estas conversiones, así como métodos adicionales de detección de picos y normalización de espectros.

En conclusión, esta herramienta representa una contribución práctica a la comunidad astronómica, particularmente para investigadores que trabajan con espectroscopía. Al transformar espectros complejos en valores numéricos de ancho equivalente, facilita la comparación sistemática entre diferentes observaciones y cuerpos celestes. La modularidad del diseño asegura que pueda adaptarse y extenderse para abordar necesidades específicas futuras, manteniendo su relevancia a medida que evolucionan las técnicas espectroscópicas y las bases de datos astronómicas.



\bibliography{referencias}

\section{Apéndice: Imágenes Auxiliares}

\begin{figure}[H]
\centering
\includegraphics[width=0.45\textwidth]{./Fig/rm_img1.png}
\caption{Representación de sumas de Riemann en el cálculo de integrales. Se observa un espectro idealizado equivalente al visto anteriormente y 6 rectángulos que aproximan el área bajo la curva. Sin embargo, la correlación no es perfecta.}
\label{fig:rm_graph1}
\end{figure}

\begin{figure}[H]
\centering
\includegraphics[width=0.45\textwidth]{./Fig/rm_img2.png}
\caption{Representación de sumas de Riemann en el cálculo de integrales. Se observa un espectro idealizado equivalente al visto anteriormente y 20 rectángulos que aproximan el área bajo la curva. La correlación mejora respecto a la figura~\ref{fig:rm_graph1}, pero aún no es perfecta.}
\label{fig:rm_graph2}
\end{figure}

\begin{figure}[H]
\centering
\includegraphics[width=0.45\textwidth]{./Fig/tp_img1.png}
\caption{Representación del método de los trapecios en el cálculo de integrales. Se observa un espectro idealizado equivalente al visto anteriormente y 6 trapecios que aproximan el área bajo la curva. Comparando con las figuras~\ref{fig:rm_graph1} y~\ref{fig:rm_graph2}, se aprecia una mejora considerable.}
\label{fig:tp_graph1}
\end{figure}

\begin{figure}[H]
\centering
\includegraphics[width=0.45\textwidth]{./Fig/tp_img2.png}
\caption{Representación del método de los trapecios en el cálculo de integrales. Se observa un espectro idealizado equivalente al visto anteriormente y 20 trapecios que aproximan el área bajo la curva. La aproximación mejora notablemente respecto a las sumas de Riemann.}
\label{fig:tp_graph2}
\end{figure}

\section{Apéndice: Inputs Necesarios de Parte del Usuario}\label{an:inputs_necesarios}

\subsection*{Parámetros de Instanciación}
\begin{itemize}[leftmargin=*]
    \item \textbf{\texttt{master}}: Ventana contenedora principal (marco padre).
    \item \textbf{\texttt{catalog}}: Objeto catálogo para acceso a la base de datos.
    \item \textbf{\texttt{table}}: Nombre de la tabla de la base de datos.
\end{itemize}

\subsection*{Widgets de Entrada de Usuario}
\begin{itemize}[leftmargin=*]
    \item \textbf{\texttt{independent\_variables\_selector}}: 
          \begin{itemize}
              \item \textit{Tipo}: Menú desplegable (CTkOptionMenu).
              \item \textit{Función}: Selección de la variable independiente.
              \item \textit{Contenido}: Variables de la base de datos.
          \end{itemize}
    
    \item \textbf{\texttt{take\_inverse\_independent}}:
          \begin{itemize}
              \item \textit{Tipo}: Casilla de verificación (CTkCheckBox).
              \item \textit{Función}: Indicar si se toma el inverso de la variable independiente. Es decir, transforma entre longitud de onda y numero de onda (con $cm^{-1}$)
          \end{itemize}
    
    \item \textbf{\texttt{dependent\_variables\_selector}}:
          \begin{itemize}
              \item \textit{Tipo}: Menú desplegable (CTkOptionMenu).
              \item \textit{Función}: Selección de la variable dependiente.
              \item \textit{Contenido}: Variables de la base de datos.
          \end{itemize}
    
    \item \textbf{\texttt{begin\_entry}}:
          \begin{itemize}
              \item \textit{Tipo}: Campo de texto (CTkEntry).
              \item \textit{Función}: Definir el valor inicial del rango de datos.
          \end{itemize}
    
    \item \textbf{\texttt{end\_entry}}:
          \begin{itemize}
              \item \textit{Tipo}: Campo de texto (CTkEntry).
              \item \textit{Función}: Definir el valor final del rango de datos.
          \end{itemize}
    
    \item \textbf{\texttt{threshold\_entry}}:
          \begin{itemize}
              \item \textit{Tipo}: Campo de texto (CTkEntry).
              \item \textit{Función}: Establecer el valor de umbral para cálculos.
          \end{itemize}
    
    \item \textbf{\texttt{generate\_image}}:
          \begin{itemize}
              \item \textit{Tipo}: Casilla de verificación (CTkCheckBox).
              \item \textit{Función}: Activar/desactivar generación de imagen gráfica.
          \end{itemize}
\end{itemize}

\subsection*{Botón de Acción}
\begin{itemize}[leftmargin=*]
    \item \textbf{\texttt{calculate\_buttons}}:
          \begin{itemize}
              \item \textit{Tipo}: Botón (CTkButton).
              \item \textit{Función}: Ejecutar el método \texttt{calculate()}.
          \end{itemize}
\end{itemize}

\section{Apéndice: Diagramas del Método de Simetría}\label{an:diagramas_symmetry}

\subsection{Diagrama Principal del Método calculate()}

\begin{figure}[H]
    \centering
    \includegraphics[width=0.3\textwidth]{./Fig/diagrama_principal.png}
    \caption{Diagrama de flujo principal del método \texttt{calculate()} de la clase \texttt{SymmetryMethod}. Muestra el flujo completo desde la consulta de datos hasta la generación de resultados.}
    \label{fig:diagrama_principal}
\end{figure}

La Figura \ref{fig:diagrama_principal} muestra el flujo principal del método \texttt{calculate()}, que incluye:
\begin{itemize}
    \item Consulta de datos desde la base de datos con filtros aplicados.
    \item Transformación opcional de variables independientes.
    \item Detección de centros de picos usando la función \texttt{test\_maker}.
    \item Expansión iterativa de picos hasta completarlos.
    \item Cálculo de áreas de cada pico.
    \item Generación de imágenes y archivos de resultados.
\end{itemize}

\subsection{Método expand() de la Clase Peak}

\begin{figure}[H]
    \centering
    \includegraphics[width=0.45\textwidth]{./Fig/diagrama_expand.png}
    \caption{Diagrama de flujo del método \texttt{expand()} de la clase \texttt{Peak}. Muestra la lógica de expansión de picos basada en pendientes y simetría.}
    \label{fig:diagrama_expand}
\end{figure}

La Figura \ref{fig:diagrama_expand} ilustra el algoritmo de expansión de picos, que:
\begin{itemize}
    \item Verifica si un pico ya está completo.
    \item Calcula pendientes en ambos lados del pico.
    \item Determina si las pendientes son aceptables según la tolerancia.
    \item Expande el pico en la dirección apropiada.
    \item Maneja casos simétricos y asimétricos.
    \item Establece límites de expansión máxima.
\end{itemize}

\subsection{Método calculate\_area() de la Clase Peak}

\begin{figure}[H]
    \centering
    \includegraphics[width=0.45\textwidth]{./Fig/diagrama_calculate_area.png}
    \caption{Diagrama de flujo del método \texttt{calculate\_area()} de la clase \texttt{Peak}. Muestra cómo se calcula el área bajo la curva para picos simétricos y asimétricos.}
    \label{fig:diagrama_calculate_area}
\end{figure}

La Figura \ref{fig:diagrama_calculate_area} detalla el cálculo de áreas, que incluye:
\begin{itemize}
    \item Verificación de simetría del pico.
    \item Cálculo de áreas para picos simétricos (suma de ambos lados).
    \item Cálculo de áreas para picos asimétricos (duplicación del lado válido).
    \item Manejo de índices de puntos adyacentes.
\end{itemize}

\subsection{Función calculate\_segment\_area()}

\begin{figure}[H]
    \centering
    \includegraphics[width=0.45\textwidth]{./Fig/diagrama_segment_area.png}
    \caption{Diagrama de flujo de la función interna \texttt{calculate\_segment\_area()}. Muestra el cálculo trapezoidal del área entre puntos consecutivos.}
    \label{fig:diagrama_segment_area}
\end{figure}

La Figura \ref{fig:diagrama_segment_area} muestra el cálculo trapezoidal:
\begin{itemize}
    \item Inicialización del área acumulada.
    \item Iteración sobre pares consecutivos de puntos.
    \item Cálculo del área trapezoidal para cada segmento.
    \item Acumulación del área total.
\end{itemize}

\subsection{Flujo General de Detección de Picos}

\begin{figure}[H]
    \centering
    \includegraphics[width=0.45\textwidth]{./Fig/diagrama_deteccion_picos.png}
    \caption{Flujo general de detección de picos. Resume el proceso completo desde datos originales hasta la generación de resultados.}
    \label{fig:diagrama_deteccion_picos}
\end{figure}

La Figura \ref{fig:diagrama_deteccion_picos} resume el proceso completo:
\begin{itemize}
    \item Aplicación del umbral a los datos originales.
    \item Identificación de mínimos locales.
    \item Creación de objetos Peak.
    \item Expansión de límites de picos.
    \item Cálculo de áreas.
    \item Generación de resultados finales.
\end{itemize}

\section{Apéndice: Resultados}\label{an:results}

\begin{longtable}{|c|c|c|c|}
\toprule
  {Wave ($cm^{-1}$)} & {Area} & {Wavelength ($nm$)} & {Area\_transformed} \\
\midrule
\endfirsthead

\toprule
  {Wave ($cm^{-1}$)} & {Area} & {Wavelength $nm$} & {Area\_transformed} \\
\midrule
\endhead

\midrule
\multicolumn{4}{r}{{Continúa en la siguiente página}} \\
\endfoot

\bottomrule
\endlastfoot

4348.944 & 0.091 & 2299.409 & 172.555 \\
4351.05 & 0.064 & 2298.296 & 122.032 \\
4356.219 & 0.009 & 2295.569 & 17.329 \\
4356.227 & 0.009 & 2295.565 & 17.194 \\
4357.515 & 0.003 & 2294.886 & 6.166 \\
4362.072 & 0.003 & 2292.488 & 6.095 \\
4366.833 & 0.068 & 2289.989 & 128.753 \\
4367.005 & 0.004 & 2289.899 & 6.919 \\
4375.621 & 0.003 & 2285.39 & 5.956 \\
4375.715 & 0.003 & 2285.341 & 6.008 \\
4378.299 & 0.004 & 2283.992 & 6.981 \\
4380.148 & 0.095 & 2283.028 & 183.093 \\
4380.329 & 0.068 & 2282.934 & 131.291 \\
4380.657 & 0.003 & 2282.763 & 6.563 \\
4384.35 & 0.003 & 2280.84 & 6.295 \\
4384.355 & 0.003 & 2280.837 & 6.294 \\
4384.374 & 0.004 & 2280.827 & 6.737 \\
4384.385 & 0.018 & 2280.821 & 34.011 \\
4391.316 & 0.003 & 2277.222 & 6.511 \\
4393.855 & 0.056 & 2275.906 & 108.532 \\
4394.314 & 0.099 & 2275.668 & 191.034 \\
4394.749 & 0.004 & 2275.443 & 6.951 \\
4400.576 & 0.003 & 2272.43 & 6.033 \\
4404.408 & 0.07 & 2270.453 & 136.519 \\
4407.887 & 0.066 & 2268.661 & 127.887 \\
4408.723 & 0.003 & 2268.231 & 6.083 \\
4415.55 & 0.085 & 2264.724 & 165.616 \\
4420.015 & 0.074 & 2262.436 & 145.446 \\
4421.403 & 0.105 & 2261.726 & 204.785 \\
4422.684 & 0.036 & 2261.071 & 70.151 \\
4512.685 & 0.092 & 2215.976 & 187.04 \\
4535.344 & 0.003 & 2204.904 & 6.746 \\
4535.363 & 0.003 & 2204.895 & 6.985 \\
4535.373 & 0.003 & 2204.89 & 6.844 \\
4549.032 & 0.003 & 2198.27 & 7.024 \\
4813.396 & 0.097 & 2077.535 & 224.594 \\
4815.343 & 0.004 & 2076.695 & 8.261 \\
4817.27 & 0.004 & 2075.865 & 8.532 \\
4819.178 & 0.004 & 2075.043 & 8.7 \\
4824.679 & 0.003 & 2072.677 & 7.571 \\
4854.406 & 0.004 & 2059.984 & 8.794 \\
4866.796 & 0.097 & 2054.74 & 230.494 \\
4868.66 & 0.091 & 2053.953 & 214.607 \\
4872.31 & 0.101 & 2052.414 & 240.059 \\
4874.094 & 0.088 & 2051.663 & 209.699 \\
4877.58 & 0.093 & 2050.197 & 222.412 \\
4879.284 & 0.08 & 2049.481 & 190.196 \\
4880.958 & 0.063 & 2048.778 & 150.041 \\
4881.97 & 0.004 & 2048.354 & 8.926 \\
4883.152 & 0.131 & 2047.858 & 311.345 \\
4884.314 & 0.094 & 2047.37 & 224.033 \\
4885.455 & 0.085 & 2046.892 & 202.49 \\
4894.111 & 0.076 & 2043.272 & 182.312 \\
4895.529 & 0.003 & 2042.68 & 7.807 \\
4896.921 & 0.082 & 2042.1 & 195.826 \\
4898.284 & 0.083 & 2041.531 & 198.191 \\
4899.621 & 0.003 & 2040.974 & 7.998 \\
4900.93 & 0.092 & 2040.429 & 221.61 \\
4902.211 & 0.081 & 2039.896 & 194.626 \\
4903.465 & 0.003 & 2039.374 & 7.378 \\
4905.887 & 0.114 & 2038.367 & 273.695 \\
4925.23 & 0.078 & 2030.362 & 188.112 \\
4927.554 & 0.101 & 2029.404 & 244.725 \\
4929.852 & 0.004 & 2028.459 & 8.912 \\
4932.119 & 0.004 & 2027.526 & 9.107 \\
4939.202 & 0.11 & 2024.619 & 267.598 \\
4939.807 & 0.094 & 2024.371 & 229.901 \\
4941.121 & 0.079 & 2023.832 & 192.289 \\
4941.763 & 0.003 & 2023.569 & 8.215 \\
4943.694 & 0.118 & 2022.779 & 289.542 \\
4944.886 & 0.004 & 2022.291 & 8.904 \\
4945.594 & 0.133 & 2022.002 & 325.479 \\
4946.731 & 0.004 & 2021.537 & 8.847 \\
4947.466 & 0.004 & 2021.237 & 9.013 \\
4948.552 & 0.143 & 2020.793 & 350.961 \\
4950.349 & 0.004 & 2020.06 & 9.023 \\
4952.119 & 0.146 & 2019.337 & 357.551 \\
4952.436 & 0.003 & 2019.208 & 8.188 \\
4953.867 & 0.004 & 2018.625 & 9.111 \\
4954.669 & 0.004 & 2018.298 & 9.118 \\
4956.398 & 0.004 & 2017.594 & 9.022 \\
4958.099 & 0.004 & 2016.902 & 9.112 \\
4959.771 & 0.112 & 2016.222 & 275.327 \\
4961.413 & 0.12 & 2015.555 & 296.613 \\
4962.225 & 0.058 & 2015.225 & 142.115 \\
4964.92 & 0.003 & 2014.131 & 8.402 \\
4965.366 & 0.003 & 2013.95 & 7.811 \\
4965.385 & 0.003 & 2013.942 & 7.724 \\
4966.875 & 0.003 & 2013.338 & 8.077 \\
4966.933 & 0.003 & 2013.315 & 7.631 \\
4968.444 & 0.004 & 2012.703 & 8.778 \\
4968.801 & 0.11 & 2012.558 & 271.33 \\
4969.21 & 0.004 & 2012.392 & 8.849 \\
4970.694 & 0.004 & 2011.791 & 9.126 \\
4972.152 & 0.004 & 2011.202 & 9.054 \\
4972.574 & 0.065 & 2011.031 & 160.088 \\
4973.583 & 0.004 & 2010.623 & 9.15 \\
4974.201 & 0.004 & 2010.373 & 9.187 \\
4974.992 & 0.004 & 2010.054 & 9.235 \\
4975.566 & 0.123 & 2009.822 & 304.034 \\
4975.658 & 0.003 & 2009.784 & 7.465 \\
4976.903 & 0.146 & 2009.282 & 361.86 \\
4977.73 & 0.004 & 2008.948 & 9.151 \\
4978.029 & 0.074 & 2008.827 & 183.06 \\
4978.21 & 0.004 & 2008.754 & 9.162 \\
4979.061 & 0.004 & 2008.411 & 9.158 \\
4979.489 & 0.004 & 2008.238 & 9.138 \\
4979.794 & 0.003 & 2008.115 & 8.48 \\
4980.739 & 0.004 & 2007.734 & 9.135 \\
4981.96 & 0.004 & 2007.242 & 9.233 \\
4986.521 & 0.004 & 2005.406 & 9.193 \\
4988.195 & 0.004 & 2004.733 & 8.729 \\
4990.672 & 0.003 & 2003.738 & 8.171 \\
4996.625 & 0.003 & 2001.351 & 8.626 \\
4998.065 & 0.047 & 2000.774 & 118.645 \\
\bottomrule
\end{longtable}

\end{document}

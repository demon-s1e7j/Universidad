\documentclass[12pt]{exam}
\usepackage{amsthm}
\usepackage{libertine}
\usepackage[utf8]{inputenc}
\usepackage[margin=1in]{geometry}
\usepackage{amsmath,amssymb}
\usepackage{multicol}
\usepackage[shortlabels]{enumitem}
\usepackage{siunitx}
\usepackage{cancel}
\usepackage{graphicx}
\usepackage{pgfplots}
\usepackage{listings}
\usepackage{xcolor}

\definecolor{codegreen}{rgb}{0,0.6,0}
\definecolor{codegray}{rgb}{0.5,0.5,0.5}
\definecolor{codepurple}{rgb}{0.58,0,0.82}
\definecolor{backcolour}{rgb}{0.95,0.95,0.92}

\lstdefinestyle{mystyle}{
  backgroundcolor=\color{backcolour},   
  commentstyle=\color{codegreen},
  keywordstyle=\color{magenta},
  numberstyle=\tiny\color{codegray},
  stringstyle=\color{codepurple},
  basicstyle=\ttfamily\footnotesize,
  breakatwhitespace=false,         
  breaklines=true,                 
  captionpos=b,                    
  keepspaces=true,                 
  numbers=left,                    
  numbersep=5pt,                  
  showspaces=false,                
  showstringspaces=false,
  showtabs=false,                  
  tabsize=2
}

%\lstset{style=mystyle}
\usepackage{tikz}


\pgfplotsset{width=10cm,compat=1.9}
\usepgfplotslibrary{external}
\tikzexternalize

\newcommand{\class}{Laboratorio Intermedio} % This is the name of the course 
\newcommand{\examnum}{Tarea Estadistica} % This is the name of the assignment
\newcommand{\examdate}{\today} % This is the due date
\newcommand{\timelimit}{}





\begin{document}
\pagestyle{plain}
\thispagestyle{empty}

\noindent
\begin{tabular*}{\textwidth}{l @{\extracolsep{\fill}} r @{\extracolsep{6pt}} l}
  \textbf{\class} & \textbf{Name:} & \textit{Sergio Montoya Ramirez}\\ %Your name here instead, obviously 
  \textbf{\examnum} && \textit{Angelica Lopez Duarte}\\
  \textbf{\examdate} &&
\end{tabular*}\\
\rule[2ex]{\textwidth}{2pt}
% ---

\section{Punto 2.1}

Podemos usar el codigo

\lstinputlisting[language=Python]{./code/punto_2_1.py}

lo que nos da:
\begin{lstlisting}[language=Bash]
$ uv run punto_2_1.py
promedio=26.0625
desviacion_rar=0.29166666666666663
desviacion_2_3=0.49280538030458104
error_rar=0.10311973892303816
error_2_3=0.17423301310929235
\end{lstlisting}

\section{Punto 2.6}
\subsection{}

Numero de datos: 5

\begin{itemize}
  \item $\alpha = 0.01913 \approx 0.019$
  \item $\bar{\delta} = 3.27346 \approx 3.273$
\end{itemize}


\textbf{Resultado: }$3.273 \pm 0.019$

\subsection{}

Numero de datos: 50

\begin{itemize}
  \item $\alpha = 0.002506 \approx 0.0025$
  \item $\bar{\delta} = 3.26513 \approx 3.2651$
\end{itemize}


\textbf{Resultado: }$3.25513 \pm 0.019$

\subsection{}

Numero de datos: 500

\begin{itemize}
  \item $\alpha = 0.000270 \approx 0.000270$
  \item $\bar{\delta} = 3.26681 \approx 3.26681$
\end{itemize}


\textbf{Resultado: }$3.273 \pm 0.019$

\section{Punto 3.4}

Podemos crear la funcion (sin integrarla) en sympy como
$$\frac{\sqrt{2} e^{- \frac{\left(- \mu + x\right)^{2}}{2 \sigma^{2}}}}{2 \sqrt{\pi} \sigma}$$

y con eso podemos implementar un codigo:

\lstinputlisting[language=Python]{./code/punto_3_2.py}

Que nos da como resultado:

\begin{tabular}{lll}
  \hline
  Centrado en media   & Medidas dentro del rango   & Medidas fuera del rango   \\
  \hline
  $\pm 1 \sigma$        & 68.27\%                     & 31.73\%                    \\
  $\pm 1.65 \sigma$     & 90.11\%                     & 9.89\%                     \\
  $\pm 2 \sigma$        & 95.45\%                     & 4.55\%                     \\
  $\pm 2.58 \sigma$     & 99.01\%                     & 0.99\%                     \\
  $\pm 3 \sigma$        & 99.73\%                     & 0.27\%                     \\
  $\pm 4 \sigma$        & 99.99\%                     & 0.01\%                     \\
  $\pm 5 \sigma$        & 100.00\%                    & 0.00\%                     \\
  \hline
\end{tabular}

Que coincide con lo que esperamos

\section{Punto 4.1}

\subsection{$Z = 2A$}
\begin{itemize}
\item $\frac{dZ}{dA} = 2$
\item $\delta Z = \left|\frac{dZ}{dA}\right|\_{\bar{A}} \delta A = |2|(0.005) \approx 0.01$
\item $Z \approx 18.54800 \pm 0.01000$
\end{itemize}
\vspace{0.5cm}

\subsection{$Z = A/2$}
\begin{itemize}
\item $\frac{dZ}{dA} = \frac{1}{2}$
\item $\delta Z = \left|\frac{dZ}{dA}\right|\_{\bar{A}} \delta A = |0.5|(0.005) \approx 0.0025$
\item $Z \approx 4.63700 \pm 0.00250$
\end{itemize}
\vspace{0.5cm}

\subsection{$Z = \frac{A-1}{A+1}$}
\begin{itemize}
\item $\frac{dZ}{dA} = - \frac{A - 1}{\left(A + 1\right)^{2}} + \frac{1}{A + 1}$
\item $\delta Z = \left|\frac{dZ}{dA}\right|\_{\bar{A}} \delta A = |0.018947|(0.005) \approx 9.4737e-05$
\item $Z \approx 0.80533 \pm 0.00009$
\end{itemize}
\vspace{0.5cm}

\subsection{$Z = \frac{A^2}{A-2}$}
\begin{itemize}
\item $\frac{dZ}{dA} = - \frac{A^{2}}{\left(A - 2\right)^{2}} + \frac{2 A}{A - 2}$
\item $\delta Z = \left|\frac{dZ}{dA}\right|\_{\bar{A}} \delta A = |0.9244|(0.005) \approx 0.004622$
\item $Z \approx 11.82390 \pm 0.00462$
\end{itemize}
\vspace{0.5cm}

\subsection{$Z = \arcsin(\frac{1}{A})$}
\begin{itemize}
\item $\frac{dZ}{dA} = - \frac{1}{A^{2} \sqrt{1 - \frac{1}{A^{2}}}}$
\item $\delta Z = \left|\frac{dZ}{dA}\right|\_{\bar{A}} \delta A = |-0.011695|(0.005) \approx 5.8476e-05$
\item $Z \approx 0.10804 \pm 0.00006$
\end{itemize}
\vspace{0.5cm}

\subsection{$Z = \sqrt{A}$}
\begin{itemize}
\item $\frac{dZ}{dA} = \frac{1}{2 \sqrt{A}}$
\item $\delta Z = \left|\frac{dZ}{dA}\right|\_{\bar{A}} \delta A = |0.16419|(0.005) \approx 0.00082093$
\item $Z \approx 3.04532 \pm 0.00082$
\end{itemize}
\vspace{0.5cm}

\subsection{$Z = \ln(\frac{1}{\sqrt{A}})$}
\begin{itemize}
\item $\frac{dZ}{dA} = - \frac{1}{2 A}$
\item $\delta Z = \left|\frac{dZ}{dA}\right|\_{\bar{A}} \delta A = |-0.053914|(0.005) \approx 0.00026957$
\item $Z \approx -1.11361 \pm 0.00027$
\end{itemize}
\vspace{0.5cm}

\subsection{$Z = \exp(A^2)$}
\begin{itemize}
\item $\frac{dZ}{dA} = 2 A e^{A^{2}}$
\item $\delta Z = \left|\frac{dZ}{dA}\right|\_{\bar{A}} \delta A = |4.1754e+38|(0.005) \approx 2.0877e+36$
\item $Z \approx 2.251e+37 \pm 2.088e+36$
\end{itemize}
\vspace{0.5cm}

\subsection{$Z = A + \sqrt{\frac{1}{A}}$}
\begin{itemize}
\item $\frac{dZ}{dA} = 1 - \frac{\sqrt{\frac{1}{A}}}{2 A}$
\item $\delta Z = \left|\frac{dZ}{dA}\right|\_{\bar{A}} \delta A = |0.9823|(0.005) \approx 0.0049115$
\item $Z \approx 9.60237 \pm 0.00491$
\end{itemize}
\vspace{0.5cm}

\subsection{$Z = 10^A$}
\begin{itemize}
\item $\frac{dZ}{dA} = 10^{A} \log{\left(10 \right)}$
\item $\delta Z = \left|\frac{dZ}{dA}\right|\_{\bar{A}} \delta A = |4.3273e+09|(0.005) \approx 2.1636e+07$
\item $Z \approx 1.879e+09 \pm 2.164e+07$
\end{itemize}
\vspace{0.5cm}

\section{Punto 4.4}

Podemos reescribir la formula que nos pidieron en sympy y despejar la ecuacion 4.10 y con eso encontrar los resultados. Si lo hacemos para un valor generico (Es decir, $\theta_i$ y $\theta_t$) los resultados son
\begin{align*}
  R &= \frac{\tan^{2}{\left(\theta_{i} - \theta_{t} \right)}}{\tan^{2}{\left(\theta_{i} + \theta_{t} \right)}}\\
\end{align*}

\[
\delta_R = \sqrt{
  \begin{aligned}
    &\delta_{\theta_i}^{2} \left(
      \frac{
        \left(2 \tan^{2}{\left(\theta_{i} - \theta_{t} \right)} + 2\right)
        \tan{\left(\theta_{i} - \theta_{t} \right)}
      }{
        \tan^{2}{\left(\theta_{i} + \theta_{t} \right)}
      } 
      + 
      \frac{
        \left(- 2 \tan^{2}{\left(\theta_{i} + \theta_{t} \right)} - 2\right)
        \tan^{2}{\left(\theta_{i} - \theta_{t} \right)}
      }{
        \tan^{3}{\left(\theta_{i} + \theta_{t} \right)}
      }
    \right)^{2}
    \\
    &+ \delta_{\theta_t}^{2} \left(
      \frac{
        \left(- 2 \tan^{2}{\left(\theta_{i} - \theta_{t} \right)} - 2\right)
        \tan{\left(\theta_{i} - \theta_{t} \right)}
      }{
        \tan^{2}{\left(\theta_{i} + \theta_{t} \right)}
      } 
      + 
      \frac{
        \left(- 2 \tan^{2}{\left(\theta_{i} + \theta_{t} \right)} - 2\right)
        \tan^{2}{\left(\theta_{i} - \theta_{t} \right)}
      }{
        \tan^{3}{\left(\theta_{i} + \theta_{t} \right)}
      }
    \right)^{2}
  \end{aligned}
}
\]

Ahora reemplazando a los valores que nos dieron para $\theta_i$ y $\theta_t$ como puede ver en el siguiente script:

\lstinputlisting[language=Python]{./code/punto_4_4.py}

al ejecutar este script nos devuelve:
\begin{equation*}
  0.00118 \pm 9 \cdot 10^{-5}
\end{equation*}

\section{Punto 6.1}

\end{document}

\documentclass[a4paper, amsfonts, amssymb, amsmath, reprint, showkeys, nofootinbib, twoside]{revtex4-1}
\usepackage[spanish]{babel}
\usepackage[utf8]{inputenc}
\usepackage{float}
\usepackage[colorinlistoftodos, color=green!40, prependcaption]{todonotes}
\usepackage{amsthm}
\usepackage{mathtools}
\usepackage{physics}
\usepackage{xcolor}
\usepackage{graphicx}
\usepackage[left=23mm,right=13mm,top=35mm,columnsep=15pt]{geometry} 
\usepackage{adjustbox}
\usepackage{placeins}
\usepackage[T1]{fontenc}
\usepackage{lipsum}
\usepackage{csquotes}
\usepackage[table]{xcolor}
\usepackage[normalem]{ulem}
\useunder{\uline}{\ul}{}
\usepackage[pdftex, pdftitle={Article}, pdfauthor={Author}]{hyperref} % For hyperlinks in the PDF
%\setlength{\marginparwidth}{2.5cm}
\bibliographystyle{apsrev4-1}

\begin{document}

%El título del experimento realizado es importante.
\title{Relación carga masa del electrón}


\author{Angelica Lopez Duarte}
\email[Correo institucional: ]{TODO: Pon tu correo}

\author{Sergio Montoya}
\email[Correo institucional: ]{s.montoyar2@uniandes.edu.co}
%Si necesitan poner un segundo autor, deben eliminar los porcentajes (%) iniciales.
  
%\author{Second Author}
%\email{Second.Author@institution.edu}

\affiliation{Universidad de los Andes, Bogotá, Colombia.}

\date{\today} % Si lo dejan vacío no les saldrá fecha. La fecha que se muestra es del día en que se compila.

\begin{abstract}

En este se describen brevemente los objetivos y los resultados del trabajo, por lo tanto se debe dar información completa pero corta del contenido del trabajo. Se debe indicar qué fue lo que se hizo, cómo se hizo y cuáles fueron los resultados obtenidos de forma EXPLÍCITA. Por ejemplo: se obtuvo un valor para la constante de Planck de $h=(5.9\pm 0.5)\times 10^{-34}\,\text{J s}$ \footnote{Utilizar esta forma para reportar los datos, note que las unidades están en un cuadro de texto. TODO debe llevar unidades e incertidumbre, siempre.}. A lo largo de todo el informe por favor utilizar el diccionario de overleaf. Tener ideas claras y concisas de lo que se hizo y de los resultados obtenidos. Que genere interés en leer el resto del artículo. El informe, que está en formato de artículo científico debe ser auto-contenido. Un lector que no haya ido al laboratorio, o que no sepa del experimento debe ser capaz de leerlo y entender todo lo que allí de plantea.

\end{abstract}

\maketitle

\section{Analisis}

\subsection{Campo respecto a Corriente}

Lo primero que debemos realizar para esta practica de laboratorio es determinar la relación entre la corriente que pasa por nuestras bobinas y el campo que se genera. Para esto, tomando los datos con el teslametro se obtuvo la figura \ref{fig:act1_fig1}. 

\begin{figure}[h]
  \centering
  \includegraphics[width=0.4\textwidth]{./img/Act1/CampoVsCorriente.png}
  \caption{Grafica de campo generado por las bobinas respecto a la corriente que pasa con ellas. Con este grafico se puede mostrar la relación de linealidad que existe entre las dos variables.}
  \label{fig:act1_fig1}
\end{figure}

\subsection{Voltaje Hall respecto a Campo}

En este caso necesitamos encontrar $R_H$ para lo cual podemos usar ley de Ohm y convertirla en 
\begin{equation}
  R_H = \frac{m \omega}{B_z}
  \label{eq:act2_eq1}
\end{equation}

Para ambos casos, con esto entonces, necesitamos tomar los datos para diferentes $B_z$ de modo tal que podamos utilizar el valor $\omega = 1\times 10^{-3}m$. Para esto necesitamos obtener los valores de regresiones lineales tanto para \textit{germanio-p} como para \textit{germanion-n}. Estos valores los puede ver en la grafica \ref{fig:act2_fig1}

\begin{figure}
  \begin{center}
    \includegraphics[width=0.4\textwidth]{./img/Act2/VoltajeHallVsCorriente_Ip.png}
  \end{center}
  \caption{Grafica de voltaje Hall respecto a la Corriente $I_p$ junto con regresiones lineales que, como se puede ver,  se ajustan increiblemente bien a los datos. Estas graficas fueron sacadas de datos tomados a diferentes corrientes las cuales puede observar por la distribución de colores.}\label{fig:act2_fig1}
\end{figure}

con estos datos y sus respectivas regresiones lineales podemos calcular $R_H$ con la ecuación \ref{eq:act2_eq1} lo que nos da la tabla \ref{tab:act1_tab1}

\begin{table*}[!h]
\centering
  \caption{Coeficientes de Hall calculados por medio de la regresion lineal encontrada en el primer punto y las pendientes que se hallaron posteriormente en esta sección.}
\label{tab:act1_tab1}
\centering
\begin{tabular}[t]{l|l|r|r|r}
\hline
Tipo & Corriente (A) & Pendiente (V/A) & B\_z (T) & R\_H (m³/C)\\
\hline
\cellcolor{gray!10}{Germanio n} & \cellcolor{gray!10}{0} & \cellcolor{gray!10}{1.240659} & \cellcolor{gray!10}{12.41424} & \cellcolor{gray!10}{0.000100}\\
\hline
Germanio n & 0.0046 & -2.039286 & 13.52365 & -0.000151\\
\hline
\cellcolor{gray!10}{Germanio n} & \cellcolor{gray!10}{0.0078} & \cellcolor{gray!10}{2.401648} & \cellcolor{gray!10}{14.29541} & \cellcolor{gray!10}{0.000168}\\
\hline
Germanio n & 0.01 & -2.896429 & 14.82600 & -0.000195\\
\hline
\cellcolor{gray!10}{Germanio p} & \cellcolor{gray!10}{0} & \cellcolor{gray!10}{0.246266} & \cellcolor{gray!10}{12.41424} & \cellcolor{gray!10}{0.000020}\\
\hline
Germanio p & 0.0046 & -0.357143 & 13.52365 & -0.000026\\
\hline
\cellcolor{gray!10}{Germanio p} & \cellcolor{gray!10}{0.0078} & \cellcolor{gray!10}{-0.711538} & \cellcolor{gray!10}{14.29541} & \cellcolor{gray!10}{-0.000050}\\
\hline
Germanio p & 0.01 & -1.007143 & 14.82600 & -0.000068\\
\hline
\end{tabular}
\end{table*}

\subsection{Voltaje Hall Respecto a Corriente $I_p$}
Ademas de eso, podemos realizar una operación relativamente similar pero esta vez apartir de la corriente $I_p$ que tiene nuestro sistema. Si la mantenemos constante y lo que variamos el campo podemos usar

\begin{equation}
  R_H = \frac{m \times \omega}{I_p}.
  \label{eq:sec2_eq2}
\end{equation}

Para esto, como en el punto anterior necesitamos las pendientes las cuales puede observar en la figura\ref{fig:act2_fig2}. Los resultados de esto se pueden ver en la tabla \ref{tab:act1_tab2}.

\begin{figure}
  \begin{center}
    \includegraphics[width=0.4\textwidth]{./img/Act2/VoltajeHallVsCorriente.png}
  \end{center}
  \caption{Grafica de voltaje Hall respecto a la Corriente $I_p$ junto con regresiones lineales que, como se puede ver,  se ajustan increiblemente bien a los datos. Estas graficas fueron sacadas de datos tomados a diferentes corrientes las cuales puede observar por la distribución de colores.}\label{fig:act2_fig2}
\end{figure}


\begin{table*}[!h]
\centering
  \caption{Coeficientes de Hall calculados por medio de la corriente Ip y las pendientes que se hallaron posteriormente en esta sección.}
\label{tab:act1_tab2}
\centering
\begin{tabular}[t]{l|l|r|r|r}
\hline
Tipo & Corriente I\_p (A) & Pendiente (V/A) & B\_z (T) & R\_H (m³/C)\\
\hline
\cellcolor{gray!10}{Germanio n} & \cellcolor{gray!10}{-0.025} & \cellcolor{gray!10}{0.387267} & \cellcolor{gray!10}{6.384831} & \cellcolor{gray!10}{-0.015491}\\
\hline
Germanio n & -0.01 & 0.171309 & 10.002474 & -0.017131\\
\hline
\cellcolor{gray!10}{Germanio n} & \cellcolor{gray!10}{0} & \cellcolor{gray!10}{-0.032249} & \cellcolor{gray!10}{12.414236} & \cellcolor{gray!10}{-Inf}\\
\hline
Germanio n & 0.01 & -0.137070 & 14.825998 & -0.013707\\
\hline
\cellcolor{gray!10}{Germanio p} & \cellcolor{gray!10}{-0.025} & \cellcolor{gray!10}{0.279648} & \cellcolor{gray!10}{6.384831} & \cellcolor{gray!10}{-0.011186}\\
\hline
Germanio p & -0.01 & 0.150420 & 10.002474 & -0.015042\\
\hline
\cellcolor{gray!10}{Germanio p} & \cellcolor{gray!10}{0} & \cellcolor{gray!10}{0.012553} & \cellcolor{gray!10}{12.414236} & \cellcolor{gray!10}{Inf}\\
\hline
Germanio p & 0.01 & -0.105551 & 14.825998 & -0.010555\\
\hline
\end{tabular}
\end{table*}

\subsection{Magnetorresistencia}

El voltaje longitudinal que medimos cuando no hay campo se tiene una resistencia obtenible por medio de la ley de Ohm. Sin embargo, el voltaje va cambiando a medida que se cambia el campo (cuando este esta presente) lo que implica un cambio en la resistencia por causas magneticas. Para este caso entonces se grafica el cambio de esta resistencia respecto a la corriente. Estos resultados se pueden ver en las figuras \ref{fig:act3_fig1} y \ref{fig:act3_fig2}.

\begin{figure}
  \begin{center}
    \includegraphics[width=0.4\textwidth]{./img/Act3/R_m_n_I_separado.png}
  \end{center}
  \caption{Grafica del cambio de la resistencia relativa respecto a la corriente que pasa por las espiras para el material \textit{Germanio n}}\label{fig:act3_fig1}
\end{figure}

\begin{figure}
  \begin{center}
    \includegraphics[width=0.4\textwidth]{./img/Act3/R_m_p_I_separado.png}
  \end{center}
  \caption{Grafica del cambio de la resistencia relativa respecto a la corriente que pasa por las espiras para el material \textit{Germanio p}}\label{fig:act3_fig2}
\end{figure}

\subsection{Voltaje Respecto Temperatura}

En este caso lo que nos interesa ver es encontrar el como varia el voltaje de Hall al variar la temperatura para ambos doppajes. Estos resultados nos dieron las figura \ref{fig:act4_fig1}

\begin{figure}
  \begin{center}
    \includegraphics[width=0.4\textwidth]{./img/Act4/temperatura.png}
  \end{center}
  \caption{Grafica de Voltaje Hall respecto a Temperatura. En esta se pueden observar las diferentes regiones por las que pasa el material siendo la zona extrinseca donde la pendiente de la grafica es practicamente una recta y la intrinseca cuando la pendiente hace menos pronunciada.}\label{fig:act4_fig1}
\end{figure}

\section{Conclusiones}

Durante el desarrollo de este experimento se verifico la existencia del efecto hall asi como su relación con la temperatura y el campo. Para hacer esto se realizaron varias mediciones. En primera instancia se probo calcular $R_H$ para dos tipos de materiales y por medio de dos rutas. Primero, se tomo con una corriente que pasaba por el material de manera quieta y se calculo estos indices encontrando la pendiente que exhibian estos datos. Los resultados de esta parte pueden verse en la tabla \ref{tab:act1_tab1}. Por otro lado, tambien se tomaron datos esta vez variando la corriente que pasaba dentro del material y cambiando el campo al que estaba sometido. Con esta información se encontro los datos de la tabla \ref{tab:act1_tab2}. Posteriormente se calculo y observo el cambio en la resistencia al cambiar el campo. Para esto, se tomaron datos del voltaje longitudinal y se calculo (por medio de ley de Ohm) el valor que toman dada la corriente que estaba pasando y se comparo de manera porcentual con la resistencia que existia cuando no habia campo. Con esta información se encontraron las figura \ref{fig:act3_fig1} y \ref{fig:act3_fig2}. Por ultimo, se observaron los cambios en el voltaje de Hall que experimentaba la muestra al cambiarle la temperatura. Con esta información se pudo observar la presencia de dos zonas diferenciadas por la forma de su derivada.

\section{Introducción}

Se da la información básica para ubicar el problema (marco teórico) resaltando la importancia y los métodos utilizados para resolverlo. Se hace un estado del arte del problema experimental abordado: cuándo se realizó este experimento por primera vez(en la historia), bajo qué condiciones, cómo han evolucionado los métodos hasta el montaje con el que va a trabajar y qué alternativas hay actualmente para hacer el experimento. Todo el estado del arte debe tener las referencias claras de los artículos CIENTÍFICOS \footnote{Puede utilizar notas al pie de página para hacer aclaraciones que considere necesarias en cualquier parte del artículo. Por ejemplo en esta: Se colocó el científico todo en mayúscula para hacer énfasis en que deben ser artículos de revistas científicas y no notas periodísticas o un blog de internet.} que evidencien una revisión bibliográfica extensa. Se hace énfasis en que esta debe ser básicamente un texto corto que resuma claramente el fenómeno físico y el experimento para estudiarlo. En esta  sección se deben explicar las ecuaciones (No hacer pasos intermedios de álgebra en la parte de marco teórico /estado del arte) que se van a utilizar.

En su mayoría, las referencias deben estar en esta sección. Se deben mencionar como: según el autor Lipari \cite{Articulo1} en su artículo \emph{Proton and neutrino extragalactic astronomy} se puede deducir que $\dots$ como dice Serway en \cite{serway2006fisica}.

También puede haber referencias las cuales se usaron pero no se parafrasearon ni se quiere hacer mención específica de la persona. \nocite{Articulo2} La segunda referencia de la bibliografía es un ejemplo de este caso.

Esta sección debe tener cohesión. Los temas de los que se hablan deben estar conectados, el hilo conductor a lo largo de todo el artículo es muy importante.

\section{Montaje experimental}

Descripción BREVE del método, procedimiento y montaje  experimental. Debe contener figuras, diagramas explicativos, esquema de circuitos eléctricos si aplica y las condiciones experimentales para la toma de los datos (por ejemplo entre qué valores se va a variar el voltaje en una medición). Este es el nuevo artículo \cite{Flectcher1994}. Se recomienda tomar foto del montaje experimental y luego editarla para que sean claras las partes y procedimientos. NO hacer una lista con los materiales usados, tampoco colocar pasos triviales del procedimiento como por ejemplo \emph{conectar todo de forma adecuada}.


\begin{table*}
\caption{\label{tab:table3}Esta es una tabla amplia que ocupa toda la página con un ancho diseño de dos columnas. Se da formato con el entorno \texttt{table*}. También demuestra el uso de \textbackslash \texttt{multicolumn} en filas con entradas que abarcan
más de una columna. Las tablas deben ser de resultados obtenidos, no de datos crudos. El \emph{caption} debe tener información detallada suficiente para entender lo que se obtuvo sin tener que entrar a leer todo el documento. La tabla debe estar referenciada dentro del texto. En ocasiones se usan notas al pie adicional al \emph{caption} para resaltar detalles de datos distintos. NO COLOCAR TABLAS COMO FIGURAS. No usar notación científica de EXCEL.}
\begin{ruledtabular}
\begin{tabular}{ccccc}
&\multicolumn{2}{c}{$D_{4h}^1$}&\multicolumn{2}{c}{$D_{4h}^5$}\\
Ion&1st alternative&2nd alternative&lst alternative
&2nd alternative\\ \hline
K&$(2e)+(2f)$&$(4i)$ &$(2c)+(2d)$&$(4f)$ \\
Mn&$(2g)$\footnote{The $z$ parameter of these positions is $z\sim\frac{1}{4}$.}
&$(a)+(b)+(c)+(d)$&$(4e)$&$(2a)+(2b)$\\
Cl&$(a)+(b)+(c)+(d)$&$(2g)$\footnotemark[1]
&$(4e)^{\text{a}}$\\
He&$(8r)^{\text{a}}$&$(4j)^{\text{a}}$&$(4g)^{\text{a}}$\\
Ag& &$(4k)^{\text{a}}$& &$(4h)^{\text{a}}$\\
\end{tabular}
\end{ruledtabular}
\end{table*}


\section{Resultados y análisis}

Resultados: tablas, figuras, regresiones, valores experimentales.
\\
Análisis: Texto argumentando y discutiendo los resultados. Análisis de error.

\begin{itemize}
    \item Tablas de datos: Deben estar numeradas, tituladas y rotuladas (encabezados con variables y  unidades  coherentes).  Adicionalmente  deben  estar  comentadas  o  referenciadas  en  el texto. Ver tabla \ref{tab:table3} como ejemplo. 
    \item Gráficas: Todas las gráfica deben estar numeradas, tituladas; ejes claros y de tamaño legible, escalas, variables y unidades coherentes. La mayoría de las gráficas tienen ajustes a diferentes tipos de curvas, de ser así, debe presentar en la gráfica la ecuación del ajuste realizado  con  las  variables  correspondientes  a  la  gráfica.  Todas las gráficas deben estar comentadas o referenciadas en el texto; es decir, en alguna parte del texto debe haber un \emph{cómo se ven en la figura 3} o \emph{los resultados se encuentran en la tabla 2}. No usar, \emph{en la siguiente figura}, \emph{en la siguiente tabla}, \emph{en la figura de arriba}. Si tienen varias gráficas y ven que se pueden agrupar en 1 sola,¡háganlo! No hacer 1 gráfica por cada toma de datos. Ver figura \ref{fig:my_label}.
    
    \begin{figure}[h]
    \centering
    \includegraphics[scale=0.5]{GraphEj.png}
    \caption{El \emph{caption} de esta figura debe tener una descripción detallada de lo que se está mostrando como resultado. Un \emph{caption} que diga ``Gráfica de T vs P'' no dice absolutamente nada de cómo fueron obtenidos los datos y los resultados que se pueden concluir a partir de estos. Siempre los datos deben llevar las respectivas barras de error o indicar que el tamaño de estas es muy pequeño para ser visualizadas. Usar colores como los que se ve en la figura para diferenciar fácilmente los datos. Nunca unir los puntos experimentales, las líneas sólidas están reservadas para los ajustes. No usar notación científica de Excel.}
    \label{fig:my_label}
    \end{figure}

    \item Analizar los resultados expuestos en las tablas y gráficas. Comparar estos resultados con la teoría, objetivos e hipótesis propuestas en la introducción. Se debe reportar si sus resultados y lo que aprendió en la práctica lo lleva a cumplir los objetivos propuestos, si son valores deseables o no; escriba oraciones completas que expresen sus ideas. Si se realizó ajuste, este debe estar referenciado en la gráfica y en el texto (con su respectiva incertidumbre). El cálculo de la incertidumbre del resultado obtenido debe estar situado junto al resultado obtenido, no debe aparecer el resultado al inicio y la incertidumbre al final.
    \item Cuando consultan un valor experimental que se mide con mucha precisión, por ejemplo una constante física, se denomina: \emph{el valor reportado en la literatura} y colocan la referencia (aconsejo la NIST). No usar más \emph{el valor teórico}.
    \item No vuelvan nunca a calcular un error porcentual. Con la incertidumbre y el valor reportado en la literatura pueden hacer un mejor análisis.
    \item Recuerden que todas las incertidumbres llevan 1 cifra significativa (máximo 2 en casos especiales).

\end{itemize}





\section{Conclusiones}

Se deben contestar las preguntas planteadas inicialmente o dar las razones por las cuales no es posible hacerlo. Las conclusiones deben ser necesariamente una consecuencia del experimento realizado, es  decir  que  no  se  deben  tocar  aspectos  que  no  se  hayan  expuesto  en  la  sección  de resultados y análisis. Si escribe algo que no se encuentra en la sección de resultados y análisis,  esto quiere decir que hace falta incluir material en resultados y análisis. Concluir únicamente aspectos pertinentes al trabajo obtenido en el laboratorio; se deben evitar las generalizaciones que no hablan concretamente de lo que lograron o midieron en la práctica experimental. NO USAR ITEMIZE. No reportar resultados nuevos en esta sección, pueden comparar los resultados con los obtenidos por otros autores (por ejemplo sus compañeros o los que encontraron en su revisión bibliográfica).


% IMPORTANTE: Comentar o eliminar la siguiente linea cuando vayan a realizar el INFORME
\bibliographystyle{apalike}


\bibliography{Referencias}
\nocite{Ejemplo}

\section*{Apéndice de cálculo de errores}

Se deben indicar explícitamente los pasos de análisis de error que se hicieron para llegar a al(los) resultado(s). Ejemplo: la propagación de error, incertidumbre en un ajuste de mínimos cuadrados, análisis estadístico, redondeo de cifras significativas, entre otros.

Las fórmulas de cómo se obtuvieron cada uno de los valores reportados debe ser incluido como si el análisis estadístico se hiciera manualmente.
\end{document}

\documentclass[12pt]{article}

\usepackage{graphicx}
\usepackage{epstopdf}
%\usepackage[spanish]{babel}
\usepackage[english]{babel}
\usepackage[latin5]{inputenc}
\usepackage{hyperref}
\usepackage[left=3cm,top=3cm,right=3cm,nohead,nofoot]{geometry}
\usepackage{braket}
\usepackage{datenumber}
\usepackage{float}
\usepackage{amsmath}
%\newdate{date}{10}{05}{2013}
%\date{\displaydate{date}}

% Personalizaciones para que se parezca más a tu ejemplo

\begin{document}

\begin{center}
  \Huge
  Theory Foundations of Quantum Error-Correction

  \vspace{3mm}
  \Large Sergio Montoya Ramirez

  \large
  202112171


  \vspace{2mm}
  \Large
  Director: Juli\'an Rinc\'on

  \normalsize
  \vspace{2mm}

  \today
\end{center}


\normalsize

\section{Introduction}
%Introducción a la propuesta de Monograf�a. Debe incluir un breve resumen del estado del arte del problema a tratar. Tambi�n deben aparecer citadas todas las referencias de la bibliograf�a (a menos de que se citen m�s adelante, en los objetivos o metodolog�a, por ejemplo)

Quantum computing (QC) is a promising paradigm that offers potentially exponential advantages over classical computing for specific problems, such as factorization, quantum simulation, and optimization. A key property enabling these benefits is the linearity of quantum mechanics, which, in principle, prevents errors from accumulating between gates (Kitaev, 2002). However, the inherent quantum nature and extreme sensitivity of these systems to errors, which can be introduced by gates or external conditions (e.g., environmental fluctuations, electromagnetic interference), mean that linearity alone is insufficient to ensure reliable results, particularly for large-scale operations like factorization (Calderbank \& Shor, 1996). Furthermore, the no-cloning theorem prohibits the copying of qubits, rendering traditional error correction methods, such as redundancy, inapplicable.

Quantum error-correction (QEC) is a field developed to address this challenge. It was demonstrated that quantum information could be encoded non-locally by distributing it across multiple qubits (Calderbank \& Shor, 1996). Codes that employ this strategy are known as quantum error-correcting codes (QECCs). The efficacy of this method has been supported by experimental improvements in physical quantum computers (AbuGhanem, 2025). Nevertheless, quantum error-correction remains an open area of research, and advancements in QEC are critical for the continued development of QC (AbuGhanem, 2025).

Significant theoretical advances have established the foundations of QEC, which will continue to guide future development in the field. These foundational elements include:

\begin{itemize}
  \item \textbf{General theory for quantum error-correction:} This framework provides a mathematical formalization of key concepts, including noisy quantum channels, error models, and the operational principles for correcting errors.
  
  \item \textbf{Hamming bound:} This bound determines the minimum size of a code required to correct a specific type of noise. For a non-degenerate code encoding \(k\) logical qubits into \(n\) physical qubits to correct up to \(t\) errors, the following inequality must be satisfied:
  \[
  \sum_{j=0}^{t} \binom{n}{j} 3^j 2^k \le 2^n
  \]
  (Nielsen \& Chuang, 2010).
  
\item \textbf{Stabilizer formalism:} This formalism uses group theory to describe quantum states in terms of the operators that stabilize them. For instance, a state \(\ket{\psi}\) is stabilized by \(X_1X_2\) if and only if \(X_1X_2\ket{\psi} = \ket{\psi}\) (Gottesman, 1998)
  
  \item \textbf{Stabilizer codes:} An important class of QECCs is built upon the stabilizer formalism. Many prominent codes, such as Shor codes and Calderbank-Shor-Steane (CSS) codes, are stabilizer codes (Nielsen \& Chuang, 2010).
\end{itemize}

This project will explore the landscape of QECCs. This thesis will investigate how quantum codes can be theoretically implemented and simulated, analyzing their performance under various noise models and operational constraints. Through both theoretical analysis and computational simulation, we aim to provide a comprehensive guide of QEC and how impactfull it is for QC.

\section{General Objective}

%Objetivo general del trabajo. Empieza con un verbo en infinitivo.

To develop a theoretical analysis of quantum error-correction field, accompanying it with computational and theoretical error analyses for determining its improvements based on the noise and parameters.

\section{Specific Objectives}

%Objetivos espec�ficos del trabajo. Empiezan con un verbo en infinitivo.

\begin{enumerate}
  \item To theoretically understand the stabilizer formalism and codes.
  \item To establish the theoretical foundations of quantum error-correction and quantum error-correcting Codes.
  \item To design, implement, and analyze classical simulations of representative QECCs using quantum computing frameworks.
  \item To perform a comparative error analysis of the implemented QECCs quantifying their quality.
\end{enumerate}

\section{Metodology}

%Exponer DETALLADAMENTE la metodolog�a que se usar� en la Monograf�a. 

%Monograf�a te�rica o computacional: �C�mo se har�n los c�lculos te�ricos? �C�mo se har�n las simulaciones? �Qu� requerimientos computacionales se necesitan? �Qu� espacios f�sicos o virtuales se van a utilizar?

%Monograf�a experimental: Recordar que para ser aprobada, los aparatos e insumos experimentales que se usar�n en la Monograf�a deben estar previamente disponibles en la Universidad, o garantizar su disponibilidad para el tiempo en el que se realizar� la misma. �Qu� montajes experimentales se van a usar y que material se requiere? �En qu� espacio f�sico se llevar�n a cabo los experimentos? Si se usan aparatos externos, �qu� permisos se necesitan? Si hay que realizar pagos a terceros, �c�mo se financiar� esto?
The project will build upon the foundation provided in the "Introducci\'on a la computaci\'on cuantica" course offered by the university in which I enroll. It will involve a state-of-the-art revision and a thorough study and understanding of key concepts in the field of quantum error-correction. This revision will rely on bibliographic study and the reproduction of proofs, resulting in a document that explains and proves key theorems in QEC, alongside theoretical explanations and simulations of representative codes. Some of the concepts included are:

\begin{itemize}
\item Hamming Bound.
\item Quantum Error Models and Linear Error Accumulation.
\item Discretization of Errors.
\item Independent Error Models.
\item Shor Code.
\item Calderbank-Shor-Steane Codes.
\item Stabilizer Codes.
\item Stabilizer Formalism.
\end{itemize}

Following the recommendations from (McGeoch, 2012), the simulation environment will be designed to separate parameters related to the QECC itself (for example, the number of qubits in the encoding) from those related to the specific noise model (for example, the probability of an error occurring). This separation will improve the correspondence and interpretability between parameters and results. The test suite will be created using Qiskit, IBM's framework for quantum computing, and will rely primarily on classical simulations due to the prohibitive cost and limited availability of physical quantum machines.

\section{Ethical Considerations}
%A partir del periodo 2017-20 debe incluirse en el formato de propuesta de monografía una sección titulada Consideraciones éticas. Esta sección debe incluir los detalles relacionados con aspectos éticos involucrados en el proyecto. Por ejemplo, se puede describir el protocolo establecido para el manejo de datos de manera que se asegure que no habrá manipulación de la información, ni habrá plagio de los mismos. También se puede tener en cuenta si hay algún conflicto de intereses involucrado en el desarrollo del proyecto o se puede detallar si el trabajo está relacionado con las actividades y poblaciones humanas mencionadas en el siguiente link https://ciencias.uniandes.edu.co/investigacion/comite-de-etica. Es importante tener en cuenta que esta sección debe incluir una frase explícita sobre si el proyecto debe pasar o no a estudio del comité de ética de la Facultad de Ciencias.

This project will not employ any methodology involving vulnerable populations or living subjects. Furthermore, it will not utilize any form of confidential or sensitive data for its processing or results. Any information or data external to the authors will be properly cited and accredited in accordance with the specifications provided by the university. Consequently, no phase of this study requires explicit approval from an ethics committee, and thus, submission to such a committee is deemed unnecessary.


\section{Schedule}

\begin{table}[H]
  \begin{tabular}{|c|cccccccccccccccc| }
    \hline
    Tasks $\backslash$ Weeks & 1 & 2 & 3 & 4 & 5 & 6 & 7 & 8 & 9 & 10 & 11 & 12 & 13 & 14 & 15 & 16  \\
    \hline
    1 & X & X & X & X &   &   &   &   &   & X & X & X & X &   &   &   \\
    2 & X & X & X &   &   & X & X &   & X & X &   &   &   &   &   &   \\
    3 &   &   & X &   &   & X &   &   & X &   & X &   &   &   &   &   \\
    4 &   &   &   & X & X &   & X & X &   &   &   & X & X &   &   &   \\
    5 &   &   &   &   & X & X &   & X & X &   &   &   & X & X &   &   \\
    6 & X & X & X & X & X &   & X & X &   & X & X &   & X & X &   & X  \\
    7 &   &   &   &   &   &   &   & X &   &   &   &   &   &   &   &   \\
    8 &   &   &   &   &   &   &   &   &   & X & X & X & X & X & X & X \\
    \hline
  \end{tabular}
\end{table}
\vspace{1mm}
\begin{itemize}
\item Task 1: Literature review, retrieving information crucial to the field.
\item Task 2: Test suite design and recalibration.
\item Task 3: Code description, explaining how a code works.
\item Task 4: Code simulation, testing the code to check its quality and performance.
\item Task 5: Error analysis, for quality measurement.
\item Task 6: Present advances in simulations, theoretical definitions, and analysis
\item Task 7: 30 \% presentation.
\item Task 8: Synthesis of knowledge acquired, writing of the final document, and preparation for the final defense.
\end{itemize}


\section{Subject Matter Expert}

%Nombres de por lo menos 3 profesores que conozcan del tema. Uno de ellos debe ser profesor de planta de la Universidad de los Andes.

\begin{itemize}
  \item C\'esar Neyit Galindo Mart\'inez (Universidad de los Andes)
  \item Alonso Botero Mej\'ia (Universidad de los Andes)
  \item Alejandra Valencia (Universidad de los Andes)
\end{itemize}

\begin{thebibliography}{10}

  \bibitem{Kitaev} Kitaev, Aleksej Yu, Alexander Shen, and Mihail N. Vyalyj. Classical and Quantum Computation. Graduate Studies in Mathematics 47. American Mathematical Society, 2002.

  \bibitem{Shor} Calderbank, A. R., and Peter W. Shor. Good Quantum Error-Correcting Codes Exist. Physical Review A, vol. 54, no. 2, Aug. 1996, pp. 1098-105. DOI.org (Crossref), https://doi.org/10.1103/PhysRevA.54.1098.

  \bibitem{Nielsen} Nielsen, Michael A., and Isaac L. Chuang. Quantum Computation and Quantum Information. 10th anniversary edition, Cambridge university press, 2010. BnF ISBN.

  \bibitem{IBM} AbuGhanem, M. IBM quantum computers: evolution, performance, and future directions. J Supercomput 81, 687 (2025). https://doi.org/10.1007/s11227-025-07047-7

  \bibitem{Gottesman} Gottesman, Daniel. Theory of Fault-Tolerant Quantum Computation. Physical Review A 57, no. 1 (1998): 127-37. https://doi.org/10.1103/PhysRevA.57.127.

  \bibitem{CompExperimental} McGeoch, Catherine C. A Guide to Experimental Algorithmics. Cambridge University Press, 2012.

  \bibitem{Quality} Knill, Emanuel, Raymond Laflamme, and Lorenza Viola. Theory of Quantum Error Correction for General Noise. Physical Review Letters 84, no. 11 (2000): 2525-28. https://doi.org/10.1103/PhysRevLett.84.2525.

\end{thebibliography}

\section*{Firma del Director}
\vspace{1.5cm}

\section*{Firma del Codirector	}



\end{document} 

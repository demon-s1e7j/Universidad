  \documentclass[12pt]{exam}
\usepackage{amsthm}
\usepackage{libertine}
\usepackage[utf8]{inputenc}
\usepackage[margin=1in]{geometry}
\usepackage{amsmath,amssymb}
\usepackage{multicol}
\usepackage[shortlabels]{enumitem}
\usepackage{siunitx}
\usepackage{cancel}
\usepackage{graphicx}
\usepackage{pgfplots}
\usepackage{listings}
\usepackage{tikz}


\pgfplotsset{width=10cm,compat=1.9}
\usepgfplotslibrary{external}
\tikzexternalize

\newcommand{\class}{La ciencia del Sexo} % This is the name of the course 
\newcommand{\examnum}{Lecturas en Dialogo Semana 7} % This is the name of the assignment
\newcommand{\examdate}{\today} % This is the due date
\newcommand{\timelimit}{}





\begin{document}
\pagestyle{plain}
\thispagestyle{empty}

\noindent
\begin{tabular*}{\textwidth}{l @{\extracolsep{\fill}} r @{\extracolsep{6pt}} l}
	\textbf{\class} & \textbf{Name:} & Sergio Montoya\\ %Your name here instead, obviously 
	\textbf{\examnum} && Juliana Galindo\\
	\textbf{\examdate} && Maria Bernal\\
			   && Ana Sofía Gómez\\
			   && Luciano Guapucal\\
			   && Laura Sofia Molano Montenegro
\end{tabular*}\\
\rule[2ex]{\textwidth}{2pt}
% ---

\section{Stepan Nancy 1; Maria Bernal \& Sergio Montoya}

\begin{itemize}
  \item Se podría explicar mejor como se entrelazaron el movimiento feminista y la eugenesia.
  \item Se podría mencionar brevemente cuales eran los contextos en América Latina.
  \item Se podría haber profundizado en las metodologías o fuentes utilizadas, y precisar mas las intervenciones centrales.
\end{itemize}

\section{Stepan Nancy 2; Juliana Galindo}

\begin{itemize}
  \item En general se puede profundizar mas.
  \item Se podría ser mas especifico a la hora de dar el contexto. En particular resultaría valioso que se hablara de la época.
  \item No escribieron todos los conceptos
  \item En la sección de argumentos pusieron conceptos
  \item En la parte de controversia podrían mejorar la redacción.
\end{itemize}

\section{Santiago Castro; Luciano Guapucal \& Ana Sofía Gómez }

\begin{itemize}
  \item Es un folleto muy densamente poblado de información. En particular, había bastante texto junto y ocasionaba que nos perdiéramos a la hora de leerlo.
  \item Hicieron un trabajo didáctico bastante impresionante (Hicieron una sopa de Letras WoW)
\end{itemize}

\section{Párrafo Artículo}

A la luz de las lecturas, esta investigación es un ejemplo de la promoción de prácticas eugenésicas. En este sentido, los fenotipos de los “potenciales” criminales se volverían una categoría para aplicar políticas eugenésicas con el fin de reducir la criminalidad y moldear el tipo de sujetos deseables en la sociedad. Consideramos que esta investigación puede replicar sesgos derivados de la criminalización de ciertas poblaciones marginalizadas. Es decir, hay una sobrerrepresentación de comunidades racionalizadas, empobrecidas, migrantes en las cárceles que no da cuenta de la peligrosidad de estas poblaciones, sino de las persecuciones sistemáticas hacia las mismas. Los fenotipos de estas poblaciones pueden estar más presentes en los resultados del artículo de Xiaolin Wu y Xi Zhang (2016) perpetuando los estigmas y sesgos hacia estas comunidades.  

\end{document}

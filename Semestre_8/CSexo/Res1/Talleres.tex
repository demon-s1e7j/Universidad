  \documentclass[12pt]{exam}
\usepackage{amsthm}
\usepackage{libertine}
\usepackage[utf8]{inputenc}
\usepackage{hyperref}
\usepackage[margin=1in]{geometry}
\usepackage{amsmath,amssymb}
\usepackage{multicol}
\usepackage[shortlabels]{enumitem}
\usepackage{siunitx}
\usepackage{cancel}
\usepackage[spanish]{babel}
\usepackage{graphicx}
\usepackage{pgfplots}
\usepackage{listings}
\usepackage{tikz}


\pgfplotsset{width=10cm,compat=1.9}
\usepgfplotslibrary{external}
\tikzexternalize

\newcommand{\class}{La Ciencia del Sexo} % This is the name of the course 
\newcommand{\examnum}{Reseña 1} % This is the name of the assignment
\newcommand{\examdate}{\today} % This is the due date
\newcommand{\timelimit}{}





\begin{document}
\pagestyle{plain}
\thispagestyle{empty}

\noindent
\begin{tabular*}{\textwidth}{l @{\extracolsep{\fill}} r @{\extracolsep{6pt}} l}
	\textbf{\class} & \textbf{Name:} & \textit{Sergio Montoya}\\ %Your name here instead, obviously 
	\textbf{\examnum} &&\\
	\textbf{\examdate} &&
\end{tabular*}\\
\rule[2ex]{\textwidth}{2pt}
% ---

El texto del que vamos a hablar es \textit{The Genetics of Sex Determination: Rethinking Concepts and Theories}. Este puede ser encontrado en el siguiente \href{http://genderedinnovations.stanford.edu/case-studies/genetics.html}{enlace}. En este caso, lo dividiremos por \textbf{¿Quien lo hizo?}, \textbf{¿Que es lo que dice?}, \textbf{¿Como lo sostiene?}, \textbf{¿Que tiene que ver con la clase?}

\section*{¿Quien lo Hizo?}

Este texto fue realizado por \textit{Gendered Innovations}. Este es un grupo de científices, ingenieres y expertes en sexo y genero con el objetivo de desarrollar métodos prácticos para ciencia e ingeniería sobre sexo, genero y análisis intersecciónal. Este grupo fue dirigido por la profesora Londa Schiebinger. Sin embargo, cuenta con la colaboración de múltiples expertos (Para mas información sobre las personas implicadas en este proyecto puede visitar la sección de \href{genderedinnovations.stanford.edu/people.html}{"Contributors"} de la pagina web de \textit{Gendered Innovations}). Es de considerar que esta es una iniciativa que parte de la universidad de Stanford, donde la profesora Londa Schiebinger desempeña su docencia. Ademas, cuenta con el apoyo de la comisión europea y la \textit{National Science Foundation}.

\section*{¿Que es lo que dice?}

El texto sostiene que en el estudio de la diferenciación sexual se ha tomado los genitales de las hembras como el por \textit{"default"} o por defecto en español. Esto a conllevado una negligencia en la investigación de los cuerpos con vagina y ovarios. Sin embargo, este texto aboga por que la diferenciación sexual de las hembras es tan activa como la de los machos. Ademas, propone dos innovaciones de genero que se han estudiado y el como pueden resultar como métodos prácticos para la investigación.

\section*{¿Como lo sostiene?}

El texto inicia hablando de como en términos hormonales se ha visto que cuando se le realiza una gonadectomía a fetos de conejo cuando aun están en el útero y antes de la diferenciación sexual estos presentan un fenotipo de hembra sin importar su genética. Ademas nombra otros estudios, con el objetivo de ejemplificar los motivos de por que se tiene la hipótesis de que tener fenotipo de hembra es el por defecto de la naturaleza en términos hormonales.

Luego de esto, se habla de como con el descubrimiento del cromosoma $Y$ se ha especulado que en el esta el gen que determina el sexo (Que se termino denominando $SRY$). Con esto planea mostrar el como los esfuerzos se han centrado en el estudio de como se diferencia un cuerpo de macho y cuales son los procesos biológicos que hace que tengan sus características. 

Luego, utilizando los dos puntos anteriores sintetiza la tesis del texto. Este es básicamente que la diferenciación sexual de los testículos se a considerado como la clave para la diferenciación sexual de los mamíferos en general. Sin embargo, critica esta posición desde el comienzo y nombra el como no resulta una posición unánime durante la historia de la ciencia. Nombrando el como se ha protestado sobre esta idea desde ya $1986$.  Básicamente, la critica nace de que el proceso de diferenciación de los ovarios es tan activa como la de los testículos y en consecuencia se debería estudiar con la misma rigurosidad.

Posterior a esto, hace mención a dos innovaciones de genero que han traído resultados interesantes y en los que podría resultar productivo el hacer estudios sobre ellos. 

La primera innovación es tomar ambas diferenciaciones como activas y estudiar el como los ovarios y la vagina en general se generan. Esto se apoya en descubrimientos sobre como el gen $SRY$ puede ser el marcador para la creación de testículos pero se espera que un equivalente también exista para la creación de ovarios. 

La segunda innovación se centra mas en como estos órganos se mantienen y funcionan. Esto pues biologes que estudiaban el genoma con relación al fallo en los ovarios han hecho descubrimientos sobre un gen que en adultos resulta necesario para que el tejido de los ovarios no se convierta en tejido de testículos. Básicamente, proponen centrarse en el como los cuerpos mantienen estos órganos y el como se relacionan después de el proceso de identificación general.

Para mi, fue un texto realmente convincente e interesante. Creo que si ha habido un sobre enfoque en los testículos y los cuerpos estereotípicamente masculinos en este estudio. De hecho, recuerdo una ocasión en el colegio (debía tener aproximadamente 13 años) en donde una compañera dijo: "Las vaginas son penes incompletos". Para mi esto fue un momento sumamente transformador pues no podía entender como alguien podía pensar aquello. Ademas, fue una de las muestras mas directas de los juicios de valor que se asignan a los distintos cuerpos. Por lo tanto, este texto me resulto esclarecedor no solo para entender el por que mi compañera decía eso si no ademas para entender las implicaciones que ha tenido en la ciencia y como se podría cambiar esto.

\section*{¿Que tiene que ver con la Clase?}

Para mi este texto tiene que ver con la epistemología de la que se hablo durante la semana tres. Creo que es una gran muestra para iniciar de como los estudios que se creían objetivos e independientes realmente nacían de un juicio completamente arbitrario y social y como al tomar en consideración esto y aproximarse de manera radicalmente opuesta se pueden llegar a resultados igual de interesantes e importantes. 


\end{document}

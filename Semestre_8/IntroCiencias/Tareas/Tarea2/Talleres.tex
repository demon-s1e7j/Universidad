  \documentclass[12pt]{exam}
\usepackage{amsthm}
\usepackage{libertine}
\usepackage[utf8]{inputenc}
\usepackage[margin=1in]{geometry}
\usepackage{amsmath,amssymb}
\usepackage{multicol}
\usepackage[shortlabels]{enumitem}
\usepackage{siunitx}
\usepackage{cancel}
\usepackage{graphicx}
\usepackage{pgfplots}
\usepackage[spanish]{babel}
\usepackage{listings}
\usepackage{tikz}
\usepackage[style=apa,backend=biber]{biblatex}
\usepackage{helvet}
\renewcommand{\familydefault}{\sfdefault}
\renewcommand{\baselinestretch}{1.5}
\addbibresource{bibliography.bib}

\pgfplotsset{width=10cm,compat=1.9}
\usepgfplotslibrary{external}
\tikzexternalize

\newcommand{\class}{Introducción a las ciencias} % This is the name of the course 
\newcommand{\examnum}{Tarea 2} % This is the name of the assignment
\newcommand{\examdate}{\today} % This is the due date
\newcommand{\timelimit}{}
\newcommand{\JEP}{\textit{JEP}}





\begin{document}
\pagestyle{plain}
\thispagestyle{empty}

\noindent
\begin{tabular*}{\textwidth}{l @{\extracolsep{\fill}} r @{\extracolsep{6pt}} l}
	\textbf{\class} & \textbf{Nombre:} & \textit{Sergio Montoya}\\ %Your name here instead, obviously 
	\textbf{\examnum} &&\\
	\textbf{\examdate} &&
\end{tabular*}\\
\rule[2ex]{\textwidth}{2pt}
% ---

Colombia es, en esencia, un estado pluralista (\cite[art.~I]{constitucion}). Esta condición, implica que esta compuesto de múltiples naciones. Cada una de ellas con una demografía, cultura y necesidades distintas.
Apelar a la ciencia que necesita Colombia no es hablar de la ciencia que se debería hacer en este estado. En cambio, es considerar los estudios que pueden tener un mayor impacto en la población Colombiana. Esta diferencia entre la ciencia que queremos producir y la que necesita la población Colombiana, no sera considerada en este texto por simplicidad y espacio. En cambio, nos centraremos en una sola de estas ramas con alto impacto y discutiremos el por que esta es importante y necesaria para el desarrollo del estado y sus naciones.

En particular, nos centraremos en la antropología como ciencia con un impacto profundo en procesos de paz, reparación y no repetición del conflicto armado en Colombia. Ademas, hablaremos de como ayuda a la creación de políticas con alto impacto. Para esto, iniciaremos hablando del trauma social e histórico que ha permeado en la cultura de todas las naciones que componen Colombia. Luego de esto, se hablara de la gran centralización que tiene Colombia y el como la antropología puede ayudar fuertemente pues es la ciencia que, entre otros, estudia la otredad. Con esto, se concluirá que la antropología es una ciencia con alto impacto y que afecta de manera transversal a las distintas naciones que componen el estado Colombiano.

Colombia es un país con una historia violenta que supero las 6 décadas. Esta historia ha dejado grandes heridas transgeneracionales (\cite{tomo1}). Estas heridas están extendidas en todo el territorio y son diversas para los distintos grupos sociales que existe. Cada espacio a sufrido un tipo particular de violencia proveniente de un complejo sistema con relaciones con política, economía, narcotráfico y otros (\cite{tomo1}). Las particularidades de cada uno de estos temas hace que sea difícil hilar una historia única que reúna todas las demás. Por esto, es importante considerar que la comisión de la verdad saco $11$ tomos centrandoce en cada uno de ellos en una población y los tipos particulares de violencia que sufre esa población (\cite{tomos}). Esta historia violenta a creado en Colombia la idea de un enemigo interno con el que es imposible construir país (\cite{tomo1}). Todo esto se a descubierto por medio del gran proceso de la comisión de la verdad después de la firma del tratado de paz. Esta información es valiosa y arranca el primer paso para una sanación nacional en la que todas las comunidades afectadas entienden lo que paso y se trabaja por resarcir a las victimas sin revictimizarlas. Este proceso, resultaría completamente imposible sin la ayuda de la antropología como ciencia que nos permite entender y dar contexto humano a la historia que tiene este país. Como tal, podemos ver el gran impacto que tiene la antropología como ciencia para todos estos grupos poblaciones que componen una buena parte de la ciudadanía de este estado.

En términos legales, Colombia es un estado centralizado en el congreso. Esta es una entidad con la visión de representar a la sociedad y ejercer la legislación de la nación (\cite{senadoMisionVision}). Sin embargo, el senado tiene una historia de centralización que ha llevado que el proceso de descentralización que se a intentado llevar a cabo desde la constitución de $1991$ de manera poco fructuosa (\cite{centro}). Esto implica en una nación tan diversa como lo es Colombia que los legisladores cuentan con pocas herramientas para entender las necesidades particulares de los distintos lugares de Colombia. En ese sentido, nos enfrentamos a una situación en la que las personas que resulta mas importante que entiendan al país para poder legislar sobre el se enfrentan a una gran otredad que les interrumpe entender de manera correcta las condiciones. Por lo tanto, la antropología que es una ciencia que estudia la otredad tiene un gran impacto en el poder traducir estas necesidades a los términos del centro del país que es donde se concentra el poder político.

En conclusión, la antropología tiene un impacto directo en la manera en la que las distintas naciones que componen Colombia pueden comunicarse entre ellas para la legislación actual y para dar sentido a lo ocurrido en un pasado. Es una mezcla que atraviesa transversalmente a todas las naciones y que toma en consideración que Colombia no es solo esta región central con gran poder político, legislativo y económico. Si no también, somos las victimas del conflicto armado, los campesinos, los pueblos indígenas y múltiples otras realidades que desde el centro del país vemos como el otro. Con esto, conseguimos mostrar que la antropología es una ciencia que tendría grandes efectos en todas las naciones del país y que su implementación y mayor conocimiento seria una de las ramas que mas podría afectar el día a día de los ciudadanos. Por lo tanto, una de las ciencias que necesita Colombia es la antropología. Es importante aclarar que no se trata de un esfuerzo único. La historia, la sociología, la ciencia política, la economía todas estas son ciencias que podrían tener un gran impacto en la vida de las personas y que dentro de nuestra definición de necesidad también se podría abogar por ellas. El objetivo final no es encontrar una respuesta única si no que desde todos los ángulos se aporte para el beneficio de las personas que hay en el país.

\printbibliography
\end{document}

  \documentclass[12pt]{exam}
\usepackage{amsthm}
\usepackage{libertine}
\usepackage[utf8]{inputenc}
\usepackage[margin=1in]{geometry}
\usepackage{amsmath,amssymb}
\usepackage{multicol}
\usepackage[scaled]{helvet}
\usepackage[T1]{fontenc}
\usepackage[shortlabels]{enumitem}
\usepackage[spanish]{babel}
\usepackage{siunitx}
\usepackage{cancel}
\usepackage{graphicx}
\usepackage{pgfplots}
\usepackage{listings}
\usepackage{tikz}


\pgfplotsset{width=10cm,compat=1.9}
\usepgfplotslibrary{external}
\tikzexternalize

\newcommand{\class}{Introducción a las Ciencias} % This is the name of the course 
\newcommand{\examnum}{Texto Reflexivo 1} % This is the name of the assignment
\newcommand{\examdate}{\today} % This is the due date
\newcommand{\timelimit}{}
\renewcommand\familydefault{\sfdefault}





\begin{document}
\pagestyle{plain}
\thispagestyle{empty}

\noindent
\begin{tabular*}{\textwidth}{l @{\extracolsep{\fill}} r @{\extracolsep{6pt}} l}
	\textbf{\class} & \textbf{Nombre:} & \textit{Sergio Montoya}\\ %Your name here instead, obviously 
	\textbf{\examnum} &&\\
	\textbf{\examdate} &&
\end{tabular*}\\
\rule[2ex]{\textwidth}{2pt}
% ---

Estas semanas hemos tenido varios conversatorios. En particular, aquí hablaremos de los primeros dos. El objetivo de este texto es reflexionar sobre lo conversado en estas conversaciones. En particular, hablaremos de los vínculos entre ciencia, política, individuo y sociedad. Para esto, dividiremos este texto en 3 reflexiones. La primera reflexión hablara de el como los profesores \textit{Bernardo Gómez}, \textit{Xavier Caicedo} y \textit{Juan Armando Sánchez} se negaban a ubicarse en Colombia como entidad política. Hablaban de una ciencia muy universalista y se negaron a contestar la pregunta "¿Cual es la ciencia que necesita Colombia?". Por ultimo, hablaremos de la visión de individuo que se vende en estas conversaciones por mucho que en las palabras se hable de la ciencia como grupo. Esto se ejemplifica en que las preguntas se hacen a un nivel personal. La política impregna todo lo que toca y las ciencias también están teniendo una tendencia a la exaltación de individuos. Con esto, se espera dar una buena visión de las cosas que me hicieron reflexionar respecto a los conversatorios.

\section*{No soy de aquí, ni soy de allá}

\textit{¿Cual es la ciencia que necesita Colombia?} con esta pregunta se inicio la conversación con los profesores. La verdad es que no esperaba un análisis sociológico profundo de las necesidades de Colombia como estado y sus distintas naciones y el como muchos de sus problemas podrían ser solucionados con ciencia. Sin embargo, si esperaba que al menos se contestara la pregunta y se considerara a Colombia dentro de la respuesta. Al final, Colombia era primordial en la pregunta. Ahora bien, lo que me encontré fue una respuesta universalista y sin alma que me sorprendió y decepciono a partes iguales. Hablaron de como el regular a la ciencia a llevado históricamente a un progreso científico mas lento y el como aun con los recursos limitados que tenemos no era justo decidir que ciencia se hace. Esta respuesta me pareció extraña así que en la ronda de preguntas me decidí por preguntarles de nuevo explicando que la pregunta debía nacer de interesarse en Colombia. En este caso, por segunda vez, decidieron ignorar por completo a Colombia en la respuesta y hablaron de el como la ciencia no esta limitada por fronteras. Ya en este punto, deje de insistir con una respuesta. 

Para mi, resulto extraño la imagen de una ciencia universalista pues creo que habla en contra de la epistemología mas moderna e importante para el desarrollo científico. Creo que la idea de que la ciencia es objetiva, global e independiente es en parte lo que causa un estancamiento en la ciencia. Todos estamos condenados a ser nosotros mismos. Como tal, esto implica no solo ser yo como individuo si no tener un contexto y una realidad que define tus experiencias vitales. Por lo tanto, considero que el reflexionar sobre el contexto particular es una habilidad sumamente importante y que no se expreso en estas charlas. Se que se puede hablar de como parte del contexto es un mundo globalizado en el que vivimos. Esto es cierto y claramente estamos en una época en donde las influencias externas son mucho mas grandes que antes. Sin embargo, es sumamente reduccionista hablar de que el contexto es global pues hay múltiples culturas radicalmente diferentes en territorios tan pequeños como un departamento. Por lo tanto, aunque esta cultura global si existe y se puede aplicar no implica que no se pueda reflexionar de las necesidades particulares que tienen los distintos grupos a los que se pertenece.

\section*{La Ciencia es en Grupos, pero la Experiencia Individual}

Durante la presentación con el Decano, se hablo mucho de su experiencia haciendo ciencia. Era una experiencia interesante pero me resulto especialmente llamativo el como toda la charla se centro en el individuo. Nos han hablado múltiples veces de que la ciencia a día de hoy se hace en grandes grupos y que los científicos solitarios ya no existen. Sin embargo, cuando hablamos de estas experiencias lo hacemos desde una visión sumamente individual que premia a unos pocos el trabajo de muchos. Aun mas, estos pocos suelen ejercer trabajos mas administrativos que científicos \textit{per-se}. Pongamos un ejemplo en la historia, el proyecto manhattan fue uno de los proyectos científicos mas grandes e importantes que han existido en la historia. Sin embargo, por mucho que fue un trabajo en grupo y sumamente extenso se resalta siempre el trabajo de aquellos pocos 20 físicos que resultaron tener mucho mas éxito mediático. En si, esto no es novedoso. La idea de que una persona toma el crédito y reconocimiento de un grupo es algo de lo que ya se ha hablado en el pasado. Sin embargo, creo que el enfoque individual de esta experiencia me hizo pensar en lo mucho que se reproduce en nuestra cultura. 

Hay dos motivos principales para que esta relación individualista me parezca tan problemática. Primero, hay un sentido de justicia en reconocer el trabajo de estos grandes grupos y distintas personas haciendo los procesos pequeños y grandes. Se que justicia es un termino filosóficamente difuso. Sin embargo, creo que todos somos conscientes de que resulta importante que las personas reciban el reconocimiento que tienen por su trabajo. Un poco parecido a un sistema de plusvalía en el reconocimiento. Se que esto suena muy marxista y no quiero incurrir mucho mas en una dialéctica hegeliana pues no creo que una lucha en estos casos aporte mucho. Sin embargo, si veo importante el pensar en esto como un proceso de justicia. Por el otro lado, observo como grandes laboratorios han sido fuente de éxito en los últimos tiempos y me sorprende que no sea un sistema mucho mas generalizado. Es decir, creo que todos reconocemos el \textit{CERN} o el \textit{FERMILAB} y el hecho de que se beneficien de que múltiples grupos trabajen ahí y produzcan descubrimientos bajo su sombrilla a beneficiado fuertemente a la ciencia.

\section*{Conclusión}

Estos conversatorios resultaron ser realmente interesantes y profundos para mi. Creo que no los vi con una óptica tan común dado que soy una persona mucho mas avanzada en su carrera que lo que el curso estaba pensado para tener. Sin embargo, me resultaron profundas inquietudes con lo que se hablo desde un punto de vista mucho mas social y filosófico. Creo que mis criticas respecto a la correlación entre un científico y su contexto y las narrativas individualistas en las que participamos sin darnos cuenta son una muestra de mi deseo de una ciencia mucho mas enfocada en lo social y consciente de si misma. Para mi es importante el desempeñarme como una persona con una visión ética y social respecto a lo que se vive y en lo que se trabaja y aunque en estos conversatorios se notaba como no se hablo de los aspectos sociales de la ciencia para mi fue sumamente enriquecedor este proceso reflexivo.

\end{document}

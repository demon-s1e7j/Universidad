  \documentclass[12pt]{exam}
\usepackage{amsthm}
\usepackage{libertine}
\usepackage[utf8]{inputenc}
\usepackage[margin=1in]{geometry}
\usepackage{amsmath,amssymb}
\usepackage{multicol}
\usepackage[scaled]{helvet}
\usepackage[T1]{fontenc}
\usepackage[shortlabels]{enumitem}
\usepackage[spanish]{babel}
\usepackage{siunitx}
\usepackage{cancel}
\usepackage{graphicx}
\usepackage{pgfplots}
\usepackage{listings}
\usepackage{tikz}


\pgfplotsset{width=10cm,compat=1.9}
\usepgfplotslibrary{external}
\tikzexternalize

\newcommand{\class}{Introducción a las Ciencias} % This is the name of the course 
\newcommand{\examnum}{Texto Reflexivo 1} % This is the name of the assignment
\newcommand{\examdate}{\today} % This is the due date
\newcommand{\timelimit}{}
\renewcommand\familydefault{\sfdefault}





\begin{document}
\pagestyle{plain}
\thispagestyle{empty}

\noindent
\begin{tabular*}{\textwidth}{l @{\extracolsep{\fill}} r @{\extracolsep{6pt}} l}
	\textbf{\class} & \textbf{Nombre:} & \textit{Sergio Montoya}\\ %Your name here instead, obviously 
	\textbf{\examnum} &&\\
	\textbf{\examdate} &&
\end{tabular*}\\
\rule[2ex]{\textwidth}{2pt}
% ---

Estas semanas hemos tenido varios conversatorios. En particular, aquí hablaremos de los primeros dos. El objetivo de este texto es reflexionar sobre lo conversado en estas conversaciones. En particular, hablaremos de los vínculos entre ciencia, política, individuo y sociedad. Para esto, dividiremos este texto en 3 reflexiones. La primera reflexión hablara de el como los profesores \textit{Bernardo Gómez}, \textit{Xavier Caicedo} y \textit{Juan Armando Sánchez} se negaban a ubicarse en Colombia como entidad política. Hablaban de una ciencia muy universalista y se negaron a contestar la pregunta "¿Cual es la ciencia que necesita Colombia?". Por ultimo, hablaremos de la visión de individuo que se vende en estas conversaciones por mucho que en las palabras se hable de la ciencia como grupo. Esto se ejemplifica en que las preguntas se hacen a un nivel personal. La política impregna todo lo que toca y las ciencias también están teniendo una tendencia a la exaltación de individuos. Con esto, se espera dar una buena visión de las cosas que me hicieron reflexionar respecto a los conversatorios.

\section*{No soy de aquí, ni soy de allá}

\textit{¿Cual es la ciencia que necesita Colombia?} con esta pregunta se inicio la conversación con los profesores. La verdad es que no esperaba un análisis sociológico profundo de las necesidades de Colombia como estado y sus distintas naciones y el como muchos de sus problemas podrían ser solucionados con ciencia. Sin embargo, si esperaba que al menos se contestara la pregunta y se considerara a Colombia dentro de la respuesta. Al final, Colombia era primordial en la pregunta. Ahora bien, lo que me encontré fue una respuesta universalista y sin alma que me sorprendió y decepciono a partes iguales. Hablaron de como el regular a la ciencia a llevado históricamente a un progreso científico mas lento y el como aun con los recursos limitados que tenemos no era justo decidir que ciencia se hace. Esta respuesta me pareció extraña así que en la ronda de preguntas me decidí por preguntarles de nuevo explicando que la pregunta debía nacer de interesarse en Colombia. En este caso, por segunda vez, decidieron ignorar por completo a Colombia en la respuesta y hablaron de el como la ciencia no esta limitada por fronteras. Ya en este punto, deje de insistir con ello.


\end{document}

  \documentclass[12pt]{exam}
\usepackage{amsthm}
\usepackage{libertine}
\usepackage[utf8]{inputenc}
\usepackage[margin=1in]{geometry}
\usepackage{amsmath,amssymb}
\usepackage{multicol}
\usepackage[scaled]{helvet}
\usepackage[T1]{fontenc}
\usepackage[shortlabels]{enumitem}
\usepackage[spanish]{babel}
\usepackage{siunitx}
\usepackage{cancel}
\usepackage{graphicx}
\usepackage{pgfplots}
\usepackage{listings}
\usepackage{tikz}


\pgfplotsset{width=10cm,compat=1.9}
\usepgfplotslibrary{external}
\tikzexternalize

\newcommand{\class}{Introducción a las Ciencias} % This is the name of the course 
\newcommand{\examnum}{Texto Reflexivo 2} % This is the name of the assignment
\newcommand{\examdate}{\today} % This is the due date
\newcommand{\timelimit}{}
\renewcommand\familydefault{\sfdefault}





\begin{document}
\pagestyle{plain}
\thispagestyle{empty}

\noindent
\begin{tabular*}{\textwidth}{l @{\extracolsep{\fill}} r @{\extracolsep{6pt}} l}
	\textbf{\class} & \textbf{Nombre:} & \textit{Sergio Montoya}\\ %Your name here instead, obviously 
	\textbf{\examnum} &&\\
	\textbf{\examdate} &&
\end{tabular*}\\
\rule[2ex]{\textwidth}{2pt}
% ---

La ciencia y la industria tienen vínculos profundos en una sociedad de consumo y eficiencia como la que vivimos actualmente. Esto no tiene en si nada de malo, la ciencia avanza rápidamente y este vinculo se ve incluso en los instrumentos que utilizamos día a día. El como, los paneles solares funcionan a partir del efecto foto eléctrico que se explico hace mas de 100 años y el como esta en boca de todos como una gran innovación es solo una muestra de como los avances científicos pueden llevar a innovaciones a medida que el tiempo pasa. Con esto en mente este texto sera una reflexión de como me cambia el estar trabajando en una start up y al mismo tiempo llevando dos carreras de ciencias y cuales son sus relaciones. Para esto iniciare con un recorrido por mi vida laboral y explicare los crecimientos personales que estos me han traído. Luego de esto hablare de como mucho de esto tiene que ver con la universidad y las carreras que estudio. Esto lo hago pues no soy ajeno al emprendimiento y me resultaría deshonesto hablarlo en abstracto cuando tengo una experiencia de primera mano de la que hablar.

Mi historia trabajando ya tiene un par de años. Inicio, escaneando documentos y cambiándoles el nombre manualmente cuando estaba en 8vo o algo así. En ese momento, yo no tenia conocimiento especialmente interesante pero el trabajo me permitía darme algunos lujos y apoyar con la economía de mi casa. Esta es una experiencia que recuerdo con cariño pues veo con ternura el como me inventaba sistemas para hacer las cosas mas rápido y el como para esa misma compañía ahora años mas tarde trabajo como consultor en momentos especiales y logro automatizar procesos que antes ni soñaba.

Luego de unos años haciendo esto, entre a la universidad y deje de trabajar. En mi primer semestre tuve una clase llamada \textit{Introducción a la Programación}. Esta clase, fue mi primera vista formal al mundo de los computadores y ciertamente me cambio la vida. Desde entonces he vivido obsesionado con las ciencias computacionales y ya voy a cumplir 4 años en los que no he pasado ni un solo día en el que no escriba al menos una linea de código. Es algo que me apasiona y que en el momento en el que lo entendí me revoluciono la manera de ver todo. Luego de esto, en estas vacaciones lo primero que hice fue acercarme de nuevo a la empresa en donde antes trabajaba y les propuse la automatización de los procesos que yo hacia antes. En ese momento, convertí lo que era un trabajo repetitivo y bastante aburrido en un reto intelectual que logre completar y del que me siento orgulloso. Desde entonces, he desempeñado un rol mas o menos frecuente en esta empresa automatizando procesos lentos y/o repetitivos. Esa empresa me ha enseñado muchísimo pues cada proyecto que llevaba se siente como un reto mas y mas grande. He aprendido a los golpes a mantener código, a estudiar con documentación a traducir proyectos y en general a tener buenas practicas laborales.

Luego de un tiempo en el que mi único trabajo fue en esta empresa y estudiando la mayoría del tiempo un compañero de física que había conocido hace un tiempo se me acerco con una oferta. Quería que dirigiera el espacio de desarrollo de su start up. Este era un proyecto que a primera vista se veía sencillo. No pensé que me tomaría mucho tiempo, pero me enfrente a uno de los grandes cambios de mi carrera. Por primera vez, tenia que trabajar ya no solo con mi código si no manejar un equipo de personas y ver todos sus trabajos para que el producto saliera lo mejor posible. Esta fue una experiencia muy enriquecedora y creativa. Me permitió aprender muchísimo de managment y también descubrí que mi lugar no esta en lo administrativo. Soy una persona técnica que le encanta mancharse las manos y estar luchando contra un compilador por horas. Este proyecto, aunque termino y creo fuertemente en sus beneficios la verdad es que nunca floreció. Conseguir clientes a resultado realmente difícil y siendo muy honesto yo me he separado de la empresa desde hace un tiempo pues aunque le tengo cariño por motivos personales la verdad es que mis carreras y mi vida personal me han ido alejando cada vez mas.

Luego de esto y sintiéndome mucho mas seguro de como hacer cosas estuve trabajando dos meses en una compañía de software para IPS. En particular, mi trabajo era hacer un sistema de control de la información. Este fue un proyecto que me obligo a meter todo lo que había aprendido de la universidad por primera vez a mis códigos. En este punto me encontré haciendo análisis estadístico de grandes cantidades de datos para estas empresas y mirando como resultaba mejor presentar la información. La verdad, este no es un trabajo del que me sienta muy orgulloso. Creo que mi vena científica hizo que muchas de las gráficas y análisis sobrestimaran al usuario y resultaran en una pagina web mas fea y en la que las conclusiones no resultaban tan valiosas como podrían haber sido. He crecido desde entonces como programador y como científico y creo que podría hacer un trabajo mucho mejor si me tuviera que enfrentar de nuevo a esto. Sin embargo, esta experiencia también me enseño lo que era un trabajo de oficina tradicional. Hasta ahora todos mis trabajos habían sido remotos o muy locos. Espacios creativos y amplios de discusión y en los que nadie me exigía nada. Podía ir en chanclas o despeinado o llegar tarde y nadie me iba a decir nada. En este trabajo por primera vez me enfrente a tener que hacer reuniones a mitad de mañana todos los días y a jugar con la burocracia de una empresa. Esto me resulto realmente cansado pero me enseño mucho de mi mismo.

Luego de todo esto, en una clase conocí a mi actual jefe y compañero. El tenia una start up ya formada y me encentro trabajando en ella en estos momentos. Esta es una experiencia que apenas comienza y que me siento expectante de que es lo que me puede enseñar. El proyecto se siente como si utilizara todos los conocimientos que adquirí con los años pues tengo roles administrativos, técnicos y de visualización de datos. Esto me resulta sumamente tentador pues es un nuevo reto y hasta ahora siempre he aprendido algo y crecido con estas experiencias.

Luego de contar un poco mi historia quiero hablar de como el pensamiento científico me ha ayudado en mis trabajos. En particular, creo que me ha dado una gran curiosidad por entender lo que estoy haciendo. Me aproximo a los problemas con genuino interés y no paro de preguntarme cosas hasta que lo entiendo al nivel mas básico posible. Por el lado de las matemáticas los algoritmos y demostraciones de la matemática computacional me han cambiado la vida pues me permiten tener un asomo de lo genial que resultan estos procesos y se juntan dos de mis grandes pasiones. Por el lado de la física me ha permitido tener una habilidad bastante decente para analizar grandes cantidades de datos sin miedo. Con la seguridad de que es lo que estoy haciendo y por que esta información resulta interesante o valiosa. Ademas, la universidad en particular me ha abierto muchas puertas. No habría conseguido muchos trabajos si no hubiera conocido a las personas correctas en los momentos correctos y eso me a ayudado muchísimo a crecer como individuo y como miembro de la sociedad.

En conclusión, mi experiencia con el trabajo, los emprendimientos y la innovación no es ajeno. Durante mi vida he tenido varias experiencias laborales en start ups y eh trabajado en muchos lugares que me han hecho crecer y madurar como persona y como científico. Ademas, creo que la universidad a jugado un rol central en mi crecimiento en estos aspectos pues me ha permitido no solo conocer a las personas correctas si no aprender y juntarme con aquellos a quienes les interesa lo que a mi.

\end{document}

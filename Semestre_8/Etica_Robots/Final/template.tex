\documentclass{report}

\documentclass[12pt]{article}
\usepackage{array}
\usepackage{color}
\usepackage{amsthm}
\usepackage{eufrak}
\usepackage{lipsum}
\usepackage{pifont}
\usepackage{yfonts}
\usepackage{amsmath}
\usepackage{amssymb}
\usepackage{ccfonts}
\usepackage{comment} \usepackage{amsfonts}
\usepackage{fancyhdr}
\usepackage{graphicx}
\usepackage{listings}
\usepackage{mathrsfs}
\usepackage{setspace}
\usepackage{textcomp}
\usepackage{blindtext}
\usepackage{enumerate}
\usepackage{microtype}
\usepackage{xfakebold}
\usepackage{kantlipsum}
%\usepackage{draftwatermark}
\usepackage[spanish]{babel}
\usepackage[margin=1.5cm, top=2cm, bottom=2cm]{geometry}
\usepackage[framemethod=tikz]{mdframed}
\usepackage[colorlinks=true,citecolor=blue,linkcolor=red,urlcolor=magenta]{hyperref}

%//////////////////////////////////////////////////////
% Watermark configuration
%//////////////////////////////////////////////////////
%\SetWatermarkScale{4}
%\SetWatermarkColor{black}
%\SetWatermarkLightness{0.95}
%\SetWatermarkText{\texttt{Watermark}}

%//////////////////////////////////////////////////////
% Frame configuration
%//////////////////////////////////////////////////////
\newmdenv[tikzsetting={draw=gray,fill=white,fill opacity=0},backgroundcolor=none]{Frame}

%//////////////////////////////////////////////////////
% Font style configuration
%//////////////////////////////////////////////////////
\renewcommand{\familydefault}{\ttdefault}
\renewcommand{\rmdefault}{tt}

%//////////////////////////////////////////////////////
% Bold configuration
%//////////////////////////////////////////////////////
\newcommand{\fbseries}{\unskip\setBold\aftergroup\unsetBold\aftergroup\ignorespaces}
\makeatletter
\newcommand{\setBoldness}[1]{\def\fake@bold{#1}}
\makeatother

%//////////////////////////////////////////////////////
% Default font configuration
%//////////////////////////////////////////////////////
\DeclareFontFamily{\encodingdefault}{\ttdefault}{%
  \hyphenchar\font=\defaulthyphenchar
  \fontdimen2\font=0.33333em
  \fontdimen3\font=0.16667em
  \fontdimen4\font=0.11111em
  \fontdimen7\font=0.11111em}


\input{macros}
\input{letterfonts}

\title{\Huge{Ética para Robots}\\Proyecto Final}
\author{\huge{Sergio Montoya Ramírez}}
\date{}

\begin{document}

\maketitle
\newpage% or \cleardoublepage
% \pdfbookmark[<level>]{<title>}{<dest>}
\pdfbookmark[section]{\contentsname}{toc}
\tableofcontents
\pagebreak

\chapter{La Abolición del Trabajo: Un castigo para Sísifo}

"La IA nos va a Quitar el Trabajo". Esta es una frase que he escuchado múltiples veces en los últimos años. Dado que soy programador es especialmente frecuente que me pregunten si no me asusta que me quiten el trabajo y que todos mis conocimientos y dedicación ya no sirvan para nada. Para mi, esto es solo Sisifo sintiendo que esta cerca a la cima de nuevo. Nos hemos enfrentado desde la revolución industrial con la idea de poder automatizar la mayoría del trabajo. Sin embargo, nunca hemos conseguido nada remotamente similar. De hecho, desde la reducción de la jornada a alrededor de 40 horas semanales no ha habido ningún cambio importante en las horas de trabajo que le dedicamos al día a trabajar. En este texto quiero ahondar en como en una sociedad en la que abundan las maquinas no hemos conseguido dejar de trabajar. Ademas, quiero contar mi experiencia como programador de automatizaciones al notar que realmente no estoy aportando a que se trabaje menos.

  Para experimentar con este tema nos dividiremos en tres caminos. Primero, hablaremos de nuestro objetivo. La cima de la montaña a la que Sísifo debe llevar su piedra. En particular, hablaremos de visiones como las que tenia Kropotkin en el libro la conquista del pan. Luego de esto, pensaremos en como las maquinas y la automatización de los trabajos no aportaron a que las personas trabajen menos. Para esto, tomaremos como base el texto "Bullshit Jobs" de David Graeber. Luego de esto, hablare de mi experiencia como programador de automatizaciones y la frustración que me da no lograr que el mundo sea distinto con mi trabajo. Por ultimo, reuniremos todos estos temas para dar una pequeña reflexión o, mejor aun, pedir una pequeña reflexión pues me encantaría saber como solucionar este enredo

\section{No conquistamos el pan, lo subyugamos}

Piotr Kropotkin era un anarquista ruso de finales del siglo $XIX$. Su libro mas famoso es \textit{La Conquista del Pan}. En este libro, Kropotnik habla de como la humanidad es sumamente rica en recursos y como si quisiéramos podríamos organizarnos para trabajar un par de horas al día y dedicar el resto del tiempo a procesos enriquecedores. Tomemos en cuenta que este fue un libro que salio en $1892$. Para ese punto ya existía la visión de una sociedad en la que el trabajo fuera distribuido y así todos podríamos vivir mejor. Ademas, el nombre del libro viene de que Kropotnik dice que lo primero que debe controlar una revolución es el pan. Puesto que todas las sociedades deben alimentarse y una vez todos puedan comer entonces se trabaja desde ahí. 

Este es un mundo maravilloso, conocer la idea de abolir el trabajo fue transformador para mi. Desde entonces la imagen de un mundo en el que no se trabaja si no que los esfuerzos van dirigidos a aquello que resulte mas enriquecedor para cada uno me ha fascinado. Sin embargo, poco a poco el hecho de que se halla enunciado hace mas de 100 años y la historia me han hecho darme cuenta de que lastimosamente el pan hasta ahora no se a conquistado. Se subyugo el pan al servicio de aquellos que amazan el capital. Para mi la idea de que las maquinas nos van a quitar el trabajo es un castigo de Sísifo. Nos mantenemos siempre subiendo la colina y cuando creemos que por fin lo vamos a lograr algo se desploma.

\section{Como subyugar el pan? Dale un Bullshit Job}

David Graeber, era un antropólogo estadounidense. Tiene un libro al que llamo \textit{Bullshit Jobs}. En ese libro se hacia la misma pregunta que yo me hago en este escrito. Su respuesta, es que terminamos creando trabajos inútiles y que no resultan de provecho para la sociedad. Trabajos que el denomina \textit{Bullshit} (Perdonen el anglicismo recurrente, realmente no se como traducir este concepto). Estos son trabajos en los que realmente la persona que lo ejecuta siente que no le esta aportando nada a nadie. Estos trabajos, resultan a primera vista sorprendentes pues se cree que el capitalismo debería fomentar la eficiencia y tener a personas trabajando para esencialmente nada no resulta creíble que sea eficiente. A esta visión Graeber responde que la existencia de los \textit{Bullshit Jobs} no esta justificada en la economía si no en la política. Es muy recomendable leer el libro pues resulta esclarecedor y el estilo de Graeber es bastante simple e inteligible. 

Continuando con nuestra analogía con el mito de Sísifo para mi esta es la caída de la roca con la que mas frecuentemente nos hemos enfrentado. Por motivos políticos hemos sido incapaces de llevar una transición a la disminución del trabajo y una distribución de las riquezas de manera justa. Esto se sostiene bajo una fantasía de la meritocracia que permite estabilizar el mundo y vender que con suficiente esfuerzo se conseguirán cosas fantásticas. Sin embargo, ¿De que sirve el esfuerzo si los trabajos que aportan valor se pueden automatizar?

\section{Mi experiencia}

Esta es la sección mas personal del trabajo. Hasta ahora, era un resumen y una presentación de las ideas que van a ser transversales a mi narración. Para iniciar, quiero aclarar que soy programador y que mi trabajo se encuentra mayoritariamente en automatizar tareas simples. Mis primeras automatizaciones se presentaron por un motivo personal. Yo veía a mi papá gastando días enteros en tareas repetitivas y cansadas. Tareas que se podrían determinar como \textit{Bullshit}. Con mi poco conocimiento en programación del momento decidí automatizarle estas tareas para el. Eran cosas simples, abrir un pdf y cambiarle el nombre, descargar información de una pagina web y ponerlo en una carpeta. Ninguno era un trabajo especialmente exigente pero poco a poco he ido implementando tareas mas y mas complejas. Mi objetivo en su momento era quitarle el peso a mi padre de tener que realizar estos procesos. Sin embargo, con el tiempo empece a notar que mi trabajo no resultaba en que mi papá hiciera menos estos trabajos si no en que se encargaba de mas cosas al tiempo pues resultaba ser mas productivo.

En su momento, cuando inicie a trabajar en automatización mi sueño era que mi padre no tuviera ni que ir a trabajar. El es una persona que lleva trabajando desde muy temprano (lleva pagando su pensión desde la primera semana que cumplió 18) y la verdad yo esperaba poder hacer que el no tuviera que ir si no quería a trabajar. Soñaba en ese momento de manera muy individual con un mundo en el que el trabajo resultara opcional. En el que las horas de esfuerzo fueran una o dos y que su experiencia fuera lo mas suave posible. Sin embargo, su trabajo no dejo de ser inútil con mi ayuda ni con los procesos. Se sentía como si no importara cuanto me esforzara por hacer siempre salia algo mas.

Esta experiencia es claramente muy personal y realmente se extiende mucho mas allá que las limitadas paredes de este ejemplo. Desde entonces he trabajado en otras empresas haciendo web scrapping, bots de chat y de voz y multitud de otros trabajos con la esperanza de que los trabajos resultaran mucho mas enriquecedores. Sin embargo, una y otra vez me choco con que no consigo ayudar a nadie. Llega un punto en el que me pregunto si no resulta que el \textit{Bullshit Job} es el mio. Que no consigo nada por mucho que los productos que desarrolle resulten correctos.

\section{Conclusión: ¿Hay algo por hacer?}

Con un tono mas bien triste la pregunta se asoma inquisidora ¿Como es posible que sigamos trabajando 40 horas a la semana?¿Acaso es imposible hacer algo? Me resulta devastador no poder dar una respuesta concreta. Creía honestamente que con un programa lo suficientemente bueno las personas podrían dedicarse mucho mas a lo que quieren. Sin embargo, algo que he aprendido en este curso es que la tecnología es solo eso, un objeto. Como tal, no tiene agenda si no la de aquellos que la desarrollan y la usan. Los algoritmos y los robots no son buenos ni malos. Quizás entonces tengo que girar el foco y mirar al como nos organizamos, a la ética del trabajo y de lo que hacemos para poder encontrar soluciones. Abolir el trabajo sigue siendo mi meta, pero quiero terminar con que esto no se conseguirá con avances en la tecnología si no con un profundo rediseño de nuestra aproximación al mundo y a sus modelos.

\end{document}

\documentclass[12pt]{article}
\usepackage{array}
\usepackage{color}
\usepackage{amsthm}
\usepackage{eufrak}
\usepackage{lipsum}
\usepackage{pifont}
\usepackage{yfonts}
\usepackage{amsmath}
\usepackage{amssymb}
\usepackage{ccfonts}
\usepackage{comment} \usepackage{amsfonts}
\usepackage{fancyhdr}
\usepackage{graphicx}
\usepackage{listings}
\usepackage{mathrsfs}
\usepackage{setspace}
\usepackage{textcomp}
\usepackage{blindtext}
\usepackage{enumerate}
\usepackage{microtype}
\usepackage{xfakebold}
\usepackage{kantlipsum}
%\usepackage{draftwatermark}
\usepackage[spanish]{babel}
\usepackage[margin=1.5cm, top=2cm, bottom=2cm]{geometry}
\usepackage[framemethod=tikz]{mdframed}
\usepackage[colorlinks=true,citecolor=blue,linkcolor=red,urlcolor=magenta]{hyperref}

%//////////////////////////////////////////////////////
% Watermark configuration
%//////////////////////////////////////////////////////
%\SetWatermarkScale{4}
%\SetWatermarkColor{black}
%\SetWatermarkLightness{0.95}
%\SetWatermarkText{\texttt{Watermark}}

%//////////////////////////////////////////////////////
% Frame configuration
%//////////////////////////////////////////////////////
\newmdenv[tikzsetting={draw=gray,fill=white,fill opacity=0},backgroundcolor=none]{Frame}

%//////////////////////////////////////////////////////
% Font style configuration
%//////////////////////////////////////////////////////
\renewcommand{\familydefault}{\ttdefault}
\renewcommand{\rmdefault}{tt}

%//////////////////////////////////////////////////////
% Bold configuration
%//////////////////////////////////////////////////////
\newcommand{\fbseries}{\unskip\setBold\aftergroup\unsetBold\aftergroup\ignorespaces}
\makeatletter
\newcommand{\setBoldness}[1]{\def\fake@bold{#1}}
\makeatother

%//////////////////////////////////////////////////////
% Default font configuration
%//////////////////////////////////////////////////////
\DeclareFontFamily{\encodingdefault}{\ttdefault}{%
  \hyphenchar\font=\defaulthyphenchar
  \fontdimen2\font=0.33333em
  \fontdimen3\font=0.16667em
  \fontdimen4\font=0.11111em
  \fontdimen7\font=0.11111em}



\begin{document}
    %//////////////////////////////////////////////////////
% Heading Configuration
%//////////////////////////////////////////////////////
\pagestyle{fancy}
\thispagestyle{plain}
\fancyhead[RO,L]{\textbf{Geometría de Curvas y Superficies (MATE-2411)}}
\fancyhead[LO,L]{\textbf{Tarea 1}}
\setlength{\headheight}{16.0pt}

%//////////////////////////////////////////////////////
% Subsections Configuration
%//////////////////////////////////////////////////////
\renewcommand*\thesubsection{\arabic{subsection}}
\newcounter{counter}
\newlength{\palabra}
\settowidth{\palabra}{counter 999.}
\newcommand{\makeboxlabel}[1]{\fbox{#1.}\hfill}

%//////////////////////////////////////////////////////
% Personalized commands configuration
%//////////////////////////////////////////////////////
\newcommand{\N}{\mathbb{N}}
\newcommand{\Z}{\mathbb{Z}}
\newcommand{\Q}{\mathbb{Q}}
\newcommand{\R}{\mathbb{R}}
\newcommand{\C}{\mathbb{C}}
\newcommand{\re}{\operatorname{Re}}
\newcommand{\im}{\operatorname{Im}}
\newcommand{\Aut}{\operatorname{Aut}}
\newcommand{\GCD}{\operatorname{GCD}}
\newcommand{\LCD}{\operatorname{LCD}}
\linespread{1} %Line spacing

%//////////////////////////////////////////////////////
% Inline code configuration
%//////////////////////////////////////////////////////
\lstset{
gobble=5,
numbers=left,
frame=single,
framerule=1pt,
showtabs=False,
showspaces=False,
showstringspaces=False,
backgroundcolor=\color{gray}}

%//////////////////////////////////////////////////////
% Problem list configuration
%//////////////////////////////////////////////////////
\newenvironment{problems}
  {\begin{list}
     {{\fbseries Problem \arabic{counter}.}}
    {\usecounter{counter}
     \setlength{\labelsep}{1em}
     \setlength{\itemsep}{2pt}
     \setlength{\leftmargin}{2em}
     \setlength{\rightmargin}{0cm}
     \setlength{\itemindent}{1em} }}
{\end{list}}

%//////////////////////////////////////////////////////
% Appendix configuration
%//////////////////////////////////////////////////////
\newenvironment{Appendix}
  {\begin{list}
     {{\fbseries Lemma \arabic{counter}.}}
    {\usecounter{counter}
     \setlength{\labelsep}{1em}
     \setlength{\itemsep}{2pt}
     \setlength{\leftmargin}{2em}
     \setlength{\rightmargin}{0cm}
     \setlength{\itemindent}{1em} }}
{\end{list}}

%//////////////////////////////////////////////////////
% Notes configuration
%//////////////////////////////////////////////////////
\newenvironment{notes}
  {\begin{list}
     {{\fbseries Note \arabic{counter}.}}
    {\usecounter{counter}
     \setlength{\labelsep}{1em}
     \setlength{\itemsep}{2pt}
     \setlength{\leftmargin}{2em}
     \setlength{\rightmargin}{0cm}
     \setlength{\itemindent}{1em} }}
{\end{list}}

%//////////////////////////////////////////////////////
% Activity Information
%//////////////////////////////////////////////////////
\vspace*{-1cm}
\hrule width \hsize \kern 1mm \hrule width \hsize height 2pt
\begin{center}
   \parbox[c]{.32\textwidth}{
   \hspace{1cm}\\
   Sergio Montoya Ramirez\\
   202112171}
%   Luis Ernesto Tejón Rojas\\
%   202113150}
   \hspace*{\fill}
   \parbox[c]{.35\textwidth}{\centering
   Universidad de Los Andes\\
   Tarea 1\\
   Geometría de Curvas y Superficies\\
   }
   \hspace*{\fill}
   \parbox[c]{.3\textwidth}{
   \begin{flushleft}
      Bogotá D.C., Colombia\\
      \today
   \end{flushleft}}
\end{center}
\hrule width \hsize height 2pt \kern 1mm \hrule width \hsize

\bigskip


\bigskip



    \section*{Primera Pregunta}
    \subsection*{Enunciado}
    \begin{align*}
      F\left( n \right) &= 12\cdot F\left( n - 1 \right) - 35 \cdot F\left( n - 2 \right)\\
      F\left( 0 \right) &= 3\\
      F\left( 1 \right) &= 19
    .\end{align*}
    \subsection*{Solución}

    \subsubsection*{Poner en forma estándar}
    \[
    F\left( n \right) - 12\cdot F\left( n - 1 \right) + 35\cdot F\left( n - 2 \right)  = 0\\
    .\] 

    \subsubsection*{Calcular Raíces del Polinomio Característico}
    \begin{align*}
      \lambda^2 - 12 \lambda + 35 &= 0\\
      \lambda &= \frac{-b \pm \sqrt{b^2 - 4ac} }{2a} \\
      &= \frac{12 \pm \sqrt{144 - 140 } }{2} \\
      &= \frac{12 \pm 2}{2} \\
      &= 6 \pm 1 \\
      &= \begin{cases}
        5 \\
	7
      \end{cases}
    .\end{align*}

    \subsubsection*{Solución Homogénea}
    $H\left( n \right) = c_1\cdot 5^{n} + c_2\cdot 7^{n}$
    \subsubsection*{Solución Particular}
    \begin{enumerate}
      \item $a$
	\begin{align*}
	  F\left( n \right) - 12F\left( n -1 \right) - 35F\left( n - 2 \right) &= 0 \\
	  a - 12\cdot a - 35\cdot a &= 0 \\
	  a\left( -46 \right) &= 0 \\
	  a &= 0
	.\end{align*}
      \item $a \cdot n$ 
	\begin{align*}
	  F\left( n \right) - 12F\left( n -1 \right) - 35F\left( n - 2 \right) &= 0 \\
	  a\cdot n - 12\cdot a \left( n - 1 \right) - 35 \cdot a \left( n - 2 \right) &= 0 \\
	  an\left( -49 \right) &= 0 \\
	  a &= 0
	.\end{align*}
    \end{enumerate}
    \subsubsection*{Solución Total}
    \begin{align*}
      F\left( n \right) &= H\left( n \right) + P\left( n \right)  \\
      F\left( n \right) &= c_1\cdot 5^{n} + c_2 7^{n}
    .\end{align*}

    \subsubsection*{Calculo Constantes}
    \begin{align*}
      F\left( 0 \right) &= 3 \\
      c_1 + c_2 &= 3 \\
      F\left( 1 \right) &= 19 \\
      c_1\cdot 5 + c_2 \cdot 7 &= 19 \\
      c_1 &= 3 - c_2 \\
      \left( 3 - c_2 \right)\cdot 5 + c_2 \cdot 7 &= 19 \\
      15 - 5c_2 + 7c_2 &= 19 \\
      2c_2 &= 4 \\
      c_2 &= 2 \\
      c_1 &= 3 - c_2 \\
      c_1 &= 1 \\
      c_2 &= 2
    .\end{align*}
    \subsection{Respuesta}
    $F\left( n \right) = 1\cdot 5^{n} + 2 \cdot 7^{n} $

%///////////////////////////////////////////////////////////////////////////////////////////////////////////
    \section*{Segunda Pregunta}
    \subsection*{Enunciado}
    \begin{align*}
      F\left( n \right) &= 7\cdot F\left( n -1 \right) + 7^n\\
      F\left( 0 \right) &= 14
    .\end{align*}
    \subsection*{Solución}
    \subsubsection*{Poner en forma Estándar}
    $F\left( n \right) - 7\cdot F\left( n - 1 \right) = 7^{n}$

    \subsubsection*{Raíces del Polinomio Característico}
    \begin{align*}
      F\left( n \right) - 7\cdot F\left( n - 1 \right) &= 0 \\
      \lambda - 7 &= 0 \\
      \lambda &= 7
    .\end{align*}
    \subsubsection*{Solución Homogénea}
    $H\left( n \right) = a\left( 7^{n} \right) $

    \subsubsection*{Solución Particular}
    \begin{enumerate}
      \item $n\cdot b\cdot 7^{n}$ (Probamos directamente desde aquí por que sabemos que no puede ser igual a la solución Homogénea)
	\begin{align*}
	  F\left( n \right) - 7\cdot F\left( n - 1 \right) &= 7^{n}\\
	  n\cdot b\cdot 7^{n} - 7\cdot \left( n - 1 \right) b \cdot 7^{n - 1} &= 7^{n} \\
	  n\cdot b \cdot 7^{n} - \left( n - 1 \right) \cdot b \cdot 7^{n}&= 7^{n} \\
	  n\cdot b \cdot 7^{n} - b\cdot n\cdot 7^{n} + b \cdot 7^{n} &= 7^{n} \\
	  b\cdot 7^{n} &= 7^{n} \\
	  b &= 1
	.\end{align*}
    \end{enumerate}

    \subsubsection*{Solución Total}
    \begin{align*}
      F\left( n \right) &= H\left( n \right) + P\left( n \right)  \\
      F\left( n \right)  &= a\left( 7^{n} \right) + n\cdot 7^{n}
    .\end{align*}

    \subsubsection*{Calculo de constantes}
    \begin{align*}
      F\left( 0 \right) &= 14 \\
      a\left( 7^{0} \right) + 0 \cdot 7^{0} &= 14 \\
      a &= 14
    .\end{align*}

    \subsection{Respuesta}
    \[
      F\left( n \right) = 14\left( 7^{n} \right) + n\cdot 7^{n} 
    .\] 

%///////////////////////////////////////////////////////////////////////////////////////////////////////////
    \section*{Tercera Pregunta}
    \subsection*{Enunciado}
    \begin{align*}
      F\left( n \right) &= 5*F\left( \frac{n}{5} \right) + 4\cdot \log_5\left( n \right)  \\
      F\left( 1 \right) &= \frac{7}{4}
    .\end{align*}
    \subsection*{Solución}
    \subsubsection*{Transformación}
    Sea $n=5^{m}$ entonces esto queda:
    \begin{align*}
      F\left( n \right) &= 5\cdot F\left( \frac{n}{5} \right) + 4\cdot \log_5\left( n \right)  \\
      F\left( 5^{m} \right) &= 5\cdot F\left( \frac{5^{m}}{5} \right) + 4 \cdot \log_5\left( 5^{m} \right)  \\
      F\left( 5^{m} \right) &= 5\cdot F\left( 5^{m - 1} \right) + 4\cdot m \\
      G\left( m \right) &= 5\cdot G\left( m - 1 \right) + 4 \cdot m
    .\end{align*}

    A partir de este punto trabajaremos con esta transformación y pasaremos de nuevo a $F$ al final del desarrollo.

    \subsubsection*{Forma Estándar}
    $G\left( m \right) - 5\cdot G\left( m - 1 \right) = 4\cdot m$
    \subsubsection*{Raíces del Polinomio Característico}
    \begin{align*}
      G\left( m \right) - 5\cdot G\left( m - 1 \right) = 0\\
      \lambda - 5 &= 0 \\
      \lambda &= 5
    .\end{align*}
    \subsubsection*{Solución Homogénea}
    $H\left( m \right) = a\cdot 5^{n}$
    \subsubsection*{Solución Particular}
    \begin{enumerate}
      \item $b\cdot m$
	\begin{align*}
	  G\left( m \right) - 5 \cdot G\left( m - 1 \right) &= 4 \cdot m \\
	  b\cdot m - 5\cdot b \cdot m &= 4\cdot m \\
	  -4\cdot b \cdot m &= 4\cdot m \\
	  b &= -1 \\
	.\end{align*}
    \end{enumerate}
    \subsubsection*{Solución Total}
    \begin{align*}
      G\left( m \right) &= H\left( m \right) + P\left( n \right)  \\
      G\left( m \right) &= a\cdot 5^{m} - m \\
    .\end{align*}
    \subsubsection*{Valor Original}
    \begin{align*}
      m &= \log_5\left( n \right) \\
      G\left( m \right) &= a\cdot 5^{m} - m \\
      G\left( \log_5\left( n \right)  \right) &= a\cdot 5^{\log_{5}\left( n \right) } - \log_5\left( n \right)  \\
      F\left( n \right) &= a \cdot n - \log_5\left( n \right)
    .\end{align*}

    \subsubsection*{Calculo de Constantes}
    \begin{align*}
      F\left( 1 \right) &= \frac{7}{4} \\
      a \cdot 1 - \log_5\left( 1 \right)  &= \frac{7}{4} \\
      a &= \frac{7}{4}
    .\end{align*}

    \subsection{Respuesta}
    $F\left( n \right) = \frac{7}{4} \cdot n - \log_5\left( n \right)  $
\end{document}

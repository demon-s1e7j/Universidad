\documentclass{report}

\documentclass[12pt]{article}
\usepackage{array}
\usepackage{color}
\usepackage{amsthm}
\usepackage{eufrak}
\usepackage{lipsum}
\usepackage{pifont}
\usepackage{yfonts}
\usepackage{amsmath}
\usepackage{amssymb}
\usepackage{ccfonts}
\usepackage{comment} \usepackage{amsfonts}
\usepackage{fancyhdr}
\usepackage{graphicx}
\usepackage{listings}
\usepackage{mathrsfs}
\usepackage{setspace}
\usepackage{textcomp}
\usepackage{blindtext}
\usepackage{enumerate}
\usepackage{microtype}
\usepackage{xfakebold}
\usepackage{kantlipsum}
%\usepackage{draftwatermark}
\usepackage[spanish]{babel}
\usepackage[margin=1.5cm, top=2cm, bottom=2cm]{geometry}
\usepackage[framemethod=tikz]{mdframed}
\usepackage[colorlinks=true,citecolor=blue,linkcolor=red,urlcolor=magenta]{hyperref}

%//////////////////////////////////////////////////////
% Watermark configuration
%//////////////////////////////////////////////////////
%\SetWatermarkScale{4}
%\SetWatermarkColor{black}
%\SetWatermarkLightness{0.95}
%\SetWatermarkText{\texttt{Watermark}}

%//////////////////////////////////////////////////////
% Frame configuration
%//////////////////////////////////////////////////////
\newmdenv[tikzsetting={draw=gray,fill=white,fill opacity=0},backgroundcolor=none]{Frame}

%//////////////////////////////////////////////////////
% Font style configuration
%//////////////////////////////////////////////////////
\renewcommand{\familydefault}{\ttdefault}
\renewcommand{\rmdefault}{tt}

%//////////////////////////////////////////////////////
% Bold configuration
%//////////////////////////////////////////////////////
\newcommand{\fbseries}{\unskip\setBold\aftergroup\unsetBold\aftergroup\ignorespaces}
\makeatletter
\newcommand{\setBoldness}[1]{\def\fake@bold{#1}}
\makeatother

%//////////////////////////////////////////////////////
% Default font configuration
%//////////////////////////////////////////////////////
\DeclareFontFamily{\encodingdefault}{\ttdefault}{%
  \hyphenchar\font=\defaulthyphenchar
  \fontdimen2\font=0.33333em
  \fontdimen3\font=0.16667em
  \fontdimen4\font=0.11111em
  \fontdimen7\font=0.11111em}


\input{macros}
\input{letterfonts}
\usepackage{float}
\usepackage{tabularx}
\usepackage{booktabs}

\title{\Huge{Diseño y Análisis de Algoritmos}\\Tarea 8}
\author{\huge{Sergio Montoya Ramírez}}
\date{202112171}

\begin{document}

\maketitle
\newpage% or \cleardoublepage
% \pdfbookmark[<level>]{<title>}{<dest>}
\pdfbookmark[section]{\contentsname}{toc}
\tableofcontents
\pagebreak

\chapter{}

\section{}

\subsection{Tabla}

\noindent \begin{tabularx}{\textwidth}{@{} l X X X @{}}
\toprule
\textbf{E/S} & \textbf{Nombre} & \textbf{Tipo} & \textbf{Descripción} \\
\midrule
E
& graph
& $List<List<int>>$
& Esta es una matriz de adyacencia. Cada vértice esta representado sin colisión por el indice de la lista. Por otro lado, la entrada $graph[i][j]$ es la distancia que hay entre el nodo $i$ y el $j$. En caso de que $i = j$ entonces el valor de esta entrada es  $\infty$ \\
S
& bestRoute
& $List<int>$
& Lista con el camino mas barato de vertices que pasa por todos\\
\bottomrule
\end{tabularx}

\subsection{Pre condición}
\begin{align*}
  \forall i, j \in V: graph[i][j] = graph[j][i]\\
  \forall i, j \in V: graph[i][j] >  0\\
  graph \neq null
.\end{align*}

\subsection{Post condición}

\begin{align*}
  bestRoute = min(H)\\ \text{ Donde }H\text{ es el conjunto de caminos hamiltonianos sobre matrix}\\
  DEG(bestRoute) = DEG(V) + 1\\
  bestRoute[0] = bestRoute[-1]
.\end{align*}

\section{}

\subsection{Tabla}

\noindent \begin{tabularx}{\textwidth}{@{} l X X X @{}}
\toprule
\textbf{E/S} & \textbf{Nombre} & \textbf{Tipo} & \textbf{Descripción} \\
\midrule
E
& graph
& $List<List<int>>$
& Esta es una matriz de adyacencia. Cada vértice esta representado sin colisión por el indice de la lista. Por otro lado, la entrada $graph[i][j]$ es la distancia que hay entre el nodo $i$ y el $j$. En caso de que $i = j$ entonces el valor de esta entrada es  $\infty$ \\
E
& k
& int
& Valor que me interesa decidir si existe un camino con peso menor o igual a $k$\\
S
& bestRoute
& $List<int>$
& Lista con un camino hamiltoniano con peso menor o igual a $k$\\
S
& existsRoute
& $bool$
& Existe un camino hamiltoniano con peso menor o igual a  $k$\\
\bottomrule
\end{tabularx}

\subsection{Pre condición}
\begin{align*}
  \forall i, j \in V: graph[i][j] = graph[j][i]\\
  \forall i, j \in V: graph[i][j] >  0\\
  graph \neq null\\
  k > 0
.\end{align*}

\subsection{Post condición}

\begin{align*}
  bestRoute \in H\\ \text{ Donde }H\text{ es el conjunto de caminos hamiltonianos sobre matrix}\\
  DEG(bestRoute) = DEG(V) + 1\\
  bestRoute[0] = bestRoute[-1]\\
  \sum_{n=0}^{DEG(V)} graph[bestRoute[n]][bestRoute[n + 1]] \le k
.\end{align*}

\section{}

\subsection{Tabla}

\noindent \begin{tabularx}{\textwidth}{@{} l X X X @{}}
\toprule
\textbf{E/S} & \textbf{Nombre} & \textbf{Tipo} & \textbf{Descripción} \\
\midrule
E
& graph
& $List<List<int>>$
& Esta es una matriz de adyacencia. Cada vértice esta representado sin colisión por el indice de la lista. Por otro lado, la entrada $graph[i][j]$ es la distancia que hay entre el nodo $i$ y el $j$. En caso de que $i = j$ entonces el valor de esta entrada es  $\infty$ \\
E
& k
& int
& Valor que me interesa decidir si existe un camino con peso menor o igual a $k$\\
E
& bestRoute
& $List<int>$
& Lista con un camino hamiltoniano con peso menor o igual a $k$\\
S
& fullFill
& $bool$
& El camino hamiltoniano $bestRoute$ tiene un costo menor a  $k$\\
\bottomrule
\end{tabularx}

\subsection{Pre condición}
\begin{align*}
  \forall i, j \in V: graph[i][j] = graph[j][i]\\
  \forall i, j \in V: graph[i][j] >  0\\
  graph \neq null\\
  k > 0\\
  bestRoute \in H\\ \text{ Donde }H\text{ es el conjunto de caminos hamiltonianos sobre matrix}\\
  DEG(bestRoute) = DEG(V) + 1\\
  bestRoute[0] = bestRoute[-1]
.\end{align*}

\subsection{Post condición}

\begin{align*}
  k = true \to \sum_{n=0}^{DEG(V)} graph[bestRoute[n]][bestRoute[n + 1]] \le k\\
  k = false \to \sum_{n=0}^{DEG(V)} graph[bestRoute[n]][bestRoute[n + 1]] > k
.\end{align*}

\end{document}

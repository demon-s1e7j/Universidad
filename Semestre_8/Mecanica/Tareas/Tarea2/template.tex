\documentclass{report}

\documentclass[12pt]{article}
\usepackage{array}
\usepackage{color}
\usepackage{amsthm}
\usepackage{eufrak}
\usepackage{lipsum}
\usepackage{pifont}
\usepackage{yfonts}
\usepackage{amsmath}
\usepackage{amssymb}
\usepackage{ccfonts}
\usepackage{comment} \usepackage{amsfonts}
\usepackage{fancyhdr}
\usepackage{graphicx}
\usepackage{listings}
\usepackage{mathrsfs}
\usepackage{setspace}
\usepackage{textcomp}
\usepackage{blindtext}
\usepackage{enumerate}
\usepackage{microtype}
\usepackage{xfakebold}
\usepackage{kantlipsum}
%\usepackage{draftwatermark}
\usepackage[spanish]{babel}
\usepackage[margin=1.5cm, top=2cm, bottom=2cm]{geometry}
\usepackage[framemethod=tikz]{mdframed}
\usepackage[colorlinks=true,citecolor=blue,linkcolor=red,urlcolor=magenta]{hyperref}

%//////////////////////////////////////////////////////
% Watermark configuration
%//////////////////////////////////////////////////////
%\SetWatermarkScale{4}
%\SetWatermarkColor{black}
%\SetWatermarkLightness{0.95}
%\SetWatermarkText{\texttt{Watermark}}

%//////////////////////////////////////////////////////
% Frame configuration
%//////////////////////////////////////////////////////
\newmdenv[tikzsetting={draw=gray,fill=white,fill opacity=0},backgroundcolor=none]{Frame}

%//////////////////////////////////////////////////////
% Font style configuration
%//////////////////////////////////////////////////////
\renewcommand{\familydefault}{\ttdefault}
\renewcommand{\rmdefault}{tt}

%//////////////////////////////////////////////////////
% Bold configuration
%//////////////////////////////////////////////////////
\newcommand{\fbseries}{\unskip\setBold\aftergroup\unsetBold\aftergroup\ignorespaces}
\makeatletter
\newcommand{\setBoldness}[1]{\def\fake@bold{#1}}
\makeatother

%//////////////////////////////////////////////////////
% Default font configuration
%//////////////////////////////////////////////////////
\DeclareFontFamily{\encodingdefault}{\ttdefault}{%
  \hyphenchar\font=\defaulthyphenchar
  \fontdimen2\font=0.33333em
  \fontdimen3\font=0.16667em
  \fontdimen4\font=0.11111em
  \fontdimen7\font=0.11111em}


\input{macros}
\input{letterfonts}
\usepackage{float}

\title{\Huge{Mecánica}\\Tarea 2}
\author{\huge{Sergio Montoya Ramírez}}
\date{}

\begin{document}

\maketitle
\newpage% or \cleardoublepage
% \pdfbookmark[<level>]{<title>}{<dest>}
\pdfbookmark[section]{\contentsname}{toc}
\tableofcontents
\pagebreak

\chapter{}
\section{Random Examples}
\dfn{Limit of Sequence in $\bs{\bbR}$}{Let $\{s_n\}$ be a sequence in $\bbR$. We say $$\lim_{n\to\infty}s_n=s$$ where $s\in\bbR$ if $\forall$ real numbers $\eps>0$ $\exists$ natural number $N$ such that for $n>N$ $$s-\eps<s_n<s+\eps\text{ i.e. }|s-s_n|<\eps$$}
\qs{}{Is the set ${x-}$axis${\setminus\{\text{Origin}\}}$ a closed set}
\sol We have to take its complement and check whether that set is a open set i.e. if it is a union of open balls
\nt{We will do topology in Normed Linear Space  (Mainly $\bbR^n$ and occasionally $\bbC^n$)using the language of Metric Space}
\clm{Topology}{}{Topology is cool}
\ex{Open Set and Close Set}{
	\begin{tabular}{rl}
		Open Set:   & $\bullet$ $\phi$                                              \\
		            & $\bullet$ $\bigcup\limits_{x\in X}B_r(x)$ (Any $r>0$ will do) \\[3mm]
		            & $\bullet$ $B_r(x)$ is open                                    \\
		Closed Set: & $\bullet$ $X,\ \phi$                                          \\
		            & $\bullet$ $\overline{B_r(x)}$                                 \\
		            & $x-$axis $\cup$ $y-$axis
	\end{tabular}}
\thm{}{If $x\in$ open set $V$ then $\exists$ $\delta>0$ such that $B_{\delta}(x)\subset V$}
\begin{myproof}By openness of $V$, $x\in B_r(u)\subset V$
	\begin{center}
		\begin{tikzpicture}
			\draw[red] (0,0) circle [x radius=3.5cm, y radius=2cm] ;
			\draw (3,1.6) node[red]{$V$};
			\draw [blue] (1,0) circle (1.45cm) ;
			\filldraw[blue] (1,0) circle (1pt) node[anchor=north]{$u$};
			\draw (2.9,0.4) node[blue]{$B_r(u)$};
			\draw [green!40!black] (1.7,0) circle (0.5cm) node [yshift=0.7cm]{$B_{\delta}(x)$} ;
			\filldraw[green!40!black] (1.7,0) circle (1pt) node[anchor=west]{$x$};
		\end{tikzpicture}
	\end{center}

	Given $x\in B_r(u)\subset V$, we want $\delta>0$ such that $x\in B_{\delta} (x)\subset B_r(u)\subset V$. Let $d=d(u,x)$. Choose $\delta $ such that $d+\delta<r$ (e.g. $\delta<\frac{r-d}{2}$)

	If $y\in B_{\delta}(x)$ we will be done by showing that $d(u,y)<r$ but $$d(u,y)\leq d(u,x)+d(x,y)<d+\delta<r$$
\end{myproof}

\cor{}{By the result of the proof, we can then show...}
\mlenma{}{Suppose $\vec{v_1}, \dots, \vec{v_n} \in \RR[n]$ is subspace of $\RR^n$.}
\mprop{}{$1 + 1 = 2$.}

\section{Random}
\dfn{Normed Linear Space and Norm $\boldsymbol{\|\cdot\|}$}{Let $V$ be a vector space over $\bbR$ (or $\bbC$). A norm on $V$ is function $\|\cdot\|\ V\to \bbR_{\geq 0}$ satisfying \begin{enumerate}[label=\bfseries\tiny\protect\circled{\small\arabic*}]
		\item \label{n:1}$\|x\|=0 \iff x=0$ $\forall$ $x\in V$
		\item \label{n:2}	$\|\lambda x\|=|\lambda|\|x\|$ $\forall$ $\lambda\in\bbR$(or $\bbC$), $x\in V$
		\item \label{n:3} $\|x+y\| \leq \|x\|+\|y\|$ $\forall$ $x,y\in V$ (Triangle Inequality/Subadditivity)
	\end{enumerate}And $V$ is called a normed linear space.

	$\bullet $ Same definition works with $V$ a vector space over $\bbC$ (again $\|\cdot\|\to\bbR_{\geq 0}$) where \ref{n:2} becomes $\|\lambda x\|=|\lambda|\|x\|$ $\forall$ $\lambda\in\bbC$, $x\in V$, where for $\lambda=a+ib$, $|\lambda|=\sqrt{a^2+b^2}$ }


\ex{$\bs{p-}$Norm}{\label{pnorm}$V={\bbR}^m$, $p\in\bbR_{\geq 0}$. Define for $x=(x_1,x_2,\cdots,x_m)\in\bbR^m$ $$\|x\|_p=\Big(|x_1|^p+|x_2|^p+\cdots+|x_m|^p\Big)^{\frac1p}$$(In school $p=2$)}
\textbf{Special Case $\bs{p=1}$}: $\|x\|_1=|x_1|+|x_2|+\cdots+|x_m|$ is clearly a norm by usual triangle inequality. \par
\textbf{Special Case $\bs{p\to\infty\ (\bbR^m$ with $\|\cdot\|_{\infty})}$}: $\|x\|_{\infty}=\max\{|x_1|,|x_2|,\cdots,|x_m|\}$\\
For $m=1$ these $p-$norms are nothing but $|x|$.
Now exercise
\qs{}{\label{exs1}Prove that triangle inequality is true if $p\geq 1$ for $p-$norms. (What goes wrong for $p<1$ ?)}
\sol{\textbf{For Property \ref{n:3} for norm-2}	\subsubsection*{\textbf{When field is $\bbR:$}} We have to show\begin{align*}
		         & \sum_i(x_i+y_i)^2\leq \left(\sqrt{\sum_ix_i^2} +\sqrt{\sum_iy_i^2}\right)^2                                       \\
		\implies & \sum_i (x_i^2+2x_iy_i+y_i^2)\leq \sum_ix_i^2+2\sqrt{\left[\sum_ix_i^2\right]\left[\sum_iy_i^2\right]}+\sum_iy_i^2 \\
		\implies & \left[\sum_ix_iy_i\right]^2\leq \left[\sum_ix_i^2\right]\left[\sum_iy_i^2\right]
	\end{align*}So in other words prove $\langle x,y\rangle^2 \leq \langle x,x\rangle\langle y,y\rangle$ where
	$$\langle x,y\rangle =\sum\limits_i x_iy_i$$

	\begin{note}
		\begin{itemize}
			\item $\|x\|^2=\langle x,x\rangle$
			\item $\langle x,y\rangle=\langle y,x\rangle$
			\item $\langle \cdot,\cdot\rangle$ is $\bbR-$linear in each slot i.e. \begin{align*}
				      \langle rx+x',y\rangle=r\langle x,y\rangle+\langle x',y\rangle	\text{ and similarly for second slot}
			      \end{align*}Here in $\langle x,y\rangle$ $x$ is in first slot and $y$ is in second slot.
		\end{itemize}
	\end{note}Now the statement is just the Cauchy-Schwartz Inequality. For proof $$\langle x,y\rangle^2\leq \langle x,x\rangle\langle y,y\rangle $$ expand everything of $\langle x-\lambda y,x-\lambda y\rangle$ which is going to give a quadratic equation in variable $\lambda $ \begin{align*}
		\langle x-\lambda y,x-\lambda y\rangle & =\langle x,x-\lambda y\rangle-\lambda\langle y,x-\lambda y\rangle                                       \\
		                                       & =\langle x ,x\rangle -\lambda\langle x,y\rangle -\lambda\langle y,x\rangle +\lambda^2\langle y,y\rangle \\
		                                       & =\langle x,x\rangle -2\lambda\langle x,y\rangle+\lambda^2\langle y,y\rangle
	\end{align*}Now unless $x=\lambda y$ we have $\langle x-\lambda y,x-\lambda y\rangle>0$ Hence the quadratic equation has no root therefore the discriminant is greater than zero.

	\subsubsection*{\textbf{When field is $\bbC:$}}Modify the definition by $$\langle x,y\rangle=\sum_i\overline{x_i}y_i$$Then we still have $\langle x,x\rangle\geq 0$}

\section{Algorithms}
\begin{algorithm}[H]
\KwIn{This is some input}
\KwOut{This is some output}
\SetAlgoLined
\SetNoFillComment
\tcc{This is a comment}
\vspace{3mm}
some code here\;
$x \leftarrow 0$\;
$y \leftarrow 0$\;
\uIf{$ x > 5$} {
    x is greater than 5 \tcp*{This is also a comment}
}
\Else {
    x is less than or equal to 5\;
}
\ForEach{y in 0..5} {
    $y \leftarrow y + 1$\;
}
\For{$y$ in $0..5$} {
    $y \leftarrow y - 1$\;
}
\While{$x > 5$} {
    $x \leftarrow x - 1$\;
}
\Return Return something here\;
\caption{what}
\end{algorithm}

\chapter{}

\begin{figure}[H]
  \centering
  \includegraphics[width=0.3\textwidth]{img/pregunta_1.png}
  \caption{Péndulo Doble}
  \label{fig:pregunta_1}
\end{figure}

En este caso, tenemos $3 \cdot N = 3\cdot 2 = 6$ coordenadas cartesianas. Sin embargo, tenemos las siguientes ligaduras:
\begin{enumerate}
  \item $z = 0$
  \item  $l_1 = cte$
  \item $l_2 = cte$
\end{enumerate}

Por lo cual tenemos $3N - n = 3 \cdot 2 - 4 = 2$. Por lo tanto tenemos dos grados de libertad. Con esto entonces definamos las coordenadas generalizadas de la manera en la que nos propone la imagen \ref{fig:pregunta_1}. Con lo cual nos queda:
\begin{align*}
  x_1 &= l_1\cdot \sin\left(\varphi_1\right)  \implies \dot{x_1} = l_1 \dot{\varphi_1} \cos\left( \varphi_1 \right) \\
  y_1 &= -l_1\cdot \cos\left(\varphi_1\right) \implies \dot{y_1} = l_1\dot{\varphi_1}\sin\left( \varphi_1 \right) 
.\end{align*}

Dado que este primer caso es esencialmente un triangulo en donde estamos calculando los dos catetos (Note que el valor de $y$ es negativo)

Ahora bien, de manera similar, para la masa 2 esto seria como calcular estos mismos catetos. Sin embargo, debe iniciar desde los valores de $m_1$
\begin{align*}
  x_2 &= l_1\cdot \sin\left(\varphi_1\right) + l_2 \cdot \sin\left( \varphi_2 \right) \implies \dot{x_2} = l_1\dot{\varphi_1}\cos\left( \varphi_1 \right) + l_2 \dot{\varphi_2}\cos\left( \varphi_2 \right)  \\
  y_2 &= -l_1\cdot \cos\left(\varphi_1\right) - l_2 \cdot \cos\left( \varphi_2 \right) \implies \dot{y_2} = l_1\dot{\varphi_1}\sin\left( \varphi_1 \right) + l_2 \dot{\varphi_2}\sin\left( \varphi_2 \right)  \\
.\end{align*}

Con esto entonces, podemos calcular la energía cinética que es:
\begin{align*}
  T_1 &= \frac{1}{2}m_1v_1^2 \\
      &= \frac{1}{2}m_1\left( \dot{x}_1^2 + \dot{y}_1^2 \right)  \\
      &= \frac{1}{2}m_1\left( l_1^2\dot{\varphi_1}^2\cos^2\left( \varphi_1 \right) + l_1^2\dot{\varphi_1}^2 \sin^2\left( \varphi \right)  \right)  \\
      &= \frac{1}{2}m_1\left( l_1^2\dot{\varphi_1}^2 \right)\left( \cos^2\left( \varphi_1 \right) + \sin^2\left( \varphi_1 \right)  \right)   \\
      &= \frac{1}{2}m_1\left( l_1^2\dot{\varphi_1}^2 \right)
.\end{align*}

Y
\begin{align*}
  T_2 &= \frac{1}{2}m_2v_2^2 \\
      &= \frac{1}{2}m_2\left( \dot{x}_2^2 + \dot{y}_2^2 \right)  \\
      &= \frac{1}{2}m_2 \left( \left( l_1\dot{\varphi_1}\cos\left( \varphi_1 \right) + l_2 \dot{\varphi_2}\cos\left( \varphi_2 \right) \right)^2 + \left( l_1\dot{\varphi_1}\sin\left( \varphi_1 \right) + l_2 \dot{\varphi_2}\sin\left( \varphi_2 \right) \right)^2  \right)  \\
  \left( l_1\dot{\varphi_1}\cos\left( \varphi_1 \right) + l_2\dot{\varphi_2}\cos\left( \varphi_2 \right)  \right)^2 &= l_1^2\dot{\varphi_1}^2\cos^2\left( \varphi_1 \right) + 2l_1l_2\dot{\varphi_1}\dot{\varphi_2}\cos\left( \varphi_1 \right)\cos\left( \varphi_2 \right) + l_2^2\dot{\varphi_2}^2\cos^2\left( \varphi_2 \right)  \\
  \left( l_1\dot{\varphi_1}\sin\left( \varphi_1 \right) + l_2\dot{\varphi_2}\sin\left( \varphi_2 \right)  \right)^2 &= l_1^2\dot{\varphi_1}^2\sin^2\left( \varphi_1 \right) + 2l_1l_2\dot{\varphi_1}\dot{\varphi_2}\sin\left( \varphi_1 \right)\sin\left( \varphi_2 \right) + l_2^2\dot{\varphi_2}^2\sin^2\left( \varphi_2 \right)  \\
  l_1^2\dot{\varphi_1}^2\sin^2\left( \varphi_1 \right) + l_1^2\dot{\varphi_1}^2\cos^2\left( \varphi_1 \right)^2 &= l_1^2\dot{\varphi_1}^2\left( \sin^2\left( \varphi_1 \right) + \cos^2\left( \varphi_1 \right)  \right)  \\
														&\implies l_1^2 \dot{\varphi_1}^2 \\
  l_2^2\dot{\varphi_2}^2\sin^2\left( \varphi_2 \right) + l_2^2\dot{\varphi_2}^2\cos^2\left( \varphi_2 \right)^2 &= l_2^2\dot{\varphi_2}^2\left( \sin^2\left( \varphi_2 \right) + \cos^2\left( \varphi_2 \right)  \right)  \\
														&\implies l_2^2\dot{\varphi_2}^2 \\
  2l_1l_2\dot{\varphi_1}\dot{\varphi_2}\cos\left( \varphi_1 \right)\cos\left( \varphi_2 \right) &+ 2l_1l_2\dot{\varphi_1}\dot{\varphi_2}\sin\left( \varphi_1 \right)\sin\left( \varphi_2 \right) = 2l_1l_2\dot{\varphi_1}\dot{\varphi_2}\left( \cos\left( \varphi_1 \right)\cos\left( \varphi_2 \right) + \sin\left( \varphi_1 \right)\sin\left( \varphi_2 \right)  \right) \\
												&\implies 2l_1l_2\dot{\varphi_1}\dot{\varphi_2}\cos\left( \varphi_1 - \varphi_2 \right)  \\
												&= \frac{1}{2}m_2\left( l_1^2\dot{\varphi_1}^2 + l_2^2\dot{\varphi_2}^2  + 2l_1l_2\dot{\varphi_1}\dot{\varphi_2}\cos\left( \varphi_1 - \varphi_2 \right)\right) 
.\end{align*}

Ahora bien, para el caso de la energía potencial tenemos
\begin{align*}
  V_1 &= m_1gy_1 \\
  &= -m_1gl_1\cos\left( \varphi_1 \right)  \\
  V_2 &= m_2g y_2 \\
  &= -m_2g\left( l_1\cos\left( \varphi_{1} \right) + l_2\cos\left( \varphi_2 \right)  \right) 
.\end{align*}

Ahora bien, tenemos entonces que:
\begin{align*}
  T &= T_1 + T_2 \\
    &= \frac{1}{2}m_1\left( l_1^2\dot{\varphi}_1^2 \right) + \frac{1}{2}m_2\left( l_1^2\dot{\varphi_1}^2 + l_2^2\dot{\varphi_2}^2  + 2l_1l_2\dot{\varphi_1}\dot{\varphi_2}\cos\left( \varphi_1 - \varphi_2 \right)\right) \\
    &= \frac{1}{2}l_1^2\dot{\phi_1}^2\left( m_1 + m_2 \right) + \frac{1}{2}m_2l_2^2\dot{\phi_2}^2 + m_2l_1l_2\dot{\varphi_1}\dot{\varphi_2}\cos\left( \varphi_1 - \varphi_2 \right) 
.\end{align*}

Y
\begin{align*}
  V &= V_1 + V_2 \\
  &= -m_1gl_1\cos\left( \varphi_1 \right) -m_2gl_1\cos\left( \varphi_1 \right) - m_2gl_2\cos\left( \varphi_2 \right)  \\
  &= - gl_1\cos\left( \varphi_1 \right) \left( m_1 + m_2 \right) - m_2 gl_2\cos\left( \varphi_2 \right)
.\end{align*}

Por lo tanto dado que $L = T - V$ nos queda:
\begin{align*}
  L &= T - V \\
  L &=  \frac{1}{2}l_1^2\dot{\phi_1}^2\left( m_1 + m_2 \right) + \frac{1}{2}m_2l_2^2\dot{\phi_2}^2 + m_2l_1l_2\dot{\varphi_1}\dot{\varphi_2}\cos\left( \varphi_1 - \varphi_2 \right) -\left( - gl_1\cos\left( \varphi_1 \right) \left( m_1 + m_2 \right) - m_2 gl_2\cos\left( \varphi_2 \right) \right)  \\
   &=  \frac{1}{2}l_1^2\dot{\phi_1}^2\left( m_1 + m_2 \right) + \frac{1}{2}m_2l_2^2\dot{\phi_2}^2 + m_2l_1l_2\dot{\varphi_1}\dot{\varphi_2}\cos\left( \varphi_1 - \varphi_2 \right) +  gl_1\cos\left( \varphi_1 \right) \left( m_1 + m_2 \right) + m_2 gl_2\cos\left( \varphi_2 \right)
.\end{align*}

\chapter{}

\section{}

Para comenzar, definamos cuantas coordenadas generalizadas existen. Dado que tenemos dos partículas tenemos $3N$ en coordenadas cartesianas. Ahora bien, tenemos 5 ligaduras las cuales son:
\begin{enumerate}
  \item $z_1 = 0$ 
  \item $z_2 = 0$ 
  \item 
    \begin{align*}
      \ell' &= \ell + \pi R\\
      \ell &= \ell' - \pi R \\
      \ell &= CTE\\
      \ell &= y_1 + y_2 = CTE
    .\end{align*}
  \item $x_1 = -R$
  \item $x_2 = R$
\end{enumerate}

Con esto entonces sabemos que tenemos $3N - n = 3\cdot 2 - 5 = 1$ grados de libertad. Por lo cual solo lo describiremos como $\phi$. En particular, tendremos que 
\begin{align*}
  y_1 &= \phi \\
  y_2 &= \ell - \phi
.\end{align*}

Una vez definimos esto, podemos calcular el lagrangiano:
\begin{align*}
  T &= \frac{1}{2}m_1\left( \dot{\phi} \right)^2 + \frac{1}{2}m_2\left( \dot{\phi} \right)^2 \\
  V &= m_1g\phi + m_2g\left( \ell - \phi \right)  \\
  \mathcal{L} &= T - V \\
  &= \frac{1}{2}m_1\left( \dot{\phi} \right)^2 + \frac{1}{2}m_2\left( \dot{\phi} \right)^2 - \left( m_1g\phi + m_2g\left(\ell-\phi\right)\right)\\
  &= \left( \frac{1}{2}\dot{\phi}^2 \right)\left( m_1 + m_2 \right) - g\left( m_1\phi + m_2\ell - m_2\phi \right)  \\
.\end{align*}

Con lo cual podemos mirar las ecuaciones de movimiento:
\begin{align*}
  \frac{\delta \mathcal{L}}{\delta \phi} &= 0 \\
  \frac{d}{dt}\left( \frac{\partial \mathcal{L}}{\partial \dot{\phi}}  \right) - \frac{\partial \mathcal{L}}{\partial \phi} &= 0 \\
  \frac{\partial \mathcal{L}}{\partial \dot{\phi}} &= \frac{\partial}{\partial \dot{\phi}} \left( \frac{1}{2}\dot{\phi}^2 \right)\left( m_1 + m_2 \right) - g\left( m_1\phi + m_2\ell - m_2\phi \right) \\
						   &= \left( m_1 + m_2 \right) \dot{\phi} \\
  \frac{d}{dt}\left( \frac{\partial \mathcal{L}}{\partial \phi}  \right) &= \frac{d}{dt}\left( \left( m_1 + m_2 \right)\dot{\phi}  \right)  \\
									 &= \left( m_1 + m_2 \right) \ddot{\phi} \\
  \frac{\partial \mathcal{L}}{\partial \phi} &= \frac{\partial}{\partial \phi} \left(\left( \frac{1}{2}\dot{\phi}^2 \right)\left( m_1 + m_2 \right) - g\left( m_1\phi + m_2\ell - m_2\phi \right)  \right)  \\
  &= -g\left( m_1 - m_2 \right)  \\
  \frac{d}{dt}\left( \frac{\partial \mathcal{L}}{\partial \dot{\phi}}  \right) - \frac{\partial \mathcal{L}}{\partial \phi} &= \left( m_1 + m_2 \right) \ddot{\phi} - \left( -g\left( m_1 - m_2 \right) \right) = 0 \\
  \left( m_1 + m_2 \right) \ddot{\phi} - \left( g\left( m_2 - m_1 \right) \right) &= 0\\
  \left( m_1 + m_2 \right) \ddot{\phi} &= g\left( m_2 - m_1 \right)  \\
  \ddot{\phi} &= \frac{g\left( m_2 - m_1 \right) }{\left( m_1 + m_2 \right) }
.\end{align*}

\chapter{}

En este caso vamos a partir de que un cilindro rueda sobre otro. Para este caso tenemos dos cilindros, uno de radio $a$  y uno de radio $\alpha a$. Las medidas se pueden observar en la figura \ref{fig:img-pregunta_3_1-png}. 

\begin{figure}[H]
  \centering
  \includegraphics[width=0.4\textwidth]{img/pregunta_3_1.png}
  \caption{Dimensiones y descripción del problema}
  \label{fig:img-pregunta_3_1-png}
\end{figure}

Con esto entonces nos hace falta ver las fuerzas que las puede encontrar en la figura \ref{fig:img-pregunta_3_2-png}

\begin{figure}[H]
  \centering
  \includegraphics[width=0.4\textwidth]{img/pregunta_3_2.png}
  \caption{Fuerzas en el sistema}
  \label{fig:img-pregunta_3_2-png}
\end{figure}

En este caso, dado que el cilindro $\alpha a$ ira teniendo cada vez una menor fuerza $Z$ (dado que el angulo en el que se encuentra con respecto a su eje disminuye) entonces sabemos que eventualmente este cilindro se despegara del fijo $a$. Ademas, sabemos que estos dos perderán contacto en cuanto la fuerza normal  $F$ ya no tenga ingerencia. Es decir, cuando \[
F = 0
.\] Ademas, sabemos por el como funciona la fricción que \[
f \le \mu F
.\] donde $\mu$ es el coeficiente de fricción. Lo que implica que este movimiento eventualmente empezara a deslizar sobre la superficie. Por lo tanto, podemos dividir este análisis en 3 momentos definidos por los ángulos $\theta_1$, $\theta_2$ y $\theta_3$. Para iniciar, desde $0$ hasta $\theta_1$ el cilindro rueda sin deslizar, lo que se describe en que la fricción seria \[
f = \mu F
.\] Ahora, entre $\theta_1$ y $\theta_2$ el cilindro se deslizaría pues la fricción es muy pequeña y en $\theta_2$ el cilindro dejaría de estar en contacto con el otro y por lo tanto se sostendría $F = 0$ y de ahí en adelante caería libremente.

Ahora bien iniciemos caso a caso
\section{}

Para empezar para un movimiento que no desliza entonces sabemos que:
\begin{align*}
  a\theta &= \alpha a \left( \varphi - \theta \right) \\
  \theta &= \alpha \left( \varphi - \theta \right) \\
  \theta &= \alpha \varphi - \alpha \theta \\
  \theta + \alpha \theta &= \alpha\varphi \\
  \left( 1 + \alpha \right) \theta &= \alpha \varphi
.\end{align*}

Ahora bien, podemos definir una función desarrollando:
\begin{align*}
  \left( 1 + \alpha \right) \theta &= \alpha \varphi \\
  \theta &= \frac{\alpha \varphi}{\left( 1 + \alpha \right) } \\
  \gamma &= \theta - \frac{\alpha \varphi}{\left( 1 + \alpha \right) }\\
  \frac{\alpha \varphi}{\left( 1 + \alpha \right) } &= \theta - \gamma \\
  \alpha \varphi &= \left( \theta - \gamma \right) \left( 1 + \alpha \right)  \\
  \varphi &= \frac{\left( \theta - \gamma \right) \left( 1 + \alpha \right) }{\alpha}
.\end{align*}

Donde, como se puede notar cuando esto no desliza $\gamma = 0$. Ahora bien, por otro lado, tenemos que al estar estos dos cilindros en contacto la distancia entre los centros queda: \[
r = a + \alpha a = \left( 1 + \alpha \right) a
.\] Ahora bien,
\begin{align*}
  T &= \frac{1}{2}m\left( \dot{r}^2 + r^2\dot{\theta}^2 \right) + \frac{1}{2}\left( \frac{1}{2}m \right) \left( \alpha^2 a^2 \dot{\varphi}^2 \right)  \\
    &= \frac{1}{2}m\dot{r}^2 + \frac{1}{2}m r^2\dot{\theta}^2 + \frac{1}{4}m\left( \alpha^2 a^2 \left( \frac{\left( \theta - \gamma \right)\left( 1 + \alpha \right) }{\alpha} \right)^2 \right) \\ 
    \frac{1}{4} m \left( \alpha^2 a^2 \left( \frac{\left( \theta - \gamma \right) \left( 1 + \alpha \right) }{\alpha} \right)^2 \right) &= \frac{1}{4}m \left( \alpha^2 a^2 \frac{\left( \theta - \gamma \right)^2 \left( 1 + \alpha \right)^2}{\alpha^2} \right)  \\
    &=  \frac{1}{4}m \left( a^2 \left( 1 + \alpha \right)^2 \left( \dot{\theta}^2 - 2\dot{\gamma}\dot{\theta} + \dot{\gamma}^2 \right)  \right) \\
    T &= \frac{1}{2}m\dot{r}^2 + \frac{1}{2}m r^2\dot{\theta}^2 +\frac{1}{4}m \left( a^2 \left( 1 + \alpha \right)^2 \left( \dot{\theta}^2 - 2\dot{\gamma}\dot{\theta} + \dot{\gamma}^2 \right)  \right)
.\end{align*}

Ahora bien, podemos encontrar las fuerzas generalizadas con la ecuación
\begin{align*}
  \delta W_k &= Q_k \delta q_k \\
  Q_k &= \frac{\delta W_k}{\delta q_k} \\
  Q_k &= - \frac{\partial V}{\partial q_k}
.\end{align*}

% TODO Desarrollo V
\textbf{TODO} V

Con lo cual, podríamos usar:
\begin{align*}
  V &= mgr\cos\left( \theta \right) + \gamma fa\left( 1 + \alpha \right) - F_r  \\
  Q_k &= - \frac{\partial V}{\partial k}  \\
  Q_\theta &= - \left( - mgr \sin\left( \theta \right) + 0 + 0 \right)  \\
  &= mgr \sin\left( \theta \right)  \\
  Q_\gamma &= - \left( 0 + f\left( 1 + \alpha \right) - 0 \right)  \\
  &= -fa\left( 1 + \alpha \right)  \\
  Q_r &= - \left( mg\cos\left( \theta \right) - F\right)  \\
  &= F - mg\cos\left( \theta \right)
.\end{align*}

Con lo cual

\begin{align*}
  Q_\theta &= mgr\sin\left( \theta \right)  \\
  Q_\gamma &= - fa\left( 1 + \alpha \right)  \\
  Q_r &= F - mg\cos\left( \theta \right)
.\end{align*}

Ahora bien, queda:
\begin{align*}
  \mathcal{L} &= T - V \\
  &=  \frac{1}{2}m\dot{r}^2 + \frac{1}{2}m r^2\dot{\theta}^2 +\frac{1}{4}m \left( a^2 \left( 1 + \alpha \right)^2 \left( \dot{\theta}^2 - 2\dot{\gamma}\dot{\theta} + \dot{\gamma}^2 \right)  \right)\\
  &- \left( mgr\cos\left( \theta \right) + \gamma fa\left( 1 + \alpha \right) - Fr \right) 
.\end{align*}

Con lo cual \[
  \frac{d}{dt}\left( \frac{\partial \mathcal{L}}{\partial \dot{q_k}}  \right) - \frac{\partial \mathcal{L}}{\partial q_k} = Q_k
.\] 

Ahora para esto:
\begin{align*}
  \frac{\partial \mathcal{L}}{\partial \theta} &= mr^2\dot{\theta} + \frac{1}{2}m\left( a^2\left( 1 + \alpha \right)^2 \right)\left( \dot{\theta} - \dot{\gamma} \right)  \\
.\end{align*}

\chapter{}

Para iniciar podemos describir la energía cinética como:
\begin{align*}
  T &= \frac{1}{2}mv^2 \\
  T &= \frac{1}{2}m\left( a^2\dot{\theta}^2 + a^2\omega^2 \sin^2\left( \theta \right)  \right)  \\
  T &= \frac{1}{2}ma^2\dot{\theta}^2 + \frac{1}{2}ma^2\omega^2\sin^2\left( \theta \right)  \\
  V &= -mgy\\
  V &= -mg\left( a \cos\left( \theta \right)  \right)  \\
  \mathcal{L} &= T - V \\
  \mathcal{L} &= \frac{1}{2}ma^2\dot{\theta}^2 + \frac{1}{2}ma^2\omega^2\sin^2\left( \theta \right) + mg\left( a \cos\left( \theta \right) \right) 
.\end{align*}

Ademas, podemos notar que \[
  \frac{\partial \mathcal{L}}{\partial t}  = 0
.\] Con esto entonces, podemos utilizar la ecuación: \[
\sum_{k=1}^{f} \dot{q_k}\frac{\partial \mathcal{L}}{\partial \dot{q_k}} - \mathcal{L} = CTE
.\] Que dado que solo tenemos un grado de libertad esto se puede resumir a

\begin{align*}
  \dot{q_k}\frac{\partial \mathcal{L}}{\partial \dot{q_k}} &= \left( ma^2\dot{\theta}^2 \right) -\frac{1}{2}ma^2\dot{\theta}^2 - \frac{1}{2}ma^2\omega^2\sin^2\left( \theta \right) - mg\left( a \cos\left( \theta \right) \right)   \\
  &= \frac{1}{2}ma^2\dot{\theta}^2 - \frac{1}{2}ma^2\omega^2\sin^2\left( \theta \right) - mg\left( a \cos\left( \theta \right) \right) \\
  &= \mathcal{E}
.\end{align*}

Que esto lo podemos tomar como un lagrangiano con un eje fijo. Con lo cual
\begin{align*}
  \mathcal{V}\left( \theta \right) &= -\frac{1}{2}ma^2\omega^2\sin^2\left( \theta \right) - mga\cos\left( \theta \right)  \\
.\end{align*}

Que para comprobar la velocidad angular limite tenemos
\begin{align*}
  \mathcal{V}\left( \theta \right) &= -\frac{1}{2}ma^2\omega^2\sin^2\left( \theta \right)  - mga\cos\left( \theta \right) \\
  &= -\frac{1}{2}ma^2\left( \left( \frac{g}{a} \right)^{\frac{1}{2}} \right)^2\sin^2\left( \theta \right)  - mga\cos\left( \theta \right)  \\
  &= -\frac{1}{2}mag\sin^2\left( \theta \right)  - mga\cos\left( \theta \right)  \\
  &= -\frac{1}{2}\sin^2\left( \theta \right) - \cos\left( \theta \right)  \\
.\end{align*}

\end{document}

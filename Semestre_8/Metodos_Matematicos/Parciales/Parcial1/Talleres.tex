  \documentclass[3pt]{exam}
\usepackage{amsthm}
\usepackage{libertine}
\usepackage[utf8]{inputenc}
\usepackage[margin=1in]{geometry}
\usepackage{amsmath,amssymb}
\usepackage{multicol}
\usepackage[shortlabels]{enumitem}
\usepackage{siunitx}
\usepackage{cancel}
\usepackage{graphicx}
\usepackage{pgfplots}
\usepackage{listings}
\usepackage{tikz}


\pgfplotsset{width=10cm,compat=1.9}
\usepgfplotslibrary{external}
\tikzexternalize

\newcommand{\class}{Métodos Matemáticos} % This is the name of the course 
\newcommand{\examnum}{Ecuaciones Parcial 1} % This is the name of the assignment
\newcommand{\examdate}{\today} % This is the due date
\newcommand{\timelimit}{}
\newcommand{\Res}{\mathrm{Res}}


\begin{document}
\pagestyle{plain}
\thispagestyle{empty}

\noindent
\begin{tabular*}{\textwidth}{l @{\extracolsep{\fill}} r @{\extracolsep{6pt}} l}
	\textbf{\class} & \textbf{Nombre:} & \textit{Sergio Montoya}\\ %Your name here instead, obviously 
	\textbf{\examnum} &&\\
	\textbf{\examdate} &&
\end{tabular*}\\
\rule[2ex]{\textwidth}{2pt}
% ---

\begin{itemize}
  \item \textbf{Funciones Complejas:}
    \begin{align*}
      f\left( z \right) &= f\left( x, y \right) = u\left( x, y \right) + i v\left( x, y \right)  \\
      f'\left( z \right) &= \text{Como si }z \in \mathbb{R}
    .\end{align*}
    \begin{itemize}
      \item \textbf{Función Analítica:} Se puede hacer transformada de Fourier
      \item \textbf{Función Holomorfa:} Se puede derivar
      \item  $\text{Analitica} \Leftrightarrow \text{Holomorfa} $
    \end{itemize}
  \item \textbf{Ecuaciones Cauchy-Riemann:}
     \begin{align*}
       \frac{\partial u}{\partial x} &= \frac{\partial v}{\partial y}; \frac{\partial u}{\partial y}  = -\frac{\partial v}{\partial x} 
    .\end{align*}
  \item \textbf{Serie de Fourier:} \[
  \sum_{n=1}^{\infty} a_n\left( z - z_0 \right)^{n}
  .\] 
 \item \textbf{Serie de Laurent:} \[
 \sum_{n=-\infty}^{\infty} a_n\left( z - z_0 \right)^{n}
 .\] 
\item \textbf{Teorema de Cauchy:} \[
\oint_{C} f\left( z \right) dz = 0
.\] cuando $C$ es un contorno sin polos para $f\left( z \right) $
\item \textbf{Integral de Cauchy:} \[
f\left( a \right) = \frac{1}{2\pi i}\oint_{C} \frac{f\left( z \right) }{z - a} dz 
.\] Donde $f\left( z \right) $ es holomorfa para todo $C$
\item \textbf{Integral de Residuos:}  \[
\oint_{C} f\left( z \right) dz = 2\pi i \sum_{k=1}^{n} \Res\left( f, a_k \right)  
.\] Donde $a_k$ son los polos que hay dentro de $C$
\item \textbf{Residuo Polo Simple:} \[
Res\left( f, z_0 \right) = \lim_{z \to z_0} \left( z - z_0 \right) f\left( z \right) 
.\] 

\end{itemize}

\end{document}

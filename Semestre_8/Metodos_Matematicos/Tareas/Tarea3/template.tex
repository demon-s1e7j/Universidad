\documentclass{report} \documentclass[12pt]{article}
\usepackage{array}
\usepackage{color}
\usepackage{amsthm}
\usepackage{eufrak}
\usepackage{lipsum}
\usepackage{pifont}
\usepackage{yfonts}
\usepackage{amsmath}
\usepackage{amssymb}
\usepackage{ccfonts}
\usepackage{comment} \usepackage{amsfonts}
\usepackage{fancyhdr}
\usepackage{graphicx}
\usepackage{listings}
\usepackage{mathrsfs}
\usepackage{setspace}
\usepackage{textcomp}
\usepackage{blindtext}
\usepackage{enumerate}
\usepackage{microtype}
\usepackage{xfakebold}
\usepackage{kantlipsum}
%\usepackage{draftwatermark}
\usepackage[spanish]{babel}
\usepackage[margin=1.5cm, top=2cm, bottom=2cm]{geometry}
\usepackage[framemethod=tikz]{mdframed}
\usepackage[colorlinks=true,citecolor=blue,linkcolor=red,urlcolor=magenta]{hyperref}

%//////////////////////////////////////////////////////
% Watermark configuration
%//////////////////////////////////////////////////////
%\SetWatermarkScale{4}
%\SetWatermarkColor{black}
%\SetWatermarkLightness{0.95}
%\SetWatermarkText{\texttt{Watermark}}

%//////////////////////////////////////////////////////
% Frame configuration
%//////////////////////////////////////////////////////
\newmdenv[tikzsetting={draw=gray,fill=white,fill opacity=0},backgroundcolor=none]{Frame}

%//////////////////////////////////////////////////////
% Font style configuration
%//////////////////////////////////////////////////////
\renewcommand{\familydefault}{\ttdefault}
\renewcommand{\rmdefault}{tt}

%//////////////////////////////////////////////////////
% Bold configuration
%//////////////////////////////////////////////////////
\newcommand{\fbseries}{\unskip\setBold\aftergroup\unsetBold\aftergroup\ignorespaces}
\makeatletter
\newcommand{\setBoldness}[1]{\def\fake@bold{#1}}
\makeatother

%//////////////////////////////////////////////////////
% Default font configuration
%//////////////////////////////////////////////////////
\DeclareFontFamily{\encodingdefault}{\ttdefault}{%
  \hyphenchar\font=\defaulthyphenchar
  \fontdimen2\font=0.33333em
  \fontdimen3\font=0.16667em
  \fontdimen4\font=0.11111em
  \fontdimen7\font=0.11111em}

 \input{macros} \input{letterfonts}
\DeclareMathOperator\erf{erf}

\title{\Huge{Métodos Matemáticos}\\Tarea 3}
\author{\huge{Sergio Montoya Ramírez - 202112171}}
\date{}

\begin{document}

\maketitle
\newpage% or \cleardoublepage
% \pdfbookmark[<level>]{<title>}{<dest>} \pdfbookmark[section]{\contentsname}{toc}
\tableofcontents
\pagebreak

\chapter{}

Dado que estas funciones dependían de $n$ tanto como de $x$ se considero que lo mejor era hacer una animación en la que se cambiaba $n$. Por lo tanto, se pondrá un link al vídeo de youtube que tiene esta animación.

\section{$\frac{n}{\pi}\frac{1}{1 + n^2x^2}$}

\subsection{Animación:}
\url{https://youtu.be/eP-jSiI97WQ?si=sAfjQWGOw0YCOSjl}

\subsection{Demostración:}

\begin{align*}
  \int_{-\infty}^{\infty}\frac{n}{\pi}\frac{1}{1 + n^2x^2}\phi\left( x \right) dx &= \frac{1}{\pi}\int_{-\infty}^{\infty} \frac{n}{1 + n^2x^2} \\
  nx &= y \\
  n dx &= dy \\
  x &= \frac{y}{n} \\
  &= \frac{1}{\pi}\int_{-\infty}^{\infty} \frac{1}{1 + y^2} \phi\left( \frac{y}{n} \right) dy
.\end{align*}

Con esto entonces podemos probar con el limite:
\begin{align*}
  \lim_{n \to \infty} \frac{1}{\pi}\int_{-\infty}^{\infty} \frac{1}{1 + y^2}\phi\left( \frac{y}{n} \right) dy&= \lim_{n \to \infty} \frac{1}{\pi}\int_{-\infty}^{\infty} \frac{1}{1 + y^2}\phi\left( 0 \right) dy \\
  &= \frac{1}{\pi}\phi\left( 0 \right) \int_{-\infty}^{\infty}\frac{1}{1 + y^2}dy \\
  \int_{-\infty}^{\infty} \frac{1}{1 + y^2}dy &= \int_{-\infty}^{\infty} \frac{\sec^2\left( u \right) }{\tan^2\left( u \right) + 1} \\
  \int_{-\infty}^{\infty} \frac{1}{1 + y^2}dy &= \int_{-\infty}^{\infty} 1 du \\
  \int_{-\infty}^{\infty} \frac{1}{1 + y^2}dy &= \left[ u \right]_{-\infty}^{\infty} \\
  \int_{-\infty}^{\infty} \frac{1}{1 + y^2}dy &= \left[ \arctan\left( x \right)  \right]_{-\infty}^{\infty} \\
  \int_{-\infty}^{\infty} \frac{1}{1 + y^2}dy &= \pi \\
  &= \phi\left( 0 \right)
.\end{align*}

\section{$\frac{n}{\sqrt{\pi} }e^{-n^2x^2}$ }

\subsection{Animación:}
\url{https://youtu.be/gdPG-1n8PcA?si=7mDJtpv3Cvwjr_pb}

\subsection{Demostración:}

\begin{align*}
  \int_{-\infty}^{\infty} \frac{n}{\sqrt{\pi} }e^{-n^2x^2}\phi\left( x \right) dx &= \frac{1}{\sqrt{\pi} } \int_{-\infty}^{\infty} n e^{-n^2x^2}\phi\left( x \right) dx \\
  nx &= y \\
  x &= \frac{y}{n} \\
  n dx &= dy \\
  &= \frac{1}{\sqrt{\pi} }\int_{-\infty}^{\infty} e^{- y^2}\phi\left( \frac{y}{n} \right) dy
.\end{align*}

Con esto entonces podemos buscar el limite
\begin{align*}
  \lim_{n \to \infty} \frac{1}{\sqrt{\pi} }\int_{-\infty}^{\infty} e^{-y^2}\phi\left( \frac{y}{n} \right) dy&= \lim_{n \to \infty} \frac{1}{\sqrt{\pi} } \int_{-\infty}^{\infty}e^{-y^2}\phi\left( 0 \right) dy\\
  &= \frac{1}{\sqrt{\pi} }\phi\left( 0 \right) \int_{-\infty}^{\infty} e^{-y^2}dy \\
  \int_{-\infty}^{\infty} e^{-y^2}dy &= \frac{\sqrt{\pi} }{2} \int_{-\infty}^{\infty} \frac{2e^{-y^2}}{\sqrt{\pi} }dy \\
  \int_{-\infty}^{\infty} e^{-y^2}dy &= \left[ \frac{\sqrt{\pi}\erf\left( y \right)  }{2} \right]_{-\infty}^{\infty} \\
  \int_{-\infty}^{\infty} e^{-y^2}dy &= \sqrt{\pi}  \\
  &= \phi\left( 0 \right)
.\end{align*}

\section{$\frac{1}{n\pi}\frac{\sin^2\left( nx \right) }{x^2}$}

\subsection{Animación:}

\url{https://youtu.be/ljhxbuyKN64}

\subsection{Demostración:}
Para comenzar
\begin{align*}
  \int_{-\infty}^{\infty} \frac{1}{n\pi} \frac{\sin^2\left( nx \right) }{x^2} \phi\left( x \right) dx &= \int_{-\infty}^{\infty} \frac{1}{n\pi} \frac{n^2 \sin^2\left( y \right) }{y^2} \phi\left( \frac{y}{n} \right) dx \\
  nx &= y \\
  x &= \frac{y}{n} \\
  n dx &= dy \\
  &= \int_{-\infty}^{\infty} \frac{1}{\pi} \frac{\sin^2\left( y \right) }{y^2}\phi\left( \frac{y}{n} \right) dy \\
.\end{align*}

Por lo tanto,
\begin{align*}
  \lim_{n \to \infty} \int_{-\infty}^{\infty} \frac{1}{n\pi} \frac{\sin^2\left( nx \right) }{x^2}\phi\left( x \right) dx &= \lim_{n \to \infty}  \int_{-\infty}^{\infty} \frac{1}{\pi} \frac{\sin^2\left( y \right) }{y^2}\phi\left( \frac{y}{n} \right) dy \\
  &= \frac{1}{\pi} \phi\left( 0 \right) \int_{-\infty}^{\infty} \frac{\sin^2\left( y \right) }{y^2} dy \\
  \int_{-\infty}^{\infty} \frac{\sin^2\left( y \right) }{y^2} dy &= \pi \\
  &= \phi\left( 0 \right) 
.\end{align*}

\chapter{}

En este caso hacemos un desarrollo muy similar al punto anterior:
\begin{align*}
  \frac{x}{\varepsilon} &= y\\
  x &= y\varepsilon \\
  \frac{1}{\varepsilon} dx &= dy \\
  \lim_{\varepsilon \to \infty} \int_{-\infty}^{\infty} \frac{f\left( \frac{x}{\varepsilon} \right) }{\varepsilon} \phi\left( x \right) dx &= \lim_{\varepsilon \to \infty} \int_{-\infty}^{\infty}f\left( y \right) \phi\left( y\cdot \varepsilon \right) dy  \\
  &= \phi\left( 0 \right) \int_{-\infty}^{\infty} f\left( y \right) dy \\
  &= \phi\left( 0 \right) 
.\end{align*}

\chapter{}

\begin{enumerate}
  \item $x v.p. \frac{1}{x} = 1$ 
    \begin{align*}
      \left<xv.p \frac{1}{x}, \phi \right> &= \lim_{\varepsilon \to 0} \int_{\mathbb{R}\backslash \left( -\varepsilon, \varepsilon \right) } \frac{1}{x}x \phi\left( x \right) dx \\
      \left<xv.p \frac{1}{x}, \phi \right> &= \lim_{\varepsilon \to 0} \int_{\mathbb{R}\backslash \left( -\varepsilon, \varepsilon \right) } \phi\left( x \right) dx \\
      \left<xv.p \frac{1}{x}, \phi \right> &= \left<1, \phi \right>
    .\end{align*}
  \item $x Pf \frac{1}{\left| x \right| } = 2H - 1$
  \item $x\delta = 0$ 
    \begin{align*}
      \left<x\delta, phi \right> &= \int_{\mathbb{R}} \delta x\phi\left( x \right) dx \\
      &= 0\phi\left( 0 \right)  \\
      &= 0 \\
    .\end{align*}
  \item $x\delta' = -\delta$
    \begin{align*}
      \left<x\delta', \phi \right> &= -\left<\delta, \left( x\phi \right)' \right> \\
      &= - \left<\delta, x'\phi \right> - \left<\delta, x\phi' \right> \\
      &= -x'\left( 0 \right) \left<\delta, \phi \right> - x\left( 0 \right) \left<x, \phi' \right>\\
      &= x\left( 0 \right) \delta' - x'\left( 0 \right) \delta \\
      &= -\delta \\
    .\end{align*}
  \item  $x^{n}\delta^{\left( m \right) } = 0$ para $n > m$
  \item  $x^{n}\delta^{\left( n \right) } = \left( -1 \right) ^{n}n!\delta$ 
  \item $x^{n}\delta^{\left( n + 1 \right) } = \left( -1 \right) \left( n + 1 \right)! \delta'$
\end{enumerate}

\end{document}

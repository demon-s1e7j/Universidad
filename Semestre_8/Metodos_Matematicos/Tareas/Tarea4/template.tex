\documentclass{report}

\documentclass[12pt]{article}
\usepackage{array}
\usepackage{color}
\usepackage{amsthm}
\usepackage{eufrak}
\usepackage{lipsum}
\usepackage{pifont}
\usepackage{yfonts}
\usepackage{amsmath}
\usepackage{amssymb}
\usepackage{ccfonts}
\usepackage{comment} \usepackage{amsfonts}
\usepackage{fancyhdr}
\usepackage{graphicx}
\usepackage{listings}
\usepackage{mathrsfs}
\usepackage{setspace}
\usepackage{textcomp}
\usepackage{blindtext}
\usepackage{enumerate}
\usepackage{microtype}
\usepackage{xfakebold}
\usepackage{kantlipsum}
%\usepackage{draftwatermark}
\usepackage[spanish]{babel}
\usepackage[margin=1.5cm, top=2cm, bottom=2cm]{geometry}
\usepackage[framemethod=tikz]{mdframed}
\usepackage[colorlinks=true,citecolor=blue,linkcolor=red,urlcolor=magenta]{hyperref}

%//////////////////////////////////////////////////////
% Watermark configuration
%//////////////////////////////////////////////////////
%\SetWatermarkScale{4}
%\SetWatermarkColor{black}
%\SetWatermarkLightness{0.95}
%\SetWatermarkText{\texttt{Watermark}}

%//////////////////////////////////////////////////////
% Frame configuration
%//////////////////////////////////////////////////////
\newmdenv[tikzsetting={draw=gray,fill=white,fill opacity=0},backgroundcolor=none]{Frame}

%//////////////////////////////////////////////////////
% Font style configuration
%//////////////////////////////////////////////////////
\renewcommand{\familydefault}{\ttdefault}
\renewcommand{\rmdefault}{tt}

%//////////////////////////////////////////////////////
% Bold configuration
%//////////////////////////////////////////////////////
\newcommand{\fbseries}{\unskip\setBold\aftergroup\unsetBold\aftergroup\ignorespaces}
\makeatletter
\newcommand{\setBoldness}[1]{\def\fake@bold{#1}}
\makeatother

%//////////////////////////////////////////////////////
% Default font configuration
%//////////////////////////////////////////////////////
\DeclareFontFamily{\encodingdefault}{\ttdefault}{%
  \hyphenchar\font=\defaulthyphenchar
  \fontdimen2\font=0.33333em
  \fontdimen3\font=0.16667em
  \fontdimen4\font=0.11111em
  \fontdimen7\font=0.11111em}


\input{macros}
\input{letterfonts}

\title{\Huge{Métodos Matemáticos}\\Tarea 4}
\author{\huge{Sergio Montoya Ramírez}}
\date{202112171}

\begin{document}

\maketitle
\newpage% or \cleardoublepage
% \pdfbookmark[<level>]{<title>}{<dest>}
\pdfbookmark[section]{\contentsname}{toc}
\tableofcontents
\pagebreak

\chapter{Arfken: 19.1.1}

En este caso tenemos la definición:
\begin{align*}
  \Delta p &= \int_{0}^{2\pi}\left[ f\left( x \right) - \frac{a_0}{2} - \sum_{n=1}^{p} \left( a_n \cos\left(nx \right) + b_n \sin\left( nx \right) \right)  \right]^2 dx 
.\end{align*}

Por lo tanto:
\begin{align*}
  0 = \frac{\partial \Delta p}{\partial a_n}  &= -2 \int_{0}^{2\pi} \left[ f\left( x \right) - \frac{a_0}{2} - \sum_{n=1}^{\infty} \left( a_n \cos\left( nx \right) + b_n \sin\left( nx \right)  \right)  \right] \cos\left( nx \right) dx\\
  &= -2 \int_{0}^{2\pi} f\left( x \right) \cos\left( nx \right) dx + 2\pi a_n
.\end{align*}

Por otro lado:
\begin{align*}
  0 = \frac{\partial \Delta p}{\partial b_n} &= -2 \int_{0}^{2\pi}\left[ f\left( x \right) - \frac{a_0}{2} - \sum_{n=1}^{\infty} \left( a_n \cos\left( nx \right) + b_n \sin\left( nx \right)  \right)  \right] \sin\left( nx \right) dx \\
  &= -2 \int_{0}^{2\pi} f\left( x \right) \sin\left( nx \right) dx + 2\pi b_n
.\end{align*}

\chapter{Arfken: 19.1.5}

Tenemos: \[
a_n = \frac{1}{\pi}\int_{0}^{2\pi}f\left( s \right) \cos\left( ns \right) ds
.\] 

Lo cual implica que
\begin{align*}
  a_n &= \frac{1}{\pi}\int_{0}^{2\pi} \frac{\sin\left( nx \right) \cos\left( nx \right) }{n} dx \\
  a_n &= 0;\ n \ge 0
.\end{align*}

Por otro lado:
\begin{align*}
  b_n &= \frac{1}{\pi}\int_{0}^{2\pi}f\left( s \right) \sin\left( ns \right) ds \\
  &= \frac{1}{2\pi}\left[ \int_{0}^{\pi} \left( \pi - x \right) \sin\left( nx \right) dx - \int_{-\pi}^{0}\left( \pi + x \right) \sin\left( nx \right) dx \right]  \\
  &= \frac{1}{2\pi}\left[ - \frac{\pi}{n} \cos\left( nx \right) + \frac{x}{n}\cos\left( nx \right)  \right]_{0}^{\pi} - \frac{1}{2\pi n}\int_{0}^{\pi}\cos\left( nx \right) dx \\ 
  &- \frac{1}{2\pi n}\int_{-\pi}^{0} \cos\left( nx \right) dx - \frac{1}{2\pi}\left[ - \frac{\pi}{n}\cos\left( nx \right) - \frac{x}{n}\cos\left( nx \right)  \right]_{-\pi}^{0}\\
  &= \frac{1}{n} - \left. \frac{\sin\left( nx \right) }{2\pi n^2}\right|_{0}^{\pi} - \left. \frac{\sin\left( nx \right) }{2\pi n^2}\right|_{-\pi}^{0} = \frac{1}{n}
.\end{align*}

\chapter{Arfken: 19.1.10}

En este caso tenemos: \[
f\left( x \right) = \begin{cases}
  4x\left( 1 - x \right) & 0 \le x \le 1 \\
  4x(1 + x) & -1 \le x \le 0
\end{cases} = \sum_{n=1}^{\infty} b_n \sin\left( n\pi x \right) 
.\] 

Ahora necesitamos encontrar $b_n$

\begin{align*}
  b_n &= \int_{-1}^{1}f\left( x \right) \sin\left( nx\pi \right) dx;\ n = 1,2,\ldots \\
  &= \frac{1}{\pi}\left[ \int_{-1}^{0} 4x\left( 1 + x \right) \sin\left( n\pi x \right) dx + \int_{0}^{1} 4x\left( 1 - x \right) \sin\left( n\pi x \right) dx \right]  \\
  &= \frac{1}{\pi}\left[ \int_{-1}^{0}\left( 4x + 4x^2 \right) \sin\left( n\pi x \right) dx + \int_{0}^{1}\left( 4x - 4x^2 \right) \sin\left( n\pi x \right) dx \right] \\
  &= \frac{1}{\pi}\left[ \int_{-1}^{0} 4x \sin\left( n\pi x \right) dx + \int_{-1}^{0} 4x^2\sin\left( n\pi x \right) dx \right.\\
  &\left.+\int_{0}^{1} 4x\sin\left( n\pi x \right) dx - \int_{0}^{1} 4x^2 \sin\left( n\pi x \right) dx\right]
.\end{align*}

Con esto ya podemos simplemente notar que con $n$ par tendríamos que su valor es 0 pues estas funciones serian pares. Por otro lado, quitando $x^2\sin\left( n \pi x \right) $ por ser pares queda entonces para $n$ impar queda entonces
\begin{align*}
  b_n &= 8 \left( \frac{2}{n\pi} - \frac{6}{n^{3}\pi^{3}} \right) \\
  &= \frac{32}{n^{3}\pi^{3}}
.\end{align*}


\chapter{Arfken: 19.2.17}

\section{Parte a}

En este caso necesitamos $b_n$ lo cual es:
\begin{align*}
  b_n = \frac{2}{L}\int_0^{L} \delta\left( x - a \right) \sin\left( \frac{n\pi x}{L} \right) dx
.\end{align*}

Lo cual por la definición de $\delta$ nos queda como  \[
b_n = \frac{2}{L}\sin\left( \frac{n \pi a}{L} \right)
.\] Ahora, reemplazando en la definición de serie seno de fourier nos queda:
\begin{align*}
  f(x) &= \sum_{n=1}^{\infty} b_n \sin\left( \frac{n\pi x}{L} \right)  \\
  &= \sum_{n=1}^{\infty} \frac{2}{L}\sin\left( \frac{n\pi a}{L} \right) \sin\left( \frac{n \pi x}{L} \right)  \\
  &= \frac{2}{L}\sum_{n=1}^{\infty} \sin\left( \frac{n\pi a}{L} \right) \sin\left( \frac{n \pi x}{L} \right)
.\end{align*}

\section{Parte b}

En este caso al hacer la integral tenemos:
\begin{align*}
  \delta \left( x - a \right)  &= \sum_{n=1}^{\infty} \sin\left( \frac{n\pi a}{L} \right) \sin\left( \frac{n \pi x}{L} \right) \\
  \int_{0}^{x} \delta\left( x - a \right) dx &= \int_{0}^{x} \sum_{n=1}^{\infty} \sin\left( \frac{n\pi a}{L} \right) \sin\left( \frac{n \pi x}{L} \right) dx\\
  \begin{cases}
   0& 0 \le  x < a\\
   1& a < x < L
  \end{cases} &= \sum_{n=1}^{\infty} \sin\left( \frac{n\pi a}{L} \right) \int_{0}^{x} \sin\left( \frac{n\pi x}{L} \right)dx   \\
  &= \frac{2}{L}\sum_{n=1}^{\infty} \sin\left( \frac{n\pi a}{L} \right) \left( \frac{L}{n\pi} \right) \left[ 1 - \cos\left( \frac{n\pi x}{L} \right)  \right]  \\
  &= \frac{2}{\pi}\sum_{n=1}^{\infty} \frac{1}{n}\sin\left( \frac{n\pi a}{L} \right) - \frac{2}{\pi}\sum_{n=1}^{\infty}\frac{1}{n} \sin\left( \frac{n\pi a}{L} \right) \cos\left( \frac{n \pi x}{L} \right)
.\end{align*}

\chapter{Arfken: 19.2.20}

Ahora queda:
\begin{align*}
  \sum_{n=1}^{\infty} \frac{d^2 b_n\left( t \right) }{dt^2} \sin\left( \frac{n \pi x}{L} \right) &= v^2 \sum_{n=1}^{\infty} b_n\left( t \right) \left( \frac{n^2\pi^2}{L^2} \right) \sin\left( \frac{n\pi x}{L} \right) \\
  \frac{d^2 b_n\left( t \right) }{dt^2} + v^2 \frac{n^2\pi^2}{L^2}b_n\left( t \right) &= 0 \\
  b_n\left( t \right) &= A_n \cos\left( \frac{n\pi v}{L}t \right) + B_n \sin\left( \frac{n\pi v}{L}t \right)  \\
  u\left( x, 0 \right) &= \sum_{n=1}^{\infty} b_n\left( 0 \right) \sin\left( \frac{n\pi x}{L} \right) = f\left( x \right)  \\
  b_n\left( 0 \right) &= A_n \\
  A_n &= \frac{2}{L}\int_{0}^{L} f\left( x \right) \sin\left( \frac{n\pi x}{L} \right) dx \\
  \frac{\partial u\left( x, t \right) }{\partial t} &= \sum_{n=1}^{\infty} \left( -A_n \frac{n\pi v}{L}\sin\left( \frac{n \pi v}{L} t \right) + B_n \frac{n\pi v}{L}\cos\left( \frac{n\pi v}{L}t \right)  \right)\sin\left( \frac{n \pi x}{L} \right) \\
  &\text{Con }t = 0 \\
  \sum_{n=1}^{\infty} B_n \frac{n \pi v}{L}\sin\left( \frac{n \pi x}{L} \right) &= g\left( x \right)  \\
  B_n &= \frac{2}{n \pi v}\int_{0}^{L}g\left( x \right) \sin\left(\frac{n \pi x}{L} \right)dx
.\end{align*}

Con esto entonces ya encontramos cuanto es $b_n$ y por lo tanto llegamos a la respuesta correcta.

\chapter{Tellez: 3.5.1}

Para iniciar sacamos los valores de $a_n$

\begin{align*}
  a_0 &= \frac{1}{\pi}\int_{0}^{\pi}\sin\left( x \right) dx \\
  &= \frac{1}{\pi}\left[ -\cos\left( x \right)  \right]_{0}^{\pi} \\
  &= \frac{2}{\pi}.
\end{align*}
\begin{align*}
  a_n &= \frac{1}{\pi}\int_{0}^{\pi}\sin\left( x \right) \cos\left( nx \right) dx \\
  &= \frac{1}{2\pi} \int_{0}^{\pi}\left( \sin\left( 1 + n \right) x + \sin\left( 1 - \frac{n}{x} \right)  \right) dx \\
  &= -\frac{1}{2\pi}\left[ \frac{\cos\left( 1 + n \right) x}{1 + n} + \frac{\cos\left( 1 - n \right) x}{1 - n} \right]_{0}^{\pi} \\
  &= - \frac{1}{2\pi}\left[ \frac{\left( -1 \right)^{1 + n}}{1 + n} + \frac{\left( -1 \right)^{1 - n}}{1 - n} - \left( \frac{1}{1 + m} + \frac{1}{1 - m} \right)  \right]  \\
  &= \frac{1}{\pi}\frac{\left( -1 \right)^{n} + 1}{\left( 1 - n^2 \right) }\ \left( n \neq 1 \right).
\end{align*}

\begin{align*}
  a_1 &= \frac{1}{\pi}\int_{0}^{\pi}\sin\left( x \right) \cos\left( x \right) dx \\
  &= \frac{1}{2\pi}\int_{0}^{\pi} \sin\left( 2\pi \right) dx \\
  &= \frac{1}{2\pi}\left( \frac{1 - \cos\left( 2x \right) }{2} \right)_{0}^{\pi} \\
  &= 0
.\end{align*}

Ahora sacamos los valores de $b_n$
\begin{align*}
  b_n &= \frac{1}{\pi}\int_{0}^{\pi}\sin\left( x \right) \sin\left( nx \right) dx \\
  &= \frac{1}{2\pi} \int_{0}^{\pi}\left( \cos\left( n - 1 \right) x - \cos\left( n + 1 \right) x \right) dx \\
  &= \frac{1}{2\pi}\left[ \frac{\sin\left( n - 1 \right) x}{n - 1} - \frac{\sin\left( n + 1 \right) x}{n + 1} \right]_{0}^{\pi} \\
  &= 0\ \left( n \neq 1 \right)
.\end{align*}
\begin{align*}
  b_1 &= \frac{1}{\pi}\int_{0}^{\pi}\sin\left( x \right) \sin\left( x \right) dx \\
  &= \frac{1}{\pi}\int_{0}^{\pi} \frac{1 - \cos\left( 2x \right) }{2} \\
  &= \frac{1}{2\pi}\left[ x - \frac{\sin\left( 2x \right) }{2} \right]_{0}^{\pi} \\
  &= \frac{1}{2\pi} + \pi \\
  &= \frac{1}{2}
.\end{align*}

Ahora juntando todo:

\begin{align*}
  f\left( x \right) &= \frac{1}{\pi} + \frac{1}{\pi}\sum_{n=2}^{\infty} \frac{1}{\pi}\frac{\left( -1 \right)^{n} + 1}{\left( 1 - n^2 \right) } \frac{1}{2} \sin\left( x \right)
.\end{align*}

\chapter{Tellez: 3.5.5}

En este caso asuma $u\left( x, t \right) = X\left( t \right)T\left( t \right)  $

Ahora poniendo esto en la ecuación queda:
\begin{align*}
  \ddot{X}\left( x \right) T\left( t \right) &= \frac{1}{c^2}X\left( x \right) \ddot{T}\left( t \right)  \\
  \frac{\ddot{X}\left( x \right) T\left( t \right)}{X\left( x \right) T\left( t \right) } &= \frac{1}{c^2} \frac{X\left( x \right) \ddot{T}\left( t \right) }{X\left( x \right) T\left( t \right) } \\
  \frac{\ddot{X}\left( x \right) }{X\left( x \right) } &= \frac{1}{c^2} \frac{\ddot{T}\left( t \right) }{T\left( t \right) } = -\lambda
.\end{align*}

Con esto entonces encontramos dos ecuaciones:
\begin{align*}
  \ddot{X}\left( x \right) + \lambda X\left( x \right) &= 0 \\
  \ddot{T}\left( t \right) + \lambda c^2 T\left( t \right) &= 0
.\end{align*}

Ahora resolviendo para $X\left( x \right) $ tenemos que usar $X\left( 0 \right) = 0$ y $X\left( L \right) = 0$ y teniendo \[
X\left( x \right) = A\sin\left( \sqrt{\lambda} x \right) + B \cos\left( \sqrt{\lambda} x \right)
.\] Con lo cual
\begin{align*}
  X\left( 0 \right) &= A\sin\left( 0 \right) + B \cos\left( 0 \right)  \\
  0 &= B
.\end{align*}

Por lo tanto $X\left( x \right) = A \sin\left( \sqrt{\lambda} x \right) $ con lo que queda
\begin{align*}
  X\left( L \right)  &= A \sin\left( \sqrt{\lambda} x \right)  \\
  0 &= \sin\left( \sqrt{\lambda} x \right)  \\
  \sqrt{\lambda} L &= n\pi \\
  \lambda &= \frac{n^2\pi^2}{L^2}
.\end{align*}

Con lo que queda
\begin{align*}
  X_n\left( x \right) &= A_n \sin\left( \frac{n\pi x}{L} \right) 
.\end{align*}

Y ahora probando con $T\left( t \right) $ queda \[
  \ddot{T}\left( t \right) + \frac{n^2\pi^2c^2}{L^2}T\left( t \right) = 0
.\] Con la solución general \[
T_n\left( t \right) = B_n \cos\left( \frac{n\pi c}{L}t \right) + C_n \sin\left( \frac{n\pi c}{L}t \right) 
.\] y tomando $\frac{\partial u}{\partial t} \left( x, 0 \right) = 0$ entonces
\begin{align*}
  T_n\left( t \right) &= B_n \cos\left( \frac{n\pi c}{L}t \right) + C_n \sin\left( \frac{n\pi c}{L}t \right) \\
\dot{T}_n\left( 0 \right) &= - B_n \sin\left( 0 \right) + C_n \cos\left( 0 \right)  \\
0 &= C_n
.\end{align*}

Con lo cual queda \[
u\left( x, t \right) = \sum_{n=1}^{\infty} A_n \sin\left( \frac{n\pi x}{L} \right) \cos\left( \frac{n \pi c}{L}t \right) 
.\] 

\end{document}

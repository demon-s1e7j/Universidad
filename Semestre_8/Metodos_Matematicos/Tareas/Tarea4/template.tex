\documentclass{report}

\documentclass[12pt]{article}
\usepackage{array}
\usepackage{color}
\usepackage{amsthm}
\usepackage{eufrak}
\usepackage{lipsum}
\usepackage{pifont}
\usepackage{yfonts}
\usepackage{amsmath}
\usepackage{amssymb}
\usepackage{ccfonts}
\usepackage{comment} \usepackage{amsfonts}
\usepackage{fancyhdr}
\usepackage{graphicx}
\usepackage{listings}
\usepackage{mathrsfs}
\usepackage{setspace}
\usepackage{textcomp}
\usepackage{blindtext}
\usepackage{enumerate}
\usepackage{microtype}
\usepackage{xfakebold}
\usepackage{kantlipsum}
%\usepackage{draftwatermark}
\usepackage[spanish]{babel}
\usepackage[margin=1.5cm, top=2cm, bottom=2cm]{geometry}
\usepackage[framemethod=tikz]{mdframed}
\usepackage[colorlinks=true,citecolor=blue,linkcolor=red,urlcolor=magenta]{hyperref}

%//////////////////////////////////////////////////////
% Watermark configuration
%//////////////////////////////////////////////////////
%\SetWatermarkScale{4}
%\SetWatermarkColor{black}
%\SetWatermarkLightness{0.95}
%\SetWatermarkText{\texttt{Watermark}}

%//////////////////////////////////////////////////////
% Frame configuration
%//////////////////////////////////////////////////////
\newmdenv[tikzsetting={draw=gray,fill=white,fill opacity=0},backgroundcolor=none]{Frame}

%//////////////////////////////////////////////////////
% Font style configuration
%//////////////////////////////////////////////////////
\renewcommand{\familydefault}{\ttdefault}
\renewcommand{\rmdefault}{tt}

%//////////////////////////////////////////////////////
% Bold configuration
%//////////////////////////////////////////////////////
\newcommand{\fbseries}{\unskip\setBold\aftergroup\unsetBold\aftergroup\ignorespaces}
\makeatletter
\newcommand{\setBoldness}[1]{\def\fake@bold{#1}}
\makeatother

%//////////////////////////////////////////////////////
% Default font configuration
%//////////////////////////////////////////////////////
\DeclareFontFamily{\encodingdefault}{\ttdefault}{%
  \hyphenchar\font=\defaulthyphenchar
  \fontdimen2\font=0.33333em
  \fontdimen3\font=0.16667em
  \fontdimen4\font=0.11111em
  \fontdimen7\font=0.11111em}


%From M275 "Topology" at SJSU
\newcommand{\id}{\mathrm{id}}
\newcommand{\taking}[1]{\xrightarrow{#1}}
\newcommand{\inv}{^{-1}}

%From M170 "Introduction to Graph Theory" at SJSU
\DeclareMathOperator{\diam}{diam}
\DeclareMathOperator{\ord}{ord}
\newcommand{\defeq}{\overset{\mathrm{def}}{=}}

%From the USAMO .tex files
\newcommand{\ts}{\textsuperscript}
\newcommand{\dg}{^\circ}
\newcommand{\ii}{\item}

% % From Math 55 and Math 145 at Harvard
% \newenvironment{subproof}[1][Proof]{%
% \begin{proof}[#1] \renewcommand{\qedsymbol}{$\blacksquare$}}%
% {\end{proof}}

\newcommand{\liff}{\leftrightarrow}
\newcommand{\lthen}{\rightarrow}
\newcommand{\opname}{\operatorname}
\newcommand{\surjto}{\twoheadrightarrow}
\newcommand{\injto}{\hookrightarrow}
\newcommand{\On}{\mathrm{On}} % ordinals
\DeclareMathOperator{\img}{im} % Image
\DeclareMathOperator{\Img}{Im} % Image
\DeclareMathOperator{\coker}{coker} % Cokernel
\DeclareMathOperator{\Coker}{Coker} % Cokernel
\DeclareMathOperator{\Ker}{Ker} % Kernel
\DeclareMathOperator{\rank}{rank}
\DeclareMathOperator{\Spec}{Spec} % spectrum
\DeclareMathOperator{\Tr}{Tr} % trace
\DeclareMathOperator{\pr}{pr} % projection
\DeclareMathOperator{\ext}{ext} % extension
\DeclareMathOperator{\pred}{pred} % predecessor
\DeclareMathOperator{\dom}{dom} % domain
\DeclareMathOperator{\ran}{ran} % range
\DeclareMathOperator{\Hom}{Hom} % homomorphism
\DeclareMathOperator{\Mor}{Mor} % morphisms
\DeclareMathOperator{\End}{End} % endomorphism

\newcommand{\eps}{\epsilon}
\newcommand{\veps}{\varepsilon}
\newcommand{\ol}{\overline}
\newcommand{\ul}{\underline}
\newcommand{\wt}{\widetilde}
\newcommand{\wh}{\widehat}
\newcommand{\vocab}[1]{\textbf{\color{blue} #1}}
\providecommand{\half}{\frac{1}{2}}
\newcommand{\dang}{\measuredangle} %% Directed angle
\newcommand{\ray}[1]{\overrightarrow{#1}}
\newcommand{\seg}[1]{\overline{#1}}
\newcommand{\arc}[1]{\wideparen{#1}}
\DeclareMathOperator{\cis}{cis}
\DeclareMathOperator*{\lcm}{lcm}
\DeclareMathOperator*{\argmin}{arg min}
\DeclareMathOperator*{\argmax}{arg max}
\newcommand{\cycsum}{\sum_{\mathrm{cyc}}}
\newcommand{\symsum}{\sum_{\mathrm{sym}}}
\newcommand{\cycprod}{\prod_{\mathrm{cyc}}}
\newcommand{\symprod}{\prod_{\mathrm{sym}}}
\newcommand{\Qed}{\begin{flushright}\qed\end{flushright}}
\newcommand{\parinn}{\setlength{\parindent}{1cm}}
\newcommand{\parinf}{\setlength{\parindent}{0cm}}
% \newcommand{\norm}{\|\cdot\|}
\newcommand{\inorm}{\norm_{\infty}}
\newcommand{\opensets}{\{V_{\alpha}\}_{\alpha\in I}}
\newcommand{\oset}{V_{\alpha}}
\newcommand{\opset}[1]{V_{\alpha_{#1}}}
\newcommand{\lub}{\text{lub}}
\newcommand{\del}[2]{\frac{\partial #1}{\partial #2}}
\newcommand{\Del}[3]{\frac{\partial^{#1} #2}{\partial^{#1} #3}}
\newcommand{\deld}[2]{\dfrac{\partial #1}{\partial #2}}
\newcommand{\Deld}[3]{\dfrac{\partial^{#1} #2}{\partial^{#1} #3}}
\newcommand{\lm}{\lambda}
\newcommand{\uin}{\mathbin{\rotatebox[origin=c]{90}{$\in$}}}
\newcommand{\usubset}{\mathbin{\rotatebox[origin=c]{90}{$\subset$}}}
\newcommand{\lt}{\left}
\newcommand{\rt}{\right}
\newcommand{\paren}[1]{\left(#1\right)}
\newcommand{\bs}[1]{\boldsymbol{#1}}
\newcommand{\exs}{\exists}
\newcommand{\st}{\strut}
\newcommand{\dps}[1]{\displaystyle{#1}}

\newcommand{\sol}{\setlength{\parindent}{0cm}\textbf{\textit{Solution:}}\setlength{\parindent}{1cm} }
\newcommand{\solve}[1]{\setlength{\parindent}{0cm}\textbf{\textit{Solution: }}\setlength{\parindent}{1cm}#1 \Qed}

% Things Lie
\newcommand{\kb}{\mathfrak b}
\newcommand{\kg}{\mathfrak g}
\newcommand{\kh}{\mathfrak h}
\newcommand{\kn}{\mathfrak n}
\newcommand{\ku}{\mathfrak u}
\newcommand{\kz}{\mathfrak z}
\DeclareMathOperator{\Ext}{Ext} % Ext functor
\DeclareMathOperator{\Tor}{Tor} % Tor functor
\newcommand{\gl}{\opname{\mathfrak{gl}}} % frak gl group
\renewcommand{\sl}{\opname{\mathfrak{sl}}} % frak sl group chktex 6

% More script letters etc.
\newcommand{\SA}{\mathcal A}
\newcommand{\SB}{\mathcal B}
\newcommand{\SC}{\mathcal C}
\newcommand{\SF}{\mathcal F}
\newcommand{\SG}{\mathcal G}
\newcommand{\SH}{\mathcal H}
\newcommand{\OO}{\mathcal O}

\newcommand{\SCA}{\mathscr A}
\newcommand{\SCB}{\mathscr B}
\newcommand{\SCC}{\mathscr C}
\newcommand{\SCD}{\mathscr D}
\newcommand{\SCE}{\mathscr E}
\newcommand{\SCF}{\mathscr F}
\newcommand{\SCG}{\mathscr G}
\newcommand{\SCH}{\mathscr H}

% Mathfrak primes
\newcommand{\km}{\mathfrak m}
\newcommand{\kp}{\mathfrak p}
\newcommand{\kq}{\mathfrak q}

% number sets
\newcommand{\RR}[1][]{\ensuremath{\ifstrempty{#1}{\mathbb{R}}{\mathbb{R}^{#1}}}}
\newcommand{\NN}[1][]{\ensuremath{\ifstrempty{#1}{\mathbb{N}}{\mathbb{N}^{#1}}}}
\newcommand{\ZZ}[1][]{\ensuremath{\ifstrempty{#1}{\mathbb{Z}}{\mathbb{Z}^{#1}}}}
\newcommand{\QQ}[1][]{\ensuremath{\ifstrempty{#1}{\mathbb{Q}}{\mathbb{Q}^{#1}}}}
\newcommand{\CC}[1][]{\ensuremath{\ifstrempty{#1}{\mathbb{C}}{\mathbb{C}^{#1}}}}
\newcommand{\PP}[1][]{\ensuremath{\ifstrempty{#1}{\mathbb{P}}{\mathbb{P}^{#1}}}}
\newcommand{\HH}[1][]{\ensuremath{\ifstrempty{#1}{\mathbb{H}}{\mathbb{H}^{#1}}}}
\newcommand{\FF}[1][]{\ensuremath{\ifstrempty{#1}{\mathbb{F}}{\mathbb{F}^{#1}}}}
% expected value
\newcommand{\EE}{\ensuremath{\mathbb{E}}}
\newcommand{\charin}{\text{ char }}
\DeclareMathOperator{\sign}{sign}
\DeclareMathOperator{\Aut}{Aut}
\DeclareMathOperator{\Inn}{Inn}
\DeclareMathOperator{\Syl}{Syl}
\DeclareMathOperator{\Gal}{Gal}
\DeclareMathOperator{\GL}{GL} % General linear group
\DeclareMathOperator{\SL}{SL} % Special linear group

%---------------------------------------
% BlackBoard Math Fonts :-
%---------------------------------------

%Captital Letters
\newcommand{\bbA}{\mathbb{A}}	\newcommand{\bbB}{\mathbb{B}}
\newcommand{\bbC}{\mathbb{C}}	\newcommand{\bbD}{\mathbb{D}}
\newcommand{\bbE}{\mathbb{E}}	\newcommand{\bbF}{\mathbb{F}}
\newcommand{\bbG}{\mathbb{G}}	\newcommand{\bbH}{\mathbb{H}}
\newcommand{\bbI}{\mathbb{I}}	\newcommand{\bbJ}{\mathbb{J}}
\newcommand{\bbK}{\mathbb{K}}	\newcommand{\bbL}{\mathbb{L}}
\newcommand{\bbM}{\mathbb{M}}	\newcommand{\bbN}{\mathbb{N}}
\newcommand{\bbO}{\mathbb{O}}	\newcommand{\bbP}{\mathbb{P}}
\newcommand{\bbQ}{\mathbb{Q}}	\newcommand{\bbR}{\mathbb{R}}
\newcommand{\bbS}{\mathbb{S}}	\newcommand{\bbT}{\mathbb{T}}
\newcommand{\bbU}{\mathbb{U}}	\newcommand{\bbV}{\mathbb{V}}
\newcommand{\bbW}{\mathbb{W}}	\newcommand{\bbX}{\mathbb{X}}
\newcommand{\bbY}{\mathbb{Y}}	\newcommand{\bbZ}{\mathbb{Z}}

%---------------------------------------
% MathCal Fonts :-
%---------------------------------------

%Captital Letters
\newcommand{\mcA}{\mathcal{A}}	\newcommand{\mcB}{\mathcal{B}}
\newcommand{\mcC}{\mathcal{C}}	\newcommand{\mcD}{\mathcal{D}}
\newcommand{\mcE}{\mathcal{E}}	\newcommand{\mcF}{\mathcal{F}}
\newcommand{\mcG}{\mathcal{G}}	\newcommand{\mcH}{\mathcal{H}}
\newcommand{\mcI}{\mathcal{I}}	\newcommand{\mcJ}{\mathcal{J}}
\newcommand{\mcK}{\mathcal{K}}	\newcommand{\mcL}{\mathcal{L}}
\newcommand{\mcM}{\mathcal{M}}	\newcommand{\mcN}{\mathcal{N}}
\newcommand{\mcO}{\mathcal{O}}	\newcommand{\mcP}{\mathcal{P}}
\newcommand{\mcQ}{\mathcal{Q}}	\newcommand{\mcR}{\mathcal{R}}
\newcommand{\mcS}{\mathcal{S}}	\newcommand{\mcT}{\mathcal{T}}
\newcommand{\mcU}{\mathcal{U}}	\newcommand{\mcV}{\mathcal{V}}
\newcommand{\mcW}{\mathcal{W}}	\newcommand{\mcX}{\mathcal{X}}
\newcommand{\mcY}{\mathcal{Y}}	\newcommand{\mcZ}{\mathcal{Z}}


%---------------------------------------
% Bold Math Fonts :-
%---------------------------------------

%Captital Letters
\newcommand{\bmA}{\boldsymbol{A}}	\newcommand{\bmB}{\boldsymbol{B}}
\newcommand{\bmC}{\boldsymbol{C}}	\newcommand{\bmD}{\boldsymbol{D}}
\newcommand{\bmE}{\boldsymbol{E}}	\newcommand{\bmF}{\boldsymbol{F}}
\newcommand{\bmG}{\boldsymbol{G}}	\newcommand{\bmH}{\boldsymbol{H}}
\newcommand{\bmI}{\boldsymbol{I}}	\newcommand{\bmJ}{\boldsymbol{J}}
\newcommand{\bmK}{\boldsymbol{K}}	\newcommand{\bmL}{\boldsymbol{L}}
\newcommand{\bmM}{\boldsymbol{M}}	\newcommand{\bmN}{\boldsymbol{N}}
\newcommand{\bmO}{\boldsymbol{O}}	\newcommand{\bmP}{\boldsymbol{P}}
\newcommand{\bmQ}{\boldsymbol{Q}}	\newcommand{\bmR}{\boldsymbol{R}}
\newcommand{\bmS}{\boldsymbol{S}}	\newcommand{\bmT}{\boldsymbol{T}}
\newcommand{\bmU}{\boldsymbol{U}}	\newcommand{\bmV}{\boldsymbol{V}}
\newcommand{\bmW}{\boldsymbol{W}}	\newcommand{\bmX}{\boldsymbol{X}}
\newcommand{\bmY}{\boldsymbol{Y}}	\newcommand{\bmZ}{\boldsymbol{Z}}
%Small Letters
\newcommand{\bma}{\boldsymbol{a}}	\newcommand{\bmb}{\boldsymbol{b}}
\newcommand{\bmc}{\boldsymbol{c}}	\newcommand{\bmd}{\boldsymbol{d}}
\newcommand{\bme}{\boldsymbol{e}}	\newcommand{\bmf}{\boldsymbol{f}}
\newcommand{\bmg}{\boldsymbol{g}}	\newcommand{\bmh}{\boldsymbol{h}}
\newcommand{\bmi}{\boldsymbol{i}}	\newcommand{\bmj}{\boldsymbol{j}}
\newcommand{\bmk}{\boldsymbol{k}}	\newcommand{\bml}{\boldsymbol{l}}
\newcommand{\bmm}{\boldsymbol{m}}	\newcommand{\bmn}{\boldsymbol{n}}
\newcommand{\bmo}{\boldsymbol{o}}	\newcommand{\bmp}{\boldsymbol{p}}
\newcommand{\bmq}{\boldsymbol{q}}	\newcommand{\bmr}{\boldsymbol{r}}
\newcommand{\bms}{\boldsymbol{s}}	\newcommand{\bmt}{\boldsymbol{t}}
\newcommand{\bmu}{\boldsymbol{u}}	\newcommand{\bmv}{\boldsymbol{v}}
\newcommand{\bmw}{\boldsymbol{w}}	\newcommand{\bmx}{\boldsymbol{x}}
\newcommand{\bmy}{\boldsymbol{y}}	\newcommand{\bmz}{\boldsymbol{z}}

%---------------------------------------
% Scr Math Fonts :-
%---------------------------------------

\newcommand{\sA}{{\mathscr{A}}}   \newcommand{\sB}{{\mathscr{B}}}
\newcommand{\sC}{{\mathscr{C}}}   \newcommand{\sD}{{\mathscr{D}}}
\newcommand{\sE}{{\mathscr{E}}}   \newcommand{\sF}{{\mathscr{F}}}
\newcommand{\sG}{{\mathscr{G}}}   \newcommand{\sH}{{\mathscr{H}}}
\newcommand{\sI}{{\mathscr{I}}}   \newcommand{\sJ}{{\mathscr{J}}}
\newcommand{\sK}{{\mathscr{K}}}   \newcommand{\sL}{{\mathscr{L}}}
\newcommand{\sM}{{\mathscr{M}}}   \newcommand{\sN}{{\mathscr{N}}}
\newcommand{\sO}{{\mathscr{O}}}   \newcommand{\sP}{{\mathscr{P}}}
\newcommand{\sQ}{{\mathscr{Q}}}   \newcommand{\sR}{{\mathscr{R}}}
\newcommand{\sS}{{\mathscr{S}}}   \newcommand{\sT}{{\mathscr{T}}}
\newcommand{\sU}{{\mathscr{U}}}   \newcommand{\sV}{{\mathscr{V}}}
\newcommand{\sW}{{\mathscr{W}}}   \newcommand{\sX}{{\mathscr{X}}}
\newcommand{\sY}{{\mathscr{Y}}}   \newcommand{\sZ}{{\mathscr{Z}}}


%---------------------------------------
% Math Fraktur Font
%---------------------------------------

%Captital Letters
\newcommand{\mfA}{\mathfrak{A}}	\newcommand{\mfB}{\mathfrak{B}}
\newcommand{\mfC}{\mathfrak{C}}	\newcommand{\mfD}{\mathfrak{D}}
\newcommand{\mfE}{\mathfrak{E}}	\newcommand{\mfF}{\mathfrak{F}}
\newcommand{\mfG}{\mathfrak{G}}	\newcommand{\mfH}{\mathfrak{H}}
\newcommand{\mfI}{\mathfrak{I}}	\newcommand{\mfJ}{\mathfrak{J}}
\newcommand{\mfK}{\mathfrak{K}}	\newcommand{\mfL}{\mathfrak{L}}
\newcommand{\mfM}{\mathfrak{M}}	\newcommand{\mfN}{\mathfrak{N}}
\newcommand{\mfO}{\mathfrak{O}}	\newcommand{\mfP}{\mathfrak{P}}
\newcommand{\mfQ}{\mathfrak{Q}}	\newcommand{\mfR}{\mathfrak{R}}
\newcommand{\mfS}{\mathfrak{S}}	\newcommand{\mfT}{\mathfrak{T}}
\newcommand{\mfU}{\mathfrak{U}}	\newcommand{\mfV}{\mathfrak{V}}
\newcommand{\mfW}{\mathfrak{W}}	\newcommand{\mfX}{\mathfrak{X}}
\newcommand{\mfY}{\mathfrak{Y}}	\newcommand{\mfZ}{\mathfrak{Z}}
%Small Letters
\newcommand{\mfa}{\mathfrak{a}}	\newcommand{\mfb}{\mathfrak{b}}
\newcommand{\mfc}{\mathfrak{c}}	\newcommand{\mfd}{\mathfrak{d}}
\newcommand{\mfe}{\mathfrak{e}}	\newcommand{\mff}{\mathfrak{f}}
\newcommand{\mfg}{\mathfrak{g}}	\newcommand{\mfh}{\mathfrak{h}}
\newcommand{\mfi}{\mathfrak{i}}	\newcommand{\mfj}{\mathfrak{j}}
\newcommand{\mfk}{\mathfrak{k}}	\newcommand{\mfl}{\mathfrak{l}}
\newcommand{\mfm}{\mathfrak{m}}	\newcommand{\mfn}{\mathfrak{n}}
\newcommand{\mfo}{\mathfrak{o}}	\newcommand{\mfp}{\mathfrak{p}}
\newcommand{\mfq}{\mathfrak{q}}	\newcommand{\mfr}{\mathfrak{r}}
\newcommand{\mfs}{\mathfrak{s}}	\newcommand{\mft}{\mathfrak{t}}
\newcommand{\mfu}{\mathfrak{u}}	\newcommand{\mfv}{\mathfrak{v}}
\newcommand{\mfw}{\mathfrak{w}}	\newcommand{\mfx}{\mathfrak{x}}
\newcommand{\mfy}{\mathfrak{y}}	\newcommand{\mfz}{\mathfrak{z}}


\title{\Huge{Métodos Matemáticos}\\Tarea 4}
\author{\huge{Sergio Montoya Ramírez}}
\date{202112171}

\begin{document}

\maketitle
\newpage% or \cleardoublepage
% \pdfbookmark[<level>]{<title>}{<dest>}
\pdfbookmark[section]{\contentsname}{toc}
\tableofcontents
\pagebreak

\chapter{Arfken: 19.1.1}

En este caso tenemos la definición:
\begin{align*}
  \Delta p &= \int_{0}^{2\pi}\left[ f\left( x \right) - \frac{a_0}{2} - \sum_{n=1}^{p} \left( a_n \cos\left(nx \right) + b_n \sin\left( nx \right) \right)  \right]^2 dx 
.\end{align*}

Por lo tanto:
\begin{align*}
  0 = \frac{\partial \Delta p}{\partial a_n}  &= -2 \int_{0}^{2\pi} \left[ f\left( x \right) - \frac{a_0}{2} - \sum_{n=1}^{\infty} \left( a_n \cos\left( nx \right) + b_n \sin\left( nx \right)  \right)  \right] \cos\left( nx \right) dx\\
  &= -2 \int_{0}^{2\pi} f\left( x \right) \cos\left( nx \right) dx + 2\pi a_n
.\end{align*}

Por otro lado:
\begin{align*}
  0 = \frac{\partial \Delta p}{\partial b_n} &= -2 \int_{0}^{2\pi}\left[ f\left( x \right) - \frac{a_0}{2} - \sum_{n=1}^{\infty} \left( a_n \cos\left( nx \right) + b_n \sin\left( nx \right)  \right)  \right] \sin\left( nx \right) dx \\
  &= -2 \int_{0}^{2\pi} f\left( x \right) \sin\left( nx \right) dx + 2\pi b_n
.\end{align*}

\chapter{Arfken: 19.1.5}

Tenemos: \[
a_n = \frac{1}{\pi}\int_{0}^{2\pi}f\left( s \right) \cos\left( ns \right) ds
.\] 

Lo cual implica que
\begin{align*}
  a_n &= \frac{1}{\pi}\int_{0}^{2\pi} \frac{\sin\left( nx \right) \cos\left( nx \right) }{n} dx \\
  a_n &= 0;\ n \ge 0
.\end{align*}

Por otro lado:
\begin{align*}
  b_n &= \frac{1}{\pi}\int_{0}^{2\pi}f\left( s \right) \sin\left( ns \right) ds \\
  &= \frac{1}{2\pi}\left[ \int_{0}^{\pi} \left( \pi - x \right) \sin\left( nx \right) dx - \int_{-\pi}^{0}\left( \pi + x \right) \sin\left( nx \right) dx \right]  \\
  &= \frac{1}{2\pi}\left[ - \frac{\pi}{n} \cos\left( nx \right) + \frac{x}{n}\cos\left( nx \right)  \right]_{0}^{\pi} - \frac{1}{2\pi n}\int_{0}^{\pi}\cos\left( nx \right) dx \\ 
  &- \frac{1}{2\pi n}\int_{-\pi}^{0} \cos\left( nx \right) dx - \frac{1}{2\pi}\left[ - \frac{\pi}{n}\cos\left( nx \right) - \frac{x}{n}\cos\left( nx \right)  \right]_{-\pi}^{0}\\
  &= \frac{1}{n} - \left. \frac{\sin\left( nx \right) }{2\pi n^2}\right|_{0}^{\pi} - \left. \frac{\sin\left( nx \right) }{2\pi n^2}\right|_{-\pi}^{0} = \frac{1}{n}
.\end{align*}

\chapter{Arfken: 19.1.10}

En este caso tenemos: \[
f\left( x \right) = \begin{cases}
  4x\left( 1 - x \right) & 0 \le x \le 1 \\
  4x(1 + x) & -1 \le x \le 0
\end{cases} = \sum_{n=1}^{\infty} b_n \sin\left( n\pi x \right) 
.\] 

Ahora necesitamos encontrar $b_n$

\begin{align*}
  b_n &= \int_{-1}^{1}f\left( x \right) \sin\left( nx\pi \right) dx;\ n = 1,2,\ldots \\
  &= \frac{1}{\pi}\left[ \int_{-1}^{0} 4x\left( 1 + x \right) \sin\left( n\pi x \right) dx + \int_{0}^{1} 4x\left( 1 - x \right) \sin\left( n\pi x \right) dx \right]  \\
  &= \frac{1}{\pi}\left[ \int_{-1}^{0}\left( 4x + 4x^2 \right) \sin\left( n\pi x \right) dx + \int_{0}^{1}\left( 4x - 4x^2 \right) \sin\left( n\pi x \right) dx \right] \\
  &= \frac{1}{\pi}\left[ \int_{-1}^{0} 4x \sin\left( n\pi x \right) dx + \int_{-1}^{0} 4x^2\sin\left( n\pi x \right) dx \right.\\
  &\left.+\int_{0}^{1} 4x\sin\left( n\pi x \right) dx - \int_{0}^{1} 4x^2 \sin\left( n\pi x \right) dx\right]
.\end{align*}

Con esto ya podemos simplemente notar que con $n$ par tendríamos que su valor es 0 pues estas funciones serian pares. Por otro lado, quitando $x^2\sin\left( n \pi x \right) $ por ser pares queda entonces para $n$ impar queda entonces
\begin{align*}
  b_n &= 8 \left( \frac{2}{n\pi} - \frac{6}{n^{3}\pi^{3}} \right) \\
  &= \frac{32}{n^{3}\pi^{3}}
.\end{align*}


\chapter{Arfken: 19.2.17}

\section{Parte a}

En este caso necesitamos $b_n$ lo cual es:
\begin{align*}
  b_n = \frac{2}{L}\int_0^{L} \delta\left( x - a \right) \sin\left( \frac{n\pi x}{L} \right) dx
.\end{align*}

Lo cual por la definición de $\delta$ nos queda como  \[
b_n = \frac{2}{L}\sin\left( \frac{n \pi a}{L} \right)
.\] Ahora, reemplazando en la definición de serie seno de fourier nos queda:
\begin{align*}
  f(x) &= \sum_{n=1}^{\infty} b_n \sin\left( \frac{n\pi x}{L} \right)  \\
  &= \sum_{n=1}^{\infty} \frac{2}{L}\sin\left( \frac{n\pi a}{L} \right) \sin\left( \frac{n \pi x}{L} \right)  \\
  &= \frac{2}{L}\sum_{n=1}^{\infty} \sin\left( \frac{n\pi a}{L} \right) \sin\left( \frac{n \pi x}{L} \right)
.\end{align*}

\section{Parte b}

En este caso al hacer la integral tenemos:
\begin{align*}
  \delta \left( x - a \right)  &= \sum_{n=1}^{\infty} \sin\left( \frac{n\pi a}{L} \right) \sin\left( \frac{n \pi x}{L} \right) \\
  \int_{0}^{x} \delta\left( x - a \right) dx &= \int_{0}^{x} \sum_{n=1}^{\infty} \sin\left( \frac{n\pi a}{L} \right) \sin\left( \frac{n \pi x}{L} \right) dx\\
  \begin{cases}
   0& 0 \le  x < a\\
   1& a < x < L
  \end{cases} &= \sum_{n=1}^{\infty} \sin\left( \frac{n\pi a}{L} \right) \int_{0}^{x} \sin\left( \frac{n\pi x}{L} \right)dx   \\
  &= \frac{2}{L}\sum_{n=1}^{\infty} \sin\left( \frac{n\pi a}{L} \right) \left( \frac{L}{n\pi} \right) \left[ 1 - \cos\left( \frac{n\pi x}{L} \right)  \right]  \\
  &= \frac{2}{\pi}\sum_{n=1}^{\infty} \frac{1}{n}\sin\left( \frac{n\pi a}{L} \right) - \frac{2}{\pi}\sum_{n=1}^{\infty}\frac{1}{n} \sin\left( \frac{n\pi a}{L} \right) \cos\left( \frac{n \pi x}{L} \right)
.\end{align*}

\chapter{Arfken: 19.2.20}

Ahora queda:
\begin{align*}
  \sum_{n=1}^{\infty} \frac{d^2 b_n\left( t \right) }{dt^2} \sin\left( \frac{n \pi x}{L} \right) &= v^2 \sum_{n=1}^{\infty} b_n\left( t \right) \left( \frac{n^2\pi^2}{L^2} \right) \sin\left( \frac{n\pi x}{L} \right) \\
  \frac{d^2 b_n\left( t \right) }{dt^2} + v^2 \frac{n^2\pi^2}{L^2}b_n\left( t \right) &= 0 \\
  b_n\left( t \right) &= A_n \cos\left( \frac{n\pi v}{L}t \right) + B_n \sin\left( \frac{n\pi v}{L}t \right)  \\
  u\left( x, 0 \right) &= \sum_{n=1}^{\infty} b_n\left( 0 \right) \sin\left( \frac{n\pi x}{L} \right) = f\left( x \right)  \\
  b_n\left( 0 \right) &= A_n \\
  A_n &= \frac{2}{L}\int_{0}^{L} f\left( x \right) \sin\left( \frac{n\pi x}{L} \right) dx \\
  \frac{\partial u\left( x, t \right) }{\partial t} &= \sum_{n=1}^{\infty} \left( -A_n \frac{n\pi v}{L}\sin\left( \frac{n \pi v}{L} t \right) + B_n \frac{n\pi v}{L}\cos\left( \frac{n\pi v}{L}t \right)  \right)\sin\left( \frac{n \pi x}{L} \right) \\
  &\text{Con }t = 0 \\
  \sum_{n=1}^{\infty} B_n \frac{n \pi v}{L}\sin\left( \frac{n \pi x}{L} \right) &= g\left( x \right)  \\
  B_n &= \frac{2}{n \pi v}\int_{0}^{L}g\left( x \right) \sin\left(\frac{n \pi x}{L} \right)dx
.\end{align*}

Con esto entonces ya encontramos cuanto es $b_n$ y por lo tanto llegamos a la respuesta correcta.

\chapter{Tellez: 3.5.1}

Para iniciar sacamos los valores de $a_n$

\begin{align*}
  a_0 &= \frac{1}{\pi}\int_{0}^{\pi}\sin\left( x \right) dx \\
  &= \frac{1}{\pi}\left[ -\cos\left( x \right)  \right]_{0}^{\pi} \\
  &= \frac{2}{\pi}.
\end{align*}
\begin{align*}
  a_n &= \frac{1}{\pi}\int_{0}^{\pi}\sin\left( x \right) \cos\left( nx \right) dx \\
  &= \frac{1}{2\pi} \int_{0}^{\pi}\left( \sin\left( 1 + n \right) x + \sin\left( 1 - \frac{n}{x} \right)  \right) dx \\
  &= -\frac{1}{2\pi}\left[ \frac{\cos\left( 1 + n \right) x}{1 + n} + \frac{\cos\left( 1 - n \right) x}{1 - n} \right]_{0}^{\pi} \\
  &= - \frac{1}{2\pi}\left[ \frac{\left( -1 \right)^{1 + n}}{1 + n} + \frac{\left( -1 \right)^{1 - n}}{1 - n} - \left( \frac{1}{1 + m} + \frac{1}{1 - m} \right)  \right]  \\
  &= \frac{1}{\pi}\frac{\left( -1 \right)^{n} + 1}{\left( 1 - n^2 \right) }\ \left( n \neq 1 \right).
\end{align*}

\begin{align*}
  a_1 &= \frac{1}{\pi}\int_{0}^{\pi}\sin\left( x \right) \cos\left( x \right) dx \\
  &= \frac{1}{2\pi}\int_{0}^{\pi} \sin\left( 2\pi \right) dx \\
  &= \frac{1}{2\pi}\left( \frac{1 - \cos\left( 2x \right) }{2} \right)_{0}^{\pi} \\
  &= 0
.\end{align*}

Ahora sacamos los valores de $b_n$
\begin{align*}
  b_n &= \frac{1}{\pi}\int_{0}^{\pi}\sin\left( x \right) \sin\left( nx \right) dx \\
  &= \frac{1}{2\pi} \int_{0}^{\pi}\left( \cos\left( n - 1 \right) x - \cos\left( n + 1 \right) x \right) dx \\
  &= \frac{1}{2\pi}\left[ \frac{\sin\left( n - 1 \right) x}{n - 1} - \frac{\sin\left( n + 1 \right) x}{n + 1} \right]_{0}^{\pi} \\
  &= 0\ \left( n \neq 1 \right)
.\end{align*}
\begin{align*}
  b_1 &= \frac{1}{\pi}\int_{0}^{\pi}\sin\left( x \right) \sin\left( x \right) dx \\
  &= \frac{1}{\pi}\int_{0}^{\pi} \frac{1 - \cos\left( 2x \right) }{2} \\
  &= \frac{1}{2\pi}\left[ x - \frac{\sin\left( 2x \right) }{2} \right]_{0}^{\pi} \\
  &= \frac{1}{2\pi} + \pi \\
  &= \frac{1}{2}
.\end{align*}

Ahora juntando todo:

\begin{align*}
  f\left( x \right) &= \frac{1}{\pi} + \frac{1}{\pi}\sum_{n=2}^{\infty} \frac{1}{\pi}\frac{\left( -1 \right)^{n} + 1}{\left( 1 - n^2 \right) } \frac{1}{2} \sin\left( x \right)
.\end{align*}

\chapter{Tellez: 3.5.5}

En este caso asuma $u\left( x, t \right) = X\left( t \right)T\left( t \right)  $

Ahora poniendo esto en la ecuación queda:
\begin{align*}
  \ddot{X}\left( x \right) T\left( t \right) &= \frac{1}{c^2}X\left( x \right) \ddot{T}\left( t \right)  \\
  \frac{\ddot{X}\left( x \right) T\left( t \right)}{X\left( x \right) T\left( t \right) } &= \frac{1}{c^2} \frac{X\left( x \right) \ddot{T}\left( t \right) }{X\left( x \right) T\left( t \right) } \\
  \frac{\ddot{X}\left( x \right) }{X\left( x \right) } &= \frac{1}{c^2} \frac{\ddot{T}\left( t \right) }{T\left( t \right) } = -\lambda
.\end{align*}

Con esto entonces encontramos dos ecuaciones:
\begin{align*}
  \ddot{X}\left( x \right) + \lambda X\left( x \right) &= 0 \\
  \ddot{T}\left( t \right) + \lambda c^2 T\left( t \right) &= 0
.\end{align*}

Ahora resolviendo para $X\left( x \right) $ tenemos que usar $X\left( 0 \right) = 0$ y $X\left( L \right) = 0$ y teniendo \[
X\left( x \right) = A\sin\left( \sqrt{\lambda} x \right) + B \cos\left( \sqrt{\lambda} x \right)
.\] Con lo cual
\begin{align*}
  X\left( 0 \right) &= A\sin\left( 0 \right) + B \cos\left( 0 \right)  \\
  0 &= B
.\end{align*}

Por lo tanto $X\left( x \right) = A \sin\left( \sqrt{\lambda} x \right) $ con lo que queda
\begin{align*}
  X\left( L \right)  &= A \sin\left( \sqrt{\lambda} x \right)  \\
  0 &= \sin\left( \sqrt{\lambda} x \right)  \\
  \sqrt{\lambda} L &= n\pi \\
  \lambda &= \frac{n^2\pi^2}{L^2}
.\end{align*}

Con lo que queda
\begin{align*}
  X_n\left( x \right) &= A_n \sin\left( \frac{n\pi x}{L} \right) 
.\end{align*}

Y ahora probando con $T\left( t \right) $ queda \[
  \ddot{T}\left( t \right) + \frac{n^2\pi^2c^2}{L^2}T\left( t \right) = 0
.\] Con la solución general \[
T_n\left( t \right) = B_n \cos\left( \frac{n\pi c}{L}t \right) + C_n \sin\left( \frac{n\pi c}{L}t \right) 
.\] y tomando $\frac{\partial u}{\partial t} \left( x, 0 \right) = 0$ entonces
\begin{align*}
  T_n\left( t \right) &= B_n \cos\left( \frac{n\pi c}{L}t \right) + C_n \sin\left( \frac{n\pi c}{L}t \right) \\
\dot{T}_n\left( 0 \right) &= - B_n \sin\left( 0 \right) + C_n \cos\left( 0 \right)  \\
0 &= C_n
.\end{align*}

Con lo cual queda \[
u\left( x, t \right) = \sum_{n=1}^{\infty} A_n \sin\left( \frac{n\pi x}{L} \right) \cos\left( \frac{n \pi c}{L}t \right) 
.\] 

\end{document}

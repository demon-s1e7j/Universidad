\documentclass{report}

\documentclass[12pt]{article}
\usepackage{array}
\usepackage{color}
\usepackage{amsthm}
\usepackage{eufrak}
\usepackage{lipsum}
\usepackage{pifont}
\usepackage{yfonts}
\usepackage{amsmath}
\usepackage{amssymb}
\usepackage{ccfonts}
\usepackage{comment} \usepackage{amsfonts}
\usepackage{fancyhdr}
\usepackage{graphicx}
\usepackage{listings}
\usepackage{mathrsfs}
\usepackage{setspace}
\usepackage{textcomp}
\usepackage{blindtext}
\usepackage{enumerate}
\usepackage{microtype}
\usepackage{xfakebold}
\usepackage{kantlipsum}
%\usepackage{draftwatermark}
\usepackage[spanish]{babel}
\usepackage[margin=1.5cm, top=2cm, bottom=2cm]{geometry}
\usepackage[framemethod=tikz]{mdframed}
\usepackage[colorlinks=true,citecolor=blue,linkcolor=red,urlcolor=magenta]{hyperref}

%//////////////////////////////////////////////////////
% Watermark configuration
%//////////////////////////////////////////////////////
%\SetWatermarkScale{4}
%\SetWatermarkColor{black}
%\SetWatermarkLightness{0.95}
%\SetWatermarkText{\texttt{Watermark}}

%//////////////////////////////////////////////////////
% Frame configuration
%//////////////////////////////////////////////////////
\newmdenv[tikzsetting={draw=gray,fill=white,fill opacity=0},backgroundcolor=none]{Frame}

%//////////////////////////////////////////////////////
% Font style configuration
%//////////////////////////////////////////////////////
\renewcommand{\familydefault}{\ttdefault}
\renewcommand{\rmdefault}{tt}

%//////////////////////////////////////////////////////
% Bold configuration
%//////////////////////////////////////////////////////
\newcommand{\fbseries}{\unskip\setBold\aftergroup\unsetBold\aftergroup\ignorespaces}
\makeatletter
\newcommand{\setBoldness}[1]{\def\fake@bold{#1}}
\makeatother

%//////////////////////////////////////////////////////
% Default font configuration
%//////////////////////////////////////////////////////
\DeclareFontFamily{\encodingdefault}{\ttdefault}{%
  \hyphenchar\font=\defaulthyphenchar
  \fontdimen2\font=0.33333em
  \fontdimen3\font=0.16667em
  \fontdimen4\font=0.11111em
  \fontdimen7\font=0.11111em}


%From M275 "Topology" at SJSU
\newcommand{\id}{\mathrm{id}}
\newcommand{\taking}[1]{\xrightarrow{#1}}
\newcommand{\inv}{^{-1}}

%From M170 "Introduction to Graph Theory" at SJSU
\DeclareMathOperator{\diam}{diam}
\DeclareMathOperator{\ord}{ord}
\newcommand{\defeq}{\overset{\mathrm{def}}{=}}

%From the USAMO .tex files
\newcommand{\ts}{\textsuperscript}
\newcommand{\dg}{^\circ}
\newcommand{\ii}{\item}

% % From Math 55 and Math 145 at Harvard
% \newenvironment{subproof}[1][Proof]{%
% \begin{proof}[#1] \renewcommand{\qedsymbol}{$\blacksquare$}}%
% {\end{proof}}

\newcommand{\liff}{\leftrightarrow}
\newcommand{\lthen}{\rightarrow}
\newcommand{\opname}{\operatorname}
\newcommand{\surjto}{\twoheadrightarrow}
\newcommand{\injto}{\hookrightarrow}
\newcommand{\On}{\mathrm{On}} % ordinals
\DeclareMathOperator{\img}{im} % Image
\DeclareMathOperator{\Img}{Im} % Image
\DeclareMathOperator{\coker}{coker} % Cokernel
\DeclareMathOperator{\Coker}{Coker} % Cokernel
\DeclareMathOperator{\Ker}{Ker} % Kernel
\DeclareMathOperator{\rank}{rank}
\DeclareMathOperator{\Spec}{Spec} % spectrum
\DeclareMathOperator{\Tr}{Tr} % trace
\DeclareMathOperator{\pr}{pr} % projection
\DeclareMathOperator{\ext}{ext} % extension
\DeclareMathOperator{\pred}{pred} % predecessor
\DeclareMathOperator{\dom}{dom} % domain
\DeclareMathOperator{\ran}{ran} % range
\DeclareMathOperator{\Hom}{Hom} % homomorphism
\DeclareMathOperator{\Mor}{Mor} % morphisms
\DeclareMathOperator{\End}{End} % endomorphism

\newcommand{\eps}{\epsilon}
\newcommand{\veps}{\varepsilon}
\newcommand{\ol}{\overline}
\newcommand{\ul}{\underline}
\newcommand{\wt}{\widetilde}
\newcommand{\wh}{\widehat}
\newcommand{\vocab}[1]{\textbf{\color{blue} #1}}
\providecommand{\half}{\frac{1}{2}}
\newcommand{\dang}{\measuredangle} %% Directed angle
\newcommand{\ray}[1]{\overrightarrow{#1}}
\newcommand{\seg}[1]{\overline{#1}}
\newcommand{\arc}[1]{\wideparen{#1}}
\DeclareMathOperator{\cis}{cis}
\DeclareMathOperator*{\lcm}{lcm}
\DeclareMathOperator*{\argmin}{arg min}
\DeclareMathOperator*{\argmax}{arg max}
\newcommand{\cycsum}{\sum_{\mathrm{cyc}}}
\newcommand{\symsum}{\sum_{\mathrm{sym}}}
\newcommand{\cycprod}{\prod_{\mathrm{cyc}}}
\newcommand{\symprod}{\prod_{\mathrm{sym}}}
\newcommand{\Qed}{\begin{flushright}\qed\end{flushright}}
\newcommand{\parinn}{\setlength{\parindent}{1cm}}
\newcommand{\parinf}{\setlength{\parindent}{0cm}}
% \newcommand{\norm}{\|\cdot\|}
\newcommand{\inorm}{\norm_{\infty}}
\newcommand{\opensets}{\{V_{\alpha}\}_{\alpha\in I}}
\newcommand{\oset}{V_{\alpha}}
\newcommand{\opset}[1]{V_{\alpha_{#1}}}
\newcommand{\lub}{\text{lub}}
\newcommand{\del}[2]{\frac{\partial #1}{\partial #2}}
\newcommand{\Del}[3]{\frac{\partial^{#1} #2}{\partial^{#1} #3}}
\newcommand{\deld}[2]{\dfrac{\partial #1}{\partial #2}}
\newcommand{\Deld}[3]{\dfrac{\partial^{#1} #2}{\partial^{#1} #3}}
\newcommand{\lm}{\lambda}
\newcommand{\uin}{\mathbin{\rotatebox[origin=c]{90}{$\in$}}}
\newcommand{\usubset}{\mathbin{\rotatebox[origin=c]{90}{$\subset$}}}
\newcommand{\lt}{\left}
\newcommand{\rt}{\right}
\newcommand{\paren}[1]{\left(#1\right)}
\newcommand{\bs}[1]{\boldsymbol{#1}}
\newcommand{\exs}{\exists}
\newcommand{\st}{\strut}
\newcommand{\dps}[1]{\displaystyle{#1}}

\newcommand{\sol}{\setlength{\parindent}{0cm}\textbf{\textit{Solution:}}\setlength{\parindent}{1cm} }
\newcommand{\solve}[1]{\setlength{\parindent}{0cm}\textbf{\textit{Solution: }}\setlength{\parindent}{1cm}#1 \Qed}

% Things Lie
\newcommand{\kb}{\mathfrak b}
\newcommand{\kg}{\mathfrak g}
\newcommand{\kh}{\mathfrak h}
\newcommand{\kn}{\mathfrak n}
\newcommand{\ku}{\mathfrak u}
\newcommand{\kz}{\mathfrak z}
\DeclareMathOperator{\Ext}{Ext} % Ext functor
\DeclareMathOperator{\Tor}{Tor} % Tor functor
\newcommand{\gl}{\opname{\mathfrak{gl}}} % frak gl group
\renewcommand{\sl}{\opname{\mathfrak{sl}}} % frak sl group chktex 6

% More script letters etc.
\newcommand{\SA}{\mathcal A}
\newcommand{\SB}{\mathcal B}
\newcommand{\SC}{\mathcal C}
\newcommand{\SF}{\mathcal F}
\newcommand{\SG}{\mathcal G}
\newcommand{\SH}{\mathcal H}
\newcommand{\OO}{\mathcal O}

\newcommand{\SCA}{\mathscr A}
\newcommand{\SCB}{\mathscr B}
\newcommand{\SCC}{\mathscr C}
\newcommand{\SCD}{\mathscr D}
\newcommand{\SCE}{\mathscr E}
\newcommand{\SCF}{\mathscr F}
\newcommand{\SCG}{\mathscr G}
\newcommand{\SCH}{\mathscr H}

% Mathfrak primes
\newcommand{\km}{\mathfrak m}
\newcommand{\kp}{\mathfrak p}
\newcommand{\kq}{\mathfrak q}

% number sets
\newcommand{\RR}[1][]{\ensuremath{\ifstrempty{#1}{\mathbb{R}}{\mathbb{R}^{#1}}}}
\newcommand{\NN}[1][]{\ensuremath{\ifstrempty{#1}{\mathbb{N}}{\mathbb{N}^{#1}}}}
\newcommand{\ZZ}[1][]{\ensuremath{\ifstrempty{#1}{\mathbb{Z}}{\mathbb{Z}^{#1}}}}
\newcommand{\QQ}[1][]{\ensuremath{\ifstrempty{#1}{\mathbb{Q}}{\mathbb{Q}^{#1}}}}
\newcommand{\CC}[1][]{\ensuremath{\ifstrempty{#1}{\mathbb{C}}{\mathbb{C}^{#1}}}}
\newcommand{\PP}[1][]{\ensuremath{\ifstrempty{#1}{\mathbb{P}}{\mathbb{P}^{#1}}}}
\newcommand{\HH}[1][]{\ensuremath{\ifstrempty{#1}{\mathbb{H}}{\mathbb{H}^{#1}}}}
\newcommand{\FF}[1][]{\ensuremath{\ifstrempty{#1}{\mathbb{F}}{\mathbb{F}^{#1}}}}
% expected value
\newcommand{\EE}{\ensuremath{\mathbb{E}}}
\newcommand{\charin}{\text{ char }}
\DeclareMathOperator{\sign}{sign}
\DeclareMathOperator{\Aut}{Aut}
\DeclareMathOperator{\Inn}{Inn}
\DeclareMathOperator{\Syl}{Syl}
\DeclareMathOperator{\Gal}{Gal}
\DeclareMathOperator{\GL}{GL} % General linear group
\DeclareMathOperator{\SL}{SL} % Special linear group

%---------------------------------------
% BlackBoard Math Fonts :-
%---------------------------------------

%Captital Letters
\newcommand{\bbA}{\mathbb{A}}	\newcommand{\bbB}{\mathbb{B}}
\newcommand{\bbC}{\mathbb{C}}	\newcommand{\bbD}{\mathbb{D}}
\newcommand{\bbE}{\mathbb{E}}	\newcommand{\bbF}{\mathbb{F}}
\newcommand{\bbG}{\mathbb{G}}	\newcommand{\bbH}{\mathbb{H}}
\newcommand{\bbI}{\mathbb{I}}	\newcommand{\bbJ}{\mathbb{J}}
\newcommand{\bbK}{\mathbb{K}}	\newcommand{\bbL}{\mathbb{L}}
\newcommand{\bbM}{\mathbb{M}}	\newcommand{\bbN}{\mathbb{N}}
\newcommand{\bbO}{\mathbb{O}}	\newcommand{\bbP}{\mathbb{P}}
\newcommand{\bbQ}{\mathbb{Q}}	\newcommand{\bbR}{\mathbb{R}}
\newcommand{\bbS}{\mathbb{S}}	\newcommand{\bbT}{\mathbb{T}}
\newcommand{\bbU}{\mathbb{U}}	\newcommand{\bbV}{\mathbb{V}}
\newcommand{\bbW}{\mathbb{W}}	\newcommand{\bbX}{\mathbb{X}}
\newcommand{\bbY}{\mathbb{Y}}	\newcommand{\bbZ}{\mathbb{Z}}

%---------------------------------------
% MathCal Fonts :-
%---------------------------------------

%Captital Letters
\newcommand{\mcA}{\mathcal{A}}	\newcommand{\mcB}{\mathcal{B}}
\newcommand{\mcC}{\mathcal{C}}	\newcommand{\mcD}{\mathcal{D}}
\newcommand{\mcE}{\mathcal{E}}	\newcommand{\mcF}{\mathcal{F}}
\newcommand{\mcG}{\mathcal{G}}	\newcommand{\mcH}{\mathcal{H}}
\newcommand{\mcI}{\mathcal{I}}	\newcommand{\mcJ}{\mathcal{J}}
\newcommand{\mcK}{\mathcal{K}}	\newcommand{\mcL}{\mathcal{L}}
\newcommand{\mcM}{\mathcal{M}}	\newcommand{\mcN}{\mathcal{N}}
\newcommand{\mcO}{\mathcal{O}}	\newcommand{\mcP}{\mathcal{P}}
\newcommand{\mcQ}{\mathcal{Q}}	\newcommand{\mcR}{\mathcal{R}}
\newcommand{\mcS}{\mathcal{S}}	\newcommand{\mcT}{\mathcal{T}}
\newcommand{\mcU}{\mathcal{U}}	\newcommand{\mcV}{\mathcal{V}}
\newcommand{\mcW}{\mathcal{W}}	\newcommand{\mcX}{\mathcal{X}}
\newcommand{\mcY}{\mathcal{Y}}	\newcommand{\mcZ}{\mathcal{Z}}


%---------------------------------------
% Bold Math Fonts :-
%---------------------------------------

%Captital Letters
\newcommand{\bmA}{\boldsymbol{A}}	\newcommand{\bmB}{\boldsymbol{B}}
\newcommand{\bmC}{\boldsymbol{C}}	\newcommand{\bmD}{\boldsymbol{D}}
\newcommand{\bmE}{\boldsymbol{E}}	\newcommand{\bmF}{\boldsymbol{F}}
\newcommand{\bmG}{\boldsymbol{G}}	\newcommand{\bmH}{\boldsymbol{H}}
\newcommand{\bmI}{\boldsymbol{I}}	\newcommand{\bmJ}{\boldsymbol{J}}
\newcommand{\bmK}{\boldsymbol{K}}	\newcommand{\bmL}{\boldsymbol{L}}
\newcommand{\bmM}{\boldsymbol{M}}	\newcommand{\bmN}{\boldsymbol{N}}
\newcommand{\bmO}{\boldsymbol{O}}	\newcommand{\bmP}{\boldsymbol{P}}
\newcommand{\bmQ}{\boldsymbol{Q}}	\newcommand{\bmR}{\boldsymbol{R}}
\newcommand{\bmS}{\boldsymbol{S}}	\newcommand{\bmT}{\boldsymbol{T}}
\newcommand{\bmU}{\boldsymbol{U}}	\newcommand{\bmV}{\boldsymbol{V}}
\newcommand{\bmW}{\boldsymbol{W}}	\newcommand{\bmX}{\boldsymbol{X}}
\newcommand{\bmY}{\boldsymbol{Y}}	\newcommand{\bmZ}{\boldsymbol{Z}}
%Small Letters
\newcommand{\bma}{\boldsymbol{a}}	\newcommand{\bmb}{\boldsymbol{b}}
\newcommand{\bmc}{\boldsymbol{c}}	\newcommand{\bmd}{\boldsymbol{d}}
\newcommand{\bme}{\boldsymbol{e}}	\newcommand{\bmf}{\boldsymbol{f}}
\newcommand{\bmg}{\boldsymbol{g}}	\newcommand{\bmh}{\boldsymbol{h}}
\newcommand{\bmi}{\boldsymbol{i}}	\newcommand{\bmj}{\boldsymbol{j}}
\newcommand{\bmk}{\boldsymbol{k}}	\newcommand{\bml}{\boldsymbol{l}}
\newcommand{\bmm}{\boldsymbol{m}}	\newcommand{\bmn}{\boldsymbol{n}}
\newcommand{\bmo}{\boldsymbol{o}}	\newcommand{\bmp}{\boldsymbol{p}}
\newcommand{\bmq}{\boldsymbol{q}}	\newcommand{\bmr}{\boldsymbol{r}}
\newcommand{\bms}{\boldsymbol{s}}	\newcommand{\bmt}{\boldsymbol{t}}
\newcommand{\bmu}{\boldsymbol{u}}	\newcommand{\bmv}{\boldsymbol{v}}
\newcommand{\bmw}{\boldsymbol{w}}	\newcommand{\bmx}{\boldsymbol{x}}
\newcommand{\bmy}{\boldsymbol{y}}	\newcommand{\bmz}{\boldsymbol{z}}

%---------------------------------------
% Scr Math Fonts :-
%---------------------------------------

\newcommand{\sA}{{\mathscr{A}}}   \newcommand{\sB}{{\mathscr{B}}}
\newcommand{\sC}{{\mathscr{C}}}   \newcommand{\sD}{{\mathscr{D}}}
\newcommand{\sE}{{\mathscr{E}}}   \newcommand{\sF}{{\mathscr{F}}}
\newcommand{\sG}{{\mathscr{G}}}   \newcommand{\sH}{{\mathscr{H}}}
\newcommand{\sI}{{\mathscr{I}}}   \newcommand{\sJ}{{\mathscr{J}}}
\newcommand{\sK}{{\mathscr{K}}}   \newcommand{\sL}{{\mathscr{L}}}
\newcommand{\sM}{{\mathscr{M}}}   \newcommand{\sN}{{\mathscr{N}}}
\newcommand{\sO}{{\mathscr{O}}}   \newcommand{\sP}{{\mathscr{P}}}
\newcommand{\sQ}{{\mathscr{Q}}}   \newcommand{\sR}{{\mathscr{R}}}
\newcommand{\sS}{{\mathscr{S}}}   \newcommand{\sT}{{\mathscr{T}}}
\newcommand{\sU}{{\mathscr{U}}}   \newcommand{\sV}{{\mathscr{V}}}
\newcommand{\sW}{{\mathscr{W}}}   \newcommand{\sX}{{\mathscr{X}}}
\newcommand{\sY}{{\mathscr{Y}}}   \newcommand{\sZ}{{\mathscr{Z}}}


%---------------------------------------
% Math Fraktur Font
%---------------------------------------

%Captital Letters
\newcommand{\mfA}{\mathfrak{A}}	\newcommand{\mfB}{\mathfrak{B}}
\newcommand{\mfC}{\mathfrak{C}}	\newcommand{\mfD}{\mathfrak{D}}
\newcommand{\mfE}{\mathfrak{E}}	\newcommand{\mfF}{\mathfrak{F}}
\newcommand{\mfG}{\mathfrak{G}}	\newcommand{\mfH}{\mathfrak{H}}
\newcommand{\mfI}{\mathfrak{I}}	\newcommand{\mfJ}{\mathfrak{J}}
\newcommand{\mfK}{\mathfrak{K}}	\newcommand{\mfL}{\mathfrak{L}}
\newcommand{\mfM}{\mathfrak{M}}	\newcommand{\mfN}{\mathfrak{N}}
\newcommand{\mfO}{\mathfrak{O}}	\newcommand{\mfP}{\mathfrak{P}}
\newcommand{\mfQ}{\mathfrak{Q}}	\newcommand{\mfR}{\mathfrak{R}}
\newcommand{\mfS}{\mathfrak{S}}	\newcommand{\mfT}{\mathfrak{T}}
\newcommand{\mfU}{\mathfrak{U}}	\newcommand{\mfV}{\mathfrak{V}}
\newcommand{\mfW}{\mathfrak{W}}	\newcommand{\mfX}{\mathfrak{X}}
\newcommand{\mfY}{\mathfrak{Y}}	\newcommand{\mfZ}{\mathfrak{Z}}
%Small Letters
\newcommand{\mfa}{\mathfrak{a}}	\newcommand{\mfb}{\mathfrak{b}}
\newcommand{\mfc}{\mathfrak{c}}	\newcommand{\mfd}{\mathfrak{d}}
\newcommand{\mfe}{\mathfrak{e}}	\newcommand{\mff}{\mathfrak{f}}
\newcommand{\mfg}{\mathfrak{g}}	\newcommand{\mfh}{\mathfrak{h}}
\newcommand{\mfi}{\mathfrak{i}}	\newcommand{\mfj}{\mathfrak{j}}
\newcommand{\mfk}{\mathfrak{k}}	\newcommand{\mfl}{\mathfrak{l}}
\newcommand{\mfm}{\mathfrak{m}}	\newcommand{\mfn}{\mathfrak{n}}
\newcommand{\mfo}{\mathfrak{o}}	\newcommand{\mfp}{\mathfrak{p}}
\newcommand{\mfq}{\mathfrak{q}}	\newcommand{\mfr}{\mathfrak{r}}
\newcommand{\mfs}{\mathfrak{s}}	\newcommand{\mft}{\mathfrak{t}}
\newcommand{\mfu}{\mathfrak{u}}	\newcommand{\mfv}{\mathfrak{v}}
\newcommand{\mfw}{\mathfrak{w}}	\newcommand{\mfx}{\mathfrak{x}}
\newcommand{\mfy}{\mathfrak{y}}	\newcommand{\mfz}{\mathfrak{z}}

\usepackage{float}

\newcommand{\Lagr}{\mathcal{L}}

\title{\Huge{Métodos Matemáticos}\\Taller Examen 3}
\author{\huge{Sergio Montoya Ramírez}}
\date{}

\begin{document}

\maketitle
\newpage% or \cleardoublepage
% \pdfbookmark[<level>]{<title>}{<dest>}
\pdfbookmark[chapter]{\contentsname}{toc}
\tableofcontents
\pagebreak

\chapter{Problema Sturm-Liouville}

\section{a}
En este caso iniciamos:

\begin{align*}
\left<\Lagr f | g \right> &= \int_{a}^b \left[\frac{d}{dx}\left(p(x) \frac{df}{dx}\right) - q(x) f(x) \right]^*g(x)dx \\
&= \int_a^b \frac{d}{dx} \left(p(x) \frac{df^*}{dx}\right) g dx - \int_a^b q(x)f^*g dx\\
u = g &; du = \frac{dg}{dx} dx\\
dv = \frac{d}{dx}\left(p(x) \frac{df}{dx}\right) dx &; v = \left(p(x) \frac{df}{dx}\right)\\
\int_a^b \frac{d}{dx} \left(p(x) \frac{df^*}{dx}\right) g dx &= \left[p(x) \frac{df^*}{dx} g \right]_a^b - \int_a^b p(x) \frac{df^*}{dx} \frac{dg}{dx} dx\\
\left[p(x) \frac{df^*}{dx} g \right]_a^b &= 0\\
&= - \int_a^b p(x) \frac{df^*}{dx} \frac{dg}{dx} dx - \int_a^b q(x)f^*g dx
\end{align*}

Ahora el lado contrario es muy similar
\begin{align*}
\left< f | \Lagr g \right> &= \int_{a}^b f^* \left[\frac{d}{dx}\left(p(x) \frac{dg}{dx}\right) - q(x) g(x) \right]dx \\
&= \int_{a}^b f^* \frac{d}{dx}\left(p(x) \frac{dg}{dx}\right) - q(x)f^* g(x)dx \\
u = f^* &; du = \frac{df^*}{dx} dx\\
dv = \frac{d}{dx}\left(p(x) \frac{dg}{dx}\right) dx &; v = \left(p(x) \frac{dg}{dx}\right)\\
\int_a^b \frac{d}{dx} \left(p(x) \frac{dg}{dx}\right) f^* dx &= \left[p(x) \frac{df^*}{dx} g \right]_a^b - \int_a^b p(x) \frac{df^*}{dx} \frac{dg}{dx} dx\\
\left[p(x) \frac{df^*}{dx} g \right]_a^b &= 0\\
&= - \int_a^b p(x) \frac{df^*}{dx} \frac{dg}{dx} dx - \int_a^b q(x)f^*g dx
\end{align*}

Con lo que confirmamos que coinciden.

\section{b}

Primero iniciemos con que los valores propios son reales.

\begin{align*}
-\Lagr u_n &= \lambda_n \rho(x) u_n\\
\rho(x) u_n^*(x) \left(-\Lagr u_n\right) &= \rho(x) \left|u_n(x)\right|^2 \\
\int_a^b \rho(x) u_n^*(x) \left(-\Lagr u_n\right) dx &= \lambda_n \int_a^b \rho(x) \left|u_n(x)\right|^2 dx \\
\int_a^b \rho(x) u_n^*(x) \left(-\Lagr u_n\right) dx &= \int_a^b \rho(x) u_n(x) \left(-\Lagr u_n\right)^* dx \\
\int_a^b \rho(x) u_n(x) \left(-\Lagr u_n\right)^* dx &= \lambda_n^* \int_a^b \rho(x) \left|u_n(x)\right|^2 dx \\
\lambda_n \int_a^b \rho(x) \left|u_n(x)\right|^2 dx &= \lambda_n^* \int_a^b \rho(x) \left|u_n(x)\right|^2 dx \\
\lambda_n &= \lambda_n^*
.\end{align*}

Ahora, para mostrar que las funciones propias son orgonales. Para esto imagine dos funciones propias con valores propios distintos. Por lo tanto, las ecuaciones quedarian como:
\begin{align*}
-\Lagr u_{n} &= \lambda_{n} \rho(x) u_{n}\\
-\Lagr u_{n'} &= \lambda_{n'} \rho(x) u_{n'}\\
\end{align*}

Con lo que podemos desarrollar:

\begin{align*}
\int_a^b \rho(x) u_{n'}^* (-\Lagr u_{n}) dx &= \lambda_{n} \int_a^b \rho(x) u_{n'}^* u_{n} dx\\
\int_a^b \rho(x) u_{n}^* (-\Lagr u_{n'}) dx &= \lambda_{n} \int_a^b \rho(x) u_{n}^* u_{n'} dx\\
\int_a^b \rho(x) u_{n'}^* (-\Lagr u_{n}) dx - \int_a^b \rho(x) u_{n}^* (-\Lagr u_{n'}) dx &= \left(\lambda_n - \lambda_{n'} \right) \int_a^b \rho(x) u_{n}^* u_{n'} dx \\
\int_a^b \rho(x) u_{n'}^* (-\Lagr u_{n}) dx - \int_a^b \rho(x) u_{n}^* (-\Lagr u_{n'}) dx &= 0\\
\left(\lambda_n - \lambda_{n'} \right) \int_a^b \rho(x) u_{n}^* u_{n'} dx &= 0\\
\int_a^b \rho(x) u_{n}^* u_{n'} dx &= 0
\end{align*}

\section{c}

Para iniciar, solucionemos la ecuación diferencial
\[\frac{d^2 u}{dx^2} + \lambda_n u  = 0\]

Lo cual tiene como solución 
\[u(x) = A \sin\left(\sqrt{\lambda_n} x\right) + B \cos\left(\sqrt{\lambda_n} x\right)\]

Sin embargo, podemos encontrar las constantes como
\begin{align*}
u(0) &= 0\\
u(0) &= A\sin(0) + B\cos(0) = B = 0\\
u(L) &= 0\\
u(L) &= A\sin(\sqrt{\lambda_n}L ) = L\\
A &\neq 0\\
\sin(\sqrt{\lambda_n} L) &= 0\\
\sqrt{\lambda_n} L &= n\pi\\
\lambda_n &= \frac{n^2\pi^2}{L^2}\\
u_n(x) &= A_n \sin\left(\frac{n\pi x}{L}\right),\ n \in \mathbb{N}
\end{align*}

Ahora teniendo
\begin{align*}
\left<u_n, u_{n'}\right> &= \int_0^L u_n (x) u_{n'}(x) dx\\
n &\neq n'\\
&= \int_0^L \sin\left(\frac{n \pi x}{L}\right)\sin\left(\frac{n' \pi x}{L}\right) dx = 0\\
n &= n'\\
&= \int_n^L  \sin^2\left(\frac{n \pi x}{L}\right) dx\\
&= \frac{1}{2} \int_0^L 1 dx - \frac{1}{2}\int_0^L \cos\left(\frac{2n\pi x}{L}\right) dx\\
\frac{1}{2}\int_0^L \cos\left(\frac{2n\pi x}{L}\right) dx &= 0\\
&= \frac{L}{2} - 0\\
A_n^2 \int_0^L \sin^2\left(\frac{n\pi x}{L}\right) dx &= A_n^2 \frac{L}{2} = 1\\
A_n &= \sqrt{\frac{2}{L}}
\end{align*}

Por lo tanto, las funciones propias normalizadas es
\[u_n(x) = \sqrt{\frac{2}{L}}\sin\left(\frac{n\pi x}{L}\right)\]

Ahora para comprobar el teorema de Tellez simplemente tenemos que notar que en el intervalo $(0, L)$ hay 0 en 
\[x = \frac{L}{n}, \frac{2L}{n},\frac{3L}{n},\ldots, \frac{(n - 1)L}{n}\]

Que son $(n - 1)$ lo que confirma lo que esperabamos.

\section{d}

Si expandimos estas funciones en series nos quedan:

$$f(x) = \sum_{n=1}^\infty f_n u_n(x)$$

$$g(x) = \sum_{n=1}^\infty g_n u_n(x)$$

Ahora bien, tenemos
\begin{align*}
-\Lagr f &= \rho g\\
\int_a^b f u_n dx &= \int_a^b \rho g u_n dx\\
-(Lu_n|f)\rho &= (u_n|\rho g)\rho\\
-\Lagr u_n &= \lambda_n \rho u_n\\
\lambda_n f_n &= g_n
\end{align*}

\chapter{Aplicación 1}

\section{a}

Tenemos la ecuación:
\begin{align*}
  \frac{1}{c^2}\frac{\partial^2 f}{\partial t^2} - \frac{\partial^2 f}{\partial x^2} &= 0 \\
.\end{align*}

Con lo que podemos encontrar:
\begin{align*}
  \Lagr \left\{ \frac{\partial^2 f}{\partial t^2}  \right\} &= s^2\Lagr \left\{ f \right\} - sf\left( x, 0 \right) - \frac{\partial f}{\partial t} \left( x, 0 \right)  \\
  0 &= \frac{s^2}{c^2}\Lagr \left\{ f \right\} - \frac{sf\left( x, 0 \right) - \frac{\partial f}{\partial t} \left( x, 0 \right)}{c^2} - \frac{\partial^2 \Lagr\left\{ f \right\} }{\partial x^2} \\
  \frac{\partial^2 \Lagr\left\{ f \right\} }{\partial x^2} - \frac{s^2}{c^2}\Lagr \left\{ f \right\} &= \frac{-sf\left( x, 0 \right) - \frac{\partial f}{\partial t} \left( x, 0 \right)}{c^2} 
.\end{align*}

Esto es una ecuación diferencial de segundo orden que podemos poner como:
\begin{align*}
  L_s \Lagr\left\{ f \right\} &= \tilde{g} \\
  \tilde{g} &= \frac{-sf\left( x, 0 \right) - \frac{\partial f}{\partial t} \left( x, 0 \right)}{c^2} 
.\end{align*}

\section{b}

En este caso recordemos
\begin{align*}
  u_n\left( 0 \right) &= u_n\left( L \right) = 0 \\
  u_n\left( x \right) &= \sin\left( \frac{n \pi x}{L} \right)  \\
  \lambda_n &= \left( \frac{n\pi}{L} \right)^2\\
  \tilde{f}\left( x, s \right) &= \sum_{n=1}^{\infty} \tilde{f}_n\left( s \right) u_n\left( x \right)  \\
  \tilde{g}\left( x, s \right) &= \sum_{n=1}^{\infty} \tilde{g}_n\left( s \right) u_n\left( x \right)
.\end{align*}

Ahora, si lo metemos en $-L_s\tilde{f} = \tilde{g}$

 \begin{align*}
  -L_s u_n &= \left( \frac{s^2}{c^2} + \lambda_n \right) u_n \\
  \left( \lambda_n - \frac{s^2}{c^2} \right) \tilde{f}_n\left( s \right) &= \tilde{g}_n\left( s \right)  \\
  \tilde{f}_n\left( s \right) &= \frac{\tilde{g}_n\left( s \right) }{\lambda_n - \frac{s^2}{c^2}} \\
  \tilde{g}_n\left( s \right) &= \int_{0}^{L}\tilde{g}\left( x, s \right) u_n\left( x \right) dx \\
  \tilde{g} &= \frac{-sf\left( x, 0 \right) - \frac{\partial f}{\partial t} \left( x, 0 \right)}{c^2}\\ 
  \tilde{g}_n\left( s \right) &= -\frac{1}{c^2}\int_{0}^{L}\left[ sf\left( x, 0 \right) + f_t\left( x, 0 \right)  \right] u_n\left( x \right) dx
.\end{align*}

Ahora, para finalizar

\begin{align*}
  f_n\left( t \right) &= \Lagr^{-1}\left\{ \tilde{f}_n\left( s \right)  \right\}  \\
  \tilde{f}_n\left( s \right) &= \frac{- \frac{sF_n + F_{t, n}}{c^2}}{\lambda_n - \frac{s^2}{c^2}} \\
  F_n &= \int_{0}^{L}f\left( x, 0 \right) u_n\left( x \right) dx \\
  F_{t, n}&= \int_{0}^{L}f_t\left( x, 0 \right) u_n\left( x \right) dx \\
  f\left( x, t \right) &= \sum_{n=1}^{\infty} f_n\left( t \right) u_n\left( x \right)
.\end{align*}

\section{c}

Para iniciar, tenemos
\begin{align*}
  -L_s \tilde{f}\left( x, s \right) &= \tilde{g}\left( x, s \right)  \\
  \tilde{f}\left( x, s \right) &= \sum_{n=1}^{\infty} \tilde{f}_n\left( s \right) u_n\left( x \right)  \\
  \tilde{g}\left( x, s \right) &= \sum_{n=1}^{\infty} \tilde{g}_n\left( s \right) u_n\left( x \right)  \\
  \tilde{f}\left( x, s \right) &= \sum_{n=0}^{\infty} \frac{\tilde{g}_n\left( s \right) }{\lambda_n \left( s \right) }u_n\left( x \right)  \\
.\end{align*}

Lo que nos deja con una base para $G$ :
\begin{align*}
  \tilde{g}_n\left( s \right) &= \int_{0}^{L}\tilde{g}\left( x', s \right) u_n\left( x' \right) dx' \\
  \tilde{f}\left( x, s \right) &= \sum_{n=0}^{\infty} \frac{1}{\lambda_n\left( s \right) }u_n\left( x \right) \int_{0}^{L}\tilde{g}\left( x', s \right) u_n\left( x' \right) dx' \\
  \tilde{f}\left( x, s \right) &= \int_{0}^{L}\tilde{g}\left( x', s \right) \left( \sum_{n=0}^{\infty} \frac{u_n\left( x \right) u_n\left( x' \right) }{\lambda_n\left( s \right) } \right) dx' \\
  G_s\left( x, x' \right) &= \sum_{n=0}^{\infty} \frac{u_n\left( x \right) u_n\left( x' \right) }{\lambda_n\left( s \right) } \\
  \tilde{f}\left( x, s \right) &= \int_{0}^{L} G_s\left( x, x' \right) \tilde{g}\left( x', s \right) dx'
.\end{align*}

Ahora bien, para encontrar el valor concreto de $G$ podemos encontrar
\begin{align*}
  u_n\left( x \right) &= \sin\left( \frac{n\pi x}{L} \right)  \\
  \lambda_n\left( s \right) &= \left( \frac{n\pi}{L} \right)^2 - \frac{s^2}{c^2} \\
  G_s\left( x, x' \right) &= \sum_{n=1}^{\infty} \frac{\sin\left( \frac{n\pi x}{L} \right) \sin\left( \frac{n\pi x'}{L} \right) }{\left( \frac{n\pi}{L} \right)^2 - \frac{s^2}{c^2}}\\
  G_s\left( x, x' \right) &= \frac{1}{2L}\sum_{n=-\infty}^{\infty} \frac{\cos\left( \frac{k\pi}{L}\left| x - x' \right|  \right) - \cos\left( \frac{k\pi}{L}\left( x + x' \right)  \right) }{\left( \frac{k\pi}{L} \right) ^2 + \frac{s^2}{c^2}}
.\end{align*}

\section{d}

Teniendo
\begin{align*}
  G_s\left( x, x' \right) &= \sum_{n=1}^{\infty} \frac{\sin\left( \frac{n\pi x}{L} \right) \sin\left( \frac{n\pi x'}{L} \right) }{\left( \frac{n\pi}{L} \right)^2 - \frac{s^2}{c^2}}\\
  u_n\left( x \right) &= \sin\left( \frac{n\pi x}{L} \right)  \\
  \lambda_n &= \left( \frac{n\pi}{L} \right)^2 \\
  \frac{\partial^2 u_n\left( x \right) }{\partial x^2} &= -\lambda_n u_n\left( x \right)  \\
  -L_su_n\left( x \right) &= \left( \lambda_n - \frac{s^2}{c^2} \right) u_n\left( x \right)  \\
  -L_sG_s\left( x, x' \right) &= \sum_{n=1}^{\infty} \frac{\left( \lambda_n - \frac{s^2}{c^2} \right) u_n\left( x \right) u_n\left( x' \right) }{\lambda_n - \frac{s^2}{c^2}} \\
  -L_sG_s\left( x, x' \right) &= \sum_{n=1}^{\infty} u_n\left( x \right) u_n\left( x' \right)  \\
  -L_sG_n\left( x, x' \right) &= \sum_{n=1}^{\infty} \sin\left( \frac{n\pi x}{L} \right) \sin\left( \frac{n\pi x'}{L} \right)  \\
  -L_sG_n\left( x, x' \right) &= \frac{L}{2}\delta\left( x - x' \right)
.\end{align*}

\section{e}

Para esto iniciemos por ver la ecuación no homogénea \[
\frac{1}{c^2}\frac{\partial^2 f}{\partial t^2} - \frac{\partial^2 f}{\partial x^2} = h\left( x, t \right) 
.\] 

Para iniciar, sabemos que tanto $f$ como $h$ pueden expandirse y quedar como:
\begin{align*}
  f\left( x, t \right) &= \sum_{n=1}^{\infty} f_n\left( t \right) u_n\left( x \right)  \\
  h\left( x, t \right) &= \sum_{n=1}^{\infty} h_n\left( t \right) u_n\left( x \right) 
.\end{align*}

Si sustituimos esto queda:
\begin{align*}
  \frac{1}{c^2}\sum_{n=1}^{\infty} \ddot{f}_n\left( t \right) u_n\left( x \right) - \sum_{n=1}^{\infty} f_n\left( t \right) \frac{\partial^2 u_n\left( x \right) }{\partial x^2} &= \sum_{n=1}^{\infty} h_n\left( t \right) u_n\left( x \right)  \\
  \frac{\partial^2 u_n\left( x \right) }{\partial x^2} &= -\lambda_n u_n\left(  \right)  \\
  \implies \frac{1}{c^2}\ddot{f}_n\left( t \right) + \lambda_n f_n\left( t \right) = h_n\left( t \right) 
.\end{align*}

Donde esta ultima es la ecuación diferencial para cada termino. Esta ecuación se soluciona con la convolución de la solución homogénea y otra función de la forma:
\begin{align*}
  f_n\left( t \right) = \gamma_n\left( t \right) * h_n\left( t \right) 
.\end{align*}

Ahora, para determinar $\gamma_n$ podemos solucionar la ecuación homogenea asociada
 \begin{align*}
   \ddot{\gamma}_n\left( t \right) + c^2\lambda_n \gamma_n\left( t \right) &= 0 \\
   \gamma_n\left( t \right) &= A_n \cos\left( \omega_n t \right) + B_n \sin\left( \omega_n t \right)  \\
   \gamma_n\left( 0 \right) &= A_n \cos\left( 0 \right) + B_n \sin\left( 0 \right)  \\
   A &= 0 \\
   \dot{\gamma}_n\left( t \right) &= B_n\omega_n \cos\left( \omega_n t \right)  \\
   \dot{\gamma}_n\left( 0 \right) &= B_n\omega_n = 1  \\
   B_n &= \frac{1}{\omega_n} \\
   B_n &= \frac{L}{n\pi c} \\
   \gamma_n\left( t \right) &= \frac{\sin\left( \frac{n\pi c}{L}t \right) }{c \frac{n\pi}{L}}
.\end{align*}

Ahora para volver a plantear $f$ nos queda como:
\begin{align*}
  f_n\left( t \right) &= \int_{0}^{t} \gamma_n \left( t - t' \right) h_n\left( t' \right) dt' \\
  f\left( x, t \right) &= \sum_{n=1}^{\infty} f_n\left( t \right) u_n\left( x \right) \\
  &= \sum_{n=1}^{\infty} \left( \int_{0}^{t}\gamma_n\left( t - t' \right) h_n\left( t' \right) dt' \right) u_n\left( x \right)  \\
  G\left( x, x', t - t' \right) &= \sum_{n=1}^{\infty} \gamma\left( t - t' \right) u_n\left( x \right) u_n\left( x' \right)  \\
  \implies f\left( x, t \right) &= \int_{0}^{L}dx'\int_{0}^{t} dt'G\left( x, x', t - t' \right) h\left( x', t' \right)
.\end{align*}

\chapter{Aplicación 2}

\section{a}

En este caso, partimos de
\begin{align*}
  -\frac{\hbar}{2m}\nabla^2 \psi + V\left( r \right) \psi = E\psi
.\end{align*}

Donde el laplaciano es: \[
\nabla^2 = \frac{1}{r^2} \frac{\partial}{\partial r} \left( r^2 \frac{\partial}{\partial r} \right) - \frac{1}{r^2} \left[ \frac{1}{\sin\theta} \frac{\partial}{\partial \theta} \left( \sin\theta \frac{\partial}{\partial \theta} \right) + \frac{1}{\sin^2\theta} \frac{\partial^2}{\partial \phi^2} \right].
.\] 

Lo que nos permite proponer la solución 
\[
\psi(r, \theta, \phi) = R_{nl}(r) P_l^m(\cos\theta) e^{im\phi}.
\]

De tal modo que al sustituir y dividir por $psi$ separamos las variable en tres ecuaciones independientes
\[
\frac{1}{R} \frac{1}{r^2} \frac{d}{dr} \left( r^2 \frac{dR}{dr} \right) - \frac{l(l+1)}{r^2} = \frac{1}{P} \left[ \frac{1}{\sin\theta} \frac{d}{d\theta} \left( \sin\theta \frac{dP}{d\theta} \right) + \frac{m^2}{\sin^2\theta} \right] = \frac{2m}{\hbar^2} \left( E - V(r) \right).
\]
\begin{enumerate}
  \item \( e^{im\phi} \)
\[
\frac{d^2 Y}{d\phi^2} + m^2 Y = 0. \implies Y(\phi) = e^{im\phi},
\]
\begin{table}[H]
  \centering
  \caption{caption}
  \label{tab:label}
  \begin{tabular}{c|c}
    \hline
    Nombre & Valor\\
    \hline
    Operador & \( L = \frac{d^2}{d\phi^2} \) \\
    Función peso & \( \rho(\phi) = 1 \) \\
    Valores propios & \( \lambda = -m^2 \)\\
    \hline
  \end{tabular}
\end{table}

\item \( P_l^m(\cos\theta) \)
\[
\frac{1}{\sin\theta} \frac{d}{d\theta} \left( \sin\theta \frac{dP}{d\theta} \right) + \left[ l(l+1) - \frac{m^2}{\sin^2\theta} \right] P = 0.
\]

Que es una ecuación de de polinomios de Legendre asociados
\[
P_l^m(\cos\theta).
\]

\begin{table}[H]
  \centering
  \caption{caption}
  \label{tab:label}
  \begin{tabular}{c|c}
    \hline
    Nombre & Valor\\
    \hline
    Operador & \( L = \frac{1}{\sin\theta} \frac{d}{d\theta} \left( \sin\theta \frac{d}{d\theta} \right) - \frac{m^2}{\sin^2\theta} \) \\
    Función peso & \( \rho(\theta) = \sin\theta \)\\
    Valores propios & \( \lambda = l(l+1) \) \\
    \hline
  \end{tabular}
\end{table}

\item \( R_{nl}(r) \)

  \begin{align*}
\frac{1}{r^2} \frac{d}{dr} \left( r^2 \frac{dR}{dr} \right) + \frac{2m}{\hbar^2} \left[ E - V(r) - \frac{\hbar^2}{2m} \frac{l(l+1)}{r^2} \right] R = 0.\\
\frac{d}{dr} \left( r^2 \frac{dR}{dr} \right) + \left[ \frac{2mr^2}{\hbar^2} (E - V(r)) - l(l+1) \right] R = 0.
  \end{align*}

\begin{table}[H]
  \centering
  \caption{caption}
  \label{tab:label}
  \begin{tabular}{c|c}
    \hline
    Nombre & Valor\\
    \hline
    Operador &  \( L = \frac{1}{r^2} \frac{d}{dr} \left( r^2 \frac{d}{dr} \right) - \frac{l(l+1)}{r^2} \)\\
    Función peso & \( \rho(r) = r^2 \)\\
    Valores propios & \( \lambda = E_{n,l} \) \\
    \hline
  \end{tabular}
\end{table}
\end{enumerate}

La energía es
\[
E_{n,l} = \frac{\hbar^2}{2m} k_{n,l}^2 + V_0,
\]

Donde \( k_{n,l} \) son los valores propios del problema radial y \( V_0 \) es el valor promedio del potencial.

Ahora, para la constante de normalización necesitamos:
\begin{align*}
  \int |\psi_{n,l,m}(r, \theta, \phi)|^2 dV &= 1\\
  dV &= r^2 \sin\theta \, dr \, d\theta \, d\phi\\
&\text{Por separación de Variables}\\
  |A_{n,l,m}|^2 & \int_a^b |R_{nl}(r)|^2 r^2 dr \int_0^\pi |P_l^m(\cos\theta)|^2 \sin\theta d\theta \int_0^{2\pi} |e^{im\phi}|^2 d\phi = 1\\
  \int_0^{2\pi} |e^{im\phi}|^2 d\phi &= 2\pi\\
  \int_0^\pi |P_l^m(\cos\theta)|^2 \sin\theta d\theta &= 1\\
  \int_a^b |R_{nl}(r)|^2 r^2 dr &= N_r\\
  |A_{n,l,m}|^2 2\pi N_r &= 1 \\
  |A_{n,l,m}|^2 &= \frac{1}{2\pi N_r}
.\end{align*}

\section{b}

Partimos de la función de onda:
\[
\psi_{n_\rho, n_z, m}(\rho, \phi, z) = A_{n_\rho, n_z, m} R_{n_\rho, m}(\rho) \sin\left(\frac{(n_z + 1)\pi z}{c}\right) e^{im\phi},
\]

Con lo que podemos desarrollar para cada variable:
\begin{enumerate}
  \item \( e^{im\phi} \)
\[
\frac{d^2 Y}{d\phi^2} + m^2 Y = 0. \implies Y(\phi) = e^{im\phi},
\]
\begin{table}[H]
  \centering
  \caption{caption}
  \label{tab:label}
  \begin{tabular}{c|c}
    \hline
    Nombre & Valor\\
    \hline
    Operador & \( L = \frac{d^2}{d\phi^2} \) \\
    Función peso & \( \rho(\phi) = 1 \) \\
    Valores propios & \( \lambda = -m^2 \)\\
    \hline
  \end{tabular}
\end{table}

\item $\sin\left(\frac{(n_z+1)\pi z}{c}\right)$
\[
\frac{d^2Z}{dz^2} + \left(\frac{(n_z+1)\pi}{c}\right)^2 Z = 0 \implies Z(z) = \sin\left(\frac{(n_z+1)\pi z}{c}\right), \quad n_z \in \mathbb{N}.
\]

\begin{table}[H]
  \centering
  \caption{caption}
  \label{tab:label}
  \begin{tabular}{c|c}
    \hline
    Nombre & Valor\\
    \hline
    Operador & \( L = \frac{d^2}{dz^2} \)\\
    Función peso & \( \rho(z) = 1 \)\\
    Valores propios & \( \lambda_{n_z} = \left(\frac{(n_z+1)\pi}{c}\right)^2 \)\\
    \hline
  \end{tabular}
\end{table}

\item $R_{n_\rho, m}\left( \rho \right) $
  \begin{align*}
	\frac{1}{\rho} \frac{d}{d\rho} \left( \rho \frac{dR}{d\rho} \right) + \left[ \frac{2m}{\hbar^2} \left(E - V(\rho)\right) - \frac{m^2}{\rho^2} \right] R = 0.\\
	\frac{d}{d\rho} \left( \rho \frac{dR}{d\rho} \right) + \left[ \frac{2m\rho}{\hbar^2} \left( E - V(\rho) \right) - \frac{m^2}{\rho^2} \right] R = 0.
  .\end{align*}

\begin{table}[H]
  \centering
  \caption{caption}
  \label{tab:label}
  \begin{tabular}{c|c}
    \hline
    Nombre & Valor\\
    \hline
    Operador & \( L = \frac{1}{\rho} \frac{d}{d\rho} \left( \rho \frac{d}{d\rho} \right) - \frac{m^2}{\rho^2} \)\\
    Función peso &  \( \rho(\rho) = \rho \)\\
    Valores propios & \\
    \hline
  \end{tabular}
\end{table}
\end{enumerate}

Ahora bien, para relacionar estos valores con la energia tenemos

\begin{align*}
  E_{n_\rho, n_z, m} = E_\rho + E_z + E_\phi\\
  E_z = \frac{\hbar^2}{2m} \left(\frac{(n_z + 1)\pi}{c}\right)^2\\
  E_\phi = \frac{\hbar^2}{2m} \frac{m^2}{\rho^2}
.\end{align*}

Para el caso de la contribución radial, dado que no encontré sus valores propios lo planteare como:
\[
E_{n_\rho, n_z, m} = \frac{\hbar^2}{2m} \left[ \lambda_{n_\rho, m} + \left(\frac{(n_z + 1)\pi}{c}\right)^2 \right].
\]

Lo que nos deja a la energía como:
\begin{align*}
  E_{n_\rho, n_z, m} &= \frac{\hbar^2}{2m} \left(\frac{(n_z + 1)\pi}{c}\right)^2 + \frac{\hbar^2}{2m} \frac{m^2}{\rho^2} + \frac{\hbar^2}{2m} \left[ \lambda_{n_\rho, m} + \left(\frac{(n_z + 1)\pi}{c}\right)^2 \right] \\
  E_{n_\rho, n_z, m} &= \frac{\hbar^2}{2m} \left(   \left(\frac{(n_z + 1)\pi}{c}\right)^2 + \frac{m^2}{\rho^2} + \left[ \lambda_{n_\rho, m} + \left(\frac{(n_z + 1)\pi}{c}\right)^2 \right] \right)\\
.\end{align*}

Para finalizar, en el caso de la constante de normalización tenemos:

\begin{align*}
  \int |\psi_{n_\rho, n_z, m}(\rho, \phi, z)|^2 dV = 1\\
  dV = \rho \, d\rho \, d\phi \, dz\\
  |A_{n_\rho, n_z, m}|^2 \int_a^b |R_{n_\rho, m}(\rho)|^2 \rho \, d\rho \int_0^{2\pi} |e^{im\phi}|^2 d\phi \int_0^c \sin^2\left(\frac{(n_z + 1)\pi z}{c}\right) dz = 1\\
   \int_0^{2\pi} |e^{im\phi}|^2 d\phi = 2\pi\\
   \int_0^c \sin^2\left(\frac{(n_z + 1)\pi z}{c}\right) dz = \frac{c}{2}\\
   \int_a^b |R_{n_\rho, m}(\rho)|^2 \rho \, d\rho = N_\rho\\
   |A_{n_\rho, n_z, m}|^2 2\pi \cdot \frac{c}{2} \cdot N_\rho = 1\\
   |A_{n_\rho, n_z, m}|^2 \pi c N_\rho = 1\\
   |A_{n_\rho, n_z, m}|^2 = \frac{1}{\pi c N_\rho}
.\end{align*}

\section{c}

Vamos a hacer el cambio de variable:
\begin{align*}
  x &= \frac{\rho - a}{b - a} \quad \text{para cilíndricas}\\
  x &= \frac{r - a}{b - a} \quad \text{para esféricas}
.\end{align*}

Con esto entonces podemos sustituir y queda
\[
\frac{d}{d\rho} \to \frac{1}{b - a} \frac{d}{dx}, \quad \rho \approx a
\frac{d}{dr} \to \frac{1}{b - a} \frac{d}{dx}, \quad r \approx a.
\]

Con eso podemos simplificar los operadores diferenciales del componente radial al mismo operador para ambos casos como:
\begin{align*}
  \frac{1}{\rho} \frac{d}{d\rho} \left( \rho \frac{d}{d\rho} \right)\\
  \frac{d}{dr^2} + \frac{2}{r} \frac{d}{dr}\\
  \implies L \approx \frac{d^2}{dx^2} + \kappa,
.\end{align*}

Donde $\kappa$ es una constante que depende de los valores iniciales

Esto nos permite simplificar la ecuación radial a un problema de Sturm-Liouville:

\begin{align*}
  \frac{d^2R}{dx^2} + \left( \lambda - \kappa \right) R = 0\\
  R(0) = R(1) = 0
.\end{align*}

La solución a esta ecuación es conocida y tiene valores propios dados por:
\[
\lambda_n = \pi^2 n^2, \quad n \in \mathbb{N}.
\]

Por lo tanto, la energía para este caso seria:
\begin{align*}
E_n = \frac{\hbar^2}{2m} \left( \frac{\pi^2 n^2}{(b - a)^2} + \kappa \right).
.\end{align*}

Con esta expresión podemos encontrar los valores de energía para cada uno de los casos

\begin{align*}
   E_{n,l,m} = \frac{\hbar^2}{2m} \left( \frac{\pi^2 n^2}{(b - a)^2} + \frac{l(l+1)}{a^2} \right) \quad\text{Para coordenadas esféricas} \\
   E_{n_\rho, n_z, m} = \frac{\hbar^2}{2m} \left( \frac{\pi^2 n_\rho^2}{(b - a)^2} + \left(\frac{(n_z + 1)\pi}{c}\right)^2 + \frac{m^2}{a^2} \right) \quad \text{Para coordenadas cilíndricas}
.\end{align*}

\end{document}

\documentclass{assignmeownt}

\coursenumber{CBCO 1030}
\coursetitle{Colombia en los medios}
\doctitle{Bono}
\docauthor{Sergio Montoya Ramirez}

\begin{document}
    \maketitle
    \thispagestyle{firststyle}
    
    Para mi fue una experiencia cuando menos singular. Hace relativamente poco en la universidad hubo una exposición sobre Marta Traba. Seguido de eso para mi vino una gran ilusión por conocer mas de lo que para mi era una de las personalidades mas interesantes de su momento. Por lo tanto cuando en la exposición narraban sobre como su postura tenia problemáticas y como en palabras de la exposición: "Para esa época era estar a la vanguardia parecer \textit{Primitivo} pero no serlo". Fue una gran sorpresa para mi. Por otro lado el gran enfoque artístico que tenían me sorprendió a gran medida pues aunque me encanta la diversidad y vivo con antropólogos (uno de los cuales me mantiene hablando de comunidades indígenas en Colombia pues participo en un proyecto con las mismas) no tengo un enfoque tan fuerte en el arte y por tanto mucho de ello era información extremadamente nueva para mi. Sin embargo, notaba una gran representación del interior privilegiado hacia afuera. Era una exposición extraña pues frecuentemente hablaban de representación mas allá de estas culturas representándose a si mismas.
\end{document}

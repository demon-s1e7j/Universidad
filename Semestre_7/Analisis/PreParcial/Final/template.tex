\documentclass{report}

\documentclass[12pt]{article}
\usepackage{array}
\usepackage{color}
\usepackage{amsthm}
\usepackage{eufrak}
\usepackage{lipsum}
\usepackage{pifont}
\usepackage{yfonts}
\usepackage{amsmath}
\usepackage{amssymb}
\usepackage{ccfonts}
\usepackage{comment} \usepackage{amsfonts}
\usepackage{fancyhdr}
\usepackage{graphicx}
\usepackage{listings}
\usepackage{mathrsfs}
\usepackage{setspace}
\usepackage{textcomp}
\usepackage{blindtext}
\usepackage{enumerate}
\usepackage{microtype}
\usepackage{xfakebold}
\usepackage{kantlipsum}
%\usepackage{draftwatermark}
\usepackage[spanish]{babel}
\usepackage[margin=1.5cm, top=2cm, bottom=2cm]{geometry}
\usepackage[framemethod=tikz]{mdframed}
\usepackage[colorlinks=true,citecolor=blue,linkcolor=red,urlcolor=magenta]{hyperref}

%//////////////////////////////////////////////////////
% Watermark configuration
%//////////////////////////////////////////////////////
%\SetWatermarkScale{4}
%\SetWatermarkColor{black}
%\SetWatermarkLightness{0.95}
%\SetWatermarkText{\texttt{Watermark}}

%//////////////////////////////////////////////////////
% Frame configuration
%//////////////////////////////////////////////////////
\newmdenv[tikzsetting={draw=gray,fill=white,fill opacity=0},backgroundcolor=none]{Frame}

%//////////////////////////////////////////////////////
% Font style configuration
%//////////////////////////////////////////////////////
\renewcommand{\familydefault}{\ttdefault}
\renewcommand{\rmdefault}{tt}

%//////////////////////////////////////////////////////
% Bold configuration
%//////////////////////////////////////////////////////
\newcommand{\fbseries}{\unskip\setBold\aftergroup\unsetBold\aftergroup\ignorespaces}
\makeatletter
\newcommand{\setBoldness}[1]{\def\fake@bold{#1}}
\makeatother

%//////////////////////////////////////////////////////
% Default font configuration
%//////////////////////////////////////////////////////
\DeclareFontFamily{\encodingdefault}{\ttdefault}{%
  \hyphenchar\font=\defaulthyphenchar
  \fontdimen2\font=0.33333em
  \fontdimen3\font=0.16667em
  \fontdimen4\font=0.11111em
  \fontdimen7\font=0.11111em}


%From M275 "Topology" at SJSU
\newcommand{\id}{\mathrm{id}}
\newcommand{\taking}[1]{\xrightarrow{#1}}
\newcommand{\inv}{^{-1}}

%From M170 "Introduction to Graph Theory" at SJSU
\DeclareMathOperator{\diam}{diam}
\DeclareMathOperator{\ord}{ord}
\newcommand{\defeq}{\overset{\mathrm{def}}{=}}

%From the USAMO .tex files
\newcommand{\ts}{\textsuperscript}
\newcommand{\dg}{^\circ}
\newcommand{\ii}{\item}

% % From Math 55 and Math 145 at Harvard
% \newenvironment{subproof}[1][Proof]{%
% \begin{proof}[#1] \renewcommand{\qedsymbol}{$\blacksquare$}}%
% {\end{proof}}

\newcommand{\liff}{\leftrightarrow}
\newcommand{\lthen}{\rightarrow}
\newcommand{\opname}{\operatorname}
\newcommand{\surjto}{\twoheadrightarrow}
\newcommand{\injto}{\hookrightarrow}
\newcommand{\On}{\mathrm{On}} % ordinals
\DeclareMathOperator{\img}{im} % Image
\DeclareMathOperator{\Img}{Im} % Image
\DeclareMathOperator{\coker}{coker} % Cokernel
\DeclareMathOperator{\Coker}{Coker} % Cokernel
\DeclareMathOperator{\Ker}{Ker} % Kernel
\DeclareMathOperator{\rank}{rank}
\DeclareMathOperator{\Spec}{Spec} % spectrum
\DeclareMathOperator{\Tr}{Tr} % trace
\DeclareMathOperator{\pr}{pr} % projection
\DeclareMathOperator{\ext}{ext} % extension
\DeclareMathOperator{\pred}{pred} % predecessor
\DeclareMathOperator{\dom}{dom} % domain
\DeclareMathOperator{\ran}{ran} % range
\DeclareMathOperator{\Hom}{Hom} % homomorphism
\DeclareMathOperator{\Mor}{Mor} % morphisms
\DeclareMathOperator{\End}{End} % endomorphism

\newcommand{\eps}{\epsilon}
\newcommand{\veps}{\varepsilon}
\newcommand{\ol}{\overline}
\newcommand{\ul}{\underline}
\newcommand{\wt}{\widetilde}
\newcommand{\wh}{\widehat}
\newcommand{\vocab}[1]{\textbf{\color{blue} #1}}
\providecommand{\half}{\frac{1}{2}}
\newcommand{\dang}{\measuredangle} %% Directed angle
\newcommand{\ray}[1]{\overrightarrow{#1}}
\newcommand{\seg}[1]{\overline{#1}}
\newcommand{\arc}[1]{\wideparen{#1}}
\DeclareMathOperator{\cis}{cis}
\DeclareMathOperator*{\lcm}{lcm}
\DeclareMathOperator*{\argmin}{arg min}
\DeclareMathOperator*{\argmax}{arg max}
\newcommand{\cycsum}{\sum_{\mathrm{cyc}}}
\newcommand{\symsum}{\sum_{\mathrm{sym}}}
\newcommand{\cycprod}{\prod_{\mathrm{cyc}}}
\newcommand{\symprod}{\prod_{\mathrm{sym}}}
\newcommand{\Qed}{\begin{flushright}\qed\end{flushright}}
\newcommand{\parinn}{\setlength{\parindent}{1cm}}
\newcommand{\parinf}{\setlength{\parindent}{0cm}}
% \newcommand{\norm}{\|\cdot\|}
\newcommand{\inorm}{\norm_{\infty}}
\newcommand{\opensets}{\{V_{\alpha}\}_{\alpha\in I}}
\newcommand{\oset}{V_{\alpha}}
\newcommand{\opset}[1]{V_{\alpha_{#1}}}
\newcommand{\lub}{\text{lub}}
\newcommand{\del}[2]{\frac{\partial #1}{\partial #2}}
\newcommand{\Del}[3]{\frac{\partial^{#1} #2}{\partial^{#1} #3}}
\newcommand{\deld}[2]{\dfrac{\partial #1}{\partial #2}}
\newcommand{\Deld}[3]{\dfrac{\partial^{#1} #2}{\partial^{#1} #3}}
\newcommand{\lm}{\lambda}
\newcommand{\uin}{\mathbin{\rotatebox[origin=c]{90}{$\in$}}}
\newcommand{\usubset}{\mathbin{\rotatebox[origin=c]{90}{$\subset$}}}
\newcommand{\lt}{\left}
\newcommand{\rt}{\right}
\newcommand{\paren}[1]{\left(#1\right)}
\newcommand{\bs}[1]{\boldsymbol{#1}}
\newcommand{\exs}{\exists}
\newcommand{\st}{\strut}
\newcommand{\dps}[1]{\displaystyle{#1}}

\newcommand{\sol}{\setlength{\parindent}{0cm}\textbf{\textit{Solution:}}\setlength{\parindent}{1cm} }
\newcommand{\solve}[1]{\setlength{\parindent}{0cm}\textbf{\textit{Solution: }}\setlength{\parindent}{1cm}#1 \Qed}

% Things Lie
\newcommand{\kb}{\mathfrak b}
\newcommand{\kg}{\mathfrak g}
\newcommand{\kh}{\mathfrak h}
\newcommand{\kn}{\mathfrak n}
\newcommand{\ku}{\mathfrak u}
\newcommand{\kz}{\mathfrak z}
\DeclareMathOperator{\Ext}{Ext} % Ext functor
\DeclareMathOperator{\Tor}{Tor} % Tor functor
\newcommand{\gl}{\opname{\mathfrak{gl}}} % frak gl group
\renewcommand{\sl}{\opname{\mathfrak{sl}}} % frak sl group chktex 6

% More script letters etc.
\newcommand{\SA}{\mathcal A}
\newcommand{\SB}{\mathcal B}
\newcommand{\SC}{\mathcal C}
\newcommand{\SF}{\mathcal F}
\newcommand{\SG}{\mathcal G}
\newcommand{\SH}{\mathcal H}
\newcommand{\OO}{\mathcal O}

\newcommand{\SCA}{\mathscr A}
\newcommand{\SCB}{\mathscr B}
\newcommand{\SCC}{\mathscr C}
\newcommand{\SCD}{\mathscr D}
\newcommand{\SCE}{\mathscr E}
\newcommand{\SCF}{\mathscr F}
\newcommand{\SCG}{\mathscr G}
\newcommand{\SCH}{\mathscr H}

% Mathfrak primes
\newcommand{\km}{\mathfrak m}
\newcommand{\kp}{\mathfrak p}
\newcommand{\kq}{\mathfrak q}

% number sets
\newcommand{\RR}[1][]{\ensuremath{\ifstrempty{#1}{\mathbb{R}}{\mathbb{R}^{#1}}}}
\newcommand{\NN}[1][]{\ensuremath{\ifstrempty{#1}{\mathbb{N}}{\mathbb{N}^{#1}}}}
\newcommand{\ZZ}[1][]{\ensuremath{\ifstrempty{#1}{\mathbb{Z}}{\mathbb{Z}^{#1}}}}
\newcommand{\QQ}[1][]{\ensuremath{\ifstrempty{#1}{\mathbb{Q}}{\mathbb{Q}^{#1}}}}
\newcommand{\CC}[1][]{\ensuremath{\ifstrempty{#1}{\mathbb{C}}{\mathbb{C}^{#1}}}}
\newcommand{\PP}[1][]{\ensuremath{\ifstrempty{#1}{\mathbb{P}}{\mathbb{P}^{#1}}}}
\newcommand{\HH}[1][]{\ensuremath{\ifstrempty{#1}{\mathbb{H}}{\mathbb{H}^{#1}}}}
\newcommand{\FF}[1][]{\ensuremath{\ifstrempty{#1}{\mathbb{F}}{\mathbb{F}^{#1}}}}
% expected value
\newcommand{\EE}{\ensuremath{\mathbb{E}}}
\newcommand{\charin}{\text{ char }}
\DeclareMathOperator{\sign}{sign}
\DeclareMathOperator{\Aut}{Aut}
\DeclareMathOperator{\Inn}{Inn}
\DeclareMathOperator{\Syl}{Syl}
\DeclareMathOperator{\Gal}{Gal}
\DeclareMathOperator{\GL}{GL} % General linear group
\DeclareMathOperator{\SL}{SL} % Special linear group

%---------------------------------------
% BlackBoard Math Fonts :-
%---------------------------------------

%Captital Letters
\newcommand{\bbA}{\mathbb{A}}	\newcommand{\bbB}{\mathbb{B}}
\newcommand{\bbC}{\mathbb{C}}	\newcommand{\bbD}{\mathbb{D}}
\newcommand{\bbE}{\mathbb{E}}	\newcommand{\bbF}{\mathbb{F}}
\newcommand{\bbG}{\mathbb{G}}	\newcommand{\bbH}{\mathbb{H}}
\newcommand{\bbI}{\mathbb{I}}	\newcommand{\bbJ}{\mathbb{J}}
\newcommand{\bbK}{\mathbb{K}}	\newcommand{\bbL}{\mathbb{L}}
\newcommand{\bbM}{\mathbb{M}}	\newcommand{\bbN}{\mathbb{N}}
\newcommand{\bbO}{\mathbb{O}}	\newcommand{\bbP}{\mathbb{P}}
\newcommand{\bbQ}{\mathbb{Q}}	\newcommand{\bbR}{\mathbb{R}}
\newcommand{\bbS}{\mathbb{S}}	\newcommand{\bbT}{\mathbb{T}}
\newcommand{\bbU}{\mathbb{U}}	\newcommand{\bbV}{\mathbb{V}}
\newcommand{\bbW}{\mathbb{W}}	\newcommand{\bbX}{\mathbb{X}}
\newcommand{\bbY}{\mathbb{Y}}	\newcommand{\bbZ}{\mathbb{Z}}

%---------------------------------------
% MathCal Fonts :-
%---------------------------------------

%Captital Letters
\newcommand{\mcA}{\mathcal{A}}	\newcommand{\mcB}{\mathcal{B}}
\newcommand{\mcC}{\mathcal{C}}	\newcommand{\mcD}{\mathcal{D}}
\newcommand{\mcE}{\mathcal{E}}	\newcommand{\mcF}{\mathcal{F}}
\newcommand{\mcG}{\mathcal{G}}	\newcommand{\mcH}{\mathcal{H}}
\newcommand{\mcI}{\mathcal{I}}	\newcommand{\mcJ}{\mathcal{J}}
\newcommand{\mcK}{\mathcal{K}}	\newcommand{\mcL}{\mathcal{L}}
\newcommand{\mcM}{\mathcal{M}}	\newcommand{\mcN}{\mathcal{N}}
\newcommand{\mcO}{\mathcal{O}}	\newcommand{\mcP}{\mathcal{P}}
\newcommand{\mcQ}{\mathcal{Q}}	\newcommand{\mcR}{\mathcal{R}}
\newcommand{\mcS}{\mathcal{S}}	\newcommand{\mcT}{\mathcal{T}}
\newcommand{\mcU}{\mathcal{U}}	\newcommand{\mcV}{\mathcal{V}}
\newcommand{\mcW}{\mathcal{W}}	\newcommand{\mcX}{\mathcal{X}}
\newcommand{\mcY}{\mathcal{Y}}	\newcommand{\mcZ}{\mathcal{Z}}


%---------------------------------------
% Bold Math Fonts :-
%---------------------------------------

%Captital Letters
\newcommand{\bmA}{\boldsymbol{A}}	\newcommand{\bmB}{\boldsymbol{B}}
\newcommand{\bmC}{\boldsymbol{C}}	\newcommand{\bmD}{\boldsymbol{D}}
\newcommand{\bmE}{\boldsymbol{E}}	\newcommand{\bmF}{\boldsymbol{F}}
\newcommand{\bmG}{\boldsymbol{G}}	\newcommand{\bmH}{\boldsymbol{H}}
\newcommand{\bmI}{\boldsymbol{I}}	\newcommand{\bmJ}{\boldsymbol{J}}
\newcommand{\bmK}{\boldsymbol{K}}	\newcommand{\bmL}{\boldsymbol{L}}
\newcommand{\bmM}{\boldsymbol{M}}	\newcommand{\bmN}{\boldsymbol{N}}
\newcommand{\bmO}{\boldsymbol{O}}	\newcommand{\bmP}{\boldsymbol{P}}
\newcommand{\bmQ}{\boldsymbol{Q}}	\newcommand{\bmR}{\boldsymbol{R}}
\newcommand{\bmS}{\boldsymbol{S}}	\newcommand{\bmT}{\boldsymbol{T}}
\newcommand{\bmU}{\boldsymbol{U}}	\newcommand{\bmV}{\boldsymbol{V}}
\newcommand{\bmW}{\boldsymbol{W}}	\newcommand{\bmX}{\boldsymbol{X}}
\newcommand{\bmY}{\boldsymbol{Y}}	\newcommand{\bmZ}{\boldsymbol{Z}}
%Small Letters
\newcommand{\bma}{\boldsymbol{a}}	\newcommand{\bmb}{\boldsymbol{b}}
\newcommand{\bmc}{\boldsymbol{c}}	\newcommand{\bmd}{\boldsymbol{d}}
\newcommand{\bme}{\boldsymbol{e}}	\newcommand{\bmf}{\boldsymbol{f}}
\newcommand{\bmg}{\boldsymbol{g}}	\newcommand{\bmh}{\boldsymbol{h}}
\newcommand{\bmi}{\boldsymbol{i}}	\newcommand{\bmj}{\boldsymbol{j}}
\newcommand{\bmk}{\boldsymbol{k}}	\newcommand{\bml}{\boldsymbol{l}}
\newcommand{\bmm}{\boldsymbol{m}}	\newcommand{\bmn}{\boldsymbol{n}}
\newcommand{\bmo}{\boldsymbol{o}}	\newcommand{\bmp}{\boldsymbol{p}}
\newcommand{\bmq}{\boldsymbol{q}}	\newcommand{\bmr}{\boldsymbol{r}}
\newcommand{\bms}{\boldsymbol{s}}	\newcommand{\bmt}{\boldsymbol{t}}
\newcommand{\bmu}{\boldsymbol{u}}	\newcommand{\bmv}{\boldsymbol{v}}
\newcommand{\bmw}{\boldsymbol{w}}	\newcommand{\bmx}{\boldsymbol{x}}
\newcommand{\bmy}{\boldsymbol{y}}	\newcommand{\bmz}{\boldsymbol{z}}

%---------------------------------------
% Scr Math Fonts :-
%---------------------------------------

\newcommand{\sA}{{\mathscr{A}}}   \newcommand{\sB}{{\mathscr{B}}}
\newcommand{\sC}{{\mathscr{C}}}   \newcommand{\sD}{{\mathscr{D}}}
\newcommand{\sE}{{\mathscr{E}}}   \newcommand{\sF}{{\mathscr{F}}}
\newcommand{\sG}{{\mathscr{G}}}   \newcommand{\sH}{{\mathscr{H}}}
\newcommand{\sI}{{\mathscr{I}}}   \newcommand{\sJ}{{\mathscr{J}}}
\newcommand{\sK}{{\mathscr{K}}}   \newcommand{\sL}{{\mathscr{L}}}
\newcommand{\sM}{{\mathscr{M}}}   \newcommand{\sN}{{\mathscr{N}}}
\newcommand{\sO}{{\mathscr{O}}}   \newcommand{\sP}{{\mathscr{P}}}
\newcommand{\sQ}{{\mathscr{Q}}}   \newcommand{\sR}{{\mathscr{R}}}
\newcommand{\sS}{{\mathscr{S}}}   \newcommand{\sT}{{\mathscr{T}}}
\newcommand{\sU}{{\mathscr{U}}}   \newcommand{\sV}{{\mathscr{V}}}
\newcommand{\sW}{{\mathscr{W}}}   \newcommand{\sX}{{\mathscr{X}}}
\newcommand{\sY}{{\mathscr{Y}}}   \newcommand{\sZ}{{\mathscr{Z}}}


%---------------------------------------
% Math Fraktur Font
%---------------------------------------

%Captital Letters
\newcommand{\mfA}{\mathfrak{A}}	\newcommand{\mfB}{\mathfrak{B}}
\newcommand{\mfC}{\mathfrak{C}}	\newcommand{\mfD}{\mathfrak{D}}
\newcommand{\mfE}{\mathfrak{E}}	\newcommand{\mfF}{\mathfrak{F}}
\newcommand{\mfG}{\mathfrak{G}}	\newcommand{\mfH}{\mathfrak{H}}
\newcommand{\mfI}{\mathfrak{I}}	\newcommand{\mfJ}{\mathfrak{J}}
\newcommand{\mfK}{\mathfrak{K}}	\newcommand{\mfL}{\mathfrak{L}}
\newcommand{\mfM}{\mathfrak{M}}	\newcommand{\mfN}{\mathfrak{N}}
\newcommand{\mfO}{\mathfrak{O}}	\newcommand{\mfP}{\mathfrak{P}}
\newcommand{\mfQ}{\mathfrak{Q}}	\newcommand{\mfR}{\mathfrak{R}}
\newcommand{\mfS}{\mathfrak{S}}	\newcommand{\mfT}{\mathfrak{T}}
\newcommand{\mfU}{\mathfrak{U}}	\newcommand{\mfV}{\mathfrak{V}}
\newcommand{\mfW}{\mathfrak{W}}	\newcommand{\mfX}{\mathfrak{X}}
\newcommand{\mfY}{\mathfrak{Y}}	\newcommand{\mfZ}{\mathfrak{Z}}
%Small Letters
\newcommand{\mfa}{\mathfrak{a}}	\newcommand{\mfb}{\mathfrak{b}}
\newcommand{\mfc}{\mathfrak{c}}	\newcommand{\mfd}{\mathfrak{d}}
\newcommand{\mfe}{\mathfrak{e}}	\newcommand{\mff}{\mathfrak{f}}
\newcommand{\mfg}{\mathfrak{g}}	\newcommand{\mfh}{\mathfrak{h}}
\newcommand{\mfi}{\mathfrak{i}}	\newcommand{\mfj}{\mathfrak{j}}
\newcommand{\mfk}{\mathfrak{k}}	\newcommand{\mfl}{\mathfrak{l}}
\newcommand{\mfm}{\mathfrak{m}}	\newcommand{\mfn}{\mathfrak{n}}
\newcommand{\mfo}{\mathfrak{o}}	\newcommand{\mfp}{\mathfrak{p}}
\newcommand{\mfq}{\mathfrak{q}}	\newcommand{\mfr}{\mathfrak{r}}
\newcommand{\mfs}{\mathfrak{s}}	\newcommand{\mft}{\mathfrak{t}}
\newcommand{\mfu}{\mathfrak{u}}	\newcommand{\mfv}{\mathfrak{v}}
\newcommand{\mfw}{\mathfrak{w}}	\newcommand{\mfx}{\mathfrak{x}}
\newcommand{\mfy}{\mathfrak{y}}	\newcommand{\mfz}{\mathfrak{z}}

\newcommand{\Q}{\mathbb{Q}}

\title{\Huge{Some Class}\\Random Examples}
\author{\huge{Your Name}}
\date{}

\begin{document}

\maketitle
\newpage% or \cleardoublepage
% \pdfbookmark[<level>]{<title>}{<dest>}
\pdfbookmark[section]{\contentsname}{toc}
\tableofcontents
\pagebreak

\chapter{Parcial 1}
\section{Punto 1}
\subsection{Parte a}

Tenemos la función \[
f = 
\begin{cases}
  1 & x \in \mathbb{Q}\cap\left[ 0, 1 \right] \\
  0 & x \in \left[ 0, 1 \right] \backslash \Q,
\end{cases}
.\] 

En este caso utilizaremos el criterio de Lebesgue para la integrabilidad de Riemman
\dfn{Criterio de Lebesgue para la integrabilidad de Riemman}{
  Un $f$ acotado es Riemman-Integrable en $\left[ a, b \right] $ si y solo si el conjunto de discontinuidades de $f$ tiene una medida de Lebesgue igual a $0$
}

Sea $x \in \Q\cap\left[ 0, 1 \right] $ .Tome, $\left( x_n \right)_{n}$ una su sucesión de números irracionales entre $\left[ 0, 1 \right] $ tal que $x_n \to x$. Por lo tanto $f\left( x_n \right) = 1$ lo que se hace distinto de $0$ por lo que $f$ no es continuo en $x$.

De manera similar, para $x \in \left[ 0, 1 \right] \backslash \mathbb{Q}$ tome una sucesión con las mismas características pero esta vez de números racionales por lo que $f\left( x_n \right) = 0$ lo que significa que $f$ no es continuo en $x$.

Con esto entonces sabemos que $f$ es discontinuo en todo el conjunto $\left[ 0, 1 \right] $ por lo que el conjunto de discontinuidades tiene una medida de $ 1 - 0 = 1$ lo cual es mayor a $0$ y en consecuencia no puede ser integrable.
\subsection{Parte b}

En este caso tenemos ya de por si definido el conjunto de discontinuidades. Es decir $D = \left\{ x_n ; x_n \in \left( x_n \right)_{n \in \mathbb{N}} \right\} $. Este conjunto es claramente contable. Dado que es contable podemos definirlo como la suma de múltiples conjuntos singleton y por las propiedades de la medida de Lebesgue todos estos se suman y la suma contable de 0 es igual 0. Por lo tanto, este conjunto tiene una medida de $0$ y esta función es integrable por el criterio de Lebesgue

\section{Punto 2}
\dfn{Serie de Taylor}{La serie de Taylor de una función $f$ centrada en $a$ es \[
f\left( x \right) = \sum_{n=0}^{\infty} \frac{f^{(n)}\left( a \right) }{n!}x^{n}
.\] }
\dfn{Test de Ratio}{Sea \[
L = \lim_{n \to \infty} \left| \frac{a_{n + 1}}{a_n} \right| 
.\] 
Entonces
\begin{enumerate}
  \item $L < 1$ la serie converge de manera absoluta
  \item $L > 1$ la serie diverge de manera absoluta
  \item $L = 1$ el test no da conclusiones interesantes
\end{enumerate}
}

Con esto entonces iniciemos por aventurarnos en $f^{(n)}$
\begin{align*}
  f\left( x \right) &= \ln\left( 1 + x \right) \\
  f^{(1)}\left( x \right) &= \frac{1}{1 + x} \\
  f^{(2)}\left( x \right) &= -1 \left( 1 + x \right)^{-2} \\
  f^{(3)}\left( x \right) &= 2\left( 1 + x \right)^{-3} \\
  f^{(4)}\left( x \right) &= - 2\cdot 3\left( 1 + x \right)^{-4} 
.\end{align*}

Con esto entonces vemos el patrón que $\left( -1 \right)^{n + 1}\left( n - 1 \right)! \left( 1 + x \right) $. Ademas notemos que si $x = 0$ (como se nos pide por centrar la función en $0$ ) entonces todos estos valores queda $\left( -1 \right)^{n + 1}\left( n - 1 \right)!$. Ahora si ponemos esto en la definición de serie de Taylor conseguimos:
\begin{align*}
  f\left( x \right) &= \sum_{n=1}^{\infty} \frac{\left( -1 \right)^{n + 1}\left( n - 1 \right)!}{n!}x^{n} \\
  f\left( x \right) &= \sum_{n=1}^{\infty} \frac{\left( -1 \right)^{n + 1}}{n}x^{n} \\
.\end{align*}

Ahora, para encontrar el radio de convergencia utilizaremos el test de ratio
\begin{align*}
  \lim_{n \to \infty} \left| \frac{\left( -1 \right)^{(n + 2)}x^{n + 1}}{n + 1}\cdot \frac{n}{\left( -1 \right)^{n}x^{n}} \right| &= \lim_{n \to \infty} \left| \frac{n}{n + 1}\cdot x \right|  \\
  &= |x| \\
.\end{align*}

Como queremos que $L < 1$ entonces $\left| x \right| < 1$ y eso nos dice que el radio de convergencia es $\left( -1, 1 \right) $

\section{Punto 3}
Como nos piden tomemos el polinomio de Taylor de grado 2. De nuevo recordemos entonces que lo que buscamos es
\begin{align*}
  e^{-x} &= f\left( 0 \right) + \frac{f'\left( 0 \right) }{1!}x^{1} + \frac{f''\left( 0 \right) }{2!}x^{2} + R_n\left( x \right) \\
  e^{-0} &= 1 \\
  f'\left( x \right) &= -e^{-x} \implies -e^{-0} = -1 \\
  f''\left( x \right) &= e^{-x} \implies e^{-0} = 1 \\
  e^{-x} &= 1 - x + \frac{1}{2}x^2 + R_n\left( x \right) \\
  \int_{0}^{1} e^{-x} dx &= \int_{0}^{1} 1 - x + \frac{1}{2}x^2 + R_n\left( x \right)  \\
  &= \int_{0}^{1} 1 dx - \int_{0}^{1} x dx + \frac{1}{2}\int_{0}^{1} x^2 dx + \int_{0}^{1} R_n\left( x \right) dx\\
  &= \left[ x \right]_{0}^{1} - \left[ \frac{x^2}{2} \right]_{0}^{1} + \frac{1}{2}\left[ \frac{x^3}{3} \right]_{0}^{1} + \int_{0}^{1} R_n\left( x \right)  \\
  &= 1 - \frac{1}{2} + \frac{1}{2}\cdot \frac{1}{3} + \int_{0}^{1}R_n\left( x \right)  \\
  &= 1 - \frac{1}{2} + \frac{1}{6} + \int_{0}^{1} R_{n}\left( x \right)  \\
  &= \frac{1}{2} + \frac{1}{6} + \int_{0}^{1}R_n\left( x \right)  \\
  &= \frac{6 + 2}{12} + \int_{0}^{1}R_n\left( x \right)  \\
  &= \frac{8}{12} + \int_{0}^{1}R_n\left( x \right)  \\
.\end{align*}

Ahora bien $R_n$ es un termino genérico que llevamos escondiendo. En verdad es $R_2$ y su definición es
\dfn{$R_n$ }{
$R_n = \frac{f^{(n + 1)}\left( c \right) }{\left( n + 1 \right)!}\left( x - a \right)^{n + 1}$
}

Por lo tanto:
\begin{align*}
  R_2 &= \frac{f^{(3)}\left( c \right) }{\left( 3 \right)!}\left( x - 0 \right)^{3}\\
  &= \frac{-e^{-c}}{6}x^{3} \\
.\end{align*}

Con lo cual
\begin{align*}
  \int_{0}^{1}R_2\left( x \right) &= \int_{0}^{1} \frac{-e^{-c}}{6}x^{3} \\
  &= \frac{-e^{-c}}{6}\int_{0}^{1} x^{3} \\
  &= \frac{-e^{-c}}{6} \left[ \frac{x^{4}}{4} \right]_{0}^{1} \\
  &= \frac{-e^{-c}}{6}\cdot \frac{1}{4} \\
  &= \frac{-e^{-c}}{24} \\
  &= -\frac{1}{24\cdot e^{c}} \\
.\end{align*}

Sabiendo que $e^{c}$ es una serie creciente entonces el valor mas grande lo toma cuando $e$ es mas grande (por que estamos en negativos) por lo tanto el error aproximado es: \[
- \frac{1}{24\cdot e}
.\] con lo cual llegamos a que el resultado debe diferenciarse por esto.
\section{Punto 4}
\subsection{Parte a}

Sea \[
R = \limsup_{n \to \infty}\left| \frac{a_{n + 1}}{a_n} \right| < 1
.\] Sea $R^{*}$ tal que $R < R^{*} < 1$. Por la propia definición de $\limsup$ sabemos que existe un $N \in \mathbb{N}$ tal que si $n \ge N$ entonces: \[
\sup_{k \ge n}\left\{ \left| \frac{a_{k + 1}}{a_k} \right|  \right\} \le R^{*}
.\] ahora por la definición de supremo sabemos que para todo $n \ge  N$ se da que: \[
\left| \frac{a_{n + 1}}{a_n} \right| \le R^{*} \implies \left| a_{n + 1} \right| \le R^{*}\left| a_n \right| 
.\] y ahora con esto podemos desarrollar para $R^{*n}$ de la misma forma y con esto tener  \[
R^{*n}\left| a_N \right| \ge \left| a_{N} \right| 
.\] pero la serie \[
\sum_{n=1}^{\infty} R^{*n}\left| a_N \right| 
.\] converge como serie geométrica dado que $R^{*} < 1$ por lo tanto, por test de comparación la otra serie también converge.

\subsection{Parte b}
Mostremos que 
\begin{align*}
  \frac{n!}{n^{n}} &= \frac{n}{n}\cdot \frac{n - 1}{n} \cdot \frac{n - 2}{n} \cdot \ldots \cdot \frac{1}{n} \\
		   &< 1 \cdot 1 \cdot 1 \cdot \ldots \cdot \frac{1}{n} = \frac{1}{n}
.\end{align*}

Mostramos que esta serie siempre es menor a $\frac{1}{n}$ por lo que por test de comparación esta debe converger a $0$

\end{document}

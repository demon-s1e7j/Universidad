\documentclass{report}

\documentclass[12pt]{article}
\usepackage{array}
\usepackage{color}
\usepackage{amsthm}
\usepackage{eufrak}
\usepackage{lipsum}
\usepackage{pifont}
\usepackage{yfonts}
\usepackage{amsmath}
\usepackage{amssymb}
\usepackage{ccfonts}
\usepackage{comment} \usepackage{amsfonts}
\usepackage{fancyhdr}
\usepackage{graphicx}
\usepackage{listings}
\usepackage{mathrsfs}
\usepackage{setspace}
\usepackage{textcomp}
\usepackage{blindtext}
\usepackage{enumerate}
\usepackage{microtype}
\usepackage{xfakebold}
\usepackage{kantlipsum}
%\usepackage{draftwatermark}
\usepackage[spanish]{babel}
\usepackage[margin=1.5cm, top=2cm, bottom=2cm]{geometry}
\usepackage[framemethod=tikz]{mdframed}
\usepackage[colorlinks=true,citecolor=blue,linkcolor=red,urlcolor=magenta]{hyperref}

%//////////////////////////////////////////////////////
% Watermark configuration
%//////////////////////////////////////////////////////
%\SetWatermarkScale{4}
%\SetWatermarkColor{black}
%\SetWatermarkLightness{0.95}
%\SetWatermarkText{\texttt{Watermark}}

%//////////////////////////////////////////////////////
% Frame configuration
%//////////////////////////////////////////////////////
\newmdenv[tikzsetting={draw=gray,fill=white,fill opacity=0},backgroundcolor=none]{Frame}

%//////////////////////////////////////////////////////
% Font style configuration
%//////////////////////////////////////////////////////
\renewcommand{\familydefault}{\ttdefault}
\renewcommand{\rmdefault}{tt}

%//////////////////////////////////////////////////////
% Bold configuration
%//////////////////////////////////////////////////////
\newcommand{\fbseries}{\unskip\setBold\aftergroup\unsetBold\aftergroup\ignorespaces}
\makeatletter
\newcommand{\setBoldness}[1]{\def\fake@bold{#1}}
\makeatother

%//////////////////////////////////////////////////////
% Default font configuration
%//////////////////////////////////////////////////////
\DeclareFontFamily{\encodingdefault}{\ttdefault}{%
  \hyphenchar\font=\defaulthyphenchar
  \fontdimen2\font=0.33333em
  \fontdimen3\font=0.16667em
  \fontdimen4\font=0.11111em
  \fontdimen7\font=0.11111em}


\input{macros}
\input{letterfonts}
\newcommand{\Q}{\mathbb{Q}}

\title{\Huge{Some Class}\\Random Examples}
\author{\huge{Your Name}}
\date{}

\begin{document}

\maketitle
\newpage% or \cleardoublepage
% \pdfbookmark[<level>]{<title>}{<dest>}
\pdfbookmark[section]{\contentsname}{toc}
\tableofcontents
\pagebreak

\chapter{Parcial 1}
\section{Punto 1}
\subsection{Parte a}

Tenemos la función \[
f = 
\begin{cases}
  1 & x \in \mathbb{Q}\cap\left[ 0, 1 \right] \\
  0 & x \in \left[ 0, 1 \right] \backslash \Q,
\end{cases}
.\] 

En este caso utilizaremos el criterio de Lebesgue para la integrabilidad de Riemman
\dfn{Criterio de Lebesgue para la integrabilidad de Riemman}{
  Un $f$ acotado es Riemman-Integrable en $\left[ a, b \right] $ si y solo si el conjunto de discontinuidades de $f$ tiene una medida de Lebesgue igual a $0$
}

Sea $x \in \Q\cap\left[ 0, 1 \right] $ .Tome, $\left( x_n \right)_{n}$ una su sucesión de números irracionales entre $\left[ 0, 1 \right] $ tal que $x_n \to x$. Por lo tanto $f\left( x_n \right) = 1$ lo que se hace distinto de $0$ por lo que $f$ no es continuo en $x$.

De manera similar, para $x \in \left[ 0, 1 \right] \backslash \mathbb{Q}$ tome una sucesión con las mismas características pero esta vez de números racionales por lo que $f\left( x_n \right) = 0$ lo que significa que $f$ no es continuo en $x$.

Con esto entonces sabemos que $f$ es discontinuo en todo el conjunto $\left[ 0, 1 \right] $ por lo que el conjunto de discontinuidades tiene una medida de $ 1 - 0 = 1$ lo cual es mayor a $0$ y en consecuencia no puede ser integrable.
\subsection{Parte b}

En este caso tenemos ya de por si definido el conjunto de discontinuidades. Es decir $D = \left\{ x_n ; x_n \in \left( x_n \right)_{n \in \mathbb{N}} \right\} $. Este conjunto es claramente contable. Dado que es contable podemos definirlo como la suma de múltiples conjuntos singleton y por las propiedades de la medida de Lebesgue todos estos se suman y la suma contable de 0 es igual 0. Por lo tanto, este conjunto tiene una medida de $0$ y esta función es integrable por el criterio de Lebesgue

\section{Punto 2}
\dfn{Serie de Taylor}{La serie de Taylor de una función $f$ centrada en $a$ es \[
f\left( x \right) = \sum_{n=0}^{\infty} \frac{f^{(n)}\left( a \right) }{n!}x^{n}
.\] }
\dfn{Test de Ratio}{Sea \[
L = \lim_{n \to \infty} \left| \frac{a_{n + 1}}{a_n} \right| 
.\] 
Entonces
\begin{enumerate}
  \item $L < 1$ la serie converge de manera absoluta
  \item $L > 1$ la serie diverge de manera absoluta
  \item $L = 1$ el test no da conclusiones interesantes
\end{enumerate}
}

Con esto entonces iniciemos por aventurarnos en $f^{(n)}$
\begin{align*}
  f\left( x \right) &= \ln\left( 1 + x \right) \\
  f^{(1)}\left( x \right) &= \frac{1}{1 + x} \\
  f^{(2)}\left( x \right) &= -1 \left( 1 + x \right)^{-2} \\
  f^{(3)}\left( x \right) &= 2\left( 1 + x \right)^{-3} \\
  f^{(4)}\left( x \right) &= - 2\cdot 3\left( 1 + x \right)^{-4} 
.\end{align*}

Con esto entonces vemos el patrón que $\left( -1 \right)^{n + 1}\left( n - 1 \right)! \left( 1 + x \right) $. Ademas notemos que si $x = 0$ (como se nos pide por centrar la función en $0$ ) entonces todos estos valores queda $\left( -1 \right)^{n + 1}\left( n - 1 \right)!$. Ahora si ponemos esto en la definición de serie de Taylor conseguimos:
\begin{align*}
  f\left( x \right) &= \sum_{n=1}^{\infty} \frac{\left( -1 \right)^{n + 1}\left( n - 1 \right)!}{n!}x^{n} \\
  f\left( x \right) &= \sum_{n=1}^{\infty} \frac{\left( -1 \right)^{n + 1}}{n}x^{n} \\
.\end{align*}

Ahora, para encontrar el radio de convergencia utilizaremos el test de ratio
\begin{align*}
  \lim_{n \to \infty} \left| \frac{\left( -1 \right)^{(n + 2)}x^{n + 1}}{n + 1}\cdot \frac{n}{\left( -1 \right)^{n}x^{n}} \right| &= \lim_{n \to \infty} \left| \frac{n}{n + 1}\cdot x \right|  \\
  &= |x| \\
.\end{align*}

Como queremos que $L < 1$ entonces $\left| x \right| < 1$ y eso nos dice que el radio de convergencia es $\left( -1, 1 \right) $

\section{Punto 3}
Como nos piden tomemos el polinomio de Taylor de grado 2. De nuevo recordemos entonces que lo que buscamos es
\begin{align*}
  e^{-x} &= f\left( 0 \right) + \frac{f'\left( 0 \right) }{1!}x^{1} + \frac{f''\left( 0 \right) }{2!}x^{2} + R_n\left( x \right) \\
  e^{-0} &= 1 \\
  f'\left( x \right) &= -e^{-x} \implies -e^{-0} = -1 \\
  f''\left( x \right) &= e^{-x} \implies e^{-0} = 1 \\
  e^{-x} &= 1 - x + \frac{1}{2}x^2 + R_n\left( x \right) \\
  \int_{0}^{1} e^{-x} dx &= \int_{0}^{1} 1 - x + \frac{1}{2}x^2 + R_n\left( x \right)  \\
  &= \int_{0}^{1} 1 dx - \int_{0}^{1} x dx + \frac{1}{2}\int_{0}^{1} x^2 dx + \int_{0}^{1} R_n\left( x \right) dx\\
  &= \left[ x \right]_{0}^{1} - \left[ \frac{x^2}{2} \right]_{0}^{1} + \frac{1}{2}\left[ \frac{x^3}{3} \right]_{0}^{1} + \int_{0}^{1} R_n\left( x \right)  \\
  &= 1 - \frac{1}{2} + \frac{1}{2}\cdot \frac{1}{3} + \int_{0}^{1}R_n\left( x \right)  \\
  &= 1 - \frac{1}{2} + \frac{1}{6} + \int_{0}^{1} R_{n}\left( x \right)  \\
  &= \frac{1}{2} + \frac{1}{6} + \int_{0}^{1}R_n\left( x \right)  \\
  &= \frac{6 + 2}{12} + \int_{0}^{1}R_n\left( x \right)  \\
  &= \frac{8}{12} + \int_{0}^{1}R_n\left( x \right)  \\
.\end{align*}

Ahora bien $R_n$ es un termino genérico que llevamos escondiendo. En verdad es $R_2$ y su definición es
\dfn{$R_n$ }{
$R_n = \frac{f^{(n + 1)}\left( c \right) }{\left( n + 1 \right)!}\left( x - a \right)^{n + 1}$
}

Por lo tanto:
\begin{align*}
  R_2 &= \frac{f^{(3)}\left( c \right) }{\left( 3 \right)!}\left( x - 0 \right)^{3}\\
  &= \frac{-e^{-c}}{6}x^{3} \\
.\end{align*}

Con lo cual
\begin{align*}
  \int_{0}^{1}R_2\left( x \right) &= \int_{0}^{1} \frac{-e^{-c}}{6}x^{3} \\
  &= \frac{-e^{-c}}{6}\int_{0}^{1} x^{3} \\
  &= \frac{-e^{-c}}{6} \left[ \frac{x^{4}}{4} \right]_{0}^{1} \\
  &= \frac{-e^{-c}}{6}\cdot \frac{1}{4} \\
  &= \frac{-e^{-c}}{24} \\
  &= -\frac{1}{24\cdot e^{c}} \\
.\end{align*}

Sabiendo que $e^{c}$ es una serie creciente entonces el valor mas grande lo toma cuando $e$ es mas grande (por que estamos en negativos) por lo tanto el error aproximado es: \[
- \frac{1}{24\cdot e}
.\] con lo cual llegamos a que el resultado debe diferenciarse por esto.
\section{Punto 4}
\subsection{Parte a}

Sea \[
R = \limsup_{n \to \infty}\left| \frac{a_{n + 1}}{a_n} \right| < 1
.\] Sea $R^{*}$ tal que $R < R^{*} < 1$. Por la propia definición de $\limsup$ sabemos que existe un $N \in \mathbb{N}$ tal que si $n \ge N$ entonces: \[
\sup_{k \ge n}\left\{ \left| \frac{a_{k + 1}}{a_k} \right|  \right\} \le R^{*}
.\] ahora por la definición de supremo sabemos que para todo $n \ge  N$ se da que: \[
\left| \frac{a_{n + 1}}{a_n} \right| \le R^{*} \implies \left| a_{n + 1} \right| \le R^{*}\left| a_n \right| 
.\] y ahora con esto podemos desarrollar para $R^{*n}$ de la misma forma y con esto tener  \[
R^{*n}\left| a_N \right| \ge \left| a_{N} \right| 
.\] pero la serie \[
\sum_{n=1}^{\infty} R^{*n}\left| a_N \right| 
.\] converge como serie geométrica dado que $R^{*} < 1$ por lo tanto, por test de comparación la otra serie también converge.

\subsection{Parte b}
Mostremos que 
\begin{align*}
  \frac{n!}{n^{n}} &= \frac{n}{n}\cdot \frac{n - 1}{n} \cdot \frac{n - 2}{n} \cdot \ldots \cdot \frac{1}{n} \\
		   &< 1 \cdot 1 \cdot 1 \cdot \ldots \cdot \frac{1}{n} = \frac{1}{n}
.\end{align*}

Mostramos que esta serie siempre es menor a $\frac{1}{n}$ por lo que por test de comparación esta debe converger a $0$

\end{document}

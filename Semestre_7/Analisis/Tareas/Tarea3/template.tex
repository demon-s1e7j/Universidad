\documentclass{report}

\documentclass[12pt]{article}
\usepackage{array}
\usepackage{color}
\usepackage{amsthm}
\usepackage{eufrak}
\usepackage{lipsum}
\usepackage{pifont}
\usepackage{yfonts}
\usepackage{amsmath}
\usepackage{amssymb}
\usepackage{ccfonts}
\usepackage{comment} \usepackage{amsfonts}
\usepackage{fancyhdr}
\usepackage{graphicx}
\usepackage{listings}
\usepackage{mathrsfs}
\usepackage{setspace}
\usepackage{textcomp}
\usepackage{blindtext}
\usepackage{enumerate}
\usepackage{microtype}
\usepackage{xfakebold}
\usepackage{kantlipsum}
%\usepackage{draftwatermark}
\usepackage[spanish]{babel}
\usepackage[margin=1.5cm, top=2cm, bottom=2cm]{geometry}
\usepackage[framemethod=tikz]{mdframed}
\usepackage[colorlinks=true,citecolor=blue,linkcolor=red,urlcolor=magenta]{hyperref}

%//////////////////////////////////////////////////////
% Watermark configuration
%//////////////////////////////////////////////////////
%\SetWatermarkScale{4}
%\SetWatermarkColor{black}
%\SetWatermarkLightness{0.95}
%\SetWatermarkText{\texttt{Watermark}}

%//////////////////////////////////////////////////////
% Frame configuration
%//////////////////////////////////////////////////////
\newmdenv[tikzsetting={draw=gray,fill=white,fill opacity=0},backgroundcolor=none]{Frame}

%//////////////////////////////////////////////////////
% Font style configuration
%//////////////////////////////////////////////////////
\renewcommand{\familydefault}{\ttdefault}
\renewcommand{\rmdefault}{tt}

%//////////////////////////////////////////////////////
% Bold configuration
%//////////////////////////////////////////////////////
\newcommand{\fbseries}{\unskip\setBold\aftergroup\unsetBold\aftergroup\ignorespaces}
\makeatletter
\newcommand{\setBoldness}[1]{\def\fake@bold{#1}}
\makeatother

%//////////////////////////////////////////////////////
% Default font configuration
%//////////////////////////////////////////////////////
\DeclareFontFamily{\encodingdefault}{\ttdefault}{%
  \hyphenchar\font=\defaulthyphenchar
  \fontdimen2\font=0.33333em
  \fontdimen3\font=0.16667em
  \fontdimen4\font=0.11111em
  \fontdimen7\font=0.11111em}


\input{macros}
\input{letterfonts}

\title{\Huge{Análisis}\\Tarea 3}
\author{\huge{Sergio Montoya Ramírez}}
\date{}

\begin{document}

\maketitle
\newpage% or \cleardoublepage
% \pdfbookmark[<level>]{<title>}{<dest>}
\pdfbookmark[section]{\contentsname}{toc}
\tableofcontents
\pagebreak

\chapter{Problema 1}
\section{Enunciado}
Demuestre el siguiente teorema:
\thm{}{Sea $\left\{ a_n \right\} $ y $\left\{ b_n \right\} $ sucesiones tales que $a_n > 0$ y $b_n > 0$. Si \[
\displaystyle\lim_{n\to \infty}\frac{a_n}{b_n}= L \neq 0
.\] entonce $\sum_n a_n$ converge si y solo si $\sum_n b_n$ converge}
\section{Solución}

\dfn{Test de Comparación}{Suponga que existe un entero $N$ tal que $0 \le a_n \le b_n$ para todo $n \ge N$. Si $\displaystyle \sum_{n=1}^{\infty} b_n$ converge entonces $\displaystyle \sum_{n=1}^{\infty} a_n$ converge.}

En este caso vamos a tomar la sucesión $\frac{a_n}{b_n}$ como una sucesión que converge a $L$. Ademas, dado que $a_n > 0$ y $b_n > 0$ sabemos que $L > 0$. Ahora bien, dado que esta serie converge a $L$ sabemos que existe un $N$ tal que para todo $n \ge N$ se cumple: \[
\frac{1}{2}L \le  \frac{a_n}{b_n} \le 2L
.\] y con esto sabiendo que $b_n > 0$ entonces podemos encontrar rápidamente 
\begin{align*}
  \frac{1}{2}Lb_n \le \frac{a_n}{b_n}b_n \le 2Lb_n\\
  \frac{1}{2}Lb_n \le a_n \le 2Lb_n
.\end{align*}

Ahora con esto, podemos dividir la demostración para cada caso.
\begin{enumerate}
  \item[$\implies$] Sea $\left\{ a_n \right\} $ una sucesión que converge. Ahora bien, sabemos que para un $N$ se cumple que $\frac{1}{2}Lb_n \le a_n$ y dado que $\frac{1}{2}L$ es una constante entonces podemos saber por el test de comparación que $b_n$ converge.
  \item[$\impliedby$] Sea $\left\{ b_n \right\} $ una sucesión que converge. Ahora bien, sabemos que para un $N$ se cumple que $a_n \le 2L b_n$ y dado que $2L$ es una constante entonces podemos saber por el test de comparación que $a_n$ converge.
\end{enumerate}

\chapter{Problema 2}
\section{Enunciado}
Sea \[
x_{n+1} = \frac{1}{2}x_n + \frac{1}{x_n};\ x_0 = 2
.\] 
\begin{enumerate}
  \item[\textbf{a.}] Demuestre que $x_n^2\ge 2$. \textit{Ayuda:} considere la ecuación: \[
  x_n^2 - 2x_{n+1}x_n + 2 = 0
.\] donde la incógnita es $x_n$. ¿Que puede decir del discriminante de dicha ecuación.
  \item[\textbf{b.}] Demuestre que $x_{n+1}\le x_n$, el punto anterior puede ser de ayuda.
  \item[\textbf{c.}] Demuestre que la sucesión $\left( x_n \right)_n$ es convergente y que su limite es $\sqrt{2} $
\end{enumerate}
\section{Solución}

\begin{enumerate}
  \item[\textbf{a.}] %TODO
  \item[\textbf{b.}] %TODO
  \item[\textbf{c.}] %TODO
\end{enumerate}


\chapter{Problema 3}
\section{Enunciado}
Cada racional $x$ puede ser escrito en la forma $x = \frac{m}{n}$, donde $n > 0$, con $m$ y $n$ enteros sin divisores en común. Cuando $x = 0$, tomamos $n = 1$. Considere la función $f$ definida en $R^{1}$ por \[
f(x) =
\begin{cases}
  0&\quad x\in \mathbb{Q}'\\
  \frac{1}{n}&\quad \left( x = \frac{m}{n} \right) 
\end{cases}
.\] Pruebe que $f$ es continuo en cada punto irracional, y que $f$ tiene una discontinuidad simple en cada punto racional.
\section{Solución}

En este caso, lo que nos pide el enunciado es esencialmente lo mismo que mostrar que $\displaystyle \lim_{t \to x} f\left( t \right) = 0 $ para todo $x$. Para esto sea $\varepsilon > 0$ y sea $x$ cualquier numero  real. Sea $N$ el único entero positivo tal que $N < \frac{1}{\varepsilon} < N + 1$ y para cada entero positivo $n = 1, 2, 3, \ldots, N$ sea $k_n$ un entero tal que \[
\frac{k_n}{n} \le x \le  \frac{k_n + 1}{n}
.\] Entonces, por cada $n$ sea $\delta_n = \frac{1}{n}$ si $x = \frac{k_n}{n}$, en cualquier otro caso sea $\delta_n = \min\left( x - \frac{k_n}{n}, \frac{k_n + 1}{n} - x \right) $. Finalmente, sea $\delta = \min\left( \delta_1, \ldots, \delta_N \right) $. Ahora con todo esto definido, podemos decir que $\left| f\left( t \right)  \right| < \varepsilon$ si $0 < \left| x - t \right| < \delta$. Esto es bastante claro para un numero irracional, sin embargo, si $t$ es racional y $t = \frac{m}{n}$, tenemos necesariamente  que $n > N$ por la manera en la que escogimos $\delta_n$ para  $n \le  N$. Por lo tanto, si $t$ es racional, entonces $f(t) \le \frac{1}{N + 1} < \epsilon$. Con lo que completamos la prueba.

Una vez tenemos que $\displaystyle \lim_{t \to x} f(t) = 0$. Entonces nos siguen ambas cosas pues para cualquier irracional se cumple que $\displaystyle \lim_{t \to x} f\left( t \right) = f\left( t \right) = 0$ y lo contrario ocurre para cualquier racional.

\chapter{Problema 4}
\section{Enunciado}

\dfn{Propiedad del valor Intermedio}{Si $f(a)<c<f(b)$, entonces $f(x) = c$ para algún $x$ entre $a$ y $b$}

Sea $f$ una función real con dominio en $R^{1}$ que tiene la propiedad del valor intermedio. Suponga también, para cada racional $r$, que el conjunto de todos los $x$ con $f(x)=r$ es cerrado. Pruebe que $f$ es continuo.

\subsection{Ayuda}
Si $x_n \to x_0$ pero $f\left( x_n \right) > r > f\left( x_0 \right) $ para algún $r$ y todo $n$, entonces  $f\left( t_n \right) = r$ para algún $t_n$ entre $x_0$ y $x_n$ ; por lo tanto $t_n \to  x_0$. Encuentre una contradicción.

\section{Solución}

La contradicción que se nos pide encontrar en la ayuda esta en que $x_0$ es un punto limite del conjunto de $t$ tal que $f\left( t \right) = r$, sin embargo, $x_0$ no pertenece al conjunto. Lo anterior, es una contradicción directa con que este conjunto es cerrado.

\chapter{Problema 5}
\section{Enunciado}

Asuma que $f$ es una función real continua definida en $\left( a, b \right) $ tal que \[
f\left( \frac{x + y}{2} \right) \le \frac{f(x) + f(y)}{2}
.\] para todo $x, y \in \left( a,b \right) $. Pruebe que $f$ es convexo.
\section{Solución}


\chapter{Problema 6}
\section{Enunciado}

Sea \[
X = \left\{ f:\left[ 0, 2\pi \right] \to R: \text{$f$ continua} \right\} 
.\] Naturalmente $X$ tiene una estructura de espacio vectorial.
\begin{enumerate}
  \item[\textbf{a.}] Defina \[
      \left<f, g \right> = \int_0^{2\pi}f\left( x \right) g\left( x \right) dx	
    .\] Muestre que $\left< \cdot, \cdot\right>$ define un producto interno sobre $X$.
  \item[\textbf{b.}] A partir del producto interno definido en el punto anterior, se puede definir una norma como sigue: \[
      \left| \left| u \right|  \right|  = \sqrt{\left<u, u \right>} 
      .\] y a partir de dicha norma una métrica: \[
      d(u, v) =  \left| \left| u - v \right|  \right| 
      .\] Calcule para $m \neq n$ \[
      \left| \left| \sin\left( mx \right) - \sin\left( nx \right)  \right|  \right| 
    .\] 
  \item[\textbf{c.}] Concluya que ninguna bola cerrada centrada en el origen de $X$ es compacta. ¿Tiene $X$ dimensión finita?
\end{enumerate}


\end{document}

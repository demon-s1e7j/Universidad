\documentclass[12pt]{exam}
\usepackage{amsthm}
\usepackage{libertine}
\usepackage[utf8]{inputenc}
\usepackage[margin=1in]{geometry} \usepackage{amsmath,amssymb}
\usepackage{multicol}
\usepackage[shortlabels]{enumitem}
\usepackage{siunitx}
\usepackage{cancel}
\usepackage{graphicx}
\usepackage[spanish]{babel}
\usepackage{pgfplots}
\usepackage{listings}
\usepackage{tikz}


\pgfplotsset{width=10cm,compat=1.9}
\usepgfplotslibrary{external}
\tikzexternalize

\newcommand{\class}{Metodos Matematicos} % This is the name of the course 
\newcommand{\examnum}{Tarea 1} % This is the name of the assignment
\newcommand{\examdate}{\today} % This is the due date
\newcommand{\timelimit}{}





\begin{document}
\pagestyle{plain}
\thispagestyle{empty}

\noindent
\begin{tabular*}{\textwidth}{l @{\extracolsep{\fill}} r @{\extracolsep{6pt}} l}
	\textbf{\class} & \textbf{Nombre:} & \textit{Sergio Montoya}\\ %Your name here instead, obviously 
	\textbf{\examnum} &&\\
	\textbf{\examdate} &&
\end{tabular*}\\
\rule[2ex]{\textwidth}{2pt}
% ---

\section*{Pregunta 1}

\section*{Pregunta 2}

\section*{Pregunta 3}

\section*{Pregunta 4}

\section*{Pregunta 5}

\section*{Pregunta 6}

\section*{Pregunta 7}

\section*{Pregunta 8}

\section*{Pregunta 9}

\section*{Pregunta 10}

\subsection*{Parte A}

En este caso, partimos de que tenemos la integral de una multiplicación. Por lo tanto, podemos aplicar integración por partes. De modo que nos queda:
\begin{align*}
	\oint u dv &= uv - \oint vdu\\
	u &= \ln(z)\\
	du &= \frac{1}{z} dz\\
	dv &= f'(z) dz\\
	v &= f(z) \\
	\oint \left(\ln(z)\right)f'(z)dz &= \ln(z)f(z) - \oint \frac{f(z)}{z}dz\\
	&= \ln(z)f(z) - 2\pi i f(0)\\
	&= 2\pi i f(z_0) - 2 \pi i f(0)\\
	&= 2\pi i (f(z_0) - f(0))
\end{align*}

En este caso, el logaritmo natural equivale a $2\pi i$ pues al poner todo el circulo en la integral (que es C) este seria el resultado. Por otro lado la integral tiene un $f(0)$ por la formula de cauchy para integrales.

\subsection*{Parte B}

Sea $b$ cualquier punto distinto a $a$ en el vecindario definido. Sea $p$ la distancia entre $a$ y $b$. Si $C_p$ denota el circulo orientado positivamente $|b - a| = p$, centrado en $a$ y que pasa por $b$ la formula integral de Cauchy nos dice que:
\begin{align*}
	f(a) = \frac{1}{2\pi i}\int_{C_p}\frac{f(z)dz}{z - a}
\end{align*}

y la representación parametrica nos permitiria expresar esto como:
\begin{align*}
	f(a) = \frac{1}{2\pi}\int_0^{2\pi}f(a + pe^{i\theta})d\theta
\end{align*}

Ahora, notamos de esta ultima expresión
\begin{align*}
	|f(a)|\le \frac{1}{2\pi}\int_{0}^{2\pi}|f(a+pe^{i\theta})| d\theta
\end{align*}

Ahora, dado que $$|f(a+pe^{i\theta})|\le|f(a)|$$ encontramos que $$\int_0^{2\pi}|f(a + pe^{i\theta})|d\theta \le \int_0^{2\pi}|f(a)|d\theta=2\pi |f(a)|$$ por lo tanto $$|f(a)|\ge\frac{1}{2\pi}\int_0^{2\pi}|f(a+pe^{i\theta})|d\theta$$ ademas, por estas dos inecuaciónes queda $$|f(a)| = \frac{1}{2\pi}\int_0^{2\pi} |f(a + pe^{i\theta})|d\theta$$ lo cual nos lleva a concluir que $$|f(a+pe^{i\theta})|=|f(a)|$$ por lo tanto todos los puntos del circulo $|a - b| = p$ tiene el mismo valor. Ademas dado que $b$ puede ser cualquier punto entonces todos los puntos de este contorno valen exactamente lo mismo $f(a)$. Esta demostración fue adaptada del libro \textit{Complex Variables and Applications} de \textit{Churchill}.



\end{document}

  \documentclass[12pt]{exam}
\usepackage{amsthm}
\usepackage{libertine}
\usepackage[utf8]{inputenc}
\usepackage[margin=1in]{geometry}
\usepackage{amsmath,amssymb}
\usepackage{multicol}
\usepackage[shortlabels]{enumitem}
\usepackage[spanish]{babel}
\usepackage{siunitx}
\usepackage{cancel}
\usepackage{graphicx}
\usepackage{pgfplots}
\usepackage{listings}
\usepackage{tikz}


\pgfplotsset{width=10cm,compat=1.9}
\usepgfplotslibrary{external}
\tikzexternalize

\newcommand{\class}{Metodos Matematicos} % This is the name of the course 
\newcommand{\examnum}{Tarea 4} % This is the name of the assignment
\newcommand{\examdate}{\today} % This is the due date
\newcommand{\timelimit}{}





\begin{document}
\pagestyle{plain}
\thispagestyle{empty}

\noindent
\begin{tabular*}{\textwidth}{l @{\extracolsep{\fill}} r @{\extracolsep{6pt}} l}
	\textbf{\class} & \textbf{Name:} & \textit{Sergio Montoya}\\ %Your name here instead, obviously 
	\textbf{\examnum} &&\\
	\textbf{\examdate} &&
\end{tabular*}\\
\rule[2ex]{\textwidth}{2pt}
% ---

\section{}
\subsection{}

Saliendo de la función:\[   
f(x) = 
     \begin{cases}
       -k&\quad -\pi< x < 0 \\
       k &\quad 0 < x < \pi\\
     \end{cases}
\]

entonces nos queda que la transformada de Fourier de esta señal queda unicamente entre $-\pi$ y $\pi$
\begin{align*}
  b_n &= \frac{1}{2\pi}\int_{-\pi}^{\pi}f(x)\sin\left(\frac{n\pi x}{2}\right)dx\\
  &= \frac{k}{2\pi} \left[-\int_{-\pi}^0\frac{\sin(u)du}{n\pi} + \int_0^\pi \frac{\sin(u)du}{n\pi}\right]
\end{align*}


\end{document}

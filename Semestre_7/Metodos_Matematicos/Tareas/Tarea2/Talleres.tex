\documentclass[12pt]{exam}
\usepackage{amsthm}
\usepackage{libertine}
\usepackage[utf8]{inputenc}
\usepackage[margin=1in]{geometry} \usepackage{amsmath,amssymb}
\usepackage{multicol}
\usepackage[shortlabels]{enumitem}
\usepackage{siunitx}
\usepackage{cancel}
\usepackage{graphicx}
\usepackage[spanish]{babel}
\usepackage{pgfplots}
\usepackage{listings}
\usepackage{tikz}


\pgfplotsset{width=10cm,compat=1.9}
\usepgfplotslibrary{external}
\tikzexternalize

\newcommand{\class}{Metodos Matematicos} % This is the name of the course 
\newcommand{\examnum}{Tarea 1} % This is the name of the assignment
\newcommand{\examdate}{\today} % This is the due date
\newcommand{\timelimit}{}





\begin{document}
\pagestyle{plain}
\thispagestyle{empty}

\noindent
\begin{tabular*}{\textwidth}{l @{\extracolsep{\fill}} r @{\extracolsep{6pt}} l}
	\textbf{\class} & \textbf{Nombre:} & \textit{Sergio Montoya}\\ %Your name here instead, obviously 
	\textbf{\examnum} &&\\
	\textbf{\examdate} &&
\end{tabular*}\\
\rule[2ex]{\textwidth}{2pt}
% ---

\section*{Pregunta 1}

\section*{Pregunta 2}

\section*{Pregunta 3}

\subsection*{$\ln(z + 1)$}

En este caso, necesitamos:
\begin{align*}
  f(0) &= \ln(1 + 0) = 0\\
  f'(0) &= (1 + 0)^{-1} = 1\\
  f''(0) &= -(1 + 0)^{-2} = -1\\
  f'''(0) &= 2(1 + 0)^{-3} = 2\\
  f^{(n)}(0) &= (-1)^{n+1}(n-1)!
\end{align*}

Por lo tanto en la formula queda como:

\begin{align*}
  \ln(1 + x) = \sum_{n=1}^{\infty} (-1)^{n + 1}\frac{x^n}{n}
\end{align*}

\section*{Pregunta 4}

\section*{Pregunta 5}

\section*{Pregunta 6}

\section*{Pregunta 7}

\section*{Pregunta 8}

\section*{Pregunta 9}

\section*{Pregunta 10}
\subsection*{1.6.2.1}
\subsubsection*{a)}

En este caso nos interesa ver que $\Gamma(z + 1) = z\Gamma(z)$ por lo tanto integremos por partes
\begin{align*}
  \Gamma(z + 1) &= \int_{0}^{\infty} e^{-t} t^{z + 1 - 1} dt\\
  &= \int_0^\infty e^{-t} t^{z} dt\\
  u(t) &= t^{z}\\
  du &= zt^{z - 1}\\
  dv &= e^{-t}\\
  v(t) &= -e^{-t}\\
  \int udv &= uv - \int vdu\\
  \int_0^\infty e^{-t} t^{z} dt &= t^{z}\cdot(-e^{-t}) - \int_0^\infty -e^{-t}zt^{z - 1}dt\\
  &= -t^{z}e^{-t} + z\int_0^\infty e^{-t}t^{z-1}dt\\
  &= -t^{z}e^{-t} + z\Gamma(z)
\end{align*}

Ahora ya vamos en buena parte de la demostración. Solo nos falta mostrar que $-t^ze^{-t}$ tiende a 0.
\begin{align*}
  -t^{z}e^{-t} &= \frac{-t^z}{e^{-t}}\\
\end{align*}

Aqui en este caso podemos aplicar L'Hopital $z$ veces y con eso nos quedaria una constante (Sabemos que llegaremos a una constante pues $-t^{z}$ es un polinomio) y por tanto esta serie al intentar llegar a infinito se hara $0$. Ademas, cuando t = 0 este tambien se hara 0 por el primer termino.

\subsubsection*{b)}
\begin{align*}
  \Gamma(1) &= \int_0^\infty e^{-t}t^{1 - 1}dt\\
  &= \int_0^\infty e^{-t}t^{0}dt\\
  &= \int_0^\infty e^{-t}dt\\
  &= -e^{-t}|_0^\infty\\
  &= -e^{-\infty} - (-e^{0})\\
  &= 1
\end{align*}
\subsubsection*{c)}

En este caso para un $n$ entero vamos a hacer una pequeña inducción. El caso base lo demostramos en el punto anterior pues $1 = 1!$. Ahora suponga que funciona para $n - 1$ y entonces $\Gamma(n) = (n - 1)!$. Ahora deduzcamos utilizando el apartado \textbf{a}:
\begin{align*}
  \Gamma(n + 1) &= n\Gamma(n)\\
  \Gamma(n + 1) &= n\cdot (n - 1)!\\
  \Gamma(n + 1) &= n!
\end{align*}

Por lo tanto, esto queda demostrado.

\subsubsection*{d)}

En este caso nos interesa mostrar lo que nos piden para los naturales. Por lo tanto lo mas prudente es hacer una demostración por inducción. Mostremos que para $n = 1$ esto se cumple.
\begin{align*}
  \Gamma(z) &= \frac{\Gamma(z + 1)}{(z)}\\
  &= \frac{z\Gamma(z)}{z}\\
  &= \Gamma(z)
\end{align*}

Ahora supongamos que esto funciona para $n - 1$ por lo tanto
$$
\Gamma(z) = \frac{\Gamma(z + (n - 1))}{(z + n - 2)(z + n - 3)}
$$

Ahora lo debemos intentar demostrar para $n$
\begin{align*}
  \Gamma(z) &= \frac{\Gamma(z + n)}{(z + n - 1)(z + n - 2)\ldots z}\\
  &= \frac{(z + n - 1)\Gamma(z + n - 1)}{(z + n - 1)(z + n - 2)\ldots z}\\
  &= \frac{\Gamma(z + n - 1)}{(z + n - 2)(z + n - 3)\ldots  z}
\end{align*}

Que asumimos en el paso inductivo que era verdad por lo que sabemos que es cierto y esto se demuestra.

Ahora para la otra parte es bastante mas sencillo. Simplemente tenemos que saber que para cualquier numero real (Como $\mathbb{Re}(z)$) existe un numero natural que sea mayor que este numero. Cosa que es bastante clara por la propia definición de natural.

Por ultimo, esta prolongación no se puede hacer para $z = 0$ pues en ese caso estariamos dividiendo por $0$ cosa que da indefinido

\subsubsection*{e)}

Ahora en este caso utilizaremos la definición de residuo que es:
\begin{align*}
  Res(f(z_0)) &= \lim_{z \to z_0} (z - z_0) f(z)\\
  &= \lim_{z \to -m} (z + m) \frac{\Gamma(z + n)}{(z + n - 1)(z + n - 2)\ldots(z + m)\ldots(z + 1)z}\\
  &= (z + m)\Gamma(z)\\
  z + m &= 0\\
  &= 0
\end{align*}

\subsection*{1.6.3.1}

En este caso vamos a asumir que 
\begin{align*}
  \Gamma(x) = 2\int_0^\infty u^{2x - 1} e^{-u^2} du
  \Gamma(y) = 2\int_0^\infty v^{2y - 1} e^{-v^2} dv
\end{align*}

Esta representación es una de las muchas definiciones que puede tomar $\Gamma$. Ahora para mostrar esto note que 
\begin{align*}
  \int \alpha f(x)dx = \alpha \int f(x) dx
  \alpha = \int g(y) dy \\
  \int g(y) dy \int f(x) dx = \int \int g(y) f(x) dy dx
\end{align*}

Por lo tanto solo tendriamos que reemplazar
\begin{align*}
  2\int_0^\infty u^{2x - 1} e^{-u^2} du 2\int_0^\infty v^{2y - 1} e^{-v^2} dv &= 4 \int_0^\infty \int_0^\infty u^{2x - 1}e^{-v^2}v^{2y - 1} e^{-v^2} du dv\\
  &= 4 \int_0^\infty \int_0^\infty u^{2x - 1}v^{2y - 1} e^{-v^2 - u^2} du dv
\end{align*}



\end{document}

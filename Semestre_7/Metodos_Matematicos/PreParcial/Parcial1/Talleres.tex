\documentclass[12pt]{exam}
\usepackage{amsthm}
\usepackage{libertine}
\usepackage[utf8]{inputenc}
\usepackage[margin=1in]{geometry} \usepackage{amsmath,amssymb}
\usepackage{multicol}
\usepackage[shortlabels]{enumitem}
\usepackage{siunitx}
\usepackage{cancel}
\usepackage{graphicx}
\usepackage[spanish]{babel}
\usepackage{pgfplots}
\usepackage{listings}
\usepackage{tikz}


\pgfplotsset{width=10cm,compat=1.9}
\usepgfplotslibrary{external}
\tikzexternalize

\newcommand{\class}{Metodos Matematicos} % This is the name of the course 
\newcommand{\examnum}{PreParcial 1} % This is the name of the assignment
\newcommand{\examdate}{\today} % This is the due date
\newcommand{\timelimit}{}





\begin{document}
\pagestyle{plain}
\thispagestyle{empty}

\noindent
\begin{tabular*}{\textwidth}{l @{\extracolsep{\fill}} r @{\extracolsep{6pt}} l}
	\textbf{\class} & \textbf{Nombre:} & \textit{Sergio Montoya}\\ %Your name here instead, obviously 
	\textbf{\examnum} &&\\
	\textbf{\examdate} &&
\end{tabular*}\\
\rule[2ex]{\textwidth}{2pt}
% ---

\section*{Primera Punto}
\subsection*{Parte A}

\textbf{Definición:} Una función es armonica si satisface la ecuación de Laplace. Esta ecuación dice basicamente $\nabla^2 u = 0$ donde esto es equivalente a $u_{xx} + u_{yy} = 0$.

Por la definición anterior sabemos que lo que no interesa es saber las derivadas de esta función. Lo que nos da:
\begin{align*}
  U(x, y) &= xy + 2x + 2y\\
  U_x(x, y) &= y + 2 \\
  U_{xx} (x, y) &= 0 \\
  U_y(x, y) &= x + 2 \\
  U_{yy}(x, y) &= 0\\
  U_{xx} + U_{yy} &= 0\\
  0 + 0 &= 0.
\end{align*}

\subsection*{Parte B}

\textbf{Definición:} Para encontrar la armonica conjugada debemos encontrar un $V$ tal que cumpla las ecuaciones de Cauchy - Riemman junto con $U$.

\begin{align*}
  U_x &= V_y\\
  U_y &= -V_x\\
  V_y &= x + 2\\
  V_x &= -(y + 2)\\
  v &= \int x + 2 dy\\
  v &= yx + 2y + g(x)\\
  v_x &= y + g'(x)\\
  -y - 2 &= y + g'(x)\\
  -2y - 2 &= g'(x)\\
  g(x) &= \int -2y - 2 dx\\
  g(x) &= -2yx - 2x + c
\end{align*}

Por otro lado

\begin{align*}
  v &= yx + 2y -2yx - 2x + c\\
  v &= -yx + 2(y - x) + c\\
  v &= \int -y - 2 dx \\
  v &= -yx - 2x + g(y)\\
  v_y &= -x + g'(y)\\
  x + 2 &= -x + g'(y)\\
  2x + 2 &= g'(y)\\
  2xy + 2y + c &= g(y)\\
  v &= -yx - 2x + 2xy + 2y + c\\
  v &= xy - 2x + 2y + c
\end{align*}


\end{document}

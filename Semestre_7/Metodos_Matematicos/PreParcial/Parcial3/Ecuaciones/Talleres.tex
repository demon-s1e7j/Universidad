  \documentclass[12pt]{exam}
\usepackage{amsthm}
\usepackage{libertine}
\usepackage[utf8]{inputenc}
\usepackage[margin=1in]{geometry}
\usepackage{amsmath,amssymb}
\usepackage{multicol}
\usepackage[spanish]{babel}
\usepackage[shortlabels]{enumitem}
\usepackage{siunitx}
\usepackage{cancel}
\usepackage{graphicx}
\usepackage{pgfplots}
\usepackage{listings}
\usepackage{tikz}


\pgfplotsset{width=10cm,compat=1.9}
\usepgfplotslibrary{external}
\tikzexternalize

\newcommand{\class}{Metodos Matematicos} % This is the name of the course 
\newcommand{\examnum}{Hoja de ecuaciones} % This is the name of the assignment
\newcommand{\examdate}{\today} % This is the due date
\newcommand{\timelimit}{}




\begin{document}
\pagestyle{plain}
\thispagestyle{empty}

\noindent
\begin{tabular*}{\textwidth}{l @{\extracolsep{\fill}} r @{\extracolsep{6pt}} l}
	\textbf{\class} & \textbf{Name:} & \textit{Sergio Montoya}\\ %Your name here instead, obviously 
	\textbf{\examnum} &&\\
	\textbf{\examdate} &&
\end{tabular*}\\
\rule[2ex]{\textwidth}{2pt}
% ---

\subsubsection*{Potencial Eléctrico}

  \textbf{Potencial de una carga puntual:} $V = \frac{kq}{r}$ donde $k = 8.99 \times 10^{9} N \cdot \frac{m^2}{c^2}$

  \textbf{Potencial de N cargas puntuales:} $V = \sum_{n=1}^{N} V_n$

  \textbf{Dipolo Eléctrico:} $V_p = kq\left( \frac{1}{r \sqrt{1 + \cos\left( \theta \right) \frac{d}{r}+ \left( \frac{a}{2r} \right)^2}} - \frac{1}{r \sqrt{1 - \cos\left( \theta \right) \frac{d}{r}+ \left( \frac{a}{2r} \right)^2}} \right)$
  
  \textbf{Dipolo Eléctrico con $r \gg a$ :} $V_p= k \frac{qd\cos\left( \theta \right) }{r^2}$
\end{document}
